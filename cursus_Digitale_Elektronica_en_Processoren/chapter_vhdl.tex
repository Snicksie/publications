\chapter{VHDL}
\chapterquote{In redelijke taal weerklinkt wat in werkelijkheid gebeurt.}{Gerard Bolland, Nederlands taalkundige en filosoof (1854-1922)}
\section{Elementen van de VHDL-taal}
In deze sectie zullen we eerst de woordenschat in de VHDL taal bespreken, waarna we de belangrijkste types zullen bespreken samen met het 
\subsection{Lexicale elementen (woordenschat)}
Elk code-bestand\footnote{Onafhankelijk van de taal.} bestaat uit een sequentie van lexicale elementen die men \termen{tokens} oemt. We zullen eerst de verschillende lexicale elementen die in VHDL-code kunnen voorkomen beschrijven. Deze kunnen we dan als terminologie gebruiken. De tokens kunnen verder worden onderverdeeld in: \termen{literals} (deze vertegenwoordigen bepaalde waardes zoals getallen, tekst,...), \termen{identifiers} (dit zijn namen van variabelen, functies, componenten,... gebruikt om naar te refereren en defini\"eren), \termen{scheidingstekens}\footnote{Engels: \emph{delimiters}.} (haakjes en andere symbolen die waardes en identifiers combineren) en \termen{commentaar}. We beschrijven de verschillende soorten tokens verder in de volgende subsecties.
\subsection{Literals}
We onderscheiden vier soorten literals: getallen, karakters, karakterreeksen en bitreeksen. We overlopen de verschillende types.
\paragraph{Getal}Een getal (ook wel ``\vhdltermen{abstract literal}'' genoemd) is een numerieke waarde. Net als bij programmeertalen onderscheid men verschillende soorten numerieke waarden:
\begin{itemize}
  \item \vhdltermen{universal\_integer}: dit zijn numerieke literals die geen decimale komma (\vhdltermen{.}) bevatten, bijvoorbeeld \verb+1425+. Deze getallen zijn vergelijkbaar met het \texttt{int}-type in \texttt{Java}\footnote{}.
  \item \vhdltermen{universal\_real}: dit zijn numerieke getallen die wel een decimale komma bevatten. Bijvoorbeeld \verb+1425.1917+. Merk op dat indien er enkel nullen achter de komma staan dat het getal dezelfde waarde heeft als een \vhdltermen{universal\_integer}: \verb+1425.0+, maar daarom niet dezelfde representatie. In \texttt{Java} zouden we dit type voorstellen met bijvoorbeeld een \texttt{float} of \texttt{double}.
  \item \vhdltermen{physical types}: dit zijn fysische waarden. Fysische waarden bestaan uit een getal gevolgd door een fysische eenheid. Er wordt een spatie tussen het getal en de eenheid gezet. Bijvoorbeeld: \verb+1818 ns+.
\end{itemize}
De numerieke literals kunnen ook geschreven worden in exponenti\"ele vorm door het toevoegen van een '\vhdltermen{E}' of `\vhdltermen{e}'. Bijvoorbeeld: \verb+14e2+. Getallen worden standaard in decimale notatie uitgedrukt. Om een andere basis te gebruiken bestaat wordt volgend formaat gebruikt: \vhdltermen{base\#literal\#exp}, met de basis tussen 2 en 16. Zo kunnen we 1425 schrijven als \verb+16#591#+, of 4864 als \verb+16#13#E2+. Tot slot mag men ook vrij underscores (\vhdltermen{\_}) toevoegen in het getal om de code leesbaar te houden. Bijvoorbeeld: \verb+19_171_425+.
\begin{itemize}
  \item \vhdltermen{Karakter} (ofwel ``\vhdltermen{character literal}''). Deze literals wordent tussen enkele aanhalingstekens (\vhdltermen{'}) geplaatst.
  \item \vhdltermen{Karakterreeks} (ofwel ``\vhdltermen{string literal}''). Deze literals worden tussen dubbele aanhalingstekens (\vhdltermen{"}) geplaatst. Indien we dubbele aanhalingstekens willen plaatsen binnen de karakterreeks dienen we twee maal dubbele aanhalingstekens te plaatsen. Bijvoorbeeld:
\begin{quote}\verb+"""The answer?"" said Deep Thought. ""The answer to what?"""+\cite[\S25]{Adams81BOOK54}\end{quote}.
  \item \vhdltermen{Bitreeks} (ofwel \vhdltermen{``bit string literal}''). Dit is een sequentie van uitgebreide cijfers\footnote{In geval van hexadecimale getallen.}. Ze worden tussen dubbele aanhalingstekens geplaatst (\vhdltermen{"}). Eventueel wordt er een B (binair), O (octaal) of X (hexadecimaal) voor de dubbele aanhalingstekens geplaatst om de bitstring in een bepaald formaat te lezen. Verder kan de gebruiker underscores (\_) toevoegen om de leesbaarheid te verhogen.
 \end{itemize}
\subsection{Identifiers}
Een identifier is een referentie naar een bepaalde waarde. Identifiers beginnen met een letter en mogen letters en cijfers bevatten. Ook het underscore (\_) teken is toegelaten als dit niet het eerste of laatste karakter is. Identifiers zijn niet hoofdletter gevoelig. In VHDL'93 wordt het begrip van een identifier verder uitgebreid. Deze uitbreiding wordt hier niet beschouwd. Voor meer details \cite[p. 4]{hardi00}.
\subsection{scheidingstekens}
\vhdltermen{Scheidingstekens} (ofwel ``\vhdltermen{delimiters}''): de
scheidingstekens van VHDL worden weergegeven in tabel \ref{tbl:vHDLdelimiters}.
\begin{table}[hbt]
\centering
\begin{tabular}{ccccc}
\verb+&+&\verb+'+&\verb+(+&\verb+)+&\verb+*+\\
\verb/+/&\verb+,+&\verb+-+&\verb+.+&\verb+/+\\
\verb+:+&\verb+;+&\verb+<+&\verb+=+&\verb+>+\\
\verb+|+&\verb+[+&\verb+]+&\verb+=>+&\verb+**+\\
\verb+:=+&\verb+/=+&\verb+>=+&\verb+<=+&\verb+<>+
\end{tabular}
\caption{Scheidingstekens in VHDL.}
\label{tbl:vHDLdelimiters}
\end{table}
\subsection{Commentaar}
Commentaar: commentaar geplaatst na twee horizontale strepen \verb+--+, en eindigt op het einde van de lijn. Commentaar wordt genegeerd door de VHDL compiler.
Daarnaast is er ook sprake van \vhdltermen{enumeration literals} en van de \vhdltermen{NULL literal}, dez worden hier niet beschouwd.
\paragraph{Gereserveerde woorden}De woorden in tabel \ref{tbl:vHDLReservedWords} mogen niet gebruikt worden als identifiers.
\begin{table}[hbt]
\centering
\small{\begin{tabular}{lllllll}
\verb+abs+&\verb+access+&\verb+after+&\verb+alias+&\verb+all+&\verb+and+&\verb+architecture+\\
\verb+array+&\verb+assert+&\verb+attribute+&\verb+begin+&\verb+block+&\verb+body+&\verb+buffer+\\
\verb+bus+&\verb+case+&\verb+component+&\verb+configuration+&\verb+constant+&\verb+disconnect+&\verb+downto+\\
\verb+else+&\verb+elsif+&\verb+end+&\verb+entity+&\verb+exit+&\verb+file+&\verb+for+\\
\verb+function+&\verb+generate+&\verb+generic+&\verb+group+&\verb+guarded+&\verb+if+&\verb+impure+\\
\verb+in+&\verb+inertial+&\verb+inout+&\verb+is+&\verb+label+&\verb+library+&\verb+linkage+\\
\verb+literal+&\verb+loop+&\verb+map+&\verb+mod+&\verb+nand+&\verb+new+&\verb+next+\\
\verb+nor+&\verb+not+&\verb+null+&\verb+of+&\verb+on+&\verb+open+&\verb+or+\\
\verb+others+&\verb+out+&\verb+package+&\verb+port+&\verb+postponed+&\verb+procedure+&\verb+process+\\
\verb+pure+&\verb+range+&\verb+record+&\verb+register+&\verb+reject+&\verb+rem+&\verb+report+\\
\verb+return+&\verb+rol+&\verb+ror+&\verb+select+&\verb+severity+&\verb+shared+&\verb+signal+\\
\verb+sla+&\verb+sll+&\verb+sra+&\verb+srl+&\verb+subtype+&\verb+then+&\verb+to+\\
\verb+transport+&\verb+type+&\verb+unaffected+&\verb+units+&\verb+until+&\verb+use+&\verb+variable+\\
\verb+wait+&\verb+when+&\verb+while+&\verb+with+&\verb+xnor+&\verb+xor+
\end{tabular}}
\caption{Gereserveerde woorden in VHDL.}
\label{tbl:vHDLReservedWords}
\end{table}
\subsubsection{Data-types}
VHDL bevat enkele belangrijke datatypes die we in de volgende paragrafen vaak opnieuw zullen tegenkomen. Ook in de VHDL-secties werden sommige van deze types reeds gebruikt. Tabel \ref{tbl:vHDLTypes} geeft een overzicht van de meest populaire hoofdtypes.
\begin{table}[hbt]
\centering
\subtable[Overzicht van belangrijke types in VHDL.]{\small{\begin{tabular}{c|cl}
Type&Bereik&Operatoren (prioriteit)\\\hline
\multirow{5}{*}{\vhdltermen{INTEGER}}&&\vhdltermen{ABS}, \vhdltermen{**}\\
&-2\ 147\ 483\ 647&\vhdltermen{*}, \vhdltermen{/}, \vhdltermen{MOD}, \vhdltermen{REM}\\
&tot&\vhdltermen{+} (unair), \vhdltermen{-} (unair)\\
&2\ 147\ 483\ 647&\vhdltermen{+} (binair), \vhdltermen{-} (binair)\\
&&\vhdltermen{=}, \vhdltermen{/=}, \vhdltermen{<}, \vhdltermen{<=}, \vhdltermen{>}, \vhdltermen{>=}\\\hline
\multirow{5}{*}{\vhdltermen{REAL}}&&\vhdltermen{ABS}, \vhdltermen{**}\\
&-1.0E38&\vhdltermen{*}, \vhdltermen{/}\\
&tot&\vhdltermen{+} (unair), \vhdltermen{-} (unair)\\
&1.0E38&\vhdltermen{+} (binair), \vhdltermen{-} (binair)\\
&&\vhdltermen{=}, \vhdltermen{/=}, \vhdltermen{<}, \vhdltermen{<=}, \vhdltermen{>}, \vhdltermen{>=}\\\hline
\multirow{5}{*}{\vhdltermen{TIME}}&-2\ 147\ 483\ 647&\vhdltermen{ABS}\\
&tot&\vhdltermen{*}, \vhdltermen{/}\\
&2\ 147\ 483\ 647&\vhdltermen{+} (unair), \vhdltermen{-} (unair)\\
&\vhdltermen{fs}, \vhdltermen{ps}, \vhdltermen{ns}, \vhdltermen{us}&\vhdltermen{+} (binair), \vhdltermen{-} (binair)\\
&\vhdltermen{ms}, \vhdltermen{sec}, \vhdltermen{min}, \vhdltermen{hr}&\vhdltermen{=}, \vhdltermen{/=}, \vhdltermen{<}, \vhdltermen{<=}, \vhdltermen{>}, \vhdltermen{>=}\\\hline
\multirow{3}{*}{\vhdltermen{BIT}}&\multirow{3}{*}{'0' of '1'}&\vhdltermen{NOT}\\
&&\vhdltermen{=}, \vhdltermen{/=}, \vhdltermen{<}, \vhdltermen{<=}, \vhdltermen{>}, \vhdltermen{>=}\\
&&\vhdltermen{AND}, \vhdltermen{NAND}, \vhdltermen{OR}, \vhdltermen{NOR}, \vhdltermen{XOR}, \vhdltermen{XNOR}\\\hline
\multirow{5}{*}{\vhdltermen{BIT\_VECTOR}}&\multirow{5}{*}{onbegrensde rij \vhdltermen{BIT}}&\vhdltermen{NOT}\\
&&\vhdltermen{\&}\\
&&\vhdltermen{SLL}, \vhdltermen{SRL}, \vhdltermen{SLA}, \vhdltermen{SRA}, \vhdltermen{ROL}, \vhdltermen{ROR}\\
&&\vhdltermen{=}, \vhdltermen{/=}, \vhdltermen{<}, \vhdltermen{<=}, \vhdltermen{>}, \vhdltermen{>=}\\
&&\vhdltermen{AND}, \vhdltermen{NAND}, \vhdltermen{OR}, \vhdltermen{NOR}, \vhdltermen{XOR}, \vhdltermen{XNOR}\\\hline
\vhdltermen{CHARACTER}&256 karakters&\\\hline
\vhdltermen{STRING}&onbegrensde rij \vhdltermen{CHARACTER}&
\end{tabular}}
\label{tbl:vHDLTypes}}
\subtable[Overzicht van belangrijke afgeleide types in VHDL.]{\small{\begin{tabular}{c|cl}
Type&Bereik&Operatoren (prioriteit)\\\hline
\vhdltermen{NATURAL}&0 tot \vhdltermen{INTEGER'HIGH}&equivalent aan \vhdltermen{INTEGER}\\\hline
\vhdltermen{POSITIVE}&1 tot \vhdltermen{INTEGER'HIGH}&equivalent aan \vhdltermen{INTEGER}\\\hline
\vhdltermen{DELAY\_LENGTH}&0 \vhdltermen{fs} tot \vhdltermen{TIME'HIGH}&equivalent aan \vhdltermen{TIME}
\end{tabular}}
\label{tbl:vHDLSubTypes}}
\caption{Overzicht van belangrijke types en afgeleide types in VHDL.}
\label{tbl:vHDLTypesSubTypes}
\end{table}
Naast deze reeks van basistypes bevat VHDL ook standaard enkele afgeleide types. Het zijn types die een gereduceerd bereik uit de basistypes vertegenwoordigen. Deze subtypes staan in tabel \ref{tbl:vHDLSubTypes} samen met hun gereduceerd bereik.
\paragraph{Zelf types defini\"eren} Hoe defini\"eren we nu zelf types? Er zijn enkele manieren om zelf types te defni\"eren:
\begin{itemize}
 \item Defini\"eren door opsomming
 \item Defini\"eren door subtypering (beperk het bereik)
 \item Defini\"eren van fysische types
 \item Afgeleide types met matrices en vectoren
\end{itemize}
\paragraph{Defini\"eren met opsomming} We kunnen een type specificeren door alle mogelijke toestanden op te sommen. Deze methode wordt toegepast voor bijvoorbeeld de types \vhdltermen{bit} en \vhdltermen{byte}. De namen of karakters die de toestand bepalen worden tussen haakjes opgesomd na het sleutelwoord \vhdltermen{is}. \vhdlref{lst:typeEnum} toont mogelijke definities van \vhdltermen{bit} en \vhdltermen{byte}.
\begin{vhdlcode}[hbt]
\begin{lstlisting}
type bit is ('0','1');
type boolean is (false,true);
\end{lstlisting}
\caption{Defini\"eren van types door opsomming.}
\label{lst:constante}
\end{vhdlcode}

\begin{vhdlcode}[hbt]
\paragraph{Defini\"eren met subtypering}
\begin{lstlisting}
subtype byte is integer range 0 to 255;
subtype lowercase is character range 'a' to 'z';
\end{lstlisting}
\caption{Defini\"eren van types door subtypering.}
\label{lst:constante}
\end{vhdlcode}

\begin{vhdlcode}[hbt]
\begin{lstlisting}
type bit is ('0','1');
type boolean is (false,true);
\end{lstlisting}
\caption{Defini\"eren van fysische types.}
\label{lst:constante}
\end{vhdlcode}
\subsubsection{Data-objecten}
Een object in VHDL is een benoemd item met een specifieke waarde. We onderscheiden 3 soorten objecten: \vhdltermen{constante}, \vhdltermen{variabele} en \vhdltermen{signaal}.
\paragraph{Constante}Een constante is een object die slechts \'e\'enmaal een waarde kan toegekend worden. Ze geven een interpretatie aan de waarde, en maken de VHDL-code daarom beter verstaanbaar en aanpasbaar. We gebruiken hiervoor het sleutelwoord \vhdltermen{constant}. \vhdlref{lst:constante} toont de declaratie van verschillende constanten. Merk op data VHDL niet hoofdlettergevoelig is in programma-syntax. Alleen de waarde die in karakters en karakterreeksen wordt opgeslagen zijn hoofdlettergevoelig.
\begin{vhdlcode}[hbt]
\begin{lstlisting}
constant pi : real := 3.14159265;
constant byte_length : natural := 8;
constant word_length : natural := 4*byte_length;
\end{lstlisting}
\caption{Werken met constanten.}
\label{lst:constante}
\end{vhdlcode}
Het algemene formaat is dus \verb+constant <identifier> : <type> := <waarde>;+.
\paragraph{Variabele}De syntax van een variabele is ongeveer dezelfde. Alleen wordt het sleutelwoord \vhdltermen{variable}. Verder is het onmiddellijk toekennen van een waarde optioneel. \vhdlref{lst:variabele} toont het gebruik van variabelen in VHDL.
\begin{vhdlcode}[hbt]
\begin{lstlisting}
variable index : integer;
index := 0;
index := index + 1;
variable andere_index : natural := 12;
variable antwoord : natural := 4*andere_index-6*index;
\end{lstlisting}
\caption{Werken met variabelen.}
\label{lst:variabele}
\end{vhdlcode}
\paragraph{Signaal}Een signaal ten slotte is het VHDL-equivalent van een fysische verbinding of een groep van verbindingen in de hardware. Een signaal wordt geconstrueerd met behulp van het sleutelwoord \vhdltermen{signal}. Toewijzingen aan een signaal gebeuren met de \vhdltermen{<=} operator. De toewijzing bij een signaal werkt anders dan bij een variabele en constante. Variabelen en constanten worden onmiddellijk toegewezen. Dit betekent dat expressies als \verb/index := index + 1;/ mogelijk zijn. Signalen veranderen echter wanneer elementen uit hun toewijzing veranderen. Beschouw het voorbeeld uit \vhdlref{lst:signal}.
\begin{vhdlcode}[hbt]
\begin{lstlisting}
signal a : bit;
signal b : bit;
signal y : bit;
a <= '1';
b <= '0', '1' after 100 ns;
y <= a and b;
\end{lstlisting}
\caption{Werken met signalen.}
\label{lst:variabele}
\end{vhdlcode}
We zien drie signalen \verb+a+, \verb+b+ en \verb+y+. \verb+b+ zal na $100\mbox{ ns}$ een 1 aanleggen. Dit heeft als implicatie dat ook \verb+y+ van waarde zal veranderen. Indien we dit met variabelen en constanten zouden hebben berekend zou variabele \verb+y+ na de toewijzing een waarde toegewezen krijgen, en deze tot een volgende toewijzing behouden. Merk verder op dat we hier ook de betekenis van het codewoord \vhdltermen{after} tonen. Een signaal kan ook een initiele waarde krijgen, hiervoor gebruiken we de \vhdltermen{:=} operator.
\subsubsection{Bibliotheken}
\subsubsection{Bewerkingen}
\section{Combinatorische schakelingen in VHDL}
\label{s:combinatorischVHDL}
Nu we verschillende schakelingen hebben gebouwd zullen we deze proberen te beschrijven met VHDL. In deze sectie zullen we eerst een formeel overzicht geven van de VHDL-syntax. Vervolgens zullen we in subsectie \ref{ss:combinatorischVHDLHardware} een methode ontwikkelen om combinatorische elementen te beschrijven met deze syntax. In elk hoofdstuk zullen we de methodologie uitbreiden zodat we de nieuwe componenten ook kunnen beschrijven.
\subsection{Hardwarebeschrijving met VHDL}
\subsubsection{Structurele beschrijving}
\begin{vhdlcode}[hbt]
\centering
\begin{lstlisting}
-- 2-naar-1 Multiplexer
--
architecture Struct of MUX2 is
  signal U,V,W : bit;
  component AND2 is
    port (X,Y: in bit;
          Z: out bit);
  end component AND2;
  component OR2 is
    port (X,Y: in bit;
          Z: out bit);
  end component OR2;
  component INV is
    port (X: in bit;
          Z: out bit);
    end component INV;
begin
  Gate1: component INV  port map (X=>S,Z=>U);
  Gate2: component AND2 port map (X=>A,Y=>S,Z=>W);
  Gate3: component AND2 port map (U,B,V);
  Gate4: component OR2  port map (X=>W,Y=>V,Z=>Y);
end Struct;
\end{lstlisting}
\caption{2-naar-1-multiplexer.}
\label{vhdl:bToAMultiplexer}
\end{vhdlcode}
\subsubsection{Gedragsbeschrijving}
\begin{vhdlcode}[hbt]
\centering
\begin{lstlisting}
-- Opteller
--
library ieee;
use ieee.std_logic_signed.all;

entity adder is
  generic(n : positive := 4);
  port(Cin : in std_logic;
       X,Y : in std_logic_vector(n-1 downto 0);
       S   : in std_logic_vector(n-1 downto 0);
       Cout, Overflow : out std_logic);
end adder;

architecture behav of adder is
  signal Sum : std_logic_vector(n downto 0);
begin
  Sum <= (X(n-1) & X) + Y + Cin;
  S <= Sum(n-1 downto 0);
  Cout <= Sum(n);
  Overflow <= Sum(n) xor X(n-1) xor Y(n-1) xor Sum(n-1);
end behav;
\end{lstlisting}
\caption{$n$-bit Opteller.}
\label{vhdl:adder}
\end{vhdlcode}
\subsubsection{Repetitieve structuren}
\label{ss:combinatorischVHDLHardware}