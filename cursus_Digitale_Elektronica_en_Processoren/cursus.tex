\documentclass[a4paper,10pt,titlepage]{book}
\usepackage[dutch]{babel}
\usepackage{glossaries}
\usepackage{timing}
\usepackage{fullpage}
\usepackage{tikz}
\usepackage{index}
\usepackage{subfigure}
\usepackage{graphicx}
\usepackage{framed}
\usepackage{tabularx}
\usepackage{wrapfig}
\usepackage{listings}
\usepackage{multicol}
\usepackage{multicol}
\usepackage{multirow}
\usepackage{stmaryrd}
\usepackage{amsfonts}
\usepackage{amsmath}
\usepackage{bbding}
\usepackage{array}
\usepackage{eurosym}
\usepackage{commusoftScripts,../SharedData/brackets,../SharedData/importsreferences,../SharedData/lstlibraryciscasmlanguage,../SharedData/lstlibrarybinlanguage}
\usetikzlibrary{circuits.logic.US}
\usetikzlibrary{DEPcomponents}
\usetikzlibrary{PCBcomponents}
\usetikzlibrary{Washing}
\usetikzlibrary{flowchart}
\usepgflibrary{shapes.geometric}
\usetikzlibrary{fit,calc,positioning,decorations.pathreplacing,matrix}
\usepackage{circuitikz}
\pdfinfo{
  /Title    (Cursus Digitale Elektronica en Processoren (DEP))
  /Author   (Willem Van Onsem)
  /Subject  (Digitale Elektronica)
  /Keywords (Digitaal, Elektronica, Processoren, Schakelingen, KULeuven, Cursus)
}
\ctikzset{bipoles/length=1cm}
\lstset{ %
language=VHDL,	                % choose the language of the code
basicstyle=\footnotesize,       % the size of the fonts that are used for the code
numbers=left,                   % where to put the line-numbers
numberstyle=\footnotesize,      % the size of the fonts that are used for the line-numbers
stepnumber=1,                   % the step between two line-numbers. If it's 1 each line
                                % will be numbered
numbersep=5pt,                  % how far the line-numbers are from the code
backgroundcolor=\color{white},  % choose the background color. You must add \usepackage{color}
showspaces=false,               % show spaces adding particular underscores
showstringspaces=false,         % underline spaces within strings
showtabs=false,                 % show tabs within strings adding particular underscores
frame=tb,                       % adds a frame around the code
tabsize=2,                      % sets default tabsize to 2 spaces
%captionpos=n,                   % sets the caption-position to bottom
breaklines=true,                % sets automatic line breaking
breakatwhitespace=false,        % sets if automatic breaks should only happen at whitespace
title=\lstname,                 % show the filename of files included with \lstinputlisting;
                                % also try caption instead of title
}

\usepackage[hidelinks,bookmarks]{hyperref}
\usepackage{minitoc}
\dominitoc
\setcounter{minitocdepth}{3}
\nomtcrule

\makeindex
\makeglossaries

\newenvironment{chapterintro}{\begin{center}\begin{large}\begin{it}\begin{minipage}{12 cm}}{\end{minipage}\end{it}\end{large}\end{center}}
\newcommand{\gtermen}[2]{\newglossaryentry{#1}{description={#2}}\termen{#1}}
\newcommand{\vhdltermen}[1]{\index{\texttt{VHDL}!\texttt{#1}}\texttt{#1}}
\newcommand{\vhdlref}[1]{\texttt{VHDL}-code \ref{#1}}
\newcommand{\NN}{\mathbb{N}}
\newcommand{\clrsin}{\texttt{Clr$^*$}}
\newcommand{\prsin}{\texttt{Pr$^*$}}
\newcommand{\pdot}[2][0.4 mm]{\fill (#2) circle (#1);}
\newcommand{\stAs}[1]{$|$#1$|$}
\newcommand{\ndAs}[1]{\underline{#1}}
\title{\includegraphics[width=5cm]{../SharedData/sedes.pdf}\\Cursus:\\Digitale Elektronica \& Processoren\\\texttt{H01L1A}\\\texttt{\small versie 0.2.718}}
\author{Willem M. A. Van Onsem, BSc.\\\includegraphics[width=2.5cm]{../SharedData/kommusoftEmblema.pdf}}
\date{Katholieke Universiteit Leuven\\Academiejaar 2010-2011}
\begin{document}
\frontmatter
\begin{titlepage}
\maketitle
\end{titlepage}
\tableofcontents
\chapter*{Notities vooraf}
\begin{it}
Deze cursus omvat de volledige inhoud van het onderwerp ``Digitale Elektronica \& Processoren'' gegeven tijdens het academiejaar 2010-2011 door prof. Van Eycken. De cursus is dan ook hoofdzakelijk gebaseerd op de presentaties voor dit onderwerp. Andere bronnen worden vermeld in de referentielijst op pagina \pageref{reference}. Aanbevolen literatuur is \cite{brown2004fundamentals}, \cite{gajski1997principles}, \cite{wakerly2000digital} en \cite{ashenden2008designer}.
\paragraph{}
De cursus wordt uitgegeven onder de \texttt{CopyLeft} licentie, dit betekent dat iedereen de cursus vrij kan aanpassen en herverdelen.
\paragraph{}
De auteur garandeert de juistheid van deze cursus \underline{niet}. Hoewel deze cursus met de nodige zorg is samengesteld, is het niet ondenkbaar dat er fouten in staan. Errata/opmerkingen/suggesties kunnen altijd doorgestuurd worden naar \verb+vanonsem.willem@gmail.com+, deze worden dan in de volgende versie verbeterd.
\paragraph{Over de auteur}
valt eigenlijk niet veel te zeggen, behalve dat hij geen exemplaren signeert.
\paragraph{}
Speciale dank gaat naar (in alfabetische volgorde) ``Amy Winehouse'', ``Harold Budd'', ``Hooverphonic'', ``Moby'', ``Paul Simon'', ``Piknik'', ``Russkij Razmer'' en ``The Beatles'' voor de muziek tijdens de nachten waarin deze cursus tot stand kwam.
\section*{Betreft het examen}
Het examen bestond traditioneel tijdens het academiejaar 2010-2011 uit drie delen:
\begin{enumerate}
 \item Theorie: bestaat uit twee vragen, dit zijn meestal termen. Deze termen kunnen achteraan in de index op pagina \pageref{idx} worden teruggevonden. De meeste ook in de woordenlijst op pagina \pageref{glos}.
 \item Synthese van een datapad en controller: De student krijgt een \texttt{VHDL}-code, en wordt gevraagd naar een equivalent ASM-schema. Vervolgens dient hierbij het datapad en de controller gesynthetiseerd te worden. Soms volstaat het echter om bij de controller een toestandsdiagram weer te geven.
 \item Synthese van een toestandsdiagram: Er wordt een toestandsdiagram gegeven (meestal in tabelvorm). De student dient dit diagram te vertalen in hardware. Meestal worden er extra beperkingen opgelegd, zoals het gebruik van een bepaald type flip-flop, en poorten. Ook dient men vaak de Karnaugh-kaarten weer te geven.
\end{enumerate}
Het examen is volledig gesloten boek. De eerste twee vragen worden mondeling besproken, dit houdt echter in dat wat op papier staat de basis vormt. De laatste vraag is volledig schriftelijk. Enkel het examenmoment geldt als evaluatie. De student mag een referentieblad meebrengen over VHDL. Dit is te koop bij de VTK CuDi.
\section*{Layout en Stijl}
\paragraph{}
Alle pagina's zijn afdrukbaar met een zwart-wit printer. Dit drukt eventuele kosten bij een afdruk van deze cursus. Bovendien verhoogt het de leesbaarheid bij kleurenblinden.
\paragraph{}
Deze cursus wordt ge\"illustreerd met talloze afbeeldingen. Deze zijn allemaal tot stand gekomen met het grafisch pakket TikZ\footnote{TikZ: TikZ ist kein Zeichenprogramm.} samen met zelfgeschreven bibliotheken. Alle afbeeldingen zijn bijgevolg vectorieel. En kunnen dus eindeloos uitvergroot worden met een geschikte .pdf viewer.
\paragraph{}
\termenlayout{Terminologie} wordt in het vetjes en zonder schreven\footnote{De zogenoemde ``pootjes'' die sommige letters krijgen. (Engels: serifs)} gezet. Deze terminologie wordt ook herhaald op het einde van de cursus in de index op pagina \pageref{idx}.
\section*{Kudos}
Kudos gaan uit naar volgende personen/groepen (in alfabetische volgorde):
\begin{description}
 \item[\texttt{Conrad.be}] Leverancier van elektronische onderdelen om de circuits zelf te testen.
 \item[Prof. Christian Maes] Voor onvergetelijke lessen Statistische Thermodynamica over het grootcanonisch ensemble $\Xi\left(T,V,\mu\right)=e^{\beta PV}$.
 \item[Ingmar Dasseville] Voor enkele Haskell scripts in het kader van geautomatiseerde \LaTeX-code generatie.
 \item[Wina studentenkring] Om overduidelijke redenen... In tegenstelling tot VTK...
 \item[Het revisoren-team] bestaande uit: Ingmar Dasseville, Jonas Vanthornhout, Katie Pauwelyn en Steven Roose.
 \item[Personen die errata indienden] Ingmar Dasseville, Pieter Van Riet, Jonas Vanthornhout, Lynn Houthuys, Katie Pauwelyn, Christophe Van Ginneken, Alex Witteveen, Sander Van Loock, Philippe De Croock, Davy Vanclee
\end{description}
\section*{Link naar deze cursus}
De meest recente versie van deze cursus is op onderstaande link te vinden:
\importtikzfigure{link}{Link naar de meest recente versie van deze cursus.}
Indien er fouten gerapporteerd worden, of toevoegingen gedaan worden zullen na verloop van tijd de aanpassingen daar te vinden zijn.
\paragraph{}Deze cursus is opgedragen aan leden van mijn familie:
\begin{itemize}
 \item \textsc{Louis Van Onsem (1931-1993)}: voor drie jaar peterschap en m'n derde voornaam: \textsc{Agnes}.% en vier maanden
 \item \textsc{Gabri\"ella Simons (1923-2012)}: voor haar humor, fijne herinneringen en 17 jaar plaatsvervangend peterschap.
 \item \textsc{Constant Soetewey (1907-1945)}: voor hoogstaande absurdistische literatuur, die dit jaar opnieuw uitgegeven werd\cite{Kohler} onder het pseudoniem \textsc{Kurt K\"ohler}.
\end{itemize}
\end{it}
\commusoftQuality{1}
\chapter*{Voorbeschouwing}
\mainmatter
\part{Digitaal Ontwerp en Fysische Limieten}
\chapter{De Basis van een Digitaal Ontwerp}
\label{ch:basis}
\chapterquote{Eva werd niet als een soort accessoire van Adam geschapen, maar Adam was het eerste ontwerp voor Eva.}{Jeanne Moreau, Frans actrice (1928-)}
\begin{chapterintro}
Zoals de naam van deze cursus reeds doet vermoeden werken we met digitale schakelingen, digitale schakelingen onderscheiden zich van de traditionele analoge schakelingen omdat ze slechts een beperkt aantal waarden kunnen aannemen. In de praktijk werken we bijna uitsluitend met \termen{binaire signalen}. Deze waarden hoeven niet noodzakelijk een verschil in spanning aan te duiden. Sterker nog, het gebruik van elektronica is zelfs niet verplicht. Zo zouden we ook een verschil in stroomsterkte, druk, reflectie,... als parameter kunnen nemen. Het komt er enkel op aan om twee waarden te defini\"eren die we symbolisch zullen voorstellen als 0 en 1. Indien we echter meer dan twee verschillende waarden nodig hebben, kunnen we dit probleem oplossen door een aantal binaire waarden te groeperen. In dit eerste hoofdstuk zullen we een beperkte set aan basisoperaties op deze binaire signalen introduceren: de NOT, AND en OR. We formaliseren deze operaties in de booleaanse algebra. Ook bespreken we een methode om een booleaanse functie te synthetiseren. Tot slot bespreken we het verloop van een digitaal ontwerp en introduceren we een taal om hardware mee te beschrijven: VHDL.
\end{chapterintro}
\minitoc[n]
\section{Logische schakelingen}
\paragraph{}
Uiteraard willen we met deze binaire waarden bepaalde operaties uitvoeren. Om berekeningen te kunnen uitvoeren maken we gebruik van \termen{logische schakelingen}. Logische schakelingen zijn een reeks operaties op \'e\'en of meer binaire parameters die resulteren in een uitgang. De waarde die op de uitgang staat is hierbij afhankelijk van de ingangen. De set van \gtermen{basisoperaties}{set van operaties waaruit andere operaties zijn opgebouwd: in de booleaanse algebra zijn dit de NOT, AND en OR} dient klein en simpel te zijn, zodat elke complexe schakeling uit een combinatie van deze basisschakelingen kan bestaan.
\paragraph{}
Als basisschakelingen definieert men meestal de \termen{NOT}, \termen{AND} en \termen{OR} operaties. Deze operaties zijn eenvoudig en makkelijk te begrijpen/onthouden. Hoewel booleaanse algebra in heel wat andere cursussen reeds aan bod komt, zullen we er ook in deze cursus enkele secties aan besteden. We zullen de booleaanse algebra behandelen aan de hand van een model met lichtschakelaars.
\subsection{Logische schakelingen in huis}
We beschouwen een schakeling zoals op \figref{switchLight}.
\importtikzfigure{switchLight}{Basis van het lamp-model.}
Zoals we zien zal indien we de schakelaar $x$ indrukken, het lichtje branden. Indien we vervolgens de schakelaar loslaten zien we dat het lampje opnieuw dooft. We formaliseren dit door te schrijven:
\begin{equation}
L=x
\end{equation}
Waarbij we $L$ beschouwen als het al dan niet branden van het lampje, en $x$ bepaalt of de schakelaar al dan niet ingedrukt is. Indien we nu als ingang beschouwen of de schakelaar al dan niet ingedrukt is, en als uitgang of het lichtje al dan niet brandt, kunnen we hiermee functies gaan defini\"eren.
\paragraph{Not}
Indien we de implementatie van de schakelaar aanpassen zijn we in staat om een NOT poort te bouwen. Zoals op \figref{switchLightBasicGatesNot}. Deze schakelaar laat stroom door indien deze niet ingedrukt is. Bijgevolg kunnen we stellen dat het al dan niet branden van het lampje equivalent is met niet de schakelaar indrukken. Of indien we dit formaliseren:
\begin{equation}
L=x'=\neg x=!x=\overline{x}=\mbox{NOT }x
\end{equation}
Zoals we zien zijn er in de loop der tijd nogal wat notaties ingevoerd. Wat bovendien eigen is aan de volledige booleaanse algebra. In deze cursus zullen we het accent ($x'$) als negatie gebruiken. Literatuur buiten deze cursus kan echter andere standaarden gebruiken. Een NOT poort wordt vaak ook een \termen{inverter} genoemd.
\paragraph{And}
Soms willen we dat het lampje pas gaat branden indien twee of meer schakelaars allemaal ingedrukt zijn. In dat geval spreken we van een AND. Een AND kunnen we implementeren volgens het lamp-model zoals in \figref{switchLightBasicGatesAnd}. We noteren:
\begin{equation}
L=x\cdot y=x\mbox{ AND }y
\end{equation}
\paragraph{OR}
Een andere basisbewerking is de OR. Hierbij willen we dat het lampje brandt bij minstens \'e\'en ingedrukte schakelaar. Of we noteren:
\begin{equation}
L=x+y=x\mbox{ OR } y
\end{equation}
Een implementatie in het lamp-model is te vinden op \figref{switchLightBasicGatesOr}.
\begin{figure}[htb]
\centering
\subfigure[Not]{\begin{tikzpicture}
\draw[thick] (-0.3,0.05) -- (0.3,0.05);
\draw (-0.15,-0.05) -- (0.15,-0.05);
\draw (0,0.05) -- (0,0.5) -- (0.5,0.5);
\fill (0.5,0.5) circle (0.05);
\fill (1,0.5) circle (0.05);
\draw (1,0.5) -- (1.5,0.5) -- (1.5,0.25);
\fill[black!20] (1.5,0) circle (0.25);
\draw[thick] (1.5,0) circle (0.25);
\draw[thick] (1.677,0.177) -- (1.323,-0.177);
\draw[thick] (1.677,-0.177) -- (1.323,0.177);
\draw (0,-0.05) -- (0,-0.5) -- (1.5,-0.5) -- (1.5,-0.25);
\draw (0.4,0.45) -- (1.1,0.45);
\draw (0.7,0.45) -- (0.7,0.65);
\draw (0.8,0.45) -- (0.8,0.65);
\draw (0.6,0.75) -- (0.75,0.6) -- (0.9,0.75);
\draw (0.75,0.75) node[anchor=south]{$x$};
\begin{scope}[xshift=2.5 cm]
\draw[thick] (-0.3,0.05) -- (0.3,0.05);
\draw (-0.15,-0.05) -- (0.15,-0.05);
\draw (0,0.05) -- (0,0.5) -- (0.5,0.5);
\fill (0.5,0.5) circle (0.05);
\fill (1,0.5) circle (0.05);
\draw (1,0.5) -- (1.5,0.5) -- (1.5,0.25);
\draw[thick] (1.5,0) circle (0.25);
\draw[thick] (1.677,0.177) -- (1.323,-0.177);
\draw[thick] (1.677,-0.177) -- (1.323,0.177);
\draw (0,-0.05) -- (0,-0.5) -- (1.5,-0.5) -- (1.5,-0.25);
\begin{scope}[yshift=-0.15 cm]
\draw (0.4,0.45) -- (1.1,0.45);
\draw (0.7,0.45) -- (0.7,0.65);
\draw (0.8,0.45) -- (0.8,0.65);
\draw (0.6,0.75) -- (0.75,0.6) -- (0.9,0.75);
\draw (0.75,0.75) node[anchor=south]{$x$};
\end{scope}
\end{scope}
\end{tikzpicture}
\figlab{switchLightBasicGatesNot}
}\\
\subfigure[And]{\begin{tikzpicture}
\draw[thick] (-0.3,0.05) -- (0.3,0.05);
\draw (-0.15,-0.05) -- (0.15,-0.05);
\draw (0,0.05) -- (0,0.5) -- (0.5,0.5);
\fill (0.5,0.5) circle (0.05);
\fill (1,0.5) circle (0.05);
\draw (1,0.5) -- (1.5,0.5);
\fill (1.5,0.5) circle (0.05);
\fill (2,0.5) circle (0.05);
\draw (2,0.5) -- (2.5,0.5) -- (2.5,0.25);
\draw[thick] (2.5,0) circle (0.25);
\draw[thick] (2.677,0.177) -- (2.323,-0.177);
\draw[thick] (2.677,-0.177) -- (2.323,0.177);
\draw (0,-0.05) -- (0,-0.5) -- (2.5,-0.5) -- (2.5,-0.25);
\begin{scope}[yshift=0.25 cm]
\draw (0.4,0.45) -- (1.1,0.45);
\draw (0.7,0.45) -- (0.7,0.65);
\draw (0.8,0.45) -- (0.8,0.65);
\draw (0.6,0.75) -- (0.75,0.6) -- (0.9,0.75);
\draw (0.75,0.75) node[anchor=south]{$x$};
\end{scope}
\begin{scope}[xshift=1 cm,yshift=0.25 cm]
\draw (0.4,0.45) -- (1.1,0.45);
\draw (0.7,0.45) -- (0.7,0.65);
\draw (0.8,0.45) -- (0.8,0.65);
\draw (0.6,0.75) -- (0.75,0.6) -- (0.9,0.75);
\draw (0.75,0.75) node[anchor=south]{$y$};
\end{scope}
\begin{scope}[xshift=4 cm]
\draw[thick] (-0.3,0.05) -- (0.3,0.05);
\draw (-0.15,-0.05) -- (0.15,-0.05);
\draw (0,0.05) -- (0,0.5) -- (0.5,0.5);
\fill (0.5,0.5) circle (0.05);
\fill (1,0.5) circle (0.05);
\draw (1,0.5) -- (1.5,0.5);
\fill (1.5,0.5) circle (0.05);
\fill (2,0.5) circle (0.05);
\draw (2,0.5) -- (2.5,0.5) -- (2.5,0.25);
\draw[thick] (2.5,0) circle (0.25);
\draw[thick] (2.677,0.177) -- (2.323,-0.177);
\draw[thick] (2.677,-0.177) -- (2.323,0.177);
\draw (0,-0.05) -- (0,-0.5) -- (2.5,-0.5) -- (2.5,-0.25);
\begin{scope}[yshift=0.25 cm]
\draw (0.4,0.45) -- (1.1,0.45);
\draw (0.7,0.45) -- (0.7,0.65);
\draw (0.8,0.45) -- (0.8,0.65);
\draw (0.6,0.75) -- (0.75,0.6) -- (0.9,0.75);
\draw (0.75,0.75) node[anchor=south]{$x$};
\end{scope}
\begin{scope}[xshift=1 cm,yshift=0.1 cm]
\draw (0.4,0.45) -- (1.1,0.45);
\draw (0.7,0.45) -- (0.7,0.65);
\draw (0.8,0.45) -- (0.8,0.65);
\draw (0.6,0.75) -- (0.75,0.6) -- (0.9,0.75);
\draw (0.75,0.75) node[anchor=south]{$y$};
\end{scope}
\end{scope}
\begin{scope}[xshift=8 cm]
\draw[thick] (-0.3,0.05) -- (0.3,0.05);
\draw (-0.15,-0.05) -- (0.15,-0.05);
\draw (0,0.05) -- (0,0.5) -- (0.5,0.5);
\fill (0.5,0.5) circle (0.05);
\fill (1,0.5) circle (0.05);
\draw (1,0.5) -- (1.5,0.5);
\fill (1.5,0.5) circle (0.05);
\fill (2,0.5) circle (0.05);
\draw (2,0.5) -- (2.5,0.5) -- (2.5,0.25);
\draw[thick] (2.5,0) circle (0.25);
\draw[thick] (2.677,0.177) -- (2.323,-0.177);
\draw[thick] (2.677,-0.177) -- (2.323,0.177);
\draw (0,-0.05) -- (0,-0.5) -- (2.5,-0.5) -- (2.5,-0.25);
\begin{scope}[yshift=0.1 cm]
\draw (0.4,0.45) -- (1.1,0.45);
\draw (0.7,0.45) -- (0.7,0.65);
\draw (0.8,0.45) -- (0.8,0.65);
\draw (0.6,0.75) -- (0.75,0.6) -- (0.9,0.75);
\draw (0.75,0.75) node[anchor=south]{$x$};
\end{scope}
\begin{scope}[xshift=1 cm,yshift=0.25 cm]
\draw (0.4,0.45) -- (1.1,0.45);
\draw (0.7,0.45) -- (0.7,0.65);
\draw (0.8,0.45) -- (0.8,0.65);
\draw (0.6,0.75) -- (0.75,0.6) -- (0.9,0.75);
\draw (0.75,0.75) node[anchor=south]{$y$};
\end{scope}
\end{scope}
\begin{scope}[xshift=12 cm]
\draw[thick] (-0.3,0.05) -- (0.3,0.05);
\draw (-0.15,-0.05) -- (0.15,-0.05);
\draw (0,0.05) -- (0,0.5) -- (0.5,0.5);
\fill (0.5,0.5) circle (0.05);
\fill (1,0.5) circle (0.05);
\draw (1,0.5) -- (1.5,0.5);
\fill (1.5,0.5) circle (0.05);
\fill (2,0.5) circle (0.05);
\draw (2,0.5) -- (2.5,0.5) -- (2.5,0.25);
\fill[black!20] (2.5,0) circle (0.25);
\draw[thick] (2.5,0) circle (0.25);
\draw[thick] (2.677,0.177) -- (2.323,-0.177);
\draw[thick] (2.677,-0.177) -- (2.323,0.177);
\draw (0,-0.05) -- (0,-0.5) -- (2.5,-0.5) -- (2.5,-0.25);
\begin{scope}[yshift=0.1 cm]
\draw (0.4,0.45) -- (1.1,0.45);
\draw (0.7,0.45) -- (0.7,0.65);
\draw (0.8,0.45) -- (0.8,0.65);
\draw (0.6,0.75) -- (0.75,0.6) -- (0.9,0.75);
\draw (0.75,0.75) node[anchor=south]{$x$};
\end{scope}
\begin{scope}[xshift=1 cm,yshift=0.1 cm]
\draw (0.4,0.45) -- (1.1,0.45);
\draw (0.7,0.45) -- (0.7,0.65);
\draw (0.8,0.45) -- (0.8,0.65);
\draw (0.6,0.75) -- (0.75,0.6) -- (0.9,0.75);
\draw (0.75,0.75) node[anchor=south]{$y$};
\end{scope}
\end{scope}
\end{tikzpicture}
\figlab{switchLightBasicGatesAnd}
}\\
\subfigure[Or]{\begin{tikzpicture}
\draw[thick] (-0.3,0.05) -- (0.3,0.05);
\draw (-0.15,-0.05) -- (0.15,-0.05);
\draw (0,0.05) -- (0,0.5) -- (0.5,0.5);
\draw (0,0.5) -- (0,1.5) -- (0.5,1.5);
\fill (0.5,0.5) circle (0.05);
\fill (1,0.5) circle (0.05);
\draw (1,1.5) -- (1.5,1.5) -- (1.5,0.5);
\draw (1,0.5) -- (1.5,0.5) -- (1.5,0.25);
\fill (0.5,1.5) circle (0.05);
\fill (1,1.5) circle (0.05);
\draw[thick] (1.5,0) circle (0.25);
\draw[thick] (1.677,0.177) -- (1.323,-0.177);
\draw[thick] (1.677,-0.177) -- (1.323,0.177);
\draw (0,-0.05) -- (0,-0.5) -- (1.5,-0.5) -- (1.5,-0.25);
\begin{scope}[yshift=0.25 cm]
\draw (0.4,0.45) -- (1.1,0.45);
\draw (0.7,0.45) -- (0.7,0.65);
\draw (0.8,0.45) -- (0.8,0.65);
\draw (0.6,0.75) -- (0.75,0.6) -- (0.9,0.75);
\draw (0.75,0.75) node[anchor=south]{$y$};
\end{scope}
\begin{scope}[yshift=1.25 cm]
\draw (0.4,0.45) -- (1.1,0.45);
\draw (0.7,0.45) -- (0.7,0.65);
\draw (0.8,0.45) -- (0.8,0.65);
\draw (0.6,0.75) -- (0.75,0.6) -- (0.9,0.75);
\draw (0.75,0.75) node[anchor=south]{$x$};
\end{scope}
\begin{scope}[xshift=2.5 cm]
\draw[thick] (-0.3,0.05) -- (0.3,0.05);
\draw (-0.15,-0.05) -- (0.15,-0.05);
\draw (0,0.05) -- (0,0.5) -- (0.5,0.5);
\draw (0,0.5) -- (0,1.5) -- (0.5,1.5);
\fill (0.5,0.5) circle (0.05);
\fill (1,0.5) circle (0.05);
\draw (1,1.5) -- (1.5,1.5) -- (1.5,0.5);
\draw (1,0.5) -- (1.5,0.5) -- (1.5,0.25);
\fill (0.5,1.5) circle (0.05);
\fill (1,1.5) circle (0.05);
\fill[black!20] (1.5,0) circle (0.25);
\draw[thick] (1.5,0) circle (0.25);
\draw[thick] (1.677,0.177) -- (1.323,-0.177);
\draw[thick] (1.677,-0.177) -- (1.323,0.177);
\draw (0,-0.05) -- (0,-0.5) -- (1.5,-0.5) -- (1.5,-0.25);
\begin{scope}[yshift=0.25 cm]
\draw (0.4,0.45) -- (1.1,0.45);
\draw (0.7,0.45) -- (0.7,0.65);
\draw (0.8,0.45) -- (0.8,0.65);
\draw (0.6,0.75) -- (0.75,0.6) -- (0.9,0.75);
\draw (0.75,0.75) node[anchor=south]{$y$};
\end{scope}
\begin{scope}[yshift=1.1 cm]
\draw (0.4,0.45) -- (1.1,0.45);
\draw (0.7,0.45) -- (0.7,0.65);
\draw (0.8,0.45) -- (0.8,0.65);
\draw (0.6,0.75) -- (0.75,0.6) -- (0.9,0.75);
\draw (0.75,0.75) node[anchor=south]{$x$};
\end{scope}
\end{scope}
\begin{scope}[xshift=5 cm]
\draw[thick] (-0.3,0.05) -- (0.3,0.05);
\draw (-0.15,-0.05) -- (0.15,-0.05);
\draw (0,0.05) -- (0,0.5) -- (0.5,0.5);
\draw (0,0.5) -- (0,1.5) -- (0.5,1.5);
\fill (0.5,0.5) circle (0.05);
\fill (1,0.5) circle (0.05);
\draw (1,1.5) -- (1.5,1.5) -- (1.5,0.5);
\draw (1,0.5) -- (1.5,0.5) -- (1.5,0.25);
\fill (0.5,1.5) circle (0.05);
\fill (1,1.5) circle (0.05);
\fill[black!20] (1.5,0) circle (0.25);
\draw[thick] (1.5,0) circle (0.25);
\draw[thick] (1.677,0.177) -- (1.323,-0.177);
\draw[thick] (1.677,-0.177) -- (1.323,0.177);
\draw (0,-0.05) -- (0,-0.5) -- (1.5,-0.5) -- (1.5,-0.25);
\begin{scope}[yshift=0.1 cm]
\draw (0.4,0.45) -- (1.1,0.45);
\draw (0.7,0.45) -- (0.7,0.65);
\draw (0.8,0.45) -- (0.8,0.65);
\draw (0.6,0.75) -- (0.75,0.6) -- (0.9,0.75);
\draw (0.75,0.75) node[anchor=south]{$y$};
\end{scope}
\begin{scope}[yshift=1.25 cm]
\draw (0.4,0.45) -- (1.1,0.45);
\draw (0.7,0.45) -- (0.7,0.65);
\draw (0.8,0.45) -- (0.8,0.65);
\draw (0.6,0.75) -- (0.75,0.6) -- (0.9,0.75);
\draw (0.75,0.75) node[anchor=south]{$x$};
\end{scope}
\end{scope}
\begin{scope}[xshift=7.5 cm]
\draw[thick] (-0.3,0.05) -- (0.3,0.05);
\draw (-0.15,-0.05) -- (0.15,-0.05);
\draw (0,0.05) -- (0,0.5) -- (0.5,0.5);
\draw (0,0.5) -- (0,1.5) -- (0.5,1.5);
\fill (0.5,0.5) circle (0.05);
\fill (1,0.5) circle (0.05);
\draw (1,1.5) -- (1.5,1.5) -- (1.5,0.5);
\draw (1,0.5) -- (1.5,0.5) -- (1.5,0.25);
\fill (0.5,1.5) circle (0.05);
\fill (1,1.5) circle (0.05);
\fill[black!20] (1.5,0) circle (0.25);
\draw[thick] (1.5,0) circle (0.25);
\draw[thick] (1.677,0.177) -- (1.323,-0.177);
\draw[thick] (1.677,-0.177) -- (1.323,0.177);
\draw (0,-0.05) -- (0,-0.5) -- (1.5,-0.5) -- (1.5,-0.25);
\begin{scope}[yshift=0.1 cm]
\draw (0.4,0.45) -- (1.1,0.45);
\draw (0.7,0.45) -- (0.7,0.65);
\draw (0.8,0.45) -- (0.8,0.65);
\draw (0.6,0.75) -- (0.75,0.6) -- (0.9,0.75);
\draw (0.75,0.75) node[anchor=south]{$y$};
\end{scope}
\begin{scope}[yshift=1.1 cm]
\draw (0.4,0.45) -- (1.1,0.45);
\draw (0.7,0.45) -- (0.7,0.65);
\draw (0.8,0.45) -- (0.8,0.65);
\draw (0.6,0.75) -- (0.75,0.6) -- (0.9,0.75);
\draw (0.75,0.75) node[anchor=south]{$x$};
\end{scope}
\end{scope}
\end{tikzpicture}
\figlab{switchLightBasicGatesOr}
}
\caption{Implementatie van de basispoorten volgens het lamp-model.}
\figlab{switchLightBasicGates}
\end{figure}
\paragraph{}
Door deze drie basisbewerkingen met elkaar te combineren kunnen we eindeloos veel nieuwe bewerkingen bouwen. Zoals bijvoorbeeld de exclusieve OR, ook wel de \termen{XOR} genoemd. De XOR is een bewerking waarbij het lampje gaat branden indien juist \'e\'en van de twee schakelaars ingedrukt is. Deze bewerking kunnen we realiseren door NOT, AND en OR bewerkingen te combineren als volgt:
\begin{equation}
L=x \oplus y=x\mbox{ XOR }y=\left(x\cdot y'\right)+\left(x'\cdot y\right)
\end{equation}
Deze schakelingen kunnen we dan vervolgens in ons lamp-model omzetten zoals op \figref{switchLightXor}.
\begin{figure}[htb]
\centering
\begin{tikzpicture}
\draw[thick] (-0.3,0.05) -- (0.3,0.05);
\draw (-0.15,-0.05) -- (0.15,-0.05);
\draw (0,0.05) -- (0,0.5) -- (0.5,0.5);
\draw (0,0.5) -- (0,1.5) -- (0.5,1.5);
\fill (0.5,0.5) circle (0.05);
\fill (1,0.5) circle (0.05);
\fill (1.5,0.5) circle (0.05);
\fill (2,0.5) circle (0.05);
\draw (1,1.5) -- (1.5,1.5);
\draw (2,1.5) -- (2.5,1.5) -- (2.5,0.5);
\draw (1,0.5) -- (1.5,0.5);
\draw (2,0.5) -- (2.5,0.5) -- (2.5,0.25);
\fill (0.5,1.5) circle (0.05);
\fill (1,1.5) circle (0.05);
\fill (1.5,1.5) circle (0.05);
\fill (2,1.5) circle (0.05);
\draw[thick] (2.5,0) circle (0.25);
\draw[thick] (2.677,0.177) -- (2.323,-0.177);
\draw[thick] (2.677,-0.177) -- (2.323,0.177);
\draw (0,-0.05) -- (0,-0.5) -- (2.5,-0.5) -- (2.5,-0.25);
\begin{scope}[yshift=0 cm]
\draw (0.4,0.45) -- (1.1,0.45);
\draw (0.7,0.45) -- (0.7,0.65);
\draw (0.8,0.45) -- (0.8,0.65);
\draw (0.6,0.75) -- (0.75,0.6) -- (0.9,0.75);
\draw (0.75,0.75) node[anchor=south]{$x$};
\end{scope}
\begin{scope}[xshift=1cm,yshift=0.25 cm]
\draw (0.4,0.45) -- (1.1,0.45);
\draw (0.7,0.45) -- (0.7,0.65);
\draw (0.8,0.45) -- (0.8,0.65);
\draw (0.6,0.75) -- (0.75,0.6) -- (0.9,0.75);
\draw (0.75,0.75) node[anchor=south]{$y$};
\end{scope}
\begin{scope}[yshift=1.25 cm]
\draw (0.4,0.45) -- (1.1,0.45);
\draw (0.7,0.45) -- (0.7,0.65);
\draw (0.8,0.45) -- (0.8,0.65);
\draw (0.6,0.75) -- (0.75,0.6) -- (0.9,0.75);
\draw (0.75,0.75) node[anchor=south]{$x$};
\end{scope}
\begin{scope}[xshift=1cm,yshift=1 cm]
\draw (0.4,0.45) -- (1.1,0.45);
\draw (0.7,0.45) -- (0.7,0.65);
\draw (0.8,0.45) -- (0.8,0.65);
\draw (0.6,0.75) -- (0.75,0.6) -- (0.9,0.75);
\draw (0.75,0.75) node[anchor=south]{$y$};
\end{scope}
\end{tikzpicture}
\caption{XOR-poort in het lamp-model.}
\figlab{switchLightXor}
\end{figure}
\subsection{Waarheidstabellen}
Uiteraard kunnen we alle schakelingen voorstellen met het lamp-model. Toch is het niet echt praktisch, we gaan dus op zoek naar andere manieren om de logische formules te berekenen, en ook eenvoudig voor te stellen. Een makkelijke manier om logische schakelingen te berekenen is met behulp van \termen{waarheidstabellen}. Een waarheidstabel is een tabel waarbij we alle variabelen voorstellen, en vervolgens met eventuele tussenstappen de uiteindelijke bewerking berekenen. Hierbij maken we gebruik van waarheidstabellen die we reeds kennen: de waarheidstabellen van de basisfuncties. Indien we $n$ variabelen beschouwen betekent dit dus dat onze tabel $2^n$ rijen telt. Immers kan elke variabele ofwel 0 ofwel 1 zijn. Bovendien is het aantal functies met $n$ variabelen beperkt tot $2^{2^n}$. Dit zegt echter niet over het aantal mogelijke implementaties. In \tblref{truthTablesBasicGates} geven we de waarheidstabellen van de basisoperaties weer.
\begin{table}[htb]
\centering
\subtable[Not]{
\begin{tabular}{c|c}
$x$&$x'$\\\hline
0&1\\
1&0
\end{tabular}}
\subtable[And]{
\begin{tabular}{cc|c}
$x$&$y$&$x\cdot y$\\\hline
0&0&0\\
0&1&0\\
1&0&0\\
1&1&1
\end{tabular}}
\subtable[Or]{
\begin{tabular}{cc|c}
$x$&$y$&$x+y$\\\hline
0&0&0\\
0&1&1\\
1&0&1\\
1&1&1
\end{tabular}}
\caption{Waarheidstabellen van de basisoperaties.}
\label{tbl:truthTablesBasicGates}
\end{table}
\paragraph{}
We kunnen vervolgens aan de hand van deze waarheidstabellen de werking van een XOR-bewerking bestuderen. We dienen eenvoudigweg basisoperaties toe te passen op deelresultaten om zo uiteindelijk het finale gedrag van de bewerking te kennen zoals in \tblref{truthTableXOR}. Indien twee implementaties voor iedere rij dezelfde uitvoer genereren, zijn de implementaties equivalent, en beschrijven ze dezelfde functies. Equivalente implementaties zijn nuttig om een schakeling effici\"enter te maken. We gaan immers op zoek naar een equivalente implementatie die minder kost of sneller werkt.
\begin{table}[htb]
\centering
\begin{tabular}{cc|ccccc|c}
$x$&$y$&$x'$&$y'$&$x'\cdot y$&$x\cdot y'$&$x'\cdot y+x\cdot y'$&$x\oplus y$\\\hline
0&0&1&1&0&0&0&0\\
0&1&1&0&1&0&1&1\\
1&0&0&1&0&1&1&1\\
1&1&0&0&0&0&0&0\\
\end{tabular}
\caption{Waarheidstabel voor de implementatie van een XOR.}
\label{tbl:truthTableXOR}
\end{table}
\subsection{Logische poorten}
\label{ss:logischePoorten}
Naast het uitrekenen van operaties heeft ons lamp-model nog een nadeel. Het is erg onpraktisch om grote en complexe schakelingen voor te stellen. Een algemeen geaccepteerde notatie is deze met behulp van \termen{logische poorten}. Poorten zijn kleine componenten die enkele ingangen bevatten, en \'e\'en uitgang. Op figuren \ref{fig:basicGatesNot} tot en met \ref{fig:basicGatesOr} geven we de poorten van de basisbewerkingen weer. We kunnen alle mogelijke schakelingen bouwen met deze poorten. Toch worden vaak ook alternatieve poorten gedefinieerd om veelgebruikte bewerkingen mee toe te passen. Bovendien kunnen we de werking van de basis poorten ook veralgemenen naar meer ingangen. Zo defini\"eren we een \termen{$n$-and} als een poort waar enkel een 1 op de uitgang verschijnt indien op alle $n$ ingangen een 1 staat. \figref{basicGatesAndExtended} toont een 3-and. Een \termen{$n$-or} is een poort waar enkel een 0 op de uitgang verschijnt indien op alle $n$ ingangen een 0 staat. Zo staat op \figref{basicGatesOrExtended} een 5-or.
\begin{figure}[htb]
\centering
\subfigure[Not]{\begin{tikzpicture}[circuit logic US]
  \node[not gate] (A) {};
  \draw (A.input -| -1,0) node[anchor=east]{$x$} -- (A.input);
  \draw (A.output) -- ++(1,0) node[anchor=west]{$L$};
\end{tikzpicture}
\figlab{basicGatesNot}
}
\subfigure[And]{\begin{tikzpicture}[circuit logic US]
  \node[and gate] (A) {};
  \draw (A.input 1 -| -1,0) node[anchor=east]{$x$} -- (A.input 1)
        (A.input 2 -| -1,0) node[anchor=east]{$y$} -- (A.input 2);
  \draw (A.output) -- ++(1,0) node[anchor=west]{$L$};
\end{tikzpicture}
\figlab{basicGatesAnd}
}
\subfigure[Or]{\begin{tikzpicture}[circuit logic US]
  \node[or gate] (A) {};
  \draw (A.input 1 -| -1,0) node[anchor=east]{$x$} -- (A.input 1)
        (A.input 2 -| -1,0) node[anchor=east]{$y$} -- (A.input 2);
  \draw (A.output) -- ++(1,0) node[anchor=west]{$L$};
\end{tikzpicture}
\figlab{basicGatesOr}
}
\subfigure[3-and]{\begin{tikzpicture}[circuit logic US]
  \node[and gate,inputs={normal,normal,normal}] (A) {};
  \draw (A.input 1 -| -0.75,0) -- (A.input 1)
        (A.input 2 -| -0.75,0) -- (A.input 2)
	(A.input 3 -| -0.75,0) -- (A.input 3);
  \draw (A.output) -- ++(0.5,0);
\end{tikzpicture}
\figlab{basicGatesAndExtended}
}
\subfigure[5-or]{\begin{tikzpicture}[circuit logic US]
  \node[or gate,inputs={normal,normal,normal,normal,normal}] (A) {};
  \draw (A.input 1 -| -0.75,0) -- (A.input 1)
        (A.input 2 -| -0.75,0) -- (A.input 2)
	(A.input 3 -| -0.75,0) -- (A.input 3)
	(A.input 4 -| -0.75,0) -- (A.input 4)
	(A.input 5 -| -0.75,0) -- (A.input 5);
  \draw (A.output) -- ++(0.5,0);
\end{tikzpicture}
\figlab{basicGatesOrExtended}
}
\caption{Basispoorten en uitbreidingen.}
\figlab{basicGates}
\end{figure}
\paragraph{Complexe poorten}
Complexe poorten die in de loop der tijd een eigen symbool kregen zijn onder meer de \termen{NOR}, \termen{NAND} en XOR, deze staan afgebeeld op \figref{complexGates}, samen met een equivalent schema.
\begin{figure}[htb]
\centering
\subfigure[NAND]{\begin{tikzpicture}[circuit logic US]
  \node[nand gate] (A) at (0.5,0) {};
  \draw (A.input 1 -| -1,0) node[anchor=east]{$x$} -- (A.input 1)
        (A.input 2 -| -1,0) node[anchor=east]{$y$} -- (A.input 2);
  \draw (A.output) -- (2.5,0) node[anchor=west]{$L$};

  \node[and gate] (A2) at (0,1) {};
  \node[not gate] (A3) at (1,1) {};
  \draw (A2.input 1 -| -1,0) node[anchor=east]{$x$} -- (A2.input 1)
        (A2.input 2 -| -1,0) node[anchor=east]{$y$} -- (A2.input 2);
  \draw (A2.output) -- (A3.input);
  \draw (A3.output) -- ++(1,0) node[anchor=west]{$L$};
\end{tikzpicture}
\figlab{complexGatesNand}
}
\subfigure[NOR]{\begin{tikzpicture}[circuit logic US]
  \node[nor gate] (A) at (0.5,0) {};
  \draw (A.input 1 -| -1,0) node[anchor=east]{$x$} -- (A.input 1)
        (A.input 2 -| -1,0) node[anchor=east]{$y$} -- (A.input 2);
  \draw (A.output) -- (2.5,0) node[anchor=west]{$L$};
  \node[or gate] (A2) at (0,1) {};
  \node[not gate] (A3) at (1,1) {};
  \draw (A2.input 1 -| -1,0) node[anchor=east]{$x$} -- (A2.input 1)
        (A2.input 2 -| -1,0) node[anchor=east]{$y$} -- (A2.input 2);
  \draw (A2.output) -- (A3.input);
  \draw (A3.output) -- ++(1,0) node[anchor=west]{$L$};
\end{tikzpicture}
\figlab{complexGatesNor}
}
\subfigure[XOR]{\begin{tikzpicture}[circuit logic US]
  \node[xor gate] (A) at (0,-0.5) {};
  \draw (A.input 1 -| -2.5,0) node[anchor=east]{$x$} -- (A.input 1)
        (A.input 2 -| -2.5,0) node[anchor=east]{$y$} -- (A.input 2);
  \draw (A.output) -- ++(2,0) node[anchor=west]{$L$};
  \node[not gate] (A2) at (-1.5,0.5) {};
  \draw (A2.input -| -2.5,0) node[anchor=east]{$y$} -- (A2.input);
  \node[not gate] (A3) at (-1.5,2.5) {};
  \draw (A3.input -| -2.5,0) node[anchor=east]{$x$} -- (A3.input);
  \node[and gate] (A4) at (0,1) {};
  \draw (A2.output) -- ++(0.375,0) |- (A4.input 2);
  \draw (A3.input -| -1.875,0) |- (A4.input 1);
  \node[and gate] (A5) at (0,2) {};
  \draw (A3.output) -- ++(0.375,0) |- (A5.input 1);
  \draw (A2.input -| -2.125,0) |- (A5.input 2);
  \node[or gate] (A6) at (1.5,1.5) {};
  \draw (A4.output) -- ++(0.5,0) |- (A6.input 2);
  \draw (A5.output) -- ++(0.5,0) |- (A6.input 1);
  \draw (A6.output) -- ++(0.5,0) node[anchor=west]{$L$};
\end{tikzpicture}
\figlab{complexGatesXor}
}
\caption{Complexe poorten.}
\figlab{complexGates}
\end{figure}
De waarheidstabellen van deze complexe poorten staan in subsectie \sscref{appendixComplexePoorten}. De reden dat NAND en NOR poorten populair zijn komt hoofdzakelijk omdat het universele poorten zijn. Dat betekent dat iedere basispoort kan ge\"implementeerd worden met behulp van NAND of NOR poorten. In \tblref{nandNorUniversal} staan deze implementaties. Bovendien is het realiseren van NAND en NOR poorten in de meeste technologie\"en goedkoper dan het bouwen van AND en OR poorten.
\begin{table}[htb]
\centering
\begin{tabular}{c|c|c|c}
&NOT&AND&OR\\\hline
met NAND&
\begin{tikzpicture}[circuit logic US]
  \node[anchor=east] (I) at (-1,0) {$x$};
  \node[nand gate] (A) at (0,0) {};
  \draw (-1,0) -- (-0.75,0)  |- (A.input 1);
  \draw (-0.75,0)  |- (A.input 2);
  \draw (A.output) -- (0.75,0) node[anchor=west]{$L$};
\end{tikzpicture}
&
\begin{tikzpicture}[circuit logic US]
  %\node[anchor=east] (I) at (-1,0) {$x$};
  \node[nand gate] (A1) at (-1.5,0) {};
  \node[nand gate] (A2) at (0,0) {};
  \draw (A1.output) -- (-0.75,0)  |- (A2.input 1);
  \draw (-0.75,0)  |- (A2.input 2);
  \draw (A1.input 1 -| -2.25,0) node[anchor=east]{$x$} -- (A1.input 1)
        (A1.input 2 -| -2.25,0) node[anchor=east]{$y$} -- (A1.input 2);
  \draw (A2.output) -- (0.75,0) node[anchor=west]{$L$};
\end{tikzpicture}
&
\begin{tikzpicture}[circuit logic US]
  \node[anchor=east] (I) at (-1,0.5) {$x$};
  \node[nand gate] (A) at (0,0.5) {};
  \draw (-1,0.5) -- (-0.75,0.5)  |- (A.input 1);
  \draw (-0.75,0.5)  |- (A.input 2);
  \node[anchor=east] (I2) at (-1,-0.5) {$y$};
  \node[nand gate] (A2) at (0,-0.5) {};
  \draw (-1,-0.5) -- (-0.75,-0.5)  |- (A2.input 1);
  \draw (-0.75,-0.5)  |- (A2.input 2);
  \node[nand gate] (A3) at (1.5,0) {};
  \draw (A3.output) -- (2.25,0) node[anchor=west]{$L$};
  \draw (A.output) -- ++(0.25,0) |- (A3.input 1);
  \draw (A2.output) -- ++(0.25,0) |- (A3.input 2);
\end{tikzpicture}
\\\hline
met NOR&
\begin{tikzpicture}[circuit logic US]
  \node[anchor=east] (I) at (-1,0) {$x$};
  \node[nor gate] (A) at (0,0) {};
  \draw (-1,0) -- (-0.75,0)  |- (A.input 1);
  \draw (-0.75,0)  |- (A.input 2);
  \draw (A.output) -- (0.75,0) node[anchor=west]{$L$};
\end{tikzpicture}
&
\begin{tikzpicture}[circuit logic US]
  \node[anchor=east] (I) at (-1,0.5) {$x$};
  \node[nor gate] (A) at (0,0.5) {};
  \draw (-1,0.5) -- (-0.75,0.5)  |- (A.input 1);
  \draw (-0.75,0.5)  |- (A.input 2);
  \node[anchor=east] (I2) at (-1,-0.5) {$y$};
  \node[nor gate] (A2) at (0,-0.5) {};
  \draw (-1,-0.5) -- (-0.75,-0.5)  |- (A2.input 1);
  \draw (-0.75,-0.5)  |- (A2.input 2);
  \node[nor gate] (A3) at (1.5,0) {};
  \draw (A3.output) -- (2.25,0) node[anchor=west]{$L$};
  \draw (A.output) -- ++(0.25,0) |- (A3.input 1);
  \draw (A2.output) -- ++(0.25,0) |- (A3.input 2);
\end{tikzpicture}
&
\begin{tikzpicture}[circuit logic US]
  %\node[anchor=east] (I) at (-1,0) {$x$};
  \node[nor gate] (A1) at (-1.5,0) {};
  \node[nor gate] (A2) at (0,0) {};
  \draw (A1.output) -- (-0.75,0)  |- (A2.input 1);
  \draw (-0.75,0)  |- (A2.input 2);
  \draw (A1.input 1 -| -2.25,0) node[anchor=east]{$x$} -- (A1.input 1)
        (A1.input 2 -| -2.25,0) node[anchor=east]{$y$} -- (A1.input 2);
  \draw (A2.output) -- (0.75,0) node[anchor=west]{$L$};
\end{tikzpicture}
\end{tabular}
\caption{Implementatie van de basispoorten met behulp van NAND en NOR poorten.}
\label{tbl:nandNorUniversal}
\end{table}
\paragraph{Ge\"inverteerde ingangen}Tot slot nog een andere conventie die vaak gebruikt wordt. NOT poorten worden heel vaak gebruikt, ook bij ingangen van andere poorten. De NOT poort neemt nogal wat plaats in, daardoor is het de gewoonte om soms cirkels te tekenen aan de ingangen van een bepaalde poort: \termen{ge\"inverteerde ingangen}. Deze cirkels stellen een NOT poort voor. Concrete voorbeelden staan op \figref{basicGatesInvertedInput}. In het algemeen kunnen we dus zeggen dat een cirkel duidt op het inverteren. Inverteren in het schema betekent echter niet noodzakelijk dat we bij de fysische implementatie gebruik moeten maken van een inverter (meestal is het zelfs omgekeerd).
\begin{figure}[htb]
\centering
\subfigure{\begin{tikzpicture}[circuit logic US]
  \node[and gate,inputs={inverted,inverted}] (A) {};
  \draw (A.input 1 -| -0.75,0) -- (A.input 1)
        (A.input 2 -| -0.75,0) -- (A.input 2);
  \draw (A.output) -- ++(0.5,0);
\end{tikzpicture}
}
\subfigure{\begin{tikzpicture}[circuit logic US]
  \node[or gate, inputs={normal,inverted}] (A) {};
  \draw (A.input 1 -| -0.75,0) -- (A.input 1)
        (A.input 2 -| -0.75,0) -- (A.input 2);
  \draw (A.output) -- ++(0.5,0);
\end{tikzpicture}
}
\subfigure{\begin{tikzpicture}[circuit logic US]
  \node[and gate,inputs={normal,normal,inverted}] (A) {};
  \draw (A.input 1 -| -0.75,0) -- (A.input 1)
        (A.input 2 -| -0.75,0) -- (A.input 2)
	(A.input 3 -| -0.75,0) -- (A.input 3);
  \draw (A.output) -- ++(0.5,0);
\end{tikzpicture}
}
\subfigure{\begin{tikzpicture}[circuit logic US]
  \node[or gate,inputs={normal,inverted,inverted,normal,normal}] (A) {};
  \draw (A.input 1 -| -0.75,0) -- (A.input 1)
        (A.input 2 -| -0.75,0) -- (A.input 2)
	(A.input 3 -| -0.75,0) -- (A.input 3)
	(A.input 4 -| -0.75,0) -- (A.input 4)
	(A.input 5 -| -0.75,0) -- (A.input 5);
  \draw (A.output) -- ++(0.5,0);
\end{tikzpicture}
}
\caption{Poorten met ge\"inverteerde ingangen.}
\figlab{basicGatesInvertedInput}
\end{figure}
\subsection{Logische schakelingen}
De vorige subsectie toonde al dat we met deze poorten netwerken kunnen bouwen. Deze netwerken worden \termen{logische schakelingen} genoemd. Deze schakelingen implementeren dan uiteindelijk de functionaliteiten waarvoor we een digitaal circuit ontwerpen. We kunnen logische schakelingen beschrijven door middel van een schema zoals we dat tot nu toe altijd gedaan hebben. Een andere techniek is echter op basis van een taal: VHDL\footnote{VHDL: VHSIC Hardware Design Language; VHSIC: Very High Speed Integrated Circuit.}. Deze taal komt onder meer aan bod in sectie \ref{s:vhdl}, en verder in de verdere hoofdstukken van deze cursus.
\paragraph{Optimaliseren}
Een belangrijk probleem met schakelingen is het vinden van een optimale implementatie voor een bepaalde functie. Sommige schakelingen zijn immers equivalent. Bijgevolg zoeken we voor een probleem uit de set van equivalente schakelingen naar de schakeling die ons het meeste voordeel oplevert. Hiervoor zijn enkele parameters belangrijk: kostprijs en verwerkingskracht. We proberen de kostprijs immers te minimaliseren. De kostprijs is eenvoudig te berekenen met volgende formule:
\begin{equation}
\mbox{Kostprijs}=\mbox{\#poortingangen}+\mbox{\#poortuitgangen}-\mbox{\#inverters}
\label{eqn:kosten}
\end{equation}
Merk op dat deze formule geen eenheid heeft. Het is dan ook niet de bedoeling de kostprijs in bijvoorbeeld euro te berekenen. Het is eerder bedoeld als een ruwe metriek om verschillende schakelingen met elkaar te kunnen vergelijken.
\subparagraph{}
De verwerkingskracht wordt bepaald door het concept van de zwakste schakel. Ketens waarbij de uitgangen van poorten weer nieuwe ingangen aansturen zijn immers nefast. We streven dus naar circuits waarbij een signaal slechts door een beperkt aantal poorten heen moet. De lengte van het langste pad is echter niet makkelijk te berekenen. Immers hangt de vertraging van een poort af van onder meer het type en het aantal ingangen. Het totaal aantal ingangen daarentegen geeft in de meeste gevallen een goede ruwe schatting. We formaliseren tot:
\begin{equation}
\mbox{Minimale vertraging}\propto\mbox{\#poortingangen}%=?
\end{equation}
\paragraph{Tijdsgedrag}In de vorige paragraaf hadden we het reeds over performantie. Snelle systemen worden echter vaak beperkt door het tijdsgedrag van logische schakelingen. Wiskundig gezien levert een verandering aan de ingang immers altijd onmiddellijk een correct signaal aan de uitgang. Maar zoals we reeds gesteld hebben, heeft een poort een zekere tijd nodig om de verandering aan de ingang door te rekenen. Dit lijkt slechts een detail. Een nadelig effect hiervan is echter dat er overgangsverschijnselen kunnen ontstaan. In de eerste plaats omdat een waarde afhangt van twee ketens van poorten. En het veranderde ingangssignaal propageert zich sneller door de ene dan door de andere keten. Aan de andere kant ook omdat men nu eenmaal niet iedere poort uniform kan aanmaken. Tussen twee (dezelfde) poorten kunnen toch kleine tijdsverschillen optreden. In dat geval spreken we van een \termen{glitch}. \figref{timeBehaviorGlitchExample} toont een scenario van veranderende signalen in een logische schakeling met een glitch.
\begin{figure}[htb]
\centering
\begin{tikzpicture}[circuit logic US]
  \node[not gate] (N) at (0,0.5) {};
  \node[and gate] (A) at (0,-0.5) {};
  \node[or gate] (O) at (2,0) {};
  \draw (-1.5,0.5) node[anchor=east]{$x$} -- (N.input);
  \draw (A.input 2 -| -1.5,0) node[anchor=east]{$y$} -- (A.input 2);
  \draw (-1,0.5) |- (A.input 1);
  \draw (N.output) -- (1,0.5) node[anchor=south east]{$a$} |- (O.input 1);
  \draw (A.output) -- (1,-0.5) node[anchor=north east]{$b$} |- (O.input 2);
  \draw (O.output) -- ++(0.5,0) node[anchor=west]{$f$};
  \begin{scope}[xshift=5 cm]
  \draw[thick,->] (0,2) -- (0,-2) -- (6,-2) node[anchor=south east]{$t$};
  \foreach\y in {0,1,...,3} {
    \draw[gray] (-1,-1.2+0.8*\y) -- (6,-1.2+0.8*\y);
    \draw[dashed] (1+\y,-2) -- (1+\y,2);
  }
  \foreach\y/\ty in {0/f,1/b,2/a,3/y,4/x} {
    \draw (-0.5,-1.6+0.8*\y) node[anchor=east]{$\ty$};
    \draw (0,-1.8+0.8*\y) node[anchor=east]{0};
    \draw (0,-1.4+0.8*\y) node[anchor=east]{1};
  }
  \begin{scope}[yshift=1.4 cm,yscale=0.4]
    \draw (0,0) -- (1,0) -- (1,1) -- (3,1) -- (3,0) -- (6,0);
  \end{scope}
  \begin{scope}[yshift=0.6 cm,yscale=0.4]
    \draw (0,0) -- (2,0) -- (2,1) -- (4,1) -- (4,0) -- (6,0);
  \end{scope}
  \begin{scope}[yshift=-0.2 cm,yscale=0.4]%not is lagging 0.2
    \draw (0,1) -- (1.2,1) -- (1.2,0) -- (3.2,0) -- (3.2,1) -- (6,1);
  \end{scope}
  \begin{scope}[yshift=-1 cm,yscale=0.4]%and is lagging 0.1
    \draw (0,0) -- (2.1,0) -- (2.1,1) -- (3.1,1) -- (3.1,0) -- (6,0);
  \end{scope}
  \begin{scope}[yshift=-1.8 cm,yscale=0.4]%or is lagging 0.15
    \draw (0,1) -- (1.35,1) -- (1.35,0) -- (2.25,0) -- (2.25,1) -- (3.25,1) -- (3.25,0) -- (3.35,0) node[anchor=south west,yshift=-0.1 cm]{``glitch''} -- (3.35,1) -- (6,1);
  \end{scope}
  \end{scope}
\end{tikzpicture}
\caption{Voorbeeld van het tijdsgedrag van een logische schakeling met een ``glitch''.}
\figlab{timeBehaviorGlitchExample}
\end{figure}
\section{Booleaanse algebra}
\label{s:booleaanseAlgebra}
De tak van de wiskunde die zich bezighoudt met logische bewerkingen is de \termen{booleaanse algebra}. Deze algebra rekent uitsluitend met de twee logische waarden $\left\{0,1\right\}$ en drie logische operatoren: de ons reeds bekende NOT ($'$), AND ($\cdot$) en OR ($+$). Deze operaties hebben elk een specifieke prioriteit zodat men met een minimum aan haakjes toch een complexe expressie kan beschrijven. Zo heeft de NOT prioriteit op de AND die prioriteit heeft op de OR.\footnote{Een ezelsbruggetje om dit te onthouden is het woord NANO: \underline{N}ot \underline{AN}d \underline{O}r.}
\subsection{Theorema's en eigenschappen}
\label{ss:theoremasPropertiesBooleanAlgebra}
De booleaanse algebra maakt gebruik van theorema's om de operatoren te evalueren. Zo worden de resultaten van logische operatoren zonder variabelen gedefinieerd zoals we dat ook met een waarheidstabel kunnen doen. Deze definities staan in \tblref{booleanAlgebraNone}.
\begin{table}[htb]
\centering
\begin{tabular}{ll}
$0\cdot0=0$&$1+1=1$\\
$1\cdot1=1$&$0+0=0$\\
$0\cdot1=1\cdot0=0$&$1+0=0+1=1$\\
$0'=1$&$1'=0$
\end{tabular}
\caption{Booleaanse algebra zonder variabelen.}
\label{tbl:booleanAlgebraNone}
\end{table}
In de booleaanse algebra wordt ook met variabelen gerekend. Op die manier kan men vaak expressies manipuleren en optimaliseren. Indien we slechts \'e\'en variabele beschouwen gelden regels weergegeven in \tblref{booleanAlgebraSingle}.
\begin{table}[htb]
\centering
\begin{tabular}{ll}
$x\cdot 0=0$&$x+1=1$\\
$x\cdot 1=x$&$x+0=x$\\
$x\cdot x=x$&$x+x=x$\\
$x\cdot x'=0$&$x+x'=1$\\
$(x')'=x$&\\
\end{tabular}
\caption{Booleaanse algebra met \'e\'en variabele.}
\label{tbl:booleanAlgebraSingle}
\end{table}
Tot slot werden ook regels gedefinieerd voor meerdere variabelen. Deze regels kunnen onder meer het aantal variabelen reduceren evenals het aantal operatoren, of tot alternatieve implementaties leiden die mogelijk sneller werken. Een opsomming van deze wetten staan in \tblref{booleanAlgebraMultiple}.
\begin{table}[htb]
\centering
\begin{tabular}{ll}
\multicolumn{2}{c}{\termen{Commutativiteit}}\\
$x\cdot y=y\cdot x$&$x+y=y+x$\\
\multicolumn{2}{c}{\termen{Associativiteit}}\\
$x\cdot (y\cdot z)=(x\cdot y)\cdot z$&$x+(y+z)=(x+y)+z$\\
\multicolumn{2}{c}{\termen{Distributiviteit}}\\
$x\cdot (y+z)=x\cdot y+x\cdot z$&$x+(y\cdot z)=(x+y)\cdot (x+z)$\\
\multicolumn{2}{c}{\termen{Absorptie}}\\
$x+x\cdot y=x$&$x\cdot(x+y)=x$\\
\multicolumn{2}{c}{\termen{Wet van De Morgan}}\\
$(x\cdot y)'=x'+y'$&$(x+y)'=x'\cdot y'$
\end{tabular}
\caption{Booleaanse algebra met meerdere variabelen.}
\label{tbl:booleanAlgebraMultiple}
\end{table}
\paragraph{Dualiteit}Een opmerkelijke eigenschap bij booleaanse algebra is de dualiteit. Indien we bij een van de wetten uit \tblref{booleanAlgebraNone}, \tblref{booleanAlgebraSingle} of \tblref{booleanAlgebraMultiple}. De OR operaties door AND operaties vervangen en vice-versa, en de 1 door 0 vervangen en vice-versa, bekomen we eenvoudigweg een andere wet uit deze tabellen. Deze eigenschap noemen we de \termen{dualiteit} van de booleaanse algebra. Bovendien geldt ook dat slechts de helft van de gestelde wetten vereist is. Het andere deel kan met de wetten van De Morgan afgeleid worden.
\paragraph{Verschil met de gewone algebra}Omdat ook de booleaanse algebra een optelling en vermenigvuldiging lijkt te defini\"eren en er bovendien ook analogie\"en te trekken zijn, lijkt het soms dat booleaanse algebra niet verschilt van de standaard algebra. Toch zijn er enkele opmerkelijke verschillen. Zo bestaat er geen verschil of deling in de booleaanse algebra. Verder voldoet in de booleaanse algebra volgend stelsel voor iedere $y$:
\begin{equation}
\left\{\begin{array}{l}
y+y'=1\\
y\cdot y=y
\end{array}\right.
\end{equation}
Terwijl in de standaard algebra er geen enkele $y$ is waarvoor dit geldt. Tot slot is bij standaard algebra de optelling ook niet distributief tegenover de vermenigvuldiging, zo is $5+2\cdot4\neq(5+2)\cdot(5+4)$.
\section{Synthese met logische poorten}
\label{s:synthese}
Nu we weten hoe logische schakelingen werken, en hoe we deze door middel van de booleaanse algebra kunnen uitwerken, wordt het tijd om zelf schakelingen te bouwen. Dit doen we vertrekkend vanuit een waarheidstabel. Soms is het makkelijk om een implementatie af te leiden uit deze waarheidstabellen. Maar indien het aantal variabelen groot wordt of de functie complex is, wordt het vinden van een implementatie moeilijk. In deze cursus stellen we dan ook enkele algemene technieken voor om uit een waarheidstabel een logische functie te genereren. In sectie \ref{s:minimalisatie} zullen we bovendien enkele methodes beschrijven om deze logische implementaties te optimaliseren. Doorheen deze sectie zullen we een logische functie genereren voor de waarheidstabel in \tblref{truthTableExample}.
\begin{table}[htb]
\centering
\begin{tabular}{ccc|c}
$x$&$y$&$z$&$f$\\\hline
0&0&0&0\\
0&0&1&1\\
0&1&0&0\\
0&1&1&0\\
1&0&0&1\\
1&0&1&1\\
1&1&0&1\\
1&1&1&0\\
\end{tabular}
\caption{Waarheidstabel voor het synthese-voorbeeld.}
\label{tbl:truthTableExample}
\end{table}
In de waarheidstabel zien we uitsluitend 0 en 1 staan. In sectie \ref{par:dontcare} zullen we echter kennis maken met een derde toestand: de \termen{don't care}.
\subsection{SOP \& POS}
\termen{Sum-of-Products (SOP)} en \termen{Product-of-Sums (POS)} zijn twee technieken die op een machinale manier uit een waarheidstabel een logische functie genereren. Ze maken respectievelijk gebruik van \termen{mintermen} en \termen{maxtermen}.
\paragraph{Sum-of-Products} De Sum-of-Products maakt gebruik van mintermen. Een minterm is een logische functie die slechts waar is voor \'e\'en gegeven rij in een waarheidstabel. Dit bekomen we door het booleaans product te nemen van alle variabelen. Indien de variabelen negatief zijn wordt de negatie in het product verwerkt. De minterm van rij $i$ noteren we als $m_i$. De Sum-of-Products methode neemt de booleaanse som van \termen{1-mintermen}. Een 1-minterm is een minterm die waar is bij een rij waarbij $f$ ook waar is. \tblref{truthTableExampleMinTerms} toont voor iedere rij de minterm. Wanneer $f$ waar is, wordt ook de 1-minterm ingevuld. Vervolgens bepalen we de som van alle 1-mintermen.
\begin{table}[htb]
\begin{center}
\begin{tabular}{m{0.65\textwidth}|m{0.35\textwidth}}
\begin{center}
\begin{tabular}{ccc|l|c|l}
$x$&$y$&$z$&minterm&$f$&1-minterm\\\hline
0&0&0&$m_0=x'\cdot y'\cdot z'$&0&\\
0&0&1&$m_1=x'\cdot y'\cdot z$&1&$m_1=x'\cdot y'\cdot z$\\
0&1&0&$m_2=x'\cdot y\cdot z'$&0&\\
0&1&1&$m_3=x'\cdot y\cdot z$&0&\\
1&0&0&$m_4=x\cdot y'\cdot z'$&1&$m_4=x\cdot y'\cdot z'$\\
1&0&1&$m_5=x\cdot y'\cdot z$&1&$m_5=x\cdot y'\cdot z$\\
1&1&0&$m_6=x\cdot y\cdot z'$&1&$m_6=x\cdot y\cdot z'$\\
1&1&1&$m_7=x\cdot y\cdot z$&0&\\
\end{tabular}
\end{center}
&
\begin{center}
\begin{tikzpicture}[circuit logic US]
\draw (-2.7,1.7) node[anchor=south]{$x$} -- (-2.7,-1.7);
\draw (-2.5,1.7) node[anchor=south]{$y$} -- (-2.5,-1.7);
\draw (-2.3,1.7) node[anchor=south]{$z$} -- (-2.3,-1.7);

\node[or gate,inputs={normal,normal,normal,normal}] (O) at (0,0) {};
\draw (O.output) -- ++(0.25,0) node[anchor=west]{$f$};

\node[and gate,inputs={inverted,inverted,normal}] (A1) at (-1.5,1.2) {$m_1$};
\draw (A1.output) -- ++(0.35,0) |- (O.input 1);
\draw (A1.input 1 -| -2.7,0) -- (A1.input 1);
\draw (A1.input 2 -| -2.5,0) -- (A1.input 2);
\draw (A1.input 3 -| -2.3,0) -- (A1.input 3);
\node[and gate,inputs={normal,inverted,inverted}] (A4) at (-1.5,0.4) {$m_4$};
\draw (A4.output) -- ++(0.2,0) |- (O.input 2);
\draw (A4.input 1 -| -2.7,0) -- (A4.input 1);
\draw (A4.input 2 -| -2.5,0) -- (A4.input 2);
\draw (A4.input 3 -| -2.3,0) -- (A4.input 3);
\node[and gate,inputs={normal,inverted,normal}] (A5) at (-1.5,-0.4) {$m_5$};
\draw (A5.output) -- ++(0.2,0) |- (O.input 3);
\draw (A5.input 1 -| -2.7,0) -- (A5.input 1);
\draw (A5.input 2 -| -2.5,0) -- (A5.input 2);
\draw (A5.input 3 -| -2.3,0) -- (A5.input 3);
\node[and gate,inputs={normal,normal,inverted}] (A6) at (-1.5,-1.2) {$m_6$};
\draw (A6.output) -- ++(0.35,0) |- (O.input 4);
\draw (A6.input 1 -| -2.7,0) -- (A6.input 1);
\draw (A6.input 2 -| -2.5,0) -- (A6.input 2);
\draw (A6.input 3 -| -2.3,0) -- (A6.input 3);
\end{tikzpicture}
\end{center}
\end{tabular}\\
SOP: $f=m_1+m_4+m_5+m_6=x'\cdot y'\cdot z+x\cdot y'\cdot z'+x\cdot y'\cdot z+x\cdot y\cdot z'$
\end{center}
\caption{Sum-of-Products methode toegepast op het voorbeeld.}
\label{tbl:truthTableExampleMinTerms}
\end{table}
\paragraph{Product-of-Sums} Analoog maakt de Product-of-Sums gebruik van maxtermen. Zoals de naam reeds doet vermoeden is een maxterm een logische functie die altijd waar is behalve voor \'e\'en rij in de waarheidstabel. Indien we de maxterm voor een bepaalde term willen genereren dienen we de booleaanse optelling van de negatie van de variabelen te nemen. We noteren een maxterm voor rij $i$ met $M_i$. De Product-of-Sums methode neemt analoog het booleaanse product van de \termen{0-maxtermen}. Een 0-maxterm is analoog een maxterm wanneer $f$ onwaar is. \tblref{truthTableExampleMaxTerms} toont voor iedere rij de maxterm. Wanneer $f$ onwaar is, wordt ook de 0-maxterm ingevuld. Vervolgens bepalen we het product van alle 0-maxtermen.
\begin{table}[htb]
\begin{center}
\begin{tabular}{m{0.65\textwidth}|m{0.35\textwidth}}
\begin{center}

\begin{tabular}{ccc|l|c|l}
$x$&$y$&$z$&maxterm&$f$&0-maxterm\\\hline
0&0&0&$M_0=x+y+z$&0&$M_0=x+y+z$\\
0&0&1&$M_1=x+y+z'$&1&\\
0&1&0&$M_2=x+y'+z$&0&$M_2=x+y'+z$\\
0&1&1&$M_3=x+y'+z'$&0&$M_3=x+y'+z'$\\
1&0&0&$M_4=x'+y+z$&1&\\
1&0&1&$M_5=x'+y+z'$&1&\\
1&1&0&$M_6=x'+y'+z$&1&\\
1&1&1&$M_7=x'+y'+z'$&0&$M_7=x'+y'+z'$\\
\end{tabular}
\end{center}
&
\begin{center}
\begin{tikzpicture}[circuit logic US]
\draw (-2.7,1.7) node[anchor=south]{$x$} -- (-2.7,-1.7);
\draw (-2.5,1.7) node[anchor=south]{$y$} -- (-2.5,-1.7);
\draw (-2.3,1.7) node[anchor=south]{$z$} -- (-2.3,-1.7);

\node[and gate,inputs={normal,normal,normal,normal}] (O) at (0,0) {};
\draw (O.output) -- ++(0.25,0) node[anchor=west]{$f$};

\node[or gate,inputs={normal,normal,normal}] (A1) at (-1.5,1.2) {$M_0$};
\draw (A1.output) -- ++(0.35,0) |- (O.input 1);
\draw (A1.input 1 -| -2.7,0) -- (A1.input 1);
\draw (A1.input 2 -| -2.5,0) -- (A1.input 2);
\draw (A1.input 3 -| -2.3,0) -- (A1.input 3);
\node[or gate,inputs={normal,inverted,normal}] (A4) at (-1.5,0.4) {$M_2$};
\draw (A4.output) -- ++(0.2,0) |- (O.input 2);
\draw (A4.input 1 -| -2.7,0) -- (A4.input 1);
\draw (A4.input 2 -| -2.5,0) -- (A4.input 2);
\draw (A4.input 3 -| -2.3,0) -- (A4.input 3);
\node[or gate,inputs={normal,inverted,inverted}] (A5) at (-1.5,-0.4) {$M_3$};
\draw (A5.output) -- ++(0.2,0) |- (O.input 3);
\draw (A5.input 1 -| -2.7,0) -- (A5.input 1);
\draw (A5.input 2 -| -2.5,0) -- (A5.input 2);
\draw (A5.input 3 -| -2.3,0) -- (A5.input 3);
\node[or gate,inputs={inverted,inverted,inverted}] (A6) at (-1.5,-1.2) {$M_7$};
\draw (A6.output) -- ++(0.35,0) |- (O.input 4);
\draw (A6.input 1 -| -2.7,0) -- (A6.input 1);
\draw (A6.input 2 -| -2.5,0) -- (A6.input 2);
\draw (A6.input 3 -| -2.3,0) -- (A6.input 3);
\end{tikzpicture}
\end{center}
\end{tabular}\\
POS: $f=M_0\cdot M_2\cdot M_3\cdot M_7=\left(x+y+z\right)\cdot\left(x+y'+z\right)\cdot\left(x+y'+z'\right)\cdot\left(x'+y'+z'\right)$
\end{center}
\caption{Product-of-Sums methode toegepast op het voorbeeld.}
\label{tbl:truthTableExampleMaxTerms}
\end{table}
\subsection{Canonieke versus standaard realisatie}
\label{ss:canoniekestandaardrealisatie}
Wanneer we de Sum-of-Products of Product-of-Sums methodes toepassen, zullen alle min- of maxtermen alle variabelen bevatten. Zo'n implementatie wordt de \termen{Canonieke Vorm} genoemd. Dit is bijgevolg een relatief dure implementatie. Nochtans bestaat er een eenvoudige methode om een groot deel van de poorten te elimineren of te reduceren in aantal ingangen. Deze methode zet de canonieke vorm om in de \termen{Standaard Vorm}. De methode komt er op neer verschillende min- of maxtermen samen te nemen door er de wetten van De Morgan op toe te passen. Op die manier kunnen we dan deelexpressies van de vorm $x+x'$ uitkomen. Omdat deze deelexpressies altijd waar zijn onafhankelijk van $x$ kan bijgevolg $x$ geëlimineerd worden uit de nieuwe min- of maxterm. Op \figref{standardizationMinMaxTerms} herleiden we de canonieke vormen van het voorbeeld naar hun standaard equivalent.
\begin{figure}[htb]
\centering
\begin{tikzpicture}
\def\dx{8 cm};
\def\dxh{0.5*\dx};
\def\dy{8 cm};
\draw[thick] (\dx,0) -- (\dx,\dy+0.25 cm);
\draw (\dxh,\dy) node {Sum-of-Products};
\draw (\dxh,\dy-0.5 cm) node {\small $x'y'z+xy'z'+xy'z+xyz'$};
\draw (\dxh,\dy-2.75 cm) node{\begin{tikzpicture}[circuit logic US]
\draw (-2.7,1.7) node[anchor=south]{$x$} -- (-2.7,-1.7);
\draw (-2.5,1.7) node[anchor=south]{$y$} -- (-2.5,-1.7);
\draw (-2.3,1.7) node[anchor=south]{$z$} -- (-2.3,-1.7);

\node[or gate,inputs={normal,normal,normal,normal}] (O) at (0,0) {};
\draw (O.output) -- ++(0.25,0) node[anchor=west]{$f$};

\node[and gate,inputs={inverted,inverted,normal}] (A1) at (-1.5,1.2) {};
\draw (A1.output) -- ++(0.35,0) |- (O.input 1);
\draw (A1.input 1 -| -2.7,0) -- (A1.input 1);
\draw (A1.input 2 -| -2.5,0) -- (A1.input 2);
\draw (A1.input 3 -| -2.3,0) -- (A1.input 3);
\node[and gate,inputs={normal,inverted,inverted}] (A4) at (-1.5,0.4) {};
\draw (A4.output) -- ++(0.2,0) |- (O.input 2);
\draw (A4.input 1 -| -2.7,0) -- (A4.input 1);
\draw (A4.input 2 -| -2.5,0) -- (A4.input 2);
\draw (A4.input 3 -| -2.3,0) -- (A4.input 3);
\node[and gate,inputs={normal,inverted,normal}] (A5) at (-1.5,-0.4) {};
\draw (A5.output) -- ++(0.2,0) |- (O.input 3);
\draw (A5.input 1 -| -2.7,0) -- (A5.input 1);
\draw (A5.input 2 -| -2.5,0) -- (A5.input 2);
\draw (A5.input 3 -| -2.3,0) -- (A5.input 3);
\node[and gate,inputs={normal,normal,inverted}] (A6) at (-1.5,-1.2) {};
\draw (A6.output) -- ++(0.35,0) |- (O.input 4);
\draw (A6.input 1 -| -2.7,0) -- (A6.input 1);
\draw (A6.input 2 -| -2.5,0) -- (A6.input 2);
\draw (A6.input 3 -| -2.3,0) -- (A6.input 3);
\end{tikzpicture}};
\draw (\dxh,\dy-5 cm) node {\small $=\left(x+x'\right)y'z+\left(y+y'\right)xz'=y'z+xz'$};
\draw (\dxh,\dy-6.5 cm) node{\begin{tikzpicture}[circuit logic US]
\draw (-2.7,0.9) node[anchor=south]{$x$} -- (-2.7,-0.9);
\draw (-2.5,0.9) node[anchor=south]{$y$} -- (-2.5,-0.9);
\draw (-2.3,0.9) node[anchor=south]{$z$} -- (-2.3,-0.9);

\node[or gate] (O) at (0,0) {};
\draw (O.output) -- ++(0.25,0) node[anchor=west]{$f$};
\node[and gate,inputs={inverted,normal}] (A1) at (-1.5,0.4) {};
\draw (A1.output) -- ++(0.2,0) |- (O.input 1);
\draw (A1.input 1 -| -2.5,0) -- (A1.input 1);
\draw (A1.input 2 -| -2.3,0) -- (A1.input 2);
\node[and gate,inputs={normal,inverted}] (A2) at (-1.5,-0.4) {};
\draw (A2.output) -- ++(0.2,0) |- (O.input 2);
\draw (A2.input 1 -| -2.7,0) -- (A2.input 1);
\draw (A2.input 2 -| -2.3,0) -- (A2.input 2);
\end{tikzpicture}};
\begin{scope}[xshift=\dx]
\draw (\dxh,\dy) node {Product-of-Sums};
\draw (\dxh,\dy-0.5 cm) node {\small $\left(x+y+z\right)\left(x+y'+z\right)\left(x+y'+z'\right)\left(x'+y'+z'\right)$};
\draw (\dxh,\dy-2.75 cm) node{\begin{tikzpicture}[circuit logic US]
\draw (-2.7,1.7) node[anchor=south]{$x$} -- (-2.7,-1.7);
\draw (-2.5,1.7) node[anchor=south]{$y$} -- (-2.5,-1.7);
\draw (-2.3,1.7) node[anchor=south]{$z$} -- (-2.3,-1.7);

\node[and gate,inputs={normal,normal,normal,normal}] (O) at (0,0) {};
\draw (O.output) -- ++(0.25,0) node[anchor=west]{$f$};

\node[or gate,inputs={normal,normal,normal}] (A1) at (-1.5,1.2) {};
\draw (A1.output) -- ++(0.35,0) |- (O.input 1);
\draw (A1.input 1 -| -2.7,0) -- (A1.input 1);
\draw (A1.input 2 -| -2.5,0) -- (A1.input 2);
\draw (A1.input 3 -| -2.3,0) -- (A1.input 3);
\node[or gate,inputs={normal,inverted,normal}] (A4) at (-1.5,0.4) {};
\draw (A4.output) -- ++(0.2,0) |- (O.input 2);
\draw (A4.input 1 -| -2.7,0) -- (A4.input 1);
\draw (A4.input 2 -| -2.5,0) -- (A4.input 2);
\draw (A4.input 3 -| -2.3,0) -- (A4.input 3);
\node[or gate,inputs={normal,inverted,inverted}] (A5) at (-1.5,-0.4) {};
\draw (A5.output) -- ++(0.2,0) |- (O.input 3);
\draw (A5.input 1 -| -2.7,0) -- (A5.input 1);
\draw (A5.input 2 -| -2.5,0) -- (A5.input 2);
\draw (A5.input 3 -| -2.3,0) -- (A5.input 3);
\node[or gate,inputs={inverted,inverted,inverted}] (A6) at (-1.5,-1.2) {};
\draw (A6.output) -- ++(0.35,0) |- (O.input 4);
\draw (A6.input 1 -| -2.7,0) -- (A6.input 1);
\draw (A6.input 2 -| -2.5,0) -- (A6.input 2);
\draw (A6.input 3 -| -2.3,0) -- (A6.input 3);
\end{tikzpicture}};
\draw (\dxh,\dy-5 cm) node {\small $=\left(y+y'\right)\left(x+z\right)\left(x+x'\right)\left(y'+z'\right)=\left(x+z\right)\left(y'+z'\right)$};
\draw (\dxh,\dy-6.5 cm) node{\begin{tikzpicture}[circuit logic US]
\draw (-2.7,0.9) node[anchor=south]{$x$} -- (-2.7,-0.9);
\draw (-2.5,0.9) node[anchor=south]{$y$} -- (-2.5,-0.9);
\draw (-2.3,0.9) node[anchor=south]{$z$} -- (-2.3,-0.9);

\node[and gate] (O) at (0,0) {};
\draw (O.output) -- ++(0.25,0) node[anchor=west]{$f$};
\node[or gate] (A1) at (-1.5,0.4) {};
\draw (A1.output) -- ++(0.2,0) |- (O.input 1);
\draw (A1.input 1 -| -2.7,0) -- (A1.input 1);
\draw (A1.input 2 -| -2.3,0) -- (A1.input 2);
\node[or gate,inputs={inverted,inverted}] (A2) at (-1.5,-0.4) {};
\draw (A2.output) -- ++(0.2,0) |- (O.input 2);
\draw (A2.input 1 -| -2.5,0) -- (A2.input 1);
\draw (A2.input 2 -| -2.3,0) -- (A2.input 2);
\end{tikzpicture}};
\end{scope}
\end{tikzpicture}
\caption{Herleiden naar Standaard Vorm van voorbeeld.}
\figlab{standardizationMinMaxTerms}
\end{figure}
\paragraph{}
De Standaard Vorm is gegarandeerd de meest minimale vorm voor schakelingen met twee lagen. Het is echter wel mogelijk goedkopere schakelingen te ontwerpen die uit meerdere lagen bestaat. In dat geval dient er echter een trade-off gemaakt te worden tussen kosten enerzijds en performantie anderzijds. Zo toont \figref{alternativeStandardForm} een alternatieve implementatie van een standaardvorm. Deze is 8\% goedkoper\footnote{De relatieve kost van een circuit kan berekend worden met vergelijking \ref{eqn:kosten}.}, maar zal slechts aan 66\% van de snelheid werken. In dat geval zal de toepassing vaak uitmaken wat het meest opportuun is.
\begin{figure}[htb]
\centering
\begin{tikzpicture}[circuit logic US]
\node[or gate,inputs={normal,normal,normal}] (O) at (0,0) {};
\draw (O.output) -- ++(0.25,0) node[anchor=west]{$f$};
\node[and gate] (A1) at (-1.5,0.8) {};
\draw (A1.input 2 -| -2.5,0) node[anchor=east]{$x$} -- (A1.input 2);
\draw (A1.output) -- ++(0.2,0) |- (O.input 1);
\node[and gate] (A2) at (-1.5,0) {};
\draw (A1.input 2 -| -2,0) |- (A2.input 1);
\draw (A2.input 2 -| -2.5,0) node[anchor=east]{$y$} -- (A2.input 2);
\draw (A2.output) -- ++(0.2,0) |- (O.input 2);
\node[and gate] (A3) at (-1.5,-0.8) {};
\draw (A2.input 2 -| -2,0) |- (A3.input 1);
\draw (A3.input 2 -| -2.5,0) node[anchor=east]{$z$} -- (A3.input 2);
\draw (A3.output) -- ++(0.2,0) |- (O.input 3);
\draw (A3.input 2 -| -2.25,0) |- (A1.input 1);
\draw (-0.75,-1.5) node{$f=xy+yz+xz$};
\begin{scope}[xshift=7.5 cm]
\node[or gate] (O1) at (0,0) {};
\draw (O1.output) -- ++(0.25,0) node[anchor=west]{$f$};
\node[and gate] (A1) at (-1.5,0.6) {};
\draw (A1.input 1 -| -4,0) node[anchor=east]{$x$} -- (A1.input 1);
\draw (A1.output) -- ++(0.2,0) |- (O1.input 1);
\node[and gate] (A2) at (-1.5,-0.6) {};
\draw (A2.input 2 -| -4,0) node[anchor=east]{$z$} -- (A2.input 2);
\draw (A2.output) -- ++(0.2,0) |- (O1.input 2);
\node[or gate] (O2) at (-3,0) {};
\draw (O2.input 1 -| -4,0) node[anchor=east]{$y$} -- (O2.input 1);
\draw (A2.input 2 -| -3.5,0) |- (O2.input 2);
\draw (O2.output) -- ++(0.2,0) |- (A1.input 2);
\draw (O2.input 1 -| -3.75,0) |- (A2.input 1);
\draw (-1.5,-1.5) node{$f=x(y+z)+yz$};
\end{scope}
\end{tikzpicture}
\caption{Standaard vorm en alternatief.}
\figlab{alternativeStandardForm}
\end{figure}
\subsection{Realisaties met NAND en NOR}
Zoals reeds kort vermeld in subsectie \ref{ss:logischePoorten} zijn NAND en NOR poorten erg populair bij heel wat implementaties. Dit komt omdat ze alle basispoorten kunnen emuleren zoals blijkt uit \tblref{nandNorUniversal} op pagina \pageref{tbl:nandNorUniversal}. Bovendien is hun kostprijs lager dan een AND of OR poort\footnote{Denk aan de formule van de kostprijs: het aantal inverters wordt afgetrokken van de kostprijs.}. NAND en NOR poorten blijken bovendien ook makkelijk implementeerbaar als substituut voor AND en OR poorten in standaard vorm. \figref{standardizationNandNor} toont hoe we ons voorbeeld in standaardvorm verder kunnen optimaliseren. In stap 1 inverteren we de min- en max-termen door van een AND en OR respectievelijk een NAND en NOR te maken. We behouden echter de functie door inverters aan de ingangen van het tweede niveau toe te voegen. Immers is tweemaal inverteren niets anders dan dezelfde waarde behouden. Door vervolgens de wetten van De Morgan toe te passen in stap 2, resulteert dit in een circuit die alleen gebruik maakt van NAND en NOR poorten. Bovendien geldt dat een circuit ge\"implementeerd met NAND en NOR poorten altijd zowel goedkoper en sneller is dan zijn canonisch equivalent.
\begin{figure}[htb]
\centering
\begin{tikzpicture}[circuit logic US]
\draw (-2.7,0.9) node[anchor=south]{$x$} -- (-2.7,-4.9);
\draw (-2.5,0.9) node[anchor=south]{$y$} -- (-2.5,-4.9);
\draw (-2.3,0.9) node[anchor=south]{$z$} -- (-2.3,-4.9);

\draw (1.5,2) -- (1.5,-5);
\draw (-3.5,1.5) -- (6.5,1.5);
\draw (-1,1.5) node[anchor=south]{Sum-of-Products};
\draw (4,1.5) node[anchor=south]{Product-of-Sums};
\draw[->] (-0.5,-0.5) -- (-0.5,-1.5);
\draw (-0.3,-1) -- (0.1,-1);
\draw[->] (0.2,-1) -- (0.3,-1);
\draw (0.4,-1) -- (0.45,-1);
\draw (0.5,-1) circle (0.05);
\draw (0.55,-1) -- (0.65,-1);
\draw (0.7,-1) circle (0.05);
\draw (0.75,-1) -- (0.8,-1);
\draw[->] (-0.5,-2.5) to node[midway,anchor=west]{De Morgan} (-0.5,-3.5);
\draw[->] (4.5,-0.5) -- (4.5,-1.5);
\draw[->] (4.5,-2.5) to node[midway,anchor=west]{De Morgan} (4.5,-3.5);

\node[or gate] (O) at (0,0) {};
\draw (O.output) -- ++(0.25,0) node[anchor=west]{$f$};
\node[and gate,inputs={inverted,normal}] (A1) at (-1.5,0.4) {};
\draw (A1.output) -- ++(0.2,0) |- (O.input 1);
\draw (A1.input 1 -| -2.5,0) -- (A1.input 1);
\draw (A1.input 2 -| -2.3,0) -- (A1.input 2);
\node[and gate,inputs={normal,inverted}] (A2) at (-1.5,-0.4) {};
\draw (A2.output) -- ++(0.2,0) |- (O.input 2);
\draw (A2.input 1 -| -2.7,0) -- (A2.input 1);
\draw (A2.input 2 -| -2.3,0) -- (A2.input 2);
\begin{scope}[yshift=-2 cm]
\node[or gate,inputs={inverted,inverted}] (O) at (0,0) {};
\draw (O.output) -- ++(0.25,0) node[anchor=west]{$f$};
\node[nand gate,inputs={inverted,normal}] (A1) at (-1.5,0.4) {};
\draw (A1.output) -- ++(0.2,0) |- (O.input 1);
\draw (A1.input 1 -| -2.5,0) -- (A1.input 1);
\draw (A1.input 2 -| -2.3,0) -- (A1.input 2);
\node[nand gate,inputs={normal,inverted}] (A2) at (-1.5,-0.4) {};
\draw (A2.output) -- ++(0.2,0) |- (O.input 2);
\draw (A2.input 1 -| -2.7,0) -- (A2.input 1);
\draw (A2.input 2 -| -2.3,0) -- (A2.input 2);
\end{scope}
\begin{scope}[yshift=-4 cm]
\node[nand gate] (O) at (0,0) {};
\draw (O.output) -- ++(0.25,0) node[anchor=west]{$f$};
\node[nand gate,inputs={inverted,normal}] (A1) at (-1.5,0.4) {};
\draw (A1.output) -- ++(0.2,0) |- (O.input 1);
\draw (A1.input 1 -| -2.5,0) -- (A1.input 1);
\draw (A1.input 2 -| -2.3,0) -- (A1.input 2);
\node[nand gate,inputs={normal,inverted}] (A2) at (-1.5,-0.4) {};
\draw (A2.output) -- ++(0.2,0) |- (O.input 2);
\draw (A2.input 1 -| -2.7,0) -- (A2.input 1);
\draw (A2.input 2 -| -2.3,0) -- (A2.input 2);
\end{scope}
\begin{scope}[xshift=5 cm]
\draw (-2.7,0.9) node[anchor=south]{$x$} -- (-2.7,-4.9);
\draw (-2.5,0.9) node[anchor=south]{$y$} -- (-2.5,-4.9);
\draw (-2.3,0.9) node[anchor=south]{$z$} -- (-2.3,-4.9);

\draw (-0.3,-1) -- (0.1,-1);
\draw[->] (0.2,-1) -- (0.3,-1);
\draw (0.4,-1) -- (0.45,-1);
\draw (0.5,-1) circle (0.05);
\draw (0.55,-1) -- (0.65,-1);
\draw (0.7,-1) circle (0.05);
\draw (0.75,-1) -- (0.8,-1);

\node[and gate] (O) at (0,0) {};
\draw (O.output) -- ++(0.25,0) node[anchor=west]{$f$};
\node[or gate] (A1) at (-1.5,0.4) {};
\draw (A1.output) -- ++(0.2,0) |- (O.input 1);
\draw (A1.input 1 -| -2.7,0) -- (A1.input 1);
\draw (A1.input 2 -| -2.3,0) -- (A1.input 2);
\node[or gate,inputs={inverted,inverted}] (A2) at (-1.5,-0.4) {};
\draw (A2.output) -- ++(0.2,0) |- (O.input 2);
\draw (A2.input 1 -| -2.5,0) -- (A2.input 1);
\draw (A2.input 2 -| -2.3,0) -- (A2.input 2);
\begin{scope}[yshift=-2 cm]
\node[and gate,inputs={inverted,inverted}] (O) at (0,0) {};
\draw (O.output) -- ++(0.25,0) node[anchor=west]{$f$};
\node[nor gate] (A1) at (-1.5,0.4) {};
\draw (A1.output) -- ++(0.2,0) |- (O.input 1);
\draw (A1.input 1 -| -2.7,0) -- (A1.input 1);
\draw (A1.input 2 -| -2.3,0) -- (A1.input 2);
\node[nor gate,inputs={inverted,inverted}] (A2) at (-1.5,-0.4) {};
\draw (A2.output) -- ++(0.2,0) |- (O.input 2);
\draw (A2.input 1 -| -2.5,0) -- (A2.input 1);
\draw (A2.input 2 -| -2.3,0) -- (A2.input 2);
\end{scope}
\begin{scope}[yshift=-4 cm]
\node[nor gate] (O) at (0,0) {};
\draw (O.output) -- ++(0.25,0) node[anchor=west]{$f$};
\node[nor gate] (A1) at (-1.5,0.4) {};
\draw (A1.output) -- ++(0.2,0) |- (O.input 1);
\draw (A1.input 1 -| -2.7,0) -- (A1.input 1);
\draw (A1.input 2 -| -2.3,0) -- (A1.input 2);
\node[nor gate,inputs={inverted,inverted}] (A2) at (-1.5,-0.4) {};
\draw (A2.output) -- ++(0.2,0) |- (O.input 2);
\draw (A2.input 1 -| -2.5,0) -- (A2.input 1);
\draw (A2.input 2 -| -2.3,0) -- (A2.input 2);
\end{scope}
\end{scope}
\end{tikzpicture}
\caption{Standaardvorm van het voorbeeld met NAND en NOR's.}
\figlab{standardizationNandNor}
\end{figure}
\section{Digitaal ontwerp in grote lijnen}
Nu we de concepten van een digitaal ontwerp in grote lijnen uitgetekend hebben, kunnen we het over de praktische kant van de zaak hebben. Immers komt bij een digitaal ontwerp veel meer kijken dan alleen het bouwen van de juiste schakeling. Een klant zal immers specificaties sturen waaraan de hardware moet voldoen, bovendien wil de klant naast de hardware vaak ook nog documentatie over hoe men het toestel dient te gebruiken. Bovendien willen we vermijden dat we bij het ontwerpen telkens opnieuw het wiel moeten uitvinden. Er bestaan immers al bibliotheken die allerhande schakelingen voor bijvoorbeeld optellingen, testen op gelijke waarde,... bevatten. Deze zijn meestal met de nodige zorg samengesteld zodat het vinden van een nog effici\"entere implementatie moeilijk wordt. \figref{digitalDesignManagement} vat het hele verloop goed samen. Verder zullen we in de volgende subsecties de verschillende stappen want meer in detail bekijken.
\begin{figure}
\centering
\begin{tikzpicture}[itm/.style={draw=black,rectangle,minimum width=3 cm,minimum height=0.5 cm}]
\node[itm] (Sp) at (0,0) {Specificatie \S\ref{ss:specificatie}};
\node[itm] (B) at (-3.5,-0.5) {Bibliotheek \S\ref{ss:bibliotheek}};
\node[itm] (Sy) at (0,-1) {Synthese \S\ref{ss:synthese}};
\node[itm] (A) at (0,-2) {Analyse \S\ref{ss:analyse}};
\node[itm] (D) at (-3.5,-2) {Documentatie \S\ref{ss:documentatie}};
\draw[->] (Sp) -- (Sy);
\draw[->] (B) |- (Sy);
\draw[->] (Sy) -- (A);
\draw (A) |- ++(2.5,-0.5) |- ++(-2.5,2);
\draw[<-] (Sp) |- ++(2.5,0.5) -- ++(0,-1);
\end{tikzpicture}
\caption{Typisch verloop van een digitaal ontwerp.}
\figlab{digitalDesignManagement}
\end{figure}
\subsection{Specificatie}
\label{ss:specificatie}
Een \termen{Specificatie} is een beschrijving van de functionaliteiten die van de hardware in kwestie gevraagd worden. In tegenstelling tot specificaties in bijvoorbeeld de informatica, is de \termen{interface} ook een onderdeel van de specificaties. De interface beschrijft hoe de hardware interageert met zijn omgeving. Deze omgeving is niet noodzakelijk de gebruiker die bijvoorbeeld toetsen indrukt en het scherm uitleest. Dit kan ook bijvoorbeeld de PCI\footnote{PCI: Peripheral Component Interconnect.} bus zijn waarmee met bijvoorbeeld een computer gecommuniceerd wordt. Dikwijls zijn initieel de specificaties onvolledig. De gaten in de specificaties worden dan ook opgevuld wanneer er zich problemen stellen in bijvoorbeeld de synthese. Specificaties zijn bijgevolg een iteratief proces. Dikwijls bevatten de specificaties ook reeds implementatiesbeslissingen die onnodige beperkingen opleggen aan het ontwerp. De beschrijving wordt ofwel in natuurlijke taal, wat soms dubbelzinnig is, of met behulp van een blokschema beschreven.
\subsection{Synthese}
\label{ss:synthese}
De \termen{Synthese} is niets anders dan de vertaling van de specificaties van een hoog en abstract naar een lager niveau. Hierbij dienen uiteraard concrete beslissingen genomen te worden over de implementatie. Zo kunnen de specificaties bijvoorbeeld vermelden dat $x$ en $y$ bij elkaar opgeteld moeten worden. De synthese moet dan een concrete implementatie voorstellen. Bijvoorbeeld een 16-bit ripple-carry adder met 2 registers. Synthese gebeurt meestal op verschillende niveaus. Zo worden op het laagste niveau componenten gebouwd. Dit zijn bijvoorbeeld de poorten maar ook bijvoorbeeld flipflops of multiplexers. Deze worden op een niveau hoger gebruikt om \termen{Register-Transfer-Level (RTL) Componenten} te bouwen. Deze RTL componenten zijn bijvoorbeeld optellers, schuifregisters en tellers. Schakelingen met deze componenten zijn dan \termen{Application Specific IC (ASIC) componenten}. Deze componenten vormen dan uiteindelijk de bouwblokken voor het systeem. Bij de systeemsynthese worden dan processoren, geheugens en de ASIC componenten gecombineerd. \figref{synthesisPyramid} toont de pyramide van synthese. Elk niveau combineert hierbij de componenten ge\"introduceerd op een niveau lager.
\begin{figure}[htb]
\centering
\begin{tikzpicture}[itm/.style={draw=black,thick,rectangle,fill=white,minimum width=10 cm,minimum height=1 cm}]
\draw[thick] (-3.75,0) -- (0,7.5) -- (3.75,0) -- cycle;
\node[itm] (LA) at (0,1.5) {Ontwerp van componenten};
\node[itm] (LB) at (0,3) {\begin{tabular}{c}Ontwerp op componentniveau\\(met basiscomponenten: poorten, flipflops)\end{tabular}};
\node[itm] (LC) at (0,4.5) {\begin{tabular}{c}Architectuursynthese (op RTL-niveau)\\(met RTL-componenten: optellers, tellers, schuifregisters)\end{tabular}};
\node[itm] (LD) at (0,6) {\begin{tabular}{c}Systeemsynthese\\(bouwblokken: processoren, geheugen, ASIC)\end{tabular}};
\end{tikzpicture}
\caption{De verschillende lagen bij de synthese.}
\figlab{synthesisPyramid}
\end{figure}
\subsection{Bibliotheek}
\label{ss:bibliotheek}
We dienen niet het volledige stuk hardware vanaf transistor of poortniveau te ontwerpen. Heel wat werk is dan ook al in het verleden door andere ontwerpers gedaan. Het hergebruiken van deze ontwerpen heeft dan ook heel wat voordelen:
\begin{itemize}
 \item De ontwerpen zijn meestal al door verschillende personen geoptimaliseerd. Het vinden van een nog optimalere implementatie is daardoor quasi onbestaand.
 \item Men streeft naar telkens hogere integratieniveaus waarbij soms volledige systemen op chip te verkrijgen zijn. Bovendien worden deze systemen telkens geavanceerder. Dit komt door de \termen{Wet van Moore}. Deze stelt dat elke 24 maanden het aantal transistoren op een chip verdubbelt.
 \item Vaak kan men componenten kopen die een bepaalde functie vervullen. Dit bespaart veel ontwerpwerk. Bovendien zijn deze componenten vaak veel goedkoper dan ze zelf te ontwerpen en te produceren. Bijvoorbeeld een 16 kB geheugen.
\end{itemize}
Er bestaan bibliotheken op elk syntheseniveau. Dus bijgevolg kan men tot op het niveau die men zelf kiest putten uit de bibliotheken.
\subsection{Analyse}
\label{ss:analyse}
Na elke synthesestap is het belangrijk om te testen of aan de vereiste specificaties voldaan is. Dit wordt gedaan in de \termen{Analyse}. Niet alleen wordt hier getest of de implementatie de functionaliteit levert die gevraagd is, ook allerhande andere parameters worden in rekening gebracht:
\begin{itemize}
 \item Kostprijs: vaak wordt hier als metriek het aantal pinnen en de oppervlakte van de printplaat (PCB\footnote{PCB: Printed Circuit Board.}) gebruikt.
 \item Vermogengebruik: het vermogenverbruik wordt meestal berekent met de formule $C\cdot f\cdot V^2$ met als parameters:
 \begin{itemize}
  \item $C$ De oppervlakte van de chip. De oppervlakte is in de tijd toegenomen. In 1983 was een gemiddelde chip immers $0.25\mbox{ cm}^2$, in 2000 was dat ongeveer $4\mbox{ cm}^2$.
  \item $f$ De \termen{klokfrequentie}. De klokfrequentie is in de loop der tijd exponentieel gestegen. Met $1\mbox{ MHz}$ in 1983, en $1\mbox{ GHz}$ in 2000.
  \item $V$ De spanning die aan de chip geleverd wordt. De spanning is gedaald: van $5\mbox{ V}$ in 1983 tot $1.5\mbox{ V}$ ongeveer 17 jaar later.
 \end{itemize}
 \item Snelheid: de vertraging of \termen{doorvoer} (``\termen{throughput}''). Dit is het aantal resultaten per seconde.
 \item Testbaarheid: kunnen we alle fouten ontdekking met behulp van testvectoren?
\end{itemize}
\subsection{Documentatie}
\label{ss:documentatie}
Naast de hardware verlangt de klant meestal ook \termen{Documentatie}. Afhankelijk van het soort klant zal er andere documentatie vereist zijn. Indien we een volledig afgewerkt consumentenproduct maken, dienen we een handleiding voor de consument en de hersteller samen te stellen. Hierin beschrijven we in natuurlijke taal hoe de component aangestuurd kan worden. Indien we echter een component leveren, bijvoorbeeld een geheugenmodule zal men een handleiding maken met daarin de specifieke ontwerpdetails. Meestal zal ook intern binnen het bedrijf documentatie een vereiste zijn. Dit om de eventueel verdere ontwikkelingen te ondersteunen.
\subsection{Ontwerpen met CAD}
\termen{Computer Aided Design (CAD)} wordt vaak toegepast om een chip te ontwikkelen. Hierbij wordt met behulp van een computer een een speciaal softwarepakket meestal het volledige ontwerp begeleid. Een concreet voorbeeld hiervan is bijvoorbeeld \emph{KiCad} voor \emph{Linux}. Zoals andere CAD tools bevat \emph{KiCad} een project manager die de gebruiker door de verschillende stappen begeleid, en een set tools die de gebruiker bij iedere stap de nodige ondersteuning bieden. Uiteraard ziet het ontwerp met een CAD-tool er gelijkaardig uit aan het ontwikkelingsproces op \figref{digitalDesignManagement}. De CAD-tools bieden echter meer mogelijkheden om ontwerpen al in een vroeg stadium te testen en te simuleren. \figref{digitalDesignManagementCad} toont het ontwikkelingsproces met behulp van een CAD-tool.
\begin{figure}[htb]
\centering
\begin{tikzpicture}[itm/.style={draw=black,rectangle,minimum width=4 cm}]
\node (Sp) at (0,0) {Specificatie};
\draw (Sp.north west) -- (Sp.north east) -- ++(0.15,-0.275) -- (Sp.south east) -- (Sp.south west) -- ++(-0.15,0.275) -- cycle;
\node (Sc) at (-1,-2.25) {Schema};
\draw (Sc.north west) -- (Sc.north east) -- ++(0.15,-0.275) -- (Sc.south east) -- (Sc.south west) -- ++(-0.15,0.275) -- cycle;
\node (V) at (1,-2.25) {VHDL};
\draw (V.north west) -- (V.north east) -- ++(0.15,-0.275) -- (V.south east) -- (V.south west) -- ++(-0.15,0.275) -- cycle;
\draw (-2,-1.25) rectangle ++(4,-1.5);
\draw (0,-1.25) node[anchor=north]{Ingave ontwerp};
\draw[->] (Sp) -- (0,-1.25);
\node[itm] (Sy) at (0,-4) {Synthese};
\draw[->] (0,-2.75) -- (Sy);
\node[itm] (Fs) at (0,-6) {Functionele Simulatie};
\draw[->] (Sy) -- (Fs);
\node[draw,shape aspect=2,diamond] (O) at (0,-8) {Ontwerp OK?};
\draw (O.south) node[anchor=north west]{Ja};
\draw (O.west) node[anchor=south east]{Nee};
\draw[->] (Fs) -- (O);
\node[itm] (Fo) at (6,-2) {Fysisch ontwerp};
\draw[->] (O.south) |- ++(3,-0.5) |- (Fo.west);
\node[itm] (St) at (6,-4) {Simulatie tijdsgedrag};
\draw[->] (Fo) -- (St);
\node[draw,shape aspect=2,diamond] (T) at (6,-6) {Tijdsgedrag OK?};
\draw (T.south) node[anchor=north west]{Ja};
\draw (T.west) node[anchor=south east]{Nee};
\draw[->] (St) -- (T);
\draw (T) -- ++(-3,0);
\draw (3.5,-6) |- (-3,-10) -- (-3,-8);
\draw (O) -| (-3,-0.625) -- ++(3,0);
\node[itm] (C) at (6,-8) {Chipconfiguratie};
\draw[->] (T) -- (C);
\end{tikzpicture}
\caption{Digitaal ontwerpen met CAD.}
\figlab{digitalDesignManagementCad}
\end{figure}
\section{Taalgebaseerd hardware ontwerp: VHDL}
\label{s:vhdl}
Zoals reeds vermeld werd op \figref{digitalDesignManagementCad}, wordt \termen{VHDL} veel gebruikt om schakelingen in te voeren in een computersysteem. VHDL is de afkorting van \termen{VHSIC Hardware Description Language}. Hierbij staat VHSIC voor \termen{Very High Speed Integrated Circuit}. VHDL is een programmeertaal waarmee men het gedrag van digitale circuits probeert te beschrijven. De taal biedt mogelijkheden aan om op een eenduidige manier het gedrag van een circuit te specifi\"eren op RTL niveau. Daarnaast is de taal erg nuttig om simulaties te draaien, synthese uit te voeren (VHDL software is meestal in staat om zelf effici\"ente implementaties voor te stellen) en documentatie te genereren. VHDL is gestandaardiseerd bij de IEEE\footnote{IEEE: Institute for Electrical and Electronics Engineers.}. De eerste versie van VHDL is VHDL-87, gestandaardiseerd onder IEEE 1076. In 1993 werd de tweede versie, VHDL-93 uitgebracht in IEEE 1164. Sinds 2002 bestaat er ook een derde versie die nog niet gestandaardiseerd is. Deze standaarden omvatten echter uitsluitend de syntax. De implementatie van de VHDL compiler is volledig vrij. Er is dan ook concurrentie tussen VHDL compilers in features om de meest effici\"ente implementatie te kunnen voorstellen.
\subsection{Alternatieven en uitbreidingen}
\paragraph{VHDL-AMS}Een uitbreiding op VHDL is \termen{VHDL Analog and Mixed Signals (VHDL-AMS)}. Hierbij worden niet alleen digitale maar ook analoge signalen beschouwd. VHDL-AMS kan dus als een superset van de orginele VHDL beschouwd worden. Bovendien wordt met continue tijd gerekend in plaats van de door VHDL gebruikte discrete tijdstippen. Dit werd ge\"implementeerd door een set algebra\"ische en differenti\"ele vergelijkingen. Hoewel het in 1999 door de IEEE gestandaardiseerd werd, is VHDL-AMS sinds zijn oorsprong in 1993 nooit echt doorgebroken.
\paragraph{Verilog}De concurrent van VHDL is \termen{Verilog} (IEEE 1364). Verilog is erg populair in de Verenigde Staten maar is nooit echt doorgebroken in Europa. Beide talen stammen ook uit andere taalfamilies met andere paradigma's. Terwijl \verb+VHDL+ eerder lijkt op \verb+Ada+, is \verb+Verilog+ meer verwant met \verb+C+. Ondanks de concurrentie lijkt geen van beide talen het pleit te kunnen beslechten. Of zoals D. Pellerin \& D. Taylor het verwoorden in ``VHDL Made Easy!''\cite{pellerin1997vhdl}:
\begin{quote}
Both languages are easy to learn and hard to master. And once you have learned one of these languages, you will have no trouble transitioning to the other.
\end{quote}
\paragraph{PLD talen}
Talen zoals Abel en Palasm zijn zogenaamde \termen{PLD talen}\footnote{PLD: programmable Logic Device, zie \ref{sss:pld}.}. Deze talen specifi\"eren schakelingen op het niveau van de poorten en dit slechts voor een speciale technologie. Deze talen hebben bijgevolg ook een ander objectief. Waar VHDL net bedoeld is om zich met de details bezig te houden, en de gebruiker met de grote lijnen, zijn PDL talen bedoelt om te implementeren op deze lagere niveaus.
\subsection{Voordelen}
VHDL wordt hoofdzakelijk in de industrie gebruikt omwille van zijn overdraagbaarheid. VHDL is immers een standaard die door allerhande programma's gehanteerd wordt. Elk van deze programma's kunnen heel diverse toepassingen hebben. Op die manier kan een stuk VHDL eerst gesimuleerd worden met het ene programma, daarna gesynthetiseerd met een ander, om bijvoorbeeld geanalyseerd te worden met een derde programma. Deze programma's kunnen bovendien afkomstig zijn van verschillende fabrikanten. Hierdoor kunnen fabrikanten zich ook toespitsen op \'e\'en kant van het ontwerp zonder dat er compatibiliteitsproblemen ontstaan.
\paragraph{}
Daarnaast is VHDL ook interessant om complexe schakelingen op een hogere abstractieniveau te beschrijven. Zo kan men een schakeling die vaak terugkomt groeperen in een bepaalde module. Repetitieve structuren dienen slechts \'e\'enmaal beschreven te worden, maar kan men eindeloos blijven gebruiken.
\paragraph{}
VHDL maakt het ook mogelijk om de gebruiker te laten ontwerpen los van de eigenlijke implementatie. Zo dient de gebruiker alleen te specificeren dat twee getallen opgeteld moeten worden. Het is dan aan het programma om de componenten te selecteren die dat op de beste manier doen (qua kosten, snelheid,...).
\paragraph{}
Tot slot zorgt VHDL er ook voor dat een ontwerp makkelijk te parametriseren valt. Indien de ontwerper niet zeker is van de woordlengte van zijn processor kan hij deze in een parameter onderbrengen. Als later blijkt dat de woordlengte groter moet zijn, kan met een eenvoudige verandering van de parameterwaarde het volledige model aangepast worden.
\subsection{Nadelen}
Naast de reeks opgesomde voordelen heeft VHDL ook enkele belangrijke nadelen. Zo is het een eenvoudig te leren taal, maar de taal echt beheersen vraagt heel wat geduld. Dit komt door een deel door de niet eenvoudige syntax. Een ander probleem is dat heel wat software afwijkt van de gestandaardiseerde versies. Er bestaan dan ook onnoemelijk veel VHDL ``dialecten'' waardoor sommige features die aan de taal werden toegevoegd slechts op bepaalde softwarepakketten werken.
\paragraph{}
Een ander nadeel is dat de taal nogal langdradig is. Bij complexe circuits zullen de groeperingen zeker hun effect hebben, maar om kleine schakelingen te realiseren is nogal veel code nodig.
\paragraph{}
Een probleem met code in het algemeen is dat het erg onoverzichtelijk is. Een eenvoudig blokschema is nog altijd overzichtelijker omdat mensen nu eenmaal grafisch sterker zijn. Pas bij grote complexe schakelingen verliest het visuele zijn kracht en zal een stuk code als doeltreffender worden aanzien.
\paragraph{}
Het feit dat VHDL redelijk uitgebreid is, brengt bovendien allerhande nadelen met zich mee. Alle extra features voor bijvoorbeeld tijdsgedrag simulaties dienen immers in de taal beschreven te kunnen worden. Bijgevolg worden sommige taalconcepten hierdoor hopeloos moeilijker gemaakt.
\subsection{Beperkingen}
Naast de voor- en nadelen heeft VHDL ook enkele beperkingen. Zo is VHDL slechts tot op zeker niveau automatisch synthetiseerbaar. Hierbij ondersteund elke fabrikant van VHDL een verschillende subset. Het tweede grote nadeel is dat slechts de syntax en de semantiek van VHDL gestandaardiseerd is. Niet hoe de code geschreven moet worden. Dit houdt in dat eenzelfde gedrag op tientallen verschillende manieren beschreven kan worden. Dit zorgt er ook voor dat elk programma die VHDL leest de code op een andere manier kan implementeren. Het gevolg is dat de codestijl, die in grote mate bepaalt hoe de code ge\"implementeerd zal worden, variabel is. In het ene programma kan een bepaalde code tot de meest optimale implementatie leiden, terwijl een ander programma met dezelfde code slechts een standaardimplementatie kiest. Men moet dus eerst heel wat ervaring opdoen met een programma alvorens men weet hoe men de code moet schrijven zodat deze een effici\"ente implementatie oplevert.
\subsection{Concepten (entiteiten en architectuur)}
Na de voor en nadelen besproken te hebben, is het tijd voor een voorbeeld waarmee we de verschillende concepten zullen duidelijk maken. We ontwerpen een schakeling zoals op \figref{vhdl-example}. Hierbij willen we een schakeling ``test'' ontwerpen. Deze schakeling krijgt drie 8-bit ingangen (\texttt{In1}, \texttt{In2} en \texttt{In3}). Als uitgangen zijn er twee bits (\texttt{Out1} en \texttt{Out2}). \texttt{Out1} geeft 1 terug indien \texttt{In1} en \texttt{In2} aan elkaar gelijk zijn. Analoog geeft \texttt{Out2} 1 terug indien \texttt{In2} en \texttt{In3} gelijk zijn. Om deze vergelijker te bouwen werken we bijgevolg met een hi\"erarchisch schema. Waarbij we \texttt{Comp} op een andere plaats implementeren.
\importtikzfigure{vhdl-example}{Voorbeeldcircuit voor VHDL code.}
\paragraph{}We zullen eerst \texttt{Comp} implementeren in VHDL. De code hiervoor staat in \vhdlref{comp}. Als we de code goed bekijken onderscheiden we twee belangrijke delen \vhdltermen{entity} en \vhdltermen{architecture}.
\paragraph{Entity}
Het \texttt{entity} gedeelte beschrijft de zogenaamde ``\termen{blackbox}'' ofwel de interface. Hierdoor is VHDL in staat een blokje te tonen met de juiste in en uitgangen. Zoals we zien bevat de beschrijving het \vhdltermen{port} commando. Het \texttt{port} commando specificeert dan de in- en uitgangen. Dit doet men door eerst de namen van de in- of uitgangen te vermelden. Vervolgens plaatst men het token \vhdltermen{in} of \vhdltermen{out} om te specificeren of het om een in- of uitgang gaat. Merk dus op dat elke geleider naar de component een expliciete richting heeft. Vervolgens duidt men het type van de in- of uitgang aan. Logischerwijs bevat dit het token \vhdltermen{bit}. Een bit is niets anders dan \'e\'en lijn. Meestal echter willen we verschillende lijnen samennemen. In dat geval duiden we dit aan met een \vhdltermen{bit\_vector}, een bitvector is dus niets anders dan een lijst van geleiders naar het blok. Het is handig verschillende geleiders samen te nemen, dit maakt de code overzichtelijker. Indien bijvoorbeeld de lengte van A gewijzigd moet worden kost dit slechts een minimale ingreep.
\paragraph{Architecture}
De concrete werking wordt beschreven in het \texttt{architecture} gedeelte. Dit beschrijft het gedrag van de component op RTL-niveau. VHDL is in staat met de gegeven beschrijving in de \texttt{architecture} omgeving een implementatie op poortniveau te bouwen. Verder dient ook opgemerkt te worden dat \'e\'en \texttt{entity} verschillende \texttt{architecture}s kan hebben.\footnote{Vandaar dat we onze architectuur de naam \texttt{Behav1} noemen.}. Uiteraard moeten al deze \texttt{architecture}s met dezelfde interface werken.
\begin{vhdlcode}
\centering
\begin{lstlisting}
-- 8-bit comparator
--
entity Comp is
  port(	A,B: in bit_vector(0 to 7);
	EQ: out bit);
end entity Comp;

architecture Behav1 of Comp is
begin
  EQ <= '1' when (A=B) else '0';
end architecture Behav1;
\end{lstlisting}
\caption{8-bit comparator.}
\label{vhdl:comp}
\end{vhdlcode}
\paragraph{Component} Nu we een comparator component gebouwd hebben, zullen we deze component importeren in ons testcircuit. Hiervoor gebruiken we code beschreven in \vhdlref{test}. Dit bestand deelt dezelfde structuur als de structuur in \vhdlref{comp}: een \texttt{entity} en \texttt{architecture}. Dit wijst er dus op dat we componenten hi\"erarchisch kunnen gebruiken. We bemerken echter een nieuw token: \vhdltermen{component}. \texttt{component} is een virtuele verwijzing naar een andere entiteit. Wat die entiteit is laten we in de code nog in het midden. Het enige wat we moeten doen is de interface van deze component specificeren. Later zullen we dan in de configuratie (\vhdlref{testConfig}) een binding voorzien tussen ons virtuele component \texttt{comparator} en onze gedefinieerde entiteit \texttt{comp}. Verder merken we ook op dat in het \texttt{architecture} gedeelte eenmaal we de \texttt{component} interface gedefinieerd hebben, we er instanties van kunnen aanmaken. Zo maken we twee instanties aan: \texttt{Comp1} en \texttt{Comp2}. Vervolgens dienen we nog de verbindingen tussen de entiteit \texttt{test} en deze instanties te leggen. Dit doen we met het token \vhdltermen{map}. Hierbij mappen we de variabelen uit \texttt{entity Test} (\texttt{In1}, \texttt{In2}, \texttt{In3}, \texttt{Out1} en \texttt{Out2}) op de in- en uitgangen van het \texttt{component Comparator} (\texttt{X}, \texttt{Y} en \texttt{Z}). Hierbij dient dus opgemerkt te worden dat we het gedrag van \texttt{Test} specificeren aan de hand van reeds gedefinieerde entiteiten. Tot slot dient ook opgemerkt te worden dat in tegenstelling tot klassieke programmeertalen alle componenten tegelijk werken. Het is dus niet zo dat bij het uitrekenen van \texttt{Test} eerst \texttt{Comp1} en dan \texttt{Comp2} uitgerekend zal worden.
\begin{vhdlcode}
\centering
\begin{lstlisting}
-- Component Test met 2 comparatoren
--
entity Test is
  port(	In1,In2,In3: in bit_vector(0 to 7);
	Out1,Out2: out bit);
end entity Test;

architecture Struct1 of Test is
component Comparator is
  port(	X,Y: in bit_vector(0 to 7);
	Z: out bit);
end component Comparator;
begin
  Comp1: component Comparator port map (In1,In2,Out1);
  Comp2: component Comparator port map (In2,In3,Out2);
end architecture Struct1;
\end{lstlisting}
\caption{Voorbeeldcode.}
\label{vhdl:test}
\end{vhdlcode}
\paragraph{Configuration} We moeten nu \texttt{Comparator} met de entiteit \texttt{Comp} binden. Dit doen we in een \vhdltermen{con\-fig\-ura\-tion} omgeving. Deze configuratie wordt beschreven in VHDL-code \ref{vhdl:testConfig}. Zoals we zien kunnen we opnieuw verschillende \texttt{configuration}s aanmaken en deze een aparte naam geven. Op die manier kunnen we dus de feitelijke implementatie snel wijzigen. Verder dienen we ook te vermelden welke entiteit we zullen configureren (\texttt{Test}) en om welke architectuur het gaat (\texttt{Struct1}). Vervolgens kunnen we per instantie aangeven welke entiteit we gebruiken. Dit doen we door het token \vhdltermen{use entity}. Ook specifi\"eren we de \texttt{architecture} die we zullen gebruiken van deze \texttt{entity}. Vervolgens mappen we opnieuw met behulp van \texttt{map} de in- en uitgangen van de entiteit op het virtuele \texttt{component}.
\begin{vhdlcode}
\begin{lstlisting}
-- Configuratie: definieer koppeling component met een
-- bepaalde architectuur van een entiteit
--
configuration Build1 of Test is
  for Struct1
    for Comp1: Comparator use entity Comp(Behav1)
      port map (A => X, B => Y, EQ => Z);
    end for;
    for Comp2: Comparator use entity Comp(Behav1)
      port map (A => X, B => Y, EQ => Z);
    end for;
  end for;
end configuration Build1;
\end{lstlisting}
\caption{Configuratie van de voorbeeldcode.}
\label{vhdl:testConfig}
\end{vhdlcode}
\subsection{Gelijkenissen en verschillen met klassieke programmeertalen}
VHDL lijkt in sommige opzichten een beetje op programmeren in klassieke programmeertalen zoals \texttt{Java} en \texttt{C++}. Immers kunnen we de ingangen als parameters bij een methode zien. De methode zelf fungeert ook als een interface die duidelijk maakt welke types er ingevoerd moeten worden, en welke uitvoer verwacht mag worden, zonder de feitelijke implementatie te tonen. We zouden bovendien elk component die we in een circuit gebruiken kunnen zien als een functie-oproep naar de functie van het bijbehorende component. Deze vergelijking heeft echter enkele anomalie\"en.
\begin{itemize}
 \item \termen{Gelijktijdigheid} (``\termen{Concurrency}''): Alle hardwarecomponenten werken in parallel dit in tegenstelling tot klassieke talen waarin alles sequentieel wordt uitgevoerd.
 \item \termen{Tijdsconcept}: Alle hardware werkt continu en houdt nooit op met werken. Bovendien zal bij een simulatie de tijd uiteraard niet de re\"ele tijd zijn.
 \item \termen{Datatypes}: VHDL heeft nood aan typische hardware-types zoals bitvectoren, getallen met geparametriseerde grootte,... Dit terwijl klassieke talen meestal proberen abstractie te maken van dit hardwareniveau.
\end{itemize}

\chapter{Technologische Randvoorwaarden}
\chplab{technology}
\chapterquote{De enige limiet aan de realisatie van de toekomst is ons twijfelen van vandaag.}{Franklin D. Roosevelt, Amerikaans staatsman en president (26e) (1882-1945)}
\begin{chapterintro}
Tot nu toe hebben we altijd abstractie gemaakt van de werkelijkheid. We hebben in hoofdstuk \ref{ch:basis} poorten ge\"introduceerd, maar hebben altijd abstractie gemaakt van de concrete werking van deze poorten. Aan de hand van het lichtmodel konden we \'e\'en en ander verklaren, maar deze schakelaars moesten manueel geschakeld worden. In dit hoofdstuk zullen we een manier zien hoe we poorten kunnen implementeren, en een 0 en 1 kunnen voorstellen die gebruik maakt van elektronica. Dit is uiteraard slechts \'e\'en manier. Naast het implementeren van poorten stuiten we vaak op allerhande fysische problemen. Dit hoofdstuk geeft een overzicht van de verschillende aspecten die we in de gaten moeten houden bij het ontwerpen van een elektronische schakeling. Verder biedt het een overzicht van manieren om een digitale schakeling te realiseren. Tot slot bekijken we de ontwikkeling van digitale schakeling met een CAD tool.
\end{chapterintro}
\minitoc[n]
\section{Logische waarden voorstellen}
\label{s:logischeWaarden}
Alvorens we enige schakeling kunnen implementeren moeten we een conventie afspreken hoe we een 0 en 1 voorstellen. Deze logische waarden moeten we voorstellen door twee fysische waarden. Deze fysische waarden noemen we ``\termen{High}'' en ``\termen{Low}''. In de elektronica kiezen we meestal als fysische grootheid de spanning. De referentie-spanningen noteren we dan respectievelijk als $V_{\mbox{\tiny H}}$ en $V_{\mbox{\tiny L}}$. Meestal houden we echter niet vast aan \'e\'en spanning, maar defini\"eren we een bereik rond $V_{\mbox{\tiny H}}$ en $V_{\mbox{\tiny L}}$. Immers is het systeem niet volledig deterministisch, en er kunnen bijgevolg kleine variaties in de spanningen optreden. Daarnaast hebben we ook nog allerhande omgevingsparameters zoals de temperatuur, waardoor deze spanning sowieso enigszins zal afwijken. Tussen het bereik van $V_{\mbox{\tiny H}}$ en $V_{\mbox{\tiny L}}$ defini\"eren we ook nog een zone waar het niet duidelijk is of er een 1 of 0 op staat. Dit laat ons toe om fouten te detecteren. Immers indien de spanning dicht bij het midden ligt, is de kans op een foute interpretatie groot. Met een detectiesysteem zouden we dan bijvoorbeeld kunnen vragen om de bit opnieuw uit te sturen. We defini\"eren het bereik van $V_{\mbox{\tiny H}}$ als $\left[V_{\mbox{\tiny IH}},V_{\mbox{\tiny DD}}\right]$, dat van $V_{\mbox{\tiny L}}$ als $\left[V_{\mbox{\tiny SS}},V_{\mbox{\tiny IL}}\right]$. \figref{potentialRange} toont schematisch het bereik van deze waarden. Overigens hebben $V_{\mbox{\tiny DD}}$ en $V_{\mbox{\tiny SS}}$ nog een andere betekenis bij een circuit: het zijn de spanningen die op de voeding van het elektronische apparaat geplaatst worden. Hierbij staat de S voor ``\termen{Source}'' en D voor ``\termen{Drain}''.
\begin{figure}[hbt]
\centering
\begin{tikzpicture}
\fill[black!20] (-0.1,0) rectangle (0.1,1);
\fill[black!20] (-0.1,2) rectangle (0.1,3);
\draw[thick,->] (0,-0.1) -- (0,3.2);
\draw (-0.1,0) node[anchor=east,scale=0.75]{$V_{\mbox{\tiny SS}}$} -- (0.1,0);
\draw (-0.1,1) node[anchor=east,scale=0.75]{$V_{\mbox{\tiny IL}}$} -- (0.1,1);
\draw (-0.1,2) node[anchor=east,scale=0.75]{$V_{\mbox{\tiny IH}}$} -- (0.1,2);
\draw (-0.1,3) node[anchor=east,scale=0.75]{$V_{\mbox{\tiny DD}}$} -- (0.1,3);
\draw (1,3) node[scale=0.75,anchor=south]{\bf Fysisch};
\draw (3.5,3) node[scale=0.75,anchor=south]{\bf Positieve Logica};
\draw (6,3) node[scale=0.75,anchor=south]{\bf Negatieve Logica};
\draw (1,2.5) node[scale=0.75]{High};
\draw (1,1.5) node[scale=0.75]{Ongedefinieerd};
\draw (1,0.5) node[scale=0.75]{Low};
\draw (3.5,2.5) node[scale=0.75]{1};
\draw (3.5,1.5) node[scale=0.75]{-};
\draw (3.5,0.5) node[scale=0.75]{0};
\draw (6,2.5) node[scale=0.75]{0};
\draw (6,1.5) node[scale=0.75]{-};
\draw (6,0.5) node[scale=0.75]{1};
\draw[dotted] (2.25,3.25) -- (2.25,0);
\draw[dotted] (4.75,3.25) -- (4.75,0);
\draw[dotted] (0.1,3) -- (7.25,3);
\draw[dotted] (0.1,2) -- (7.25,2);
\draw[dotted] (0.1,1) -- (7.25,1);
\draw[dotted] (0.1,0) -- (7.25,0);
\end{tikzpicture}
\caption{Schematisch bereik van ``High'' en ``Low'' spanning.}
\figlab{potentialRange}
\end{figure}
\paragraph{Negatieve logica}Het is niet per definitie zo dat $V_{\mbox{\tiny H}}$ geassocieerd wordt met 1, en $V_{\mbox{\tiny L}}$ met 0. Dit hangt af van het type logica dat gehanteerd wordt. In geval van \termen{Positieve Logica} is dit inderdaad het geval. Soms komt het echter voordeliger uit om deze orde om te draaien, in dat geval spreekt men van \termen{Negatieve Logica}. Merk op dat bij negatieve logica de poorten fysisch anders ge\"implementeerd moeten worden. Immers kent de fysische poort alleen maar spanningen. De belangrijkste reden om negatieve logica te gebruiken, is dan ook omdat sommige implementaties hierbij goedkoper kunnen zijn. Bovendien kan men in een circuit op verschillende plaatsen een andere logica gebruiken. Een concreet voorbeeld is de reset-module in veel elektronicasystemen.
\section{Implementatie van poorten}
\subsection{Schakelaars}
In hoofdstuk \ref{ch:basis} maakten we gebruik van het lichtmodel om poorten te implementeren. Daarbij werd gebruik gemaakt van schakelaars. Ook in de echte implementatie van poorten maakt men gebruik van \termen{schakelaars}. Deze schakelaars kunnen door een derde ingang automatisch open en dicht geschakeld worden, deze ingang wordt ook het ``\termen{stuursignaal}'' genoemd. Net als de schakelaars in het lichtmodel hebben deze schakelaars twee toestanden: open en gesloten. We kunnen dit interpreteren als een elektronische weerstand die respectievelijk een weerstand van $\infty\ \Omega$ en $0\ \Omega$ heeft. \figref{switchNotationSwitch} toont hoe dergelijke schakelaars genoteerd worden. Afhankelijk van hun toestand worden ze bovendien anders genoteerd. Zoals eerder gezegd dienen schakelaars een stuursignaal te hebben. Op \figref{switchNotationControlledNmos} en \ref{fig:switchNotationControlledPmos} staan de schakelaars met stuursignaal. We merken op dat er twee varianten van schakelaars zijn. De ene variant, \termen{NMOS}, is gesloten wanneer het stuursignaal een hoog fysische waarde heeft, en open bij een lage waarde. De tweede soort, \termen{PMOS}, inverteert dit principe en is gesloten bij een laag fysische waarde, en open bij een hoge waarde.
\begin{figure}[hbt]
\centering
\subfigure[Schakelaars]{\begin{tikzpicture}
\draw[thick] (0,0) circle(0.625);
\draw (0,-0.625) node[anchor=north]{Open} -- ++(0,0.375) -- ++(-0.2,0) -- ++(-0.2,0.5);
\draw (0,0.625) -- ++(0,-0.375) -- ++(-0.2,0);

\draw[thick] (1.5,0) circle(0.625);
\draw (1.5,-0.625) node[anchor=north]{Gesloten} -- ++(0,0.375) -- ++(-0.2,0) -- ++(0,0.5) -- ++(0.2,0) -- ++(0,0.375);
\end{tikzpicture}
\figlab{switchNotationSwitch}
}
\subfigure[NMOS Schakelaars met stuursignaal] {\begin{tikzpicture}
\draw[thick] (0,0) circle(0.625);
\draw (0,-0.625) -- ++(0,0.375) -- ++(-0.2,0) -- ++(-0.2,0.5);
\draw (0,0.625) -- ++(0,-0.375) -- ++(-0.2,0);
\draw[thick] (0,-0.625) -- ++(0,-0.5);
\draw[thick] (0,0.625) -- ++(0,0.5);
\draw[thick] (-0.625,0) -- ++(-0.75,0) node[anchor=east]{L};

\draw[thick] (3,0) circle(0.625);
\draw (3,-0.625) -- ++(0,0.375) -- ++(-0.2,0) -- ++(0,0.5) -- ++(0.2,0) -- ++(0,0.375);
\draw[thick] (3,-0.625) -- ++(0,-0.5);
\draw[thick] (3,0.625) -- ++(0,0.5);
\draw[thick] (2.375,0) -- ++(-0.75,0) node[anchor=east]{H};
\end{tikzpicture}
\figlab{switchNotationControlledNmos}
}
\subfigure[PMOS Schakelaars met stuursignaal] {\begin{tikzpicture}
\draw[thick] (6,0) circle(0.625);
\draw[thick] (5.25,0) circle(0.125);
\draw (6,-0.625) -- ++(0,0.375) -- ++(-0.2,0) -- ++(0,0.5) -- ++(0.2,0) -- ++(0,0.375);
\draw[thick] (6,-0.625) -- ++(0,-0.5);
\draw[thick] (6,0.625) -- ++(0,0.5);
\draw[thick] (5.125,0) -- ++(-0.5,0) node[anchor=east]{L};

\draw[thick] (9,0) circle(0.625);
\draw[thick] (8.25,0) circle(0.125);
\draw (9,-0.625) -- ++(0,0.375) -- ++(-0.2,0) -- ++(-0.2,0.5);
\draw (9,0.625) -- ++(0,-0.375) -- ++(-0.2,0);
\draw[thick] (9,-0.625) -- ++(0,-0.5);
\draw[thick] (9,0.625) -- ++(0,0.5);
\draw[thick] (8.125,0) -- ++(-0.75,0) node[anchor=east]{H};
\end{tikzpicture}
\figlab{switchNotationControlledPmos}
}
\caption{Notatie van een schakelaar (met stuursignaal).}
\figlab{switchNotation}
\end{figure}
\subsubsection{Werking van NMOS en PMOS}
\label{ss:nmosPmosWork}
In de vorige subsectie werden twee types schakelaars ge\"introduceerd: NMOS en PMOS. Dit zijn \termen{transistoren} die in bijna elk elektronisch apparaat gebruikt worden. In deze subsectie zullen we de terminologie van een transistor bespreken, en de werking van deze componenten verklaren. Zoals de meeste transistoren hebben een NMOS en PMOS 3 aansluitingen: de \termen{Source}, \termen{Gate} en \termen{Drain}, in het Nederlands worden deze aansluitingen ook respectievelijk \termen{Collector}, \termen{Basis} en \termen{Emitter} genoemd. Een transistor is eigenlijk niets anders dan een schakelaar tussen de source en drain. De weerstand tussen de source en de drain wordt geregeld volgens de spanning die op de gate staat. \figref{mosWork} toont de concrete werking van NMOS en PMOS transistoren. Voor beide typen beschouwen we een substraat van siliciumdioxide (SiO$_2$). We kunnen dit substraat zowel positief als negatief \termen{doperen}\footnote{Het introduceren van alternatieve atomen.}. Bij NMOS doperen we het substraat hoofdzakelijk positief (p). Bij de ingangen van de source en de drain doperen we negatief (n). Er kan nauwelijks stroom vloeien tussen het negatief en positief gedopeerde substraat. Bijgevolg fungeert de p-laag als een barri\`ere tussen de source en de drain. Indien we echter een positieve spanning aanbrengen op de Gate, zullen de negatieve deeltjes in de p-laag zich aangetrokken voelen tot de gate. Er ontstaat een reorganisatie van de p-laag waardoor er een n-laagje gevormd wordt ter hoogte van de gate. Hierdoor ontstaat er een kanaal tussen de source en de drain, waardoor de schakelaar zich sluit. De spanning die tussen de gate en de source moet staan om dit te bereiken wordt de \termen{threshold spanning $V_{\mbox{\tiny T}}$} genoemd. Indien de spanning tussen de gate en de source $V_{\mbox{\tiny GS}}$ dus kleiner is dan $V_{\mbox{\tiny T}}$ is de NMOS schakelaar open, anders is de schakelaar gesloten. Een PMOS transistor werkt op analoge manier, maar dan omgekeerd.
\begin{figure}[hbt]
\centering
\subfigure[NMOS bij $V_{\mbox{\tiny GS}}<V_{\mbox{\tiny T}}$.]{\begin{tikzpicture}
\fill[black!20] (0,0) rectangle (3,0.75);
\fill[black!80] (0,0.75) node[anchor=north west,scale=0.75,white]{n+} -- (0,0.45) .. controls (0.4,0.35) and (1.15,0.45) .. (1.15,0.75) -- cycle;
\fill[black!80] (3,0.75) node[anchor=north east,scale=0.75,white]{n+} -- (3,0.45) .. controls (2.6,0.35) and (1.85,0.45) .. (1.85,0.75) -- cycle;
\draw (1,0.75) rectangle ++(1,0.3);
\draw (1.5,0.9) node[scale=0.75]{isolator};
\filldraw[fill=gray] (1,1.05) rectangle ++(1,0.3);
\draw (1.5,1.2) node[scale=0.75]{metaal};
\draw[thick] (1.5,1.35) -- ++(0,0.5) node[anchor=south,scale=0.75]{Gate};
\draw[thick] (0.75,0.75) .. controls (0.75,1.05) and (0.75,1.05) .. (0.5,1.05) node[anchor=east,scale=0.75]{Source};
\draw[thick] (2.25,0.75) .. controls (2.25,1.05) and (2.25,1.05) .. (2.5,1.05) node[anchor=west,scale=0.75]{Drain};
\draw (0,0) to node[above,midway,scale=0.75]{p} (3,0);
\draw (0,0.75) -- (3,0.75);
\end{tikzpicture}}
\subfigure[NMOS bij $V_{\mbox{\tiny GS}}\geq V_{\mbox{\tiny T}}$.]{\begin{tikzpicture}
\fill[black!20] (0,0) rectangle (3,0.75);
\fill[black!80] (0,0.75) node[anchor=north west,scale=0.75,white]{n+} -- (0,0.45) .. controls (0.4,0.35) and (1.15,0.45) .. (1.25,0.55)  .. controls (1.5,0.675) and (1.5,0.675) .. (1.75,0.55) .. controls (1.85,0.45) and (2.6,0.35) .. (3,0.45) -- (3,0.75) node[anchor=north east,scale=0.75,white]{n+} -- cycle;
\draw (1,0.75) rectangle ++(1,0.3);
\draw (1.5,0.9) node[scale=0.75]{isolator};
\filldraw[fill=gray] (1,1.05) rectangle ++(1,0.3);
\draw (1.5,1.2) node[scale=0.75]{metaal};
\draw[thick] (1.5,1.35) -- ++(0,0.5) node[anchor=south,scale=0.75]{Gate};
\draw[thick] (0.75,0.75) .. controls (0.75,1.05) and (0.75,1.05) .. (0.5,1.05) node[anchor=east,scale=0.75]{Source};
\draw[thick] (2.25,0.75) .. controls (2.25,1.05) and (2.25,1.05) .. (2.5,1.05) node[anchor=west,scale=0.75]{Drain};
\draw (0,0) to node[above,midway,scale=0.75]{p} (3,0);
\draw (0,0.75) -- (3,0.75);
\end{tikzpicture}}
\subfigure[PMOS bij $V_{\mbox{\tiny GD}}<V_{\mbox{\tiny T}}$.]{\begin{tikzpicture}
\fill[black!80] (0,0) rectangle (3,0.75);
\fill[black!20] (0,0.75) node[anchor=north west,scale=0.75,black]{p+} -- (0,0.45) .. controls (0.4,0.35) and (1.15,0.45) .. (1.25,0.55)  .. controls (1.5,0.675) and (1.5,0.675) .. (1.75,0.55) .. controls (1.85,0.45) and (2.6,0.35) .. (3,0.45) -- (3,0.75) node[anchor=north east,scale=0.75,black]{p+} -- cycle;
\draw (1,0.75) rectangle ++(1,0.3);
\draw (1.5,0.9) node[scale=0.75]{isolator};
\filldraw[fill=gray] (1,1.05) rectangle ++(1,0.3);
\draw (1.5,1.2) node[scale=0.75]{metaal};
\draw[thick] (1.5,1.35) -- ++(0,0.5) node[anchor=south,scale=0.75]{Gate};
\draw[thick] (0.75,0.75) .. controls (0.75,1.05) and (0.75,1.05) .. (0.5,1.05) node[anchor=east,scale=0.75]{Drain};
\draw[thick] (2.25,0.75) .. controls (2.25,1.05) and (2.25,1.05) .. (2.5,1.05) node[anchor=west,scale=0.75]{Source};
\draw (0,0) to node[above,midway,scale=0.75,white]{n} (3,0);
\draw (0,0.75) -- (3,0.75);
\end{tikzpicture}}
\subfigure[PMOS bij $V_{\mbox{\tiny GD}}\geq V_{\mbox{\tiny T}}$.]{\begin{tikzpicture}
\fill[black!80] (0,0) rectangle (3,0.75);
\fill[black!20] (0,0.75) node[anchor=north west,scale=0.75,black]{p+} -- (0,0.45) .. controls (0.4,0.35) and (1.15,0.45) .. (1.15,0.75) -- cycle;
\fill[black!20] (3,0.75) node[anchor=north east,scale=0.75,black]{p+} -- (3,0.45) .. controls (2.6,0.35) and (1.85,0.45) .. (1.85,0.75) -- cycle;
\draw (1,0.75) rectangle ++(1,0.3);
\draw (1.5,0.9) node[scale=0.75]{isolator};
\filldraw[fill=gray] (1,1.05) rectangle ++(1,0.3);
\draw (1.5,1.2) node[scale=0.75]{metaal};
\draw[thick] (1.5,1.35) -- ++(0,0.5) node[anchor=south,scale=0.75]{Gate};
\draw[thick] (0.75,0.75) .. controls (0.75,1.05) and (0.75,1.05) .. (0.5,1.05) node[anchor=east,scale=0.75]{Drain};
\draw[thick] (2.25,0.75) .. controls (2.25,1.05) and (2.25,1.05) .. (2.5,1.05) node[anchor=west,scale=0.75]{Source};
\draw (0,0) to node[above,midway,scale=0.75,white]{n} (3,0);
\draw (0,0.75) -- (3,0.75);
\end{tikzpicture}}
\caption{Werking van NMOS en PMOS.}
\figlab{mosWork}
\end{figure}
\subsection{Basispoorten}
Nu we automatische schakelaars kunnen gebruiken, wordt het tijd om met behulp van NMOS en PMOS de poorten te implementeren. Eerst zullen we de basispoorten met behulp van NMOS implementeren. Vervolgens zal blijken dat deze implementaties enkele nadelen hebben. Hierdoor zullen we opteren voor implementatie met \termen{CMOS}. CMOS is in feite een combinatie van NMOS en PMOS. We zullen alle schakelingen implementeren in positieve logica. In sectie \ref{s:negativeLogic} behandelen we nog enkele zaken in verband met negatieve logica.
\subsubsection{Implementatie in NMOS}
\paragraph{NOT} \figref{notNmos} toont de implementatie van een NOT-poort in NMOS. Als ingang beschouwen we $x$, als uitgang $f$ de $V_{\mbox{\tiny DD}}$ en $V_{\mbox{\tiny SS}}$ zijn lijnen voor de voeding van de poort. Indien we een laag voltage aanbrengen op $x$ is de NMOS transistor open. Hierdoor staat er een hoge spanning op $f$. Indien we echter een hoge spanning op $x$ aanbrengen, zal de transistor zich sluiten. Hierdoor vloeit er stroom tussen $V_{\mbox{\tiny DD}}$ en $V_{\mbox{\tiny SS}}$. Omdat de stroom door $f$ nog andere componenten zal aansturen, zal de stroom bijgevolg verkiezen om door de transistor te stromen, en krijgt $f$ dus een laag potentiaal. Het principe van het verlagen van de spanning door stroom door te laten wordt \termen{Pull-Down Network (PDN)} genoemd. Theoretisch gezien mag dit model mooi lijken, de werkelijkheid verschilt echter. Wanneer de NMOS transistor immers gesloten is, zal er immers nog steeds een kleine weerstand over staan. Deze weerstand noteren we als $R_{\mbox{\tiny on}}$. Dit betekent dat $f$ niet volledig gelijk zal zijn aan $L$. We berekenen de spanning op $f$ dan ook met volgende formule:
\begin{equation}
V_{\mbox{\tiny out}}\left(x=1\right)=\displaystyle\frac{R_{\mbox{\tiny on}}}{R+R_{\mbox{\tiny on}}}V_{\mbox{\tiny DD}}
\end{equation}
Een tweede groot nadeel is dat we bij $x=H$ statisch vermogen verbruiken:
\begin{equation}
P\left(x=1\right)=\displaystyle\frac{V_{\mbox{\tiny DD}}^2}{R+R_{\mbox{\tiny on}}}
\end{equation}
Dit is op energieniveau een grote kost, vooral omdat er in een gemiddelde processor makkelijk miljoenen transistoren zitten. Bovendien zou dit hoge temperaturen genereren. Dit is dan ook de hoofdreden waarom er nauwelijks elektronica ge\"implementeerd wordt met NMOS.
\begin{figure}[hbt]
\centering
\subfigure[x=L, f=H]{\begin{circuitikz}[american resistors]
\node [nmoso] (T) at (0,3) {};
\draw[<-] (0,4.75) node[anchor=west,scale=0.75]{$V_{\mbox{\tiny DD}}\ (H)$} -- (0,4.5);
\draw (0,4.5) to [R,size=0.5,l={\small $R$}] (0,3.5) -- (T.drain);
\draw (0,3.5) -- ++(1,0) node[anchor=west,scale=0.75]{$f=H$};
\draw (T.gate) -- ++(-0.5,0) node[anchor=east,scale=0.75]{$x=L$};
\draw (T.source) -- (0,2.65) node[ground]{};
\draw (0.4,2.3) node[anchor=west,scale=0.75]{$V_{\mbox{\tiny SS}}\ (L)$};
\end{circuitikz}}
\subfigure[x=H, f=L]{\begin{circuitikz}[american resistors]
\node [nmosc] (T) at (0,3) {};
\draw[<-] (0,4.75) node[anchor=west,scale=0.75]{$V_{\mbox{\tiny DD}}\ (H)$} -- (0,4.5);
\draw (0,4.5) to [R,size=0.5,l={\small $R$}] (0,3.5) -- (T.drain);
\draw (0,3.5) -- ++(1,0) node[anchor=west,scale=0.75]{$f=L$};
\draw (T.gate) -- ++(-0.5,0) node[anchor=east,scale=0.75]{$x=H$};
\draw (T.source) -- (0,2.65) node[ground]{};
\draw (0.4,2.3) node[anchor=west,scale=0.75]{$V_{\mbox{\tiny SS}}\ (L)$};
\end{circuitikz}}
\caption{NOT poort ge\"implementeerd in NMOS.}
\figlab{notNmos}
\end{figure}
\paragraph{``Open-drain poort''}Indien we de weerstand als een extern component beschouwen, en meer NMOS transistoren in parallel aan de draad hangen, kunnen we het gedrag van een AND poort nabootsen, zoals op \figref{openDrainNmos}. Vanaf het moment dat \'e\'en van de transistoren gesloten wordt, lekt de stroom door die transistor, en krijgt $f$ dus een laag niveau. Alle ingangen moeten dus een lage spanning hebben, om een hoge spanning aan de uitgang te verkrijgen. We kunnen dit principe dus zien als een AND waarbij aan alle ingangen een inverter staat. Dit systeem wordt ``\termen{Open-Drain Poort}'' genoemd. En heeft het voordeel dat we een AND kunnen bouwen aan de hand van draden. Deze implementatie is dus goedkoop indien we een AND-poort willen maken met een zeer groot aantal ingangen. Standaard wordt dit soort implementatie dan ook gebruikt in de reset-functionaliteit van elektronica. Meestal wordt er immers een reset procedure aangeroepen indien \'e\'en van de detectoren een fout registreert. In dat geval zal de detector een hoge spanning aan zijn ingang genereren. Dit resulteert een lage uitgang, bij een lage uitgang treed de reset-procedure dan in werking.
\begin{figure}[hbt]
\centering
\begin{circuitikz}[american resistors]
\node [nmosc] (T1) at (-1,2.75) {};
\node [nmosc] (T2) at (-1,0.75) {};
\draw[<-] (0,4.75) node[anchor=west,scale=0.75]{$V_{\mbox{\tiny DD}}\ (H)$} -- (0,4.5);
\draw (0,4.5) to [R,size=0.5,l={\small $R$}] (0,3.5) -- (0,-0.25);
\draw (0,3.5) -| (T1.drain);
\draw (0,1.5) -| (T2.drain);
\draw (0,2.5) -- ++(1,0) node[anchor=west,scale=0.75]{$f$};
\draw (T1.gate) -- ++(-0.25,0) node[anchor=east,scale=0.75]{$x$};
\draw (T1.source) -- ++(0,0) node[ground]{};
\draw (T2.gate) -- ++(-0.25,0) node[anchor=east,scale=0.75]{$y$};
\draw (T2.source) -- ++(0,0) node[ground]{};
\end{circuitikz}
\caption{Open-Drain Poort in NMOS.}
\figlab{openDrainNmos}
\end{figure}
\paragraph{NAND en NOR}
De vorige paragraaf gaf reeds een opzet hoe we een NOR en NAND poort kunnen bouwen. We moeten er eenvoudig weg voor zorgen dat er stroom kan vloeien tussen $V_{\mbox{\tiny DD}}$ en $V_{\mbox{\tiny SS}}$ indien respectievelijk minstens \'e\'en of alle transistoren gesloten worden. \figref{nandNorNmos} toont dan ook de implementaties voor een NAND en NOR poort. Een AND en OR poort kunnen we dan vervolgens synthetiseren door een inverter achter de poort te plaatsen. Deze implementaties maken ook meteen duidelijk waarom een NAND goedkoper is dan een AND.
\begin{figure}
\centering
\subfigure[NAND poort.]{\begin{circuitikz}[american resistors]
\node [nmosc] (T1) at (0,3) {};
\node [nmosc] (T2) at (0,2) {};
\draw[<-] (0,4.75) node[anchor=west,scale=0.75]{$V_{\mbox{\tiny DD}}\ (H)$} -- (0,4.5);
\draw (0,4.5) to [R,size=0.5,l={\small $R$}] (0,3.5) -- (T1.drain);
\draw (0,3.5) -- ++(1,0) node[anchor=west,scale=0.75]{$f$};
\draw (T1.gate) -- ++(-0.5,0) node[anchor=east,scale=0.75]{$x$};
\draw (T1.source) -- (T2.drain);
\draw (T2.gate) -- ++(-0.5,0) node[anchor=east,scale=0.75]{$y$};
\draw (T2.source) -- ++(0,0) node[ground]{};
\end{circuitikz}}
\subfigure[NOR poort.]{\begin{circuitikz}[american resistors]
\node [nmosc] (T1) at (-1,2.5) {};
\node [nmosc] (T2) at (1,2.5) {};
\draw[<-] (0,4.75) node[anchor=west,scale=0.75]{$V_{\mbox{\tiny DD}}\ (H)$} -- (0,4.5);
\draw (0,4.5) to [R,size=0.5,l={\small $R$}] (0,3.5);
\draw (0,3.5) -- ++(1,0) node[anchor=west,scale=0.75]{$f$};
\draw (T1.gate) -- ++(-0.5,0) node[anchor=east,scale=0.75]{$x$};
\draw (T1.drain) |- ++(1,0.25);
\draw (T1.source) |- ++(1,-0.25);
\draw (T2.gate) -- ++(-0.5,0) node[anchor=east,scale=0.75]{$y$};
\draw (T2.drain) |- ++(-1,0.25) -- (0,3.5);
\draw (T2.source) |- ++(-1,-0.25) -- ++(0,-0.25) node[ground]{};
\end{circuitikz}}
\caption{NAND en NOR poort ge\"implementeerd met NMOS.}
\figlab{nandNorNmos}
\end{figure}
\subsubsection{Implementatie in CMOS}
Het grote argument tegen het gebruik van NMOS is dat het veel vermogen nutteloos verbruikt. CMOS biedt hiervoor een oplossing. CMOS is eigenlijk een techniek waarbij we zowel een NMOS als een PMOS transistor gebruiken. Analoog aan het ``Pull-Down Network'' fenomeen van de NMOS, spreken we dan over het ``\termen{Pull-Up Network (PUN)}'' fenomeen bij PMOS.
\paragraph{NOT}Indien we de uitgang $f$ plaatsen tussen een NMOS en PMOS transistor, die we allebei met dezelfde ingang $x$ verbinden, kunnen we een NOT poort implementeren. De uitwerking hiervan staat op \figref{notCmos}. Op het moment dat we een hoge spanning aanleggen op de ingang sluit de NMOS transistor zich, en opent de PMOS transistor zich, hierdoor krijgt de uitgang dezelfde spanning als de ground. Indien we een lage spanning aan de ingang aanleggen is de configuratie van de transistoren omgekeerd, en wordt komt aan de uitgang een spanning $V_{\mbox{\tiny DD}}$ te liggen.
\begin{figure}[hbt]
\centering
\begin{circuitikz}[american resistors]
\node [pmosc] (T1) at (0,3.75) {};
\node [nmoso] (T2) at (0,2.75) {};
\draw[<-] (0,4.5) node[anchor=west,scale=0.75]{$V_{\mbox{\tiny DD}}\ (H)$} -- (0,4);
\draw (0,3.25) -- ++(1,0) node[anchor=west,scale=0.75]{$f=H$};
\draw (T1.source) -- (0,4.5);
\draw (T1.drain) -- (T2.drain);
\draw (T1.gate) -- (T1.gate -| -0.75,0) |- (T2.gate);
\draw (-0.75,3.25) -- ++(-0.5,0) node[anchor=east,scale=0.75]{$x=L$};
\draw (T2.source) -- ++(0,0) node[ground]{};
\begin{scope}[xshift=6 cm]
\node [pmoso] (T1) at (0,3.75) {};
\node [nmosc] (T2) at (0,2.75) {};
\draw[<-] (0,4.5) node[anchor=west,scale=0.75]{$V_{\mbox{\tiny DD}}\ (H)$} -- (0,4);
\draw (0,3.25) -- ++(1,0) node[anchor=west,scale=0.75]{$f=L$};
\draw (T1.source) -- (0,4.5);
\draw (T1.drain) -- (T2.drain);
\draw (T1.gate) -- (T1.gate -| -0.75,0) |- (T2.gate);
\draw (-0.75,3.25) -- ++(-0.5,0) node[anchor=east,scale=0.75]{$x=H$};
\draw (T2.source) -- ++(0,0) node[ground]{};
\end{scope}
\end{circuitikz}
\caption{NOT poort ge\"implementeerd met CMOS.}
\figlab{notCmos}
\end{figure}
\paragraph{NAND en NOR} Ook NAND en NOR poorten hebben hun equivalent in CMOS. \figref{nandNorCmos} toont hun implementatie. Wat opvalt is dat we redelijk eenvoudig het CMOS equivalent kunnen halen uit een NMOS implementatie. Immers is het onderste gedeelte van de NMOS-schakelaars volledig equivalent met de NMOS-implementatie. We plaatsen eenvoudigweg een PMOS circuit boven de uitgang. Dit circuit werkt met duale logica: indien de NMOS verbindingen parallel zijn, zullen we de PMOS transistoren in serie plaatsen, indien de NMOS transistoren in serie stonden, zetten we de PMOS transistoren in parallel. Verder kunnen we de weerstand ook weglaten. Deze weerstand stond er immers alleen om verschillende poorten aan eenzelfde voeding te kunnen hangen. Nu er echter geen stroom vloeit in om het even welke configuratie van de transistoren, is de weerstand dus nagenoeg nutteloos geworden, en brengt deze hoge kosten met zich mee.
\begin{figure}[hbt]
\centering
\subfigure[NAND]{\begin{circuitikz}[american resistors]
\node [pmoso] (T1) at (-1,1) {};
\node [pmoso] (T2) at (1,1) {};
\node [nmosc] (T3) at (0,-0.5) {};
\node [nmosc] (T4) at (0,-1.5) {};
\draw[->] (T1.source) |- ++(1,0.25) -- ++(0,0.5) node[anchor=west,scale=0.75]{$V_{\mbox{\tiny DD}}\ (H)$};
\draw (T2.source) |- ++(-1,0.25);
\draw (T1.drain) |- ++(1,-0.25) -- (0,0);
\draw (T1.gate) -- (T1.gate -| -2,0) node[anchor=east,scale=0.75]{$x$};
\draw (T2.drain) |- ++(-1,-0.25);
\draw (T2.gate) -- ++(-0.5,0) -- (T3.gate -| -0.75,0);
\draw (T4.gate) -- ++(-0.5,0) -- (T1.gate -| -1.75,0);
\draw (T3.gate) -- (T3.gate -| -2,0) node[anchor=east,scale=0.75]{$y$};
\draw (0,0) -- ++(2,0) node[anchor=west,scale=0.75]{$f$};
\draw (0,0) -- (T3.drain);
\draw (T3.source) -- (T4.drain);
\draw (T4.source) -- ++(0,0) node[ground]{};
\end{circuitikz}
\figlab{nandCmos}}
\subfigure[NOR]{\begin{circuitikz}[american resistors]
\node [nmosc] (T1) at (-1,-1) {};
\node [nmosc] (T2) at (1,-1) {};
\node [pmoso] (T3) at (0,0.5) {};
\node [pmoso] (T4) at (0,1.5) {};
%\draw[<-] (0,4.5) node[anchor=west,scale=0.75]{$V_{\mbox{\tiny DD}}\ (H)$} -- (0,4);
\draw[->] (T4.source) -- ++(0,0.25) node[anchor=west,scale=0.75]{$V_{\mbox{\tiny DD}}\ (H)$};
\draw (T2.drain) |- ++(-1,0.25) -- (0,0);
\draw (T1.drain) |- ++(1,0.25);
\draw (T1.source) |- ++(1,-0.25) -- ++(0,-0.25) node[ground]{};
\draw (T2.source) |- ++(-1,-0.25);
\draw (T1.gate) -- (T1.gate -| -2,0) node[anchor=east,scale=0.75]{$y$};
\draw (T2.source) |- ++(-1,-0.25);
\draw (T2.gate) -- ++(-0.5,0) -- (T3.gate -| -0.75,0);
\draw (T4.gate) -- ++(-0.5,0) -- (T1.gate -| -1.75,0);
\draw (T3.gate) -- (T3.gate -| -2,0) node[anchor=east,scale=0.75]{$x$};
\draw (0,0) -- ++(2,0) node[anchor=west,scale=0.75]{$f$};
\draw (0,0) -- (T3.drain);
\draw (T3.source) -- (T4.drain);
\end{circuitikz}
\figlab{norCmos}}
\caption{NAND en NOR poorten ge\"implementeerd met CMOS.}
\figlab{nandNorCmos}
\end{figure}
\paragraph{Uitgangen verbinden}
\label{term:kortsluiting}
Bij NMOS konden we uitgangen met elkaar verbinden, zonder dat dit problemen met zich meebracht, sterker nog, we konden sommige poorten implementeren aan de hand van verbindingen. Een groot nadeel van CMOS is dat dit niet langer mogelijk is. \figref{cmosFail} toont de reden hiervoor. Op het moment dat de ene ingang een hoog potentiaal heeft, en de andere een laag potentiaal ontstaat er immers een kortsluiting. De stroom is in staat om vanaf de $V_{\mbox{\tiny DD}}$ rechtstreeks de aarding te bereiken. Hoewel de transistoren een minimale weerstand hebben, volstaat deze meestal niet. Het gevolg zijn zeer hoge stromen die het elektronische circuit zouden kunnen beschadigen.
\begin{figure}[hbt]
\centering
\begin{circuitikz}[american resistors]
\def\dx{3};
\node [pmosc] (T1) at (0,3.75) {};
\node [nmoso] (T2) at (0,2.75) {};
\draw[<-] (0,4.5) node[anchor=west,scale=0.75]{$V_{\mbox{\tiny DD}}\ (H)$} -- (0,4);
\draw (0,3.25) -| ++(1,-2) -| (\dx+1,3.25);
\draw (T1.source) -- (0,4);
\draw (T1.drain) -- (T2.drain);
\draw (T1.gate) -- (T1.gate -| -0.75,0) |- (T2.gate);
\draw (-0.75,3.25) -- ++(-0.5,0);
\draw (T2.source) -- ++(0,0) node[ground]{};
\node [pmoso] (T3) at (\dx,3.75) {};
\node [nmosc] (T4) at (\dx,2.75) {};
\draw[<-] (\dx,4.5) node[anchor=west,scale=0.75]{$V_{\mbox{\tiny DD}}\ (H)$} -- (\dx,4);
\draw (\dx,3.25) -- ++(1,0);
\draw (T3.source) -- (\dx,4);
\draw (T3.drain) -- (T4.drain);
\draw (T3.gate) -- (T3.gate -| \dx-0.75,0) |- (T4.gate);
\draw (\dx-0.75,3.25) -- ++(-0.5,0);
\draw (T4.source) -- ++(0,0) node[ground]{};
\draw[dotted,thick,rounded corners,->] (0.35,4.1) -- (0.35,3.6) -| (1.35,1.6) -| (\dx+0.65,2.9) -| (\dx+0.35,2);
\end{circuitikz}
\caption{Kortsluiting bij wired poort implementaties met CMOS.}
\figlab{cmosFail}
\end{figure}
\subsection{Complexe poorten}
\begin{figure}[hbt]
\centering
\subfigure[AOI op poortniveau.]{\begin{tikzpicture}[circuit logic US]
\node[and gate,rotate=90] (A1) at (-0.85,0) {};
\draw (A1.input 1) -- (A1.input 1 |- 0,-1.75) node[anchor=north]{$a$};
\draw (A1.input 2) -- (A1.input 2 |- 0,-2.25) node[anchor=north]{$b$};
\node[and gate,inputs={normal,normal,normal},rotate=90] (A2) at (0.85,0) {};
\draw (A2.input 1) -- (A2.input 1 |- 0,-1.75) node[anchor=north]{$x$};
\draw (A2.input 2) -- (A2.input 2 |- 0,-2) node[anchor=north]{$y$};
\draw (A2.input 3) -- (A2.input 3 |- 0,-2.25) node[anchor=north]{$z$};
\node[nor gate,rotate=90] (NO) at (0,3) {};
\draw (NO.output) -- ++(0,1.5) node[anchor=south]{$f$};
\draw (A1.output) -- ++(0,1.4) -| (NO.input 1);
\draw (A2.output) -- ++(0,1.4) -| (NO.input 2);
\end{tikzpicture}
\figlab{aoiExample}}
\subfigure[AOI op CMOS-niveau.]{
\begin{circuitikz}
\def\dxy{1.5};
\def\dg{0.2};
\def\ds{0.1};
\node[pmoso] (PB1) at (-0.5*\dxy,2*\dxy) {};
\draw (PB1.gate) -- ++(-\dg,0) node[anchor=east,scale=0.75]{$a$};
\node[pmoso] (PB2) at (0.5*\dxy,2*\dxy) {};
\draw (PB2.gate) -- ++(-\dg,0) node[anchor=east,scale=0.75]{$b$};

\coordinate (SPBb) at (0,2.5*\dxy-\ds);
\draw (PB1.source) |- (SPBb);
\draw (PB2.source) |- (SPBb);
\draw[->] (SPBb) -- ++(0,0.5) node[anchor=west,scale=0.75]{$V_{\mbox{\tiny DD}}\ (H)$};
\coordinate (SPBe) at (0,1.5*\dxy+\ds);
\draw (PB1.drain) |- (SPBe);
\draw (PB2.drain) |- (SPBe);

\node[pmoso] (PA1) at (-\dxy,1*\dxy) {};
\draw (PA1.gate) -- ++(-\dg,0) node[anchor=east,scale=0.75]{$x$};
\node[pmoso] (PA2) at (0,1*\dxy) {};
\draw (PA2.gate) -- ++(-\dg,0) node[anchor=east,scale=0.75]{$y$};
\node[pmoso] (PA3) at (\dxy,1*\dxy) {};
\draw (PA3.gate) -- ++(-\dg,0) node[anchor=east,scale=0.75]{$z$};

\coordinate (SPAb) at (0,1.5*\dxy-\ds);
\draw (PA1.source) |- (SPAb);
\draw (PA2.source) |- (SPAb);
\draw (PA3.source) |- (SPAb);
\draw (SPBe) -- (SPAb);

\coordinate (SPAe) at (0,0.5*\dxy+\ds);
\draw (PA1.drain) |- (SPAe);
\draw (PA2.drain) |- (SPAe);
\draw (PA3.drain) |- (SPAe);

\fill (SPAe) circle (0.05 cm);
\fill (SPAb) circle (0.05 cm);

\draw (0,0.25) -- ++(2,0) node[anchor=west,scale=0.75]{$f$};

\node[nmosc] (NA1) at (0.5*\dxy,-0.5*\dxy) {};
\draw (NA1.gate) -- ++(-\dg,0) node[anchor=east,scale=0.75]{$x$};
\node[nmosc] (NA2) at (0.5*\dxy,-1.5*\dxy) {};
\draw (NA2.gate) -- ++(-\dg,0) node[anchor=east,scale=0.75]{$y$};
\node[nmosc] (NA3) at (0.5*\dxy,-2.5*\dxy) {};
\draw (NA3.gate) -- ++(-\dg,0) node[anchor=east,scale=0.75]{$z$};
\node[nmosc] (NB1) at (-0.5*\dxy,-\dxy) {};
\draw (NB1.gate) -- ++(-\dg,0) node[anchor=east,scale=0.75]{$a$};
\node[nmosc] (NB2) at (-0.5*\dxy,-2*\dxy) {};
\draw (NB2.gate) -- ++(-\dg,0) node[anchor=east,scale=0.75]{$b$};

\coordinate (SNAb) at (0,-\ds);

\draw (SPAe) -- (SNAb);

\draw (SNAb) -| (NA1.drain);
\draw (SNAb) -| (NB1.drain);
\draw (NA1.source) -- (NA2.drain);
\draw (NA2.source) -- (NA3.drain);
\draw (NB1.source) -- (NB2.drain);

\coordinate (SNAe) at (0,-3*\dxy+\ds);

\draw (NB2.source) |- (SNAe);
\draw (NA3.source) |- (SNAe);
\draw (SNAe) node[ground]{};
\end{circuitikz}
\figlab{aoiCmos}
}
\subfigure[OAI op poortniveau.]{\begin{tikzpicture}[circuit logic US]
\node[or gate,rotate=90] (A1) at (-0.85,0) {};
\draw (A1.input 1) -- (A1.input 1 |- 0,-1.75) node[anchor=north]{$a$};
\draw (A1.input 2) -- (A1.input 2 |- 0,-2.25) node[anchor=north]{$b$};
\node[or gate,inputs={normal,normal,normal},rotate=90] (A2) at (0.85,0) {};
\draw (A2.input 1) -- (A2.input 1 |- 0,-1.75) node[anchor=north]{$x$};
\draw (A2.input 2) -- (A2.input 2 |- 0,-2) node[anchor=north]{$y$};
\draw (A2.input 3) -- (A2.input 3 |- 0,-2.25) node[anchor=north]{$z$};
\node[nand gate,rotate=90] (NO) at (0,3) {};
\draw (NO.output) -- ++(0,1.5) node[anchor=south]{$f$};
\draw (A1.output) -- ++(0,1.4) -| (NO.input 1);
\draw (A2.output) -- ++(0,1.4) -| (NO.input 2);
\end{tikzpicture}
\figlab{oaiExample}}
\subfigure[OAI op CMOS-niveau.]{
\begin{circuitikz}
\def\dxy{1.5};
\def\dg{0.2};
\def\ds{0.1};
\node[nmosc] (PB1) at (-0.5*\dxy,-2*\dxy) {};
\draw (PB1.gate) -- ++(-\dg,0) node[anchor=east,scale=0.75]{$a$};
\node[nmosc] (PB2) at (0.5*\dxy,-2*\dxy) {};
\draw (PB2.gate) -- ++(-\dg,0) node[anchor=east,scale=0.75]{$b$};

\coordinate (SPBb) at (0,-2.5*\dxy+\ds);
\draw (PB1.source) |- (SPBb);
\draw (PB2.source) |- (SPBb);
\draw (SPBb) node[ground] {};
\coordinate (SPBe) at (0,-1.5*\dxy-\ds);
\draw (PB1.drain) |- (SPBe);
\draw (PB2.drain) |- (SPBe);

\node[nmosc] (PA1) at (-\dxy,-1*\dxy) {};
\draw (PA1.gate) -- ++(-\dg,0) node[anchor=east,scale=0.75]{$x$};
\node[nmosc] (PA2) at (0,-1*\dxy) {};
\draw (PA2.gate) -- ++(-\dg,0) node[anchor=east,scale=0.75]{$y$};
\node[nmosc] (PA3) at (\dxy,-1*\dxy) {};
\draw (PA3.gate) -- ++(-\dg,0) node[anchor=east,scale=0.75]{$z$};

\coordinate (SPAb) at (0,-1.5*\dxy+\ds);
\draw (PA1.source) |- (SPAb);
\draw (PA2.source) |- (SPAb);
\draw (PA3.source) |- (SPAb);
\draw (SPBe) -- (SPAb);

\coordinate (SPAe) at (0,-0.5*\dxy-\ds);
\draw (PA1.drain) |- (SPAe);
\draw (PA2.drain) |- (SPAe);
\draw (PA3.drain) |- (SPAe);

\fill (SPAe) circle (0.05 cm);
\fill (SPAb) circle (0.05 cm);

\draw (0,-0.25) -- ++(2,0) node[anchor=west,scale=0.75]{$f$};

\node[pmoso] (NA1) at (0.5*\dxy,0.5*\dxy) {};
\draw (NA1.gate) -- ++(-\dg,0) node[anchor=east,scale=0.75]{$x$};
\node[pmoso] (NA2) at (0.5*\dxy,1.5*\dxy) {};
\draw (NA2.gate) -- ++(-\dg,0) node[anchor=east,scale=0.75]{$y$};
\node[pmoso] (NA3) at (0.5*\dxy,2.5*\dxy) {};
\draw (NA3.gate) -- ++(-\dg,0) node[anchor=east,scale=0.75]{$z$};
\node[pmoso] (NB1) at (-0.5*\dxy,\dxy) {};
\draw (NB1.gate) -- ++(-\dg,0) node[anchor=east,scale=0.75]{$a$};
\node[pmoso] (NB2) at (-0.5*\dxy,2*\dxy) {};
\draw (NB2.gate) -- ++(-\dg,0) node[anchor=east,scale=0.75]{$b$};

\coordinate (SNAb) at (0,\ds);

\draw (SPAe) -- (SNAb);

\draw (SNAb) -| (NA1.drain);
\draw (SNAb) -| (NB1.drain);
\draw (NA1.source) -- (NA2.drain);
\draw (NA2.source) -- (NA3.drain);
\draw (NB1.source) -- (NB2.drain);

\coordinate (SNAe) at (0,3*\dxy-\ds);

\draw (NB2.source) |- (SNAe);
\draw (NA3.source) |- (SNAe);
\draw[->] (SNAe) -- ++(0,0.5) node[anchor=west,scale=0.75]{$V_{\mbox{\tiny DD}}\ (H)$};
\end{circuitikz}
\figlab{oaiCmos}
}
\caption{AND-OR-Inverter (AOI) en OR-AND-Inverter (OAI) in CMOS.}
\figlab{aoiOai}
\end{figure}
In hoofdstuk \ref{ch:basis} hadden we het reeds over de XOR-poort. Een poort die 1 teruggeeft als de twee ingangen niet gelijk zijn aan elkaar. Deze poort hadden we toen ge\"implementeerd met een ingewikkeld schema van poorten. Omdat we echter op transistor niveau werken, kunnen we verschillende complexe poorten toch relatief simpel implementeren. In de volgende subsecties zullen we eerst de \termen{AND-OR-Invert (AOI)} en \termen{OR-AND-Invert (OAI)} implementeren. Deze schakelingen worden relatief vaak gebruikt, om bijvoorbeeld XOR en \termen{XNOR} poorten te implementeren.
\subsubsection{AND-OR-Invert (AOI)}
Een veelgebruikt component is een AND-OR-Inverter. Dit is een component die we op poort-niveau kunnen beschrijven als een tweelagige structuur waarbij de ingangen eerst door een reeks AND-poorten gaan. De uitgangen van de AND-poorten vormen op hun beurt de ingangen van een NOR-poort. Een concreet voorbeeld hiervan staat op \figref{aoiExample}. Indien we echter de schakeling zouden bouwen zoals we dit voorstellen op poortniveau\footnote{We substitueren dus iedere poort door zijn equivalent in CMOS.}, hebben we 18 transistoren nodig. Een effici\"entere manier is het implementeren van een nieuw component de AOI zoals op \figref{aoiCmos}. Hierbij hebben we slechts 10 transistoren nodig. Bovendien hebben we de implementatie gereduceerd tot \'e\'en laag. Hierdoor wordt de vertraging van het component ook teruggedrongen.
\subsubsection{OR-AND-Invert (OAI)}
De tegenhanger van de AND-OR-Inverter is de OR-AND-Inverter. \figref{oaiExample} toont een voorbeeld van dit type component. Naar analogie met de AOI bestaat dit component op poortniveau opnieuw uit twee lagen, ditmaal gaan de ingangen door een reeks OR poorten waarbij hun uitgangen dan weer de invoer van een NAND poort vormen. Opnieuw kunnen we door een implementatie in CMOS het aantal transistoren van 18 naar 10 reduceren.
\subsubsection{Andere veelgebruikte poorten}
Naast de NOT, NAND, NOR, AOI en OAI poorten, worden er nog enkele andere poorten frequent gebruikt. De implementatie van deze poorten wordt weergegeven in \figref{alternativeGatesCmos}. In de volgende paragrafen wordt hun nut en werking kort toegelicht.
\begin{figure}[hbt]
\centering
\subfigure[Buffer]{\begin{tikzpicture}[circuit logic US]
\node [buffer gate,scale=0.75] (B) at (-1,0) {};
\node [not gate,scale=0.75] (N1) at (1,0) {};
\node [not gate,scale=0.75] (N2) at (2,0) {};
\draw (N2.output) -- ++(0.25,0);
\draw (N1.output) -- (N2.input);
\draw (N1.input) -- ++(-0.25,0);
\draw (B.output) -- ++(0.25,0);
\draw (B.input) -- ++(-0.25,0);
\draw (0,0) node{$\equiv$};
\begin{scope}[xshift=1 cm, yshift=-3 cm,xscale=0.75]
\coordinate (F0) at (-3,0);
\coordinate (I) at (-2,0);
\draw (F0) node[anchor=east,scale=0.75]{$x$} -- (I);
\node [pmoso] (P1) at (-1,0.75) {};
\draw (I) |- (P1.gate);
\draw[->] (P1.source) -- ++(0,0.5) node[anchor=west,scale=0.75]{$V_{\mbox{\tiny DD}}$};
\node [nmosc] (N1) at (-1,-0.75) {};
\draw (I) |- (N1.gate);
\draw (N1.source) node [ground] {};
\draw (N1.drain) -- (P1.drain);
\node [pmosc] (P2) at (1,0.75) {};
\coordinate (F1) at (-1,0);
\coordinate (O) at (0,0);
\draw (F1) -- (O);
\draw[->] (P2.source) -- ++(0,0.5) node[anchor=west,scale=0.75]{$V_{\mbox{\tiny DD}}$};
\draw (O) |- (P2.gate);
\node [nmoso] (N2) at (1,-0.75) {};
\draw (O) |- (N2.gate);
\draw (N2.source) node [ground] {};
\draw (N2.drain) -- (P2.drain);
\coordinate (F2) at (1,0);
\draw (F2) -- ++(1,0) node[anchor=west,scale=0.75]{$f$};
\end{scope}
\end{tikzpicture}
\figlab{bufferCmos}
}
\subfigure[AND-poort]{\begin{tikzpicture}[circuit logic US]
\node [and gate,scale=0.75] (B) at (-1,0) {};
\node [nand gate,scale=0.75] (N1) at (1,0) {};
\node [not gate,scale=0.75] (N2) at (2,0) {};
\draw (N2.output) -- ++(0.25,0);
\draw (N1.output) -- (N2.input);
\draw (N1.input 1) -- ++(-0.25,0);
\draw (N1.input 2) -- ++(-0.25,0);
\draw (B.output) -- ++(0.25,0);
\draw (B.input 1) -- ++(-0.25,0);
\draw (B.input 2) -- ++(-0.25,0);
\draw (0,0) node{$\equiv$};
\begin{scope}[xshift=0.25 cm, yshift=-3 cm,xscale=0.75]
\node [pmoso] (P1) at (-1,0.75) {};
\draw (P1.gate) -- ++(-0.5,0) node[anchor=east,scale=0.75]{$x$};
\node [pmoso] (P1b) at (-1,1.75) {};
\draw (P1b.gate) -- ++(-0.5,0) node[anchor=east,scale=0.75]{$y$};
\draw (P1b.drain) -- (P1.source);
\draw[->] (P1b.source) -- ++(0,0.5) node[anchor=west,scale=0.75]{$V_{\mbox{\tiny DD}}$};
\node [nmosc] (N1) at (-2,-0.65) {};
\draw (N1.gate) -- ++(-0.5,0) node[anchor=east,scale=0.75]{$x$};
\draw (N1.source) |- ++(1,-0.1) node [ground] {};
\coordinate (O) at (1,0);
\coordinate (Fb) at (-1,-0.2);
\coordinate (F1) at (-1,0);
\draw (P1.drain) -- (F1) -- (Fb) -| (N1.drain);
\node [nmosc] (N1b) at (0,-0.65) {};
\draw (N1b.gate) -- ++(-0.5,0) node[anchor=east,scale=0.75]{$y$};
\draw (Fb) -| (N1b.drain);
\draw (N1b.source) |- ++(-1,-0.1);
\node [pmosc] (P2) at (2,0.75) {};
\draw (F1) -- (O);
\draw[->] (P2.source) -- ++(0,0.5) node[anchor=west,scale=0.75]{$V_{\mbox{\tiny DD}}$};
\draw (O) |- (P2.gate);
\node [nmoso] (N2) at (2,-0.75) {};
\draw (O) |- (N2.gate);
\draw (N2.source) node [ground] {};
\draw (N2.drain) -- (P2.drain);
\coordinate (F2) at (2,0);
\draw (F2) -- ++(1,0) node[anchor=west,scale=0.75]{$f$};
\end{scope}
\end{tikzpicture}
\figlab{andCmos}
}
\subfigure[OR-poort]{\begin{tikzpicture}[circuit logic US]
\node [or gate,scale=0.75] (B) at (-1,0) {};
\node [nor gate,scale=0.75] (N1) at (1,0) {};
\node [not gate,scale=0.75] (N2) at (2,0) {};
\draw (N2.output) -- ++(0.25,0);
\draw (N1.output) -- (N2.input);
\draw (N1.input 1) -- ++(-0.25,0);
\draw (N1.input 2) -- ++(-0.25,0);
\draw (B.output) -- ++(0.25,0);
\draw (B.input 1) -- ++(-0.25,0);
\draw (B.input 2) -- ++(-0.25,0);
\draw (0,0) node{$\equiv$};
\begin{scope}[xshift=0.25 cm, yshift=-3 cm,xscale=0.75]
\node [pmoso] (P1) at (-2,1.65) {};
\draw (P1.gate) -- ++(-0.5,0) node[anchor=east,scale=0.75]{$x$};
\node [pmoso] (P1b) at (0,1.65) {};
\draw (P1b.gate) -- ++(-0.4,0) node[anchor=east,scale=0.75]{$y$};
\draw (P1.source) |- ++(1,0.1);
\draw[->] (P1b.source) |- ++(-1,0.1) -- ++(0,0.5) node[anchor=west,scale=0.75]{$V_{\mbox{\tiny DD}}$};
\node [nmosc] (N1) at (-1,0.25) {};
\draw (N1.gate) -- ++(-0.5,0) node[anchor=east,scale=0.75]{$x$};
\coordinate (O) at (1,0);
\coordinate (Fb) at (-1,1.2);
\draw (P1b.drain) |- (Fb);
\coordinate (F1) at (-1,1);
\draw (P1.drain) |- (Fb) -- (F1) -- (N1.drain);
\node [nmosc] (N1b) at (-1,-0.75) {};
\draw (N1.source) -- (N1b.drain);
\draw (N1b.source) node [ground] {};
\draw (N1b.gate) -- ++(-0.5,0) node[anchor=east,scale=0.75]{$y$};
\node [pmosc] (P2) at (2,0.75) {};
\draw (F1) -- ++(1,0) |- (O);
\draw[->] (P2.source) -- ++(0,0.5) node[anchor=west,scale=0.75]{$V_{\mbox{\tiny DD}}$};
\draw (O) |- (P2.gate);
\node [nmoso] (N2) at (2,-0.75) {};
\draw (O) |- (N2.gate);
\draw (N2.source) node [ground] {};
\draw (N2.drain) -- (P2.drain);
\coordinate (F2) at (2,0);
\draw (F2) -- ++(1,0) node[anchor=west,scale=0.75]{$f$};
\end{scope}
\end{tikzpicture}
\figlab{orCmos}
}
\subfigure[XOR-poort]{\begin{tikzpicture}[circuit logic US]
\node [xor gate,scale=0.75] (X) at (-1,0) {};
\draw (X.output) -- ++(0.25,0);
\draw (X.input 1) -- ++(-0.25,0);
\draw (X.input 2) -- ++(-0.25,0);
\draw (0,0) node{$\equiv$};
\node [nand gate,scale=0.75] (NA) at (3,0) {};
\draw (NA.output) -- ++(0.5,0);
\node [or gate,scale=0.75] (O1) at (2,0.5) {};
\node [or gate,scale=0.75] (O2) at (2,-0.5) {};
\draw (O1.output) -- ++(0.1,0) |- (NA.input 1);
\draw (O2.output) -- ++(0.1,0) |- (NA.input 2);
\node [not gate,scale=0.35] (N1) at (O1.input 1 -| 1,0) {};
\node [not gate,scale=0.35] (N2) at (O2.input 1 -| 1,0) {};
\draw (N1.output) -- (O1.input 1);
\draw (N2.output) -- (O2.input 1);
\draw (N1.input) -- ++(-0.5,0);
\draw (N2.input) -- ++(-0.5,0);
\draw (N2.input -| 0.8,0) |- (O1.input 2);
\draw (N1.input -| 0.65,0) |- (O2.input 2);
\begin{scope}[xshift=0.25 cm, yshift=-5.5 cm,xscale=0.75,yscale=0.8]
\def\dxs{4};
\foreach\l/\lt/\x in {A/x/0,B/y/\dxs} {
  \coordinate (F0\l) at (-3,\x+0.1);
  \coordinate (I) at (-2,\x+0.1);
  \fill (I) circle (0.06 cm);
  \coordinate (O\l0) at (1,\x+0.1);
  \draw (F0\l) node[anchor=east,scale=0.75]{$\lt$} -- (I) -- (O\l0);
  \node [pmoso] (P1\l) at (-1,0.75+\x) {};
  \draw (I) |- (P1\l.gate);
  \draw[->] (P1\l.source) -- ++(0,0.5) node[anchor=west,scale=0.75]{$V_{\mbox{\tiny DD}}$};
  \node [nmosc] (N1\l) at (-1,\x-0.75) {};
  \draw (I) |- (N1\l.gate);
  \draw (N1\l.source) node [ground] {};
  \draw (N1\l.drain) -- (P1\l.drain);
  \coordinate (F1\l) at (-1,\x-0.1);
  \coordinate (O\l1) at (0,\x-0.1);
  \draw (F1\l) -- (O\l1);
}
\begin{scope}[xshift=3 cm]
\node[pmoso] (PaAA) at (-1,\dxs+0.65) {};
\node[pmoso] (PaBA) at (1,\dxs+0.65) {};
\node[pmosc] (PaAB) at (-1,\dxs-0.95) {};
\node[pmosc] (PaBB) at (1,\dxs-0.95) {};
\draw[->] (PaAA.source) |- ++(1,0.1) -- ++(0,0.5) node[anchor=west,scale=0.75]{$V_{\mbox{\tiny DD}}$};
\draw (PaBA.source) |- ++(-1,0.1);
\draw (PaAA.drain) -- (PaAB.source);
\draw (PaBA.drain) -- (PaBB.source);

\coordinate (Mid) at (0,0.5*\dxs);
\coordinate (MidT) at (0,0.5*\dxs+0.4);
\draw (PaAB.drain) |- (MidT);
\draw (PaBB.drain) |- (MidT);
\coordinate (MidB) at (0,0.5*\dxs-0.4);
\draw (MidB) -- (MidT);
\draw (Mid) -- ++(2,0) node[anchor=west,scale=0.75]{$f$};

\node[nmosc] (NaAA) at (-1,0.95) {};
\node[nmoso] (NaBA) at (1,0.95) {};
\draw (NaAA.drain) |- (MidB);
\draw (NaBA.drain) |- (MidB);
\node[nmoso] (NaAB) at (-1,-0.65) {};
\node[nmosc] (NaBB) at (1,-0.65) {};
\draw (NaAB.source) |- ++(1,-0.1) node[ground]{};
\draw (NaBB.source) |- ++(-1,-0.1);
\coordinate (Nmt) at (0,0.25);
\coordinate (Nmb) at (0,0.05);
\draw (Nmt) -- (Nmb);
\draw (Nmb) -| (NaAB.drain);
\draw (Nmb) -| (NaBB.drain);
\draw (Nmt) -| (NaAA.source);
\draw (Nmt) -| (NaBA.source);
\draw (OB0) |- (PaAA.gate);
\draw (OB1) -- ++(2.75,0) |- (PaBB.gate);
\coordinate (yn) at (-2.25,0 |- OB0);
\coordinate (yi) at (-0.25,0 |- PaBB.gate);
\coordinate (xn) at (OA0 |- NaAA.gate);
\coordinate (xnc) at (xn -| -1.85,0);
\draw (xnc) |- ++(1.5,3.1) |- (PaBA.gate);
\draw (OA0) |- (NaAA.gate);
\draw (OA1) |- (NaAB.gate);
\draw (OA1) |- (PaAB.gate);
\draw (yi) |- (NaBA.gate);
\draw (yn) |- (0.25,-1.5) |- (NaBB.gate);
\end{scope}
\end{scope}
\end{tikzpicture}
\figlab{xorCmos}
}
\subfigure[XNOR-poort]{\begin{tikzpicture}[circuit logic US]
\node [xnor gate,scale=0.75] (X) at (-1,0) {};
\draw (X.output) -- ++(0.25,0);
\draw (X.input 1) -- ++(-0.25,0);
\draw (X.input 2) -- ++(-0.25,0);
\draw (0,0) node{$\equiv$};
\node [nor gate,scale=0.75] (NA) at (3,0) {};
\draw (NA.output) -- ++(0.5,0);
\node [and gate,scale=0.75] (O1) at (2,0.5) {};
\node [and gate,scale=0.75] (O2) at (2,-0.5) {};
\draw (O1.output) -- ++(0.1,0) |- (NA.input 1);
\draw (O2.output) -- ++(0.1,0) |- (NA.input 2);
\node [not gate,scale=0.35] (N1) at (O1.input 1 -| 1,0) {};
\node [not gate,scale=0.35] (N2) at (O2.input 1 -| 1,0) {};
\draw (N1.output) -- (O1.input 1);
\draw (N2.output) -- (O2.input 1);
\draw (N1.input) -- ++(-0.5,0);
\draw (N2.input) -- ++(-0.5,0);
\draw (N2.input -| 0.8,0) |- (O1.input 2);
\draw (N1.input -| 0.65,0) |- (O2.input 2);

\begin{scope}[xshift=0.25 cm, yshift=-5.5 cm,xscale=0.75,yscale=0.8]
\def\dxs{4};
\foreach\l/\lt/\x in {A/x/0,B/y/\dxs} {
  \coordinate (F0\l) at (-3,\x+0.1);
  \coordinate (I) at (-2,\x+0.1);
  \fill (I) circle (0.06 cm);
  \coordinate (O\l0) at (1,\x+0.1);
  \draw (F0\l) node[anchor=east,scale=0.75]{$\lt$} -- (I) -- (O\l0);
  \node [pmoso] (P1\l) at (-1,0.75+\x) {};
  \draw (I) |- (P1\l.gate);
  \draw[->] (P1\l.source) -- ++(0,0.5) node[anchor=west,scale=0.75]{$V_{\mbox{\tiny DD}}$};
  \node [nmosc] (N1\l) at (-1,\x-0.75) {};
  \draw (I) |- (N1\l.gate);
  \draw (N1\l.source) node [ground] {};
  \draw (N1\l.drain) -- (P1\l.drain);
  \coordinate (F1\l) at (-1,\x-0.1);
  \coordinate (O\l1) at (0,\x-0.1);
  \draw (F1\l) -- (O\l1);
}
\begin{scope}[xshift=3 cm]
\node[pmoso] (PaAA) at (-1,\dxs+0.65) {};
\node[pmosc] (PaBA) at (1,\dxs+0.65) {};
\node[pmosc] (PaAB) at (-1,\dxs-0.95) {};
\node[pmoso] (PaBB) at (1,\dxs-0.95) {};
\draw[->] (PaAA.source) |- ++(1,0.1) -- ++(0,0.5) node[anchor=west,scale=0.75]{$V_{\mbox{\tiny DD}}$};
\draw (PaBA.source) |- ++(-1,0.1);

\coordinate (Pmt) at (0,\dxs-0.25);
\coordinate (Pmb) at (0,\dxs-0.05);
\draw (Pmt) -- (Pmb);
\draw (Pmb) -| (PaAA.drain);
\draw (Pmb) -| (PaBA.drain);
\draw (Pmt) -| (PaAB.source);
\draw (Pmt) -| (PaBB.source);

\coordinate (Mid) at (0,0.5*\dxs);
\coordinate (MidT) at (0,0.5*\dxs+0.4);
\draw (PaAB.drain) |- (MidT);
\draw (PaBB.drain) |- (MidT);
\coordinate (MidB) at (0,0.5*\dxs-0.4);
\draw (MidB) -- (MidT);
\draw (Mid) -- ++(2,0) node[anchor=west,scale=0.75]{$f$};

\node[nmosc] (NaAA) at (-1,0.95) {};
\node[nmoso] (NaBA) at (1,0.95) {};
\draw (NaAA.drain) |- (MidB);
\draw (NaBA.drain) |- (MidB);
\node[nmoso] (NaAB) at (-1,-0.65) {};
\node[nmosc] (NaBB) at (1,-0.65) {};
\draw (NaAA.source) -- (NaAB.drain);
\draw (NaBA.source) -- (NaBB.drain);

\draw (NaAB.source) |- ++(1,-0.1) node[ground]{};
\draw (NaBB.source) |- ++(-1,-0.1);

\coordinate (MidOB1) at (OB1 -| -2.5,0);
\draw (OB1) -- (MidOB1);
\draw (PaAB.gate -| MidOB1) |- (PaAB.gate);
\draw (MidOB1) |- (Mid -| -0.25,0) |- (NaBA.gate);
\draw (OB0) |- (PaAA.gate);
\draw (OB0) |- (NaAA.gate);
\draw (OA1) |- (NaAB.gate);
\draw (OA0) -- (0.25,0 |- OA0) |- (PaBB.gate);
\draw (0.25,0 |- OA0) |- (NaBB.gate);
\draw (OA1) |- (0.25,\dxs+1.5) |- (PaBA.gate);

%\draw (OB0) |- (PaAA.gate);
%\draw (OB1) -- ++(2.75,0) |- (PaBB.gate);
%\coordinate (yn) at (-2.25,0 |- OB0);
%\coordinate (yi) at (-0.25,0 |- PaBB.gate);
%\coordinate (xn) at (OA0 |- NaAA.gate);
%\coordinate (xnc) at (xn -| -1.85,0);
%\draw (xnc) |- ++(1.5,3.1) |- (PaBA.gate);
%\draw (OA0) |- (NaAA.gate);
%\draw (OA1) |- (NaAB.gate);
%\draw (OA1) |- (PaAB.gate);
%\draw (yi) |- (NaBA.gate);
%\draw (yn) |- (0.25,-1.5) |- (NaBB.gate);
\end{scope}
\end{scope}

\end{tikzpicture}
\figlab{xnorCmos}
}
\caption{Implementatie van populaire alternatieve poorten in CMOS.}
\figlab{alternativeGatesCmos}
\end{figure}
\paragraph{Buffer}
Een \termen{buffer} of \termen{driver} is een speciaal type poort met \'e\'en ingang waarbij de uitgang dezelfde waarde heeft als de ingang. Men zou dus in principe een buffer kunnen vervangen door een draad. Toch wordt een buffer frequent gebruikt, indien een bepaalde draad verbonden wordt met vele ingangen van andere poorten. In dat geval immers, zou de spanning op deze draad verlaagd worden. Een buffer dient dus om bij complexe circuits een signaal over een groot aantal componenten te kunnen verspreiden, waar een eenvoudige vertakking zou falen. \figref{bufferCmos} toont de notatie en de implementatie van een buffer. We implementeren een buffer meestal met twee opeenvolgende inverters.
\paragraph{AND-poort}
We hebben een AND-poort reeds voldoende ge\"introduceerd om te weten wat deze component doet. Als we naar \figref{andCmos} kijken, zien we dat we deze poort implementeren door een NOT-poort aan de uitgang van een NAND-poort te plaatsen. Dit verklaart ook meteen waarom we de voorkeur zullen geven aan inverterende poorten: een NAND werkt sneller en is bovendien twee transistoren goedkoper.
\paragraph{OR-poort}
Analoog aan een AND-poort implementeren we een OR-poort ook met een inverter na een NOR-poort. \figref{orCmos} toont hierbij de implementatie. De conclusies de we trokken voor AND-poorten gelden uiteraard ook voor een OR-poort.
\paragraph{XOR-poort}
Ook de XOR-poort werd reeds ge\"implementeerd op poortniveau. Om dit te implementeren werd toen een complex netwerk van poorten opgezet. Hierbij werd echter impliciet een AND-OR-Inverter gebruikt. Dit houdt in dat we een XOR poort relatief goedkoop kunnen implementeren. \figref{xorCmos} toont dat deze implementatie een OR-AND-Inverter vraagt en twee invertoren. In totaal hebben we dus 12 transistoren nodig. Indien we deze schakeling zouden implementeren zoals op \figref{complexGatesXor} op pagina \pageref{fig:complexGatesXor} zouden we 22 transistoren gebruiken.
\paragraph{XNOR-poort}
Een laatste populaire poort is de XNOR-poort. Deze poort is niets anders dan de ge\"inverteerde van de XOR-poort. Door eenvoudigweg de AND-OR inverter te vervangen door een OR-AND-Inverter kunnen we deze poort implementeren zoals op \figref{xnorCmos}. De XNOR poort is dan ook de enige poort die even goedkoop is als zijn invers.
\section{Negatieve logica}
\label{s:negativeLogic}
We hebben reeds kort negatieve logica behandeld. Nu we echter poorten ge\"implementeerd hebben, zijn we beter in staat om te vatten wat negatieve logica doet. Indien we de NAND-poort beschouwen op \figref{nandCmos}, kunnen we rekenen met high en low. Tabel \ref{tbl:positiveNegativeLogicHighLow}
\begin{table}[hbt]
\centering
\subtable[High en Low]{
\begin{tabular}{cc|c}
$x$&$y$&$f$\\\hline
L&L&H\\
L&H&H\\
H&L&H\\
H&H&L
\end{tabular}
\label{tbl:positiveNegativeLogicHighLow}
}
\subtable[Positief]{
\begin{tabular}{cc|c}
$x$&$y$&$f$\\\hline
0&0&1\\
0&1&1\\
1&0&1\\
1&1&0
\end{tabular}
\label{tbl:positiveNegativeLogicPositive}
}
\subtable[Negatief]{
\begin{tabular}{cc|c}
$x$&$y$&$f$\\\hline
1&1&0\\
1&0&0\\
0&1&0\\
0&0&1
\end{tabular}
\label{tbl:positiveNegativeLogicNegative}
}
\caption{Verschil tussen positieve en negatieve logica.}
\label{tbl:positiveNegativeLogic}
\end{table}
toont de functie met high en low signalen. Indien we deze signalen interpreteren met positieve logica zoals in tabel \ref{tbl:positiveNegativeLogicPositive} bekomen we zoals verwacht de NAND-poort. Indien we negatieve logica toepassen zoals in tabel \ref{tbl:positiveNegativeLogicNegative} bekomen we een NOR-poort.Bijgevolg kunnen we dus stellen dat positieve en negatieve logica elkaars duale vorm zijn (zie \ref{ss:theoremasPropertiesBooleanAlgebra}).
\section{Technologie\"en}
Niet elk toestel wordt volledig ontworpen en ontwikkeld. Voor goedkope toestellen in beperkte oplages zal men vaak geen nieuwe specifieke chips ontwikkelen. Meestal maakt men gebruik van reeds bestaande componenten, die men vervolgens op een printplaat combineert. Nog een alternatief zijn programmeerbare chips. Hierbij wordt de logica in de chip geprogrammeerd. Dit laat toe programmeerbare chips in grote oplages te produceren die dan vervolgens voor allerhande verschillende toepassingen gebruikt worden. Tot slot worden sommige chips ook volledig zelf geassembleerd. We overlopen eerst kort de drie vormen, waarna we ze in detail bespreken in de volgende subsecties.
\begin{itemize}
 \item \termen{Specifieke Chips (ASIC)} (\S\ref{ss:specifiekeChips}): Hierbij maken we de chips volledig zelf. Deze techniek is echter duur, omdat er bijvoorbeeld een masker moet aangemaakt worden. Bijgevolg is deze techniek enkel winstgevend bij grote volumes.
 \item \termen{Programmeerbare Chips} (\S\ref{ss:programmeerbareChips}): Dit zijn chips waarbij de logica geprogrammeerd kan worden. Meestal is dit echter volgens het ``\termen{Write Once Read Many (WORM)}'' principe. Deze chips kunnen een redelijke complexiteit aan en bevatten anno 2010 2 miljoen logische cellen. Meestal worden deze chips dan ook gebruikt voor prototypes en voor apparaten met kleine tot middelgrote oplages\footnote{minder dan 100 000 stukken per jaar.}.
 \item \termen{Standaard Chips}: Hierbij worden simpele chips gekocht zoals poorten (SSI) en chips die eenvoudige functies vervullen (MSI/LSI). Het enige wat men nog moet doen is deze componenten met elkaar verbinden. De zogenaamde ``\termen{Glue Logic}''. Het gevolg is echter dat we enkel circuits kunnen bouwen met een beperkte complexiteit. Een typevoorbeeld van zo'n chips is bijvoorbeeld de 744 chip. Deze bevat 6 NOT poorten.
\end{itemize}
\subsection{Specifieke chips}
\label{ss:specifiekeChips}
Bij specifieke chips ontwerpen we zelf het volledige ge\"integreerde circuit. Deze techniek is dan ook zeer arbeidsintensief. Er bestaan drie verschillende technieken om specifieke chips te maken: maatwerk, standaard cellen en gate-arrays. Deze technieken worden in de volgende subsubsecties besproken.
\subsubsection{Maatwerk}
Bij \termen{maatwerk} zullen we elke transistor en verbinding zelf ontwerpen. Deze componenten stellen we dan voor als een set rechthoeken, die we op de chipoppervlakte plaatsen. Deze techniek zal in het algemeen tot het meest optimale ontwerp leiden qua snelheid, vermogenverbruik en afmetingen. Toch is deze techniek niet haalbaar voor complexe systemen. Deze techniek wordt wel vaak toegepast bij het ontwerpen van componenten voor een bibliotheek. In dat geval zal men bijvoorbeeld een opteller zo effici\"ent mogelijk implementeren, zodat complexere systemen die een opteller nodig hebben, een zeer effici\"ente implementatie kunnen gebruiken. Nog een groot nadeel van deze techniek is, dat de technologie verder evolueert, de transistoren worden telkens kleiner, en om de 18 maanden is een volledige herimplementatie nodig. Deze manier van werken wordt ook wel ``\termen{Custom Design}'' genoemd.
\subsubsection{Standaard cellen}
Een effici\"entere manier van werken is met de zogenoemde ``\termen{standaard cellen}''. Deze techniek is min of meer analoog met het maatwerk. Alleen gebruiken we hier componenten uit een bibliotheek als cellen, in plaats van transistoren. Elk van deze cellen heeft een vaste hoogte, en een variabele breedte. Bovendien wordt er in de hoogte ruimte voorzien voor bedrading. Elke logische cel heeft enkele ingangen aan de bovenkant, en enkele uitgangen aan de onderkant van de cel. Het komt er dus alleen nog op aan de cellen op een interessante manier te plaatsen, de zogenoemde ``\termen{placement}''. En het leggen van bedrading, wat ``\termen{routing}'' genoemd wordt. Deze taken dienen met de nodige zorg te gebeuren. Componenten die veel met elkaar interageren worden beter dicht bij elkaar gezet, om de snelheid op te drijven, en bovendien is het aantal lagen voor bekabeling heel beperkt. Deze techniek laat echter wel toe om snel complexe bouwblokken te ontwerpen, en bovendien laat deze methode toe dat fabrikanten cellen kunnen optimaliseren. \figref{standardcells} toont hoe het ontwerp van deze standaard cellen er ongeveer uitziet.
\begin{figure}[hbt]
\centering
\begin{tikzpicture}
\foreach \y in {0,1,2} {
  \draw[thick] (0,1.5*\y) -- (0,0.5+1.5*\y);
}
\foreach \y/\xa/\xb in {0/0/5,0/5/8,0/8/10,0/10/13,0/13/16,1/0/1,1/1/3,1/3/9,1/9/11,1/11/13,1/13/14,1/14/16,2/0/2,2/2/4,2/4/5,2/5/7,2/7/9,2/9/11,2/11/16} {
  \draw[thick] (0.3*\xa,1.5*\y) -- (0.3*\xb,1.5*\y) -- (0.3*\xb,0.5+1.5*\y) -- (0.3*\xa,0.5+1.5*\y);
}
\foreach \ya/\xa/\ym/\yb/\xb in {2/1/1.5/1/4,2/3/1.25/0/2,2/6/0.5/0/14.5,2/8/1.5/1/8,2/13.5/1.5/1/10,1/4/0.5/0/3,1/10/0.25/0/6.5} {
  \draw (0.3*\xa,1.5*\ya) -- (0.3*\xa,0.25+1.5*\ym) -| (0.3*\xb,0.5+1.5*\yb);
}
\end{tikzpicture}
\caption{Ontwerp met standaard cellen.}
\figlab{standardcells}
\end{figure}
\subsubsection{Gate-array}
Een techniek die nog minder ontwerp-kosten met zich meebrengt is de \termen{Gate Array}. Hierbij beschouwen we een tweedimensionaal rooster van identieke poorten\footnote{Meestal worden hiervoor NAND-poorten gebruikt. Bijvoorbeeld een 3-input NAND-poort.}, de zogenoemde ``\termen{Sea of Gates}''. In dit rooster heeft elke poort identieke afmetingen, en wordt er ruimte gelaten voor bedrading. Opnieuw bevinden zich alle ingangen bovenaan en de uitgangen onderaan. Enkel de bedrading is vervolgens nog uniek aan het product. Dit heeft niet alleen het voordeel van goedkoop ontwerp, de rooster kunnen in massa geproduceerd worden, om daarna elk tot een specifieke circuit uit te groeien. Enkel de metallisatielaag\footnote{De laag waar de verbindingen gelegd worden.} is dus uniek.
\subsection{Programmeerbare chips}
\label{ss:programmeerbareChips}
Prototypes waar nog fouten uitgehaald moeten worden, of elektronica in bijvoorbeeld computers waar de firmware kan veranderen, worden meestal ge\"implementeerd met \termen{programmeerbare chips}. Maar ook onder de programmeerbare chips bestaan nog verschillende technologie\"en. Vooral de programmeertechnieken zijn zeer divers. Hieronder worden de meest courante technieken opgesomd. Vervolgens worden deze technieken verder toegelicht in de subsubsecties.
\begin{itemize}
 \item \termen{Zekeringen}: Hierbij bevat de chip een groot aantal zekeringen. We kunnen dan vervolgens een deel van deze zekeringen doorbranden, om de logica te implementeren. Deze techniek is irreversibel. Immers kan een doorgebrande zekeringen niet hersteld worden. Weliswaar blijft bijprogrammeren\footnote{In een tweede iteratie andere zekeringen doorbranden.} wel mogelijk.%TODO: doorgebrande weglaten?
 \item \termen{Flash-programmeerbaar}: Hierbij kunnen we verbindingen openen en sluiten met transistoren waarbij de gate opgeladen kan worden. Hierdoor wordt herprogrammeren mogelijk, dit is echter traag, en bovendien zal na enige tijd het flash geheugen niet meer herprogrammeerbaar zijn.
 \item \termen{Geheugen-programmeerbaar}: Hierbij bevat de chip een geheugen component, meestal in \termen{SRAM}. De transistoren die de verbindingen bepalen worden dan gekoppeld aan dit geheugen. Het voordeel hierbij is dat we makkelijk en snel de chip kunnen programmeren, en dit kan zelfs dynamisch\footnote{De chip kan tijdens uitvoering zijn gedrag aanpassen.}. Het nadeel is dat dit geheugen telkens bij het aanleggen van voedingsspanning weer opnieuw ingeladen moet worden. Het typevoorbeeld van deze techniek is een \termen{Field Programmable Gate Array (FPGA)}.
\end{itemize}
\subsubsection{Programmable Logic Array (PLA)}
In sectie \ref{s:synthese} zagen we reeds dat we alle logische functies kunnen maken met een \termen{Sum-of-Products (SOP)}. Een \termen{Programmable Logic Array (PLA)} is gebaseerd op dit idee. Deze chip bevat twee rasters van poorten: een \termen{AND-matrix} en een \termen{OR-matrix}. Als ingangen kunnen we vervolgens zowel een variable als zijn inverse aanleggen. \figref{plaSchema}
\begin{figure}[hbt]
\centering
\begin{tikzpicture}[circuit logic US]
\foreach \y in {0,1,...,7} {
  \node[and gate, inputs={normal,normal,normal,normal,normal,normal,normal,normal},scale=0.5] (A\y) at (-1,0.8*\y) {$A\y$};
  \node[or gate, inputs={normal,normal,normal,normal,normal,normal,normal,normal},rotate=-90,scale=0.5] (O\y) at (0.8*\y,-1) {$O\y$};
  \draw (O\y.output) -- ++(0,-0.5) node[scale=0.75,anchor=north]{$f_\y$};
}
\foreach \t/\xa/\xb/\xc/\xd in {0/0/1/8/7,1/2/3/6/5,2/4/5/4/3,3/6/7/2/1} {
  \node[not gate,scale=0.5,rotate=-90] (N\t) at (-6.8+1.6*\t,6.6) {$N\t$};
  \draw[very thin] (N\t.input) |- ++(-0.8,0.5);
  \draw[very thin] (-7.6+1.6*\t,7.6) node[anchor=south,scale=0.75] {$x_\t$} -- (-7.6+1.6*\t,0 |- N\t.output);
  \coordinate (I\xa) at (-7.6+1.6*\t,0 |- N\t.output);
  \coordinate (I\xb) at (N\t.output);
  \draw[very thin] (I\xa) -- (I\xa |- A0.input \xc);
  \draw[very thin] (I\xb) -- (I\xb |- A0.input \xd);
}
\draw[dashed] (-8,-0.9) node[anchor=south west,scale=0.75]{AND-matrix} rectangle (-1.8,6);
\draw[dashed] (-0.4,-0.4) rectangle (6,6.5) node[anchor=north east,scale=0.75]{OR-matrix};
\foreach \y/\yf in {0/8,1/7,2/6,3/5,4/4,5/3,6/2,7/1} {
  \draw[very thin] (A\y.output) -- (O7.input \yf |- A\y.output);
  \foreach \yi/\yid in {1/7,2/6,3/5,4/4,5/3,6/2,7/1,8/0} {
    \draw[very thin] (A\y.input \yi -| I\yid) -- (A\y.input \yi);
    \draw[very thin] (O\y.input \yi) -- (O\y.input \yi |- A\yid.output);
    \fill (O\y.input \yi |- A\yid.output) circle (0.035 cm);
    \fill (A\y.input \yi -| I\yid) circle (0.035 cm);
  }
}
\foreach \y/\x/\xi in {0/0/8,0/2/6,0/4/4,0/7/1,1/0/8,1/2/6,1/4/4,1/6/2,2/1/7,2/2/6,2/5/3,2/6/2,3/1/7,3/2/6,3/5/3,3/6/2,4/0/8,4/3/5,4/5/3,4/7/1,5/0/8,5/2/6,5/4/4,5/7/1,6/0/8,6/2/6,6/5/3,6/7/1,7/1/7,7/3/5,7/4/4,7/6/2,2/3/5,2/4/4,0/6/2} {
  \draw[thick] (A\y.input \xi -| I\x) -- ++(0.05,0.05);
  \draw[thick] (A\y.input \xi -| I\x) -- ++(-0.05,0.05);
  \draw[thick] (A\y.input \xi -| I\x) -- ++(0.05,-0.05);
  \draw[thick] (A\y.input \xi -| I\x) -- ++(-0.05,-0.05);
}

\foreach \y/\x/\xi in {0/1/7,0/2/6,0/3/5,0/5/3,0/7/1,1/1/7,1/2/6,1/3/5,2/0/8,2/2/6,2/3/5,2/5/3,2/7/1,3/0/8,3/5/3,3/6/2,4/1/7,4/2/6,4/4/4,4/7/1,5/1/7,5/2/6,6/2/6,6/4/4,6/5/3,6/6/2,7/0/8,7/5/3,7/7/1} {
  \draw[thick] (O\y.input \xi |- A\x.output) -- ++(0.05,0.05);
  \draw[thick] (O\y.input \xi |- A\x.output) -- ++(-0.05,0.05);
  \draw[thick] (O\y.input \xi |- A\x.output) -- ++(0.05,-0.05);
  \draw[thick] (O\y.input \xi |- A\x.output) -- ++(-0.05,-0.05);
}
\end{tikzpicture}
\caption{Schematische voorstelling van een Programmable Logic Array (PLA).}
\figlab{plaSchema}
\end{figure}
toont een schematische voorstelling van zo'n PLA. De punten tussen twee verbindingen stellen een zekering door. Door zekeringen door te branden kunnen we dus het model aanpassen. Punten waar een kruisje bij staat zijn in dit voorbeeld doorgebrand. Bij een doorgebrande zekering in de AND-matrix, komt er een 1 op de lijn, bij een doorgebrande zekering op de OR-matrix een 0. Er bestaan nog twee varianten op een PLA:
\begin{itemize}
 \item een \termen{Programmable Array Logic (PAL)} heeft een vaste OR-matrix.
 \item een \termen{Programmable Read Only Memory (PROM)} heeft een vaste AND-matrix. Hierbij fungeert deze matrix als een adresdecoder (zie \ref{ss:decoder}).
\end{itemize}
\subsubsection{Programmable Logic Device (PLD)}
\label{sss:pld}
Een PLA is niet in staat complexe functies te beschrijven zonder de AND- en OR-matrix zeer groot te maken. Omdat immers voor iedere verbinding ook nog lijnen moeten voorzien worden, zou het component al snel te groot worden. Een \termen{Programmable Logic Device (PLD)} probeert hierop een antwoord te bieden. Hierbij wordt meer lagen logica gehanteerd. \figref{pldSchema}
\begin{figure}[hbt]
\centering
\begin{tikzpicture}[circuit logic US] %yscale=0.8
\foreach\y/\ia/\ib in {0/1/2,1/3/4,2/5/6,3/7/8,4/9/10} {
  \node[not gate,scale=0.5] (N\y) at (0,\y) {$N\y$};
  \coordinate (I\y) at (-1-0.1*\y,0.5+\y);
  \draw[very thin] (I\y) -- (-0.5,0.5+\y -| N\y.output);
  \draw[very thin] (N\y) -| ++(-0.5,0.5);
  \coordinate (B\ia) at (N\y.output);
  \coordinate (B\ib) at (N\y.output |- -0.5,0.5+\y);
}
\draw (I3) node[anchor=east,scale=0.75]{$x_1$};
\draw (I4) node[anchor=east,scale=0.75]{$x_2$};
\foreach\s in {0,1,...,3} {
  \node[or gate,inputs={normal,normal,normal,normal,normal},scale=0.5,rotate=-90] (O\s) at (3*\s+2.2,-2) {};
  \foreach\x/\xi/\iy in {0/5/0.7,1/4/0.6,2/3/0.5,3/2/0.6,4/1/0.7} {
    \node[and gate,inputs={normal,normal,normal,normal,normal,normal,normal,normal,normal,normal},scale=0.333,rotate=-90] (A\s\x) at (3*\s+0.6*\x+1,-1) {};
    \draw[very thin] (A\s\x.output) -- ++(0,-\iy+0.5) -| (O\s.input \xi);
    \foreach \g in {1,2,...,10} {
      \fill (B\g -| A\s\x.input \g) circle (0.035 cm);
      \draw[very thin] (B\g -| A\s\x.input \g) -- (A\s\x.input \g);
    }
  }
}
\foreach\s in {0,3} {
  \coordinate (OO\s) at (O\s.output |- 0,-3-0.1*\s);
  \coordinate (OOO\s) at (OO\s);
  \draw (O\s.output) -- (OO\s);
}
\foreach\s in {1,2} {
  \node[draw=black,rectangle] (D\s) at (3*\s+2.8,-3.125) {D};
  \draw[very tiny] (D\s.north) |- ++(-0.6,0.25);
  \coordinate (OO\s) at (3*\s+2.5,-4.5-0.1*\s);
  \draw[very tiny] (D\s.south) -- (D\s.south |- 0,-3.75);
  \draw[very thin] (O\s.output) -- (O\s.output |- 0,-3.75);
  \draw[very thin] (3*\s+2.5,-4.25) -- (OO\s);
  \coordinate (OOO\s) at (OO\s |- 0,-5.5);
  \draw[very thin] (OO\s) -- (OOO\s);
  \draw (3*\s+2,-3.75) -- (3*\s+3,-3.75) -- (3*\s+2.8,-4.25) -- (3*\s+2.2,-4.25) -- cycle;
}
\foreach\s in {0,1,...,2} {
  \draw[very thin] (OO\s) -| (I\s);
}
\foreach\s in {1,2,3} {
  \draw(OOO\s) node[anchor=north,scale=0.75] {$f_\s$};
}
\foreach\y in {1,2,...,10} {
  \draw[very thin] (B\y) -- (B\y -| A34.input \y);
}
\foreach \s/\x/\g in {0/0/1,0/0/2,0/0/3,0/0/4,0/0/6,0/0/7,0/0/10,0/1/1,0/1/3,0/1/4,0/1/5,0/1/6,0/1/8,0/1/10,0/2/1,0/2/2,0/2/3,0/2/4,0/2/6,0/2/8,0/2/10,0/3/1,0/3/2,0/3/3,0/3/6,0/3/7,0/3/9,0/3/10,0/4/1,0/4/3,0/4/5,0/4/6,0/4/7,0/4/10,1/0/1,1/0/2,1/0/4,1/0/6,1/0/7,1/0/9,1/0/10,1/1/2,1/1/4,1/1/6,1/1/8,1/1/9,1/1/10,1/2/1,1/2/2,1/2/3,1/2/6,1/2/7,1/2/8,1/2/9,1/3/1,1/3/3,1/3/4,1/3/6,1/3/7,1/3/8,1/3/10,1/4/1,1/4/2,1/4/3,1/4/4,1/4/5,1/4/7,1/4/9,1/4/10,2/0/2,2/0/3,2/0/4,2/0/6,2/0/7,2/0/8,2/0/10,2/1/2,2/1/4,2/1/5,2/1/6,2/1/8,2/1/9,2/1/10,2/2/1,2/2/2,2/2/3,2/2/5,2/2/6,2/2/7,2/2/8,2/2/10,2/3/1,2/3/2,2/3/3,2/3/5,2/3/7,2/3/8,2/3/9,2/3/10,3/4/1,3/4/3,3/4/4,3/4/6,3/4/8,3/4/10,3/0/2,3/0/3,3/0/4,3/0/6,3/0/7,3/0/8,3/0/9,3/1/2,3/1/3,3/1/6,3/1/7,3/1/9,3/2/1,3/2/2,3/2/4,3/2/5,3/2/7,3/2/10,3/3/1,3/3/2,3/3/3,3/3/4,3/3/6,3/3/8,3/3/9,3/4/2,3/4/4,3/4/6,3/4/7,3/4/9} {
  \draw[thick] (B\g -| A\s\x.input \g) -- ++(0.05,0.05);
  \draw[thick] (B\g -| A\s\x.input \g) -- ++(-0.05,0.05);
  \draw[thick] (B\g -| A\s\x.input \g) -- ++(0.05,-0.05);
  \draw[thick] (B\g -| A\s\x.input \g) -- ++(-0.05,-0.05);
}
\end{tikzpicture}%TODO: make more compact + multiplexer input
\caption{Schematische voorstelling van een Programmable Logic Device (PLD).??}
\figlab{pldSchema}
\end{figure}
toont het idee achter deze techniek. We beschouwen nog steeds een matrix, alleen is een groot deel van de invoer eigenlijk de uitvoer van andere geprogrammeerde functies. Zo zijn $x_1$ en $x_2$ de enig invoerlijnen, de andere lijnen zijn het resultaat van geprogrammeerde logica. Op de figuur bemerken we verder nog dat er componenten onder de uitgangen staan. Deze componenten komen later in de cursus aan bod. Het vierkant stelt een 1-bit geheugen voor (zie \ref{s:memory}), meestal een D-Flip Flop. De trapezium een multiplexer (zie \ref{ss:multiplexer}). Dit component laat ons dus toe complexere schakelingen te maken, en tegelijk het aantal zekeringen onder controle te houden.
\subsubsection{Complex Programmable Logic Device (CPLD)}
\begin{figure}[H]%TODO: check alignment, fixed floating environment used [H]
\centering
\begin{tikzpicture}[scale=0.75]
\draw (-3.375,-2.35) node[scale=0.75]{O};
\draw (-4,-2.6) rectangle (-2.75,-2.1);
\draw (-1.625,-2.35) node[scale=0.75]{I/O};
\draw (-1,-2.6) rectangle (-2.25,-2.1);
\draw (-2.5,-1.2) node[scale=0.75]{PLD};
\draw (-4,-1.7) rectangle (-1,-0.7);

\draw (3.375,-2.35) node[scale=0.75]{O};
\draw (4,-2.6) rectangle (2.75,-2.1);
\draw (1.625,-2.35) node[scale=0.75]{I/O};
\draw (1,-2.6) rectangle (2.25,-2.1);
\draw (2.5,-1.2) node[scale=0.75]{PLD};
\draw (4,-1.7) rectangle (1,-0.7);

\draw (-3.375,2.35) node[scale=0.75]{O};
\draw (-4,2.6) rectangle (-2.75,2.1);
\draw (-1.625,2.35) node[scale=0.75]{I/O};
\draw (-1,2.6) rectangle (-2.25,2.1);
\draw (-2.5,1.2) node[scale=0.75]{PLD};
\draw (-4,1.7) rectangle (-1,0.7);

\draw (3.375,2.35) node[scale=0.75]{O};
\draw (4,2.6) rectangle (2.75,2.1);
\draw (1.625,2.35) node[scale=0.75]{I/O};
\draw (1,2.6) rectangle (2.25,2.1);
\draw (2.5,1.2) node[scale=0.75]{PLD};
\draw (4,1.7) rectangle (1,0.7);
\draw (0,0) node[scale=0.75]{Schakelmatrix};
\draw (-4,-0.3) rectangle (4,0.3);
\foreach \x in {-3,-2,...,3} {
  \draw (-3.375+0.05*\x,-2.1) -- (-3.375+0.05*\x,-1.7);
  \draw (-1.625+0.05*\x,-2.1) -- (-1.625+0.05*\x,-1.7);
  \draw (-1.625+0.05*\x,-2.6) -- (-1.625+0.05*\x,-3.1);

  \draw (3.375+0.05*\x,-2.1) -- (3.375+0.05*\x,-1.7);
  \draw (1.625+0.05*\x,-2.1) -- (1.625+0.05*\x,-1.7);
  \draw (1.625+0.05*\x,-2.6) -- (1.625+0.05*\x,-3.1);

  \draw (-3.375+0.05*\x,2.1) -- (-3.375+0.05*\x,1.7);
  \draw (-1.625+0.05*\x,2.1) -- (-1.625+0.05*\x,1.7);
  \draw (-1.625+0.05*\x,2.6) -- (-1.625+0.05*\x,3.1);

  \draw (3.375+0.05*\x,2.1) -- (3.375+0.05*\x,1.7);
  \draw (1.625+0.05*\x,2.1) -- (1.625+0.05*\x,1.7);
  \draw (1.625+0.05*\x,2.6) -- (1.625+0.05*\x,3.1);
  
  \draw (-0.5+0.05*\x,0.3) |- (-1.625+0.05*\x,2.9+0.05*\x);
  \draw (0.5-0.05*\x,0.3) |- (1.625-0.05*\x,2.9+0.05*\x);
  \draw (-0.5+0.05*\x,-0.3) |- (-1.625+0.05*\x,-2.9-0.05*\x);
  \draw (0.5-0.05*\x,-0.3) |- (1.625-0.05*\x,-2.9-0.05*\x);
}
\foreach \x in {-12,-11,...,12} {
  \draw (-2.5-0.05*\x,-0.7) -- (-2.5-0.05*\x,-0.3);
  \draw (-2.5-0.05*\x,0.7) -- (-2.5-0.05*\x,0.3);
  \draw (2.5-0.05*\x,-0.7) -- (2.5-0.05*\x,-0.3);
  \draw (2.5-0.05*\x,0.7) -- (2.5-0.05*\x,0.3);
}
\end{tikzpicture}
\caption{Schematische voorstelling van een Complex Programmable Logic Device (CPLD).}
\figlab{cpldSchema}
\end{figure}
Een aantal PLD componenten samen op \'e\'en chip samen met de zogenaamde \termen{Glue Logic} om ze samen te laten werken, noemen we een \termen{Complex Programmable Logic Device (CPLD)}. \figref{cpldSchema} toont dat deze glue eigenlijk neerkomt op een \termen{schakelmatrix} om PLDs te laten samenwerken en \termen{in- en uitvoer modules} om te intrageren met de buitenwereld. Naast de PLDs zelf dient dus ook de schakelmatrix geprogrammeerd te worden. Een ander belangrijk verschil is dat CPLDs niet geprogrammeerd worden met behup van zekeringen. De chip wordt geprogrammeerd door ladingen op transistoren te plaatsen, bijgevolg kunnen de we deze chip enkele keren volledig herprogrammeren.
\subsubsection{Field Programmable Gate Array (FPGA)}
Een nog meer geavanceerde programmeerbare chip is de \termen{Field Programmable Gate Array (FPGA)}. Net zoals bij een CPLD bevat deze chip enkele logische blokken met daartussen ``glue''. Deze ``glue'' uit zich opnieuw in een schakelmatrix, maar ook in korte en lange verbindingen. Deze laatsten worden vaak ook \termen{lange lijnen} genoemd. Daarnaast bevat de chip uiteraard ook opnieuw en- en uitvoer modulen. \figref{fpgaSchemaFull}
\begin{figure}[hbt]
\centering
\subfigure[Volledige FPGA]{\begin{tikzpicture}[scale=1.15]
\filldraw[fill=black!35,draw=black] (0,0) rectangle (4.5,4.5);
\foreach \y in {1,2,...,16} {
  \fill[fill=white,draw=black] (0.25*\y+0.025,0) rectangle ++(0.20,0.225);
  \fill[fill=white,draw=black] (0.25*\y+0.025,4.275) rectangle ++(0.20,0.225);
  \fill[fill=white,draw=black] (0,0.25*\y+0.025) rectangle ++(0.225,0.20);
  \fill[fill=white,draw=black] (4.275,0.25*\y+0.025) rectangle ++(0.225,0.20);
  \foreach \xs in {1,2,3} {
    \draw (0.25*\y+0.05*\xs+0.025,0) -- ++(0,0.225);
    \draw (0.25*\y+0.05*\xs+0.025,4.275) -- ++(0,0.225);
  \draw (0,0.25*\y+0.05*\xs+0.025) -- ++(0.225,0);
    \draw (4.275,0.25*\y+0.05*\xs+0.025) -- ++(0.225,0);
  }
  \foreach \x in {1,2,4,5,...,13,15,16} {
    \filldraw[fill=white,draw=black] (0.25*\x+0.025,0.25*\y+0.025) rectangle ++(0.20,0.20);
  }
}
\foreach \x in {3,14} {
  \fill (0.25*\x+0.025,0) rectangle ++(0.20,0.225);
  \fill (0.25*\x+0.025,4.275) rectangle ++(0.20,0.225);
  \foreach \y in {0,1,...,3} {
    \filldraw[fill=white,draw=black] (0.25*\x+0.025,\y+0.275) rectangle ++(0.075,0.95);
    \filldraw[fill=white,draw=black] (0.25*\x+0.15,\y+0.275) rectangle ++(0.075,0.95);
  }
}
\draw[thick] (1.5,4.525) rectangle ++(1.25,-1.275);
\begin{scope}[xshift=5 cm]
\draw[thick] (-3.5,3.25) -- (0,0);
\draw[thick] (-2.25,3.25) -- (4.5,0);
\draw[thick] (-2.25,4.525) -- (4.5,4.5);
\fill[fill=black!35] (0,0) rectangle (4.5,4.5);
\foreach \x in {0,1,...,4} {
  \filldraw[fill=black!60,draw=black] (0.9*\x+0.15,3.75) rectangle ++(0.6,0.75);
  \draw (0.9*\x+0.45,4.125) node[white] {I/O};
}
\foreach \x in {0,2,4} {
  \foreach \y in {1,3} {
    \filldraw[fill=white,draw=black] (0.9*\x+0.15,0.9*\y+0.15) rectangle ++(0.6,0.6);
    \draw (0.9*\x+0.45,0.9*\y+0.45) node{SM};
  }
  \foreach \y in {0,2} {
    \filldraw[fill=white,draw=black] (0.9*\x+0.15,0.9*\y+0.15) rectangle ++(0.6,0.6);
    \draw (0.9*\x+0.45,0.9*\y+0.45) node{SMc};
  }
}
\foreach \x in {1,3} {
  \foreach \y in {0,2} {
    \filldraw[fill=black!70,draw=black] (0.9*\x+0.15,0.9*\y+0.15) rectangle ++(0.6,0.6);
    \draw (0.9*\x+0.45,0.9*\y+0.45) node[white]{LB};
  }
  \foreach \y in {1,3} {
    \filldraw[fill=white,draw=black] (0.9*\x+0.15,0.9*\y+0.15) rectangle ++(0.6,0.6);
    \draw (0.9*\x+0.45,0.9*\y+0.45) node{SMc};
  }
}
\foreach \y in {0,1,...,3} {
  \foreach \s in {-4,-3,...,4} {
    \draw (0,0.9*\y+0.45+0.04*\s) -- ++(0.15,0);
    \draw (4.35,0.9*\y+0.45+0.04*\s) -- ++(0.15,0);
    \foreach \x in {0,1,...,4} {
      \draw (0.9*\x+0.45+0.04*\s,0.9*\y+0.75) -- ++(0,0.3);
    }
    \foreach \x in {0,1,...,3} {
      \draw (0.9*\x+0.75,0.9*\y+0.45+0.04*\s) -- ++(0.3,0);
    }
  }
}
\foreach \x in {0,1,...,4} {
  \foreach \s in {-4,-3,...,4} {
    \draw (0.9*\x+0.45+0.04*\s,0) -- ++(0,0.15);
  }
}
\foreach \s in {-1,0,1} {
  \foreach \y in {0,2} {
    \draw (0,0.9*\y+0.04*\s+0.9) -- ++(4.5,0);
    \draw (0.9*\y+0.04*\s+0.9,0) -- ++(0,0.04*\s+2.7);
  }
}
\end{scope}
\end{tikzpicture}
\figlab{fpgaSchemaFull}}
\subfigure[Logical Block (LB)] {
\begin{tikzpicture}
\filldraw[fill=black!35,draw=black] (-0.5,-0.5) rectangle (3.5,2.5);
\filldraw[fill=white,draw=black] (0.1,0.1) rectangle ++(1.8,1.8);
\filldraw[fill=white,draw=black] (2.9,0.8) rectangle ++(-0.7,0.4);
\draw (1.9,1) -- (2.2,1);
\draw (2.05,1) |- (4,1.5) node[anchor=west,scale=0.7]{$y$};
\draw (2.9,1) -- (4,1) node[anchor=west,scale=0.7]{$y_Q$};
\draw (2.55,1) node[scale=0.7]{D-FF};
\draw (1,1) node[alignment=center,text width=2 cm,scale=0.7] {$16\times1$ LUT: logische functie van 4 variabelen};
\foreach \y/\yt in {-2/4,-1/3,0/2,1/1} {
  \draw (-1,1.2+0.4*\y) node[anchor=east,scale=0.7]{$x_\yt$} -- ++(1.1,0);
}
\end{tikzpicture}
\figlab{fpgaSchemaLB}}
\subfigure[Schakelmatrix]{
\begin{tikzpicture}
\filldraw[fill=black!35,draw=black] (0,0) rectangle (3,3);
\node[nmoso,rotate=-90,scale=0.8] (N1) at (0.75,2.5) {};
\node[nmosc,rotate=-90,scale=0.8] (N2) at (2.25,2.5) {};
\node[nmosc,scale=0.8] (N3) at (1.5,2) {};
\node[nmoso,rotate=-90,scale=0.8] (N4) at (2,1.5) {};
\node[nmoso,rotate=-90,scale=0.8] (N5) at (0.75,0.5) {};
\node[nmoso,rotate=-90,scale=0.8] (N6) at (2.25,0.5) {};
\draw (N2.drain) -- (N2.drain -| 2.75,0) |- (N6.drain);
\draw (N1.drain) -- (N2.source);
\draw (N1.source) -- (N1.source -| 0.25,0) |- (N5.source);
\draw (N5.drain) -- (N6.source);
\draw (N4.source) -- (-0.5,1.5);
\draw (N4.drain) -- (3.5,1.5);
\draw (N3.source) -- (1.5,-0.5);
\draw (N3.drain) -- (1.5,3.5);
\fill (N4.source -| 0.25,0) circle (0.035 cm);
\fill (N4.drain -| 2.75,0) circle (0.035 cm);
\fill (N3.source |- N1.source) circle (0.035 cm);
\fill (N3.drain |- N6.source) circle (0.035 cm);
\end{tikzpicture}
\figlab{fpgaSchemaSchakelMatrix}
}
\caption{Schematische voorstelling van een Field Programmable Gate Array (FPGA).}
\figlab{fpgaSchema}
\end{figure}
schematiseert dit concept. In tegenstelling tot een CPLD heeft een FPGA andere logische bouwblokken, namelijk \termen{Logical Blocks (LB)}. Dit logische blok staat weergegeven op \figref{fpgaSchemaLB}. Het bestaat typisch uit 4 ingangen ($x_1$, $x_2$, $x_3$ en $x_4$) en twee uitgangen ($y$ en $y_Q$). Inwendig bestaat het uit een 16 bit \termen{Look-Up Table (LUT)} en een \termen{Data Flip-Flop (D-FF)}. Deze Look-Up Table is in feite niets anders dan een klein geheugen. Met de 4 ingangen zijn we in staat om 16 verschillende invoer-waarden te genereren. Voor elk van deze waarden programmeren we een bit als uitvoer. Deze wordt uitgevoerd langs de $y$. De Data Flip-Flop houdt deze bit 1 klokcyclus bij, en zal hem de volgende klokcyclus op $y_Q$ zetten. Het gevolg is dus dat $y_Q$ een klokflank na-ijlt op $y$. Tot slot beschouwen we op \figref{fpgaSchemaSchakelMatrix} de implementatie van een \termen{schakelmatrix}. Hierbij verbinden we 4 lijnen met elkaar. Dit levert dus 6 mogelijke verbindingen. Door het programmeren van de chip, kunnen we de transistoren openen of sluiten, wat resulteert in het al dan niet verbinden van twee verbindingen. De transistoren die men hierbij gebruikt zijn de zogenaamde ``\termen{Pass-transistoren}''.
\paragraph{Spartan-3}
Een beroemde FPGA-chip is de \termen{Spartan-3} van \emph{Xilinx}. Deze FPGA maakt gebruik van \termen{Configurable Logic Blocks (CLB)} in plaats van LBs. Een CLB wordt onderverdeeld in 4 ``\termen{slices}''. Twee van deze slices kunnen geconfigureerd worden als schuifregisters (zie \ref{s:schuifregisters}), geheugencellen, of logische blokken, de overige twee alleen als logische blokken. Deze complexere slices bevatten twee verschillende componenten voor logica: 2 functies met 4 variabelen, en 1 functie met 5 variabelen, meestal aangevuld met flipflops. Verder bevatten deze vaak een multiplexer en een schuifregister. De specificaties van deze slices zijn verder ook terug te vinden in \cite[p.~22-23]{xilinxFpgaDs099}. Verder zien we op \figref{fpgaSchemaFull} ook dat er naast logische en in- en uitvoer blokken ook nog andere componenten op een FPGA zitten. Meestal gaat het dan om RAM en een multiplexer. Bovendien bevat de chip ook enkele klokken die het systeem aansturen. Deze zijn als een zwart blokje weergegeven. Typische frequenties bevinden zich rond de 50 MHz. Modernere FPGA's uiten zich in grotere geheugens, soms zelfs met een Stack (LIFO, zie \sscref{stack}), specifieke in- en uitvoer blokken zoals Ethernet, PCIe,... tot zelfs microprocessoren.
\section{Praktische aspecten}
Tot nu toe hebben we ons bezig gehouden met de implementatie van chips. Deze implementatie hebben we afgeleid uit de circuits. Er zijn echter ook aspecten die een invloed hebben op de werking die niet altijd van het circuit kunnen afgelezen worden. Zoals bijvoorbeeld de geometrie van de chip. De volgende subsecties beschrijven met welke praktische aspecten rekening gehouden moeten worden bij het ontwerpen.
\subsection{Spanningsniveaus en ruismarge}
\subsubsection{Ruismarge}
In sectie \ref{s:logischeWaarden} hebben we reeds kort beschreven dat we een bereik specificeren voor een \termen{High} en \termen{Low} signaal.  We definieerden ook een ongedefinieerde zone. Deze zone wordt gebruikt indien het onduidelijk wordt wat er precies aan de ingang staat. Uiteraard willen we vermijden dat we ooit in deze ruismarge terecht komen. Daarom introduceren we een \termen{Ruismarge}. Dit is een spanningsmarge die we beschikbaar stellen door een kleiner gebied van uitgangsspanningen te gebruiken dan toegelaten. We zullen dus de poorten spanningen laten genereren die zich in het hoogste gedeelte van High bevinden, of in het laagste gedeelte van Low. De ruis zal het signaal uiteraard kunnen afzwakken, maar we bijven in de acceptabele zone van High en Low. Het resultaat is dus dat bij de marges van de ingangen ruimer is dan de marges van de uitgangen. \figref{noiseMargin} geeft dit concept schematisch weer. Ook vermeldt de figuur de typische spanningsniveaus bij CMOS en TTL\footnote{TTL: Transistor-Transistor Logic.} implementatie.
\begin{figure}[hbt]
\centering
\begin{tikzpicture}
\fill[black!20] (-0.1,0) rectangle (0,1.25);
\fill[black!20] (-0.1,1.75) rectangle (0,3);
\fill[black!20] (0,0.625) rectangle (0.1,0);
\fill[black!20] (0,2.375) rectangle (0.1,3);
\draw[thick,->] (0,-0.1) -- (0,3.2);
\draw (-0.1,0) node[anchor=east,scale=0.75]{$V_{\mbox{\tiny SS}}$} -- (0.1,0);
\draw (-0.1,0.625) node[anchor=east,scale=0.75]{$V_{\mbox{\tiny OL}}$} -- (0.1,0.625);
\draw (-0.1,1.25) node[anchor=east,scale=0.75]{$V_{\mbox{\tiny IL}}$} -- (0.1,1.25);
\draw (-0.1,1.75) node[anchor=east,scale=0.75]{$V_{\mbox{\tiny IH}}$} -- (0.1,1.75);
\draw (-0.1,2.375) node[anchor=east,scale=0.75]{$V_{\mbox{\tiny OH}}$} -- (0.1,2.375);
\draw (-0.1,3) node[anchor=east,scale=0.75]{$V_{\mbox{\tiny DD}}$} -- (0.1,3);
\draw[<->] (-7,3.5) to node[midway,above,scale=0.75]{Ingang} (0,3.5);
\draw[<->] (7,3.5)  to node[midway,above,scale=0.75]{Uitgang} (0,3.5);

\draw (4,3.5) node[anchor=north,scale=0.75]{CMOS};
\draw (6,3.5) node[anchor=north,scale=0.75]{TTL};
\draw (-4,3.5) node[anchor=north,scale=0.75]{CMOS};
\draw (-6,3.5) node[anchor=north,scale=0.75]{TTL};
\draw (2.25,1.5) node[scale=0.75]{ongedefinieerd};
\draw (-2.25,1.5) node[scale=0.75]{ongedefinieerd};
\draw (2.25,2.6875) node[scale=0.75]{High (H)};
\draw (2.25,0.3125) node[scale=0.75]{Low (L)};
\draw (2.25,0.9375) node[scale=0.75]{ruismarge (NM$_L$)};
\draw (2.25,2.0625) node[scale=0.75]{ruismarge (NM$_L$)};
\draw (-2.25,2.375) node[scale=0.75]{High (H)};
\draw (-2.25,0.625) node[scale=0.75]{Low (L)};

\draw[dotted] (0.1,3) -- ++(6.9,0);
\draw[dotted] (0.1,2.375) -- ++(6.9,0);
\draw[dotted] (0.1,1.75) -- ++(6.9,0);
\draw[dotted] (0.1,1.25) -- ++(6.9,0);
\draw[dotted] (0.1,0.625) -- ++(6.9,0);
\draw[dotted] (0.1,0) -- ++(6.9,0);

\draw[dotted] (-0.8,3) -- ++(-6.2,0);
\draw[dotted] (-0.8,1.75) -- ++(-6.2,0);
\draw[dotted] (-0.8,1.25) -- ++(-6.2,0);
\draw[dotted] (-0.8,0) -- ++(-6.2,0);

\filldraw[fill=white,draw=black] (4,3) node[scale=0.75,fill=white]{5.0 V};
\filldraw[fill=white,draw=black] (4,2.375) node[scale=0.75,fill=white]{4.4 V};
\filldraw[fill=white,draw=black] (4,1.75) node[scale=0.75,fill=white]{3.5 V};
\filldraw[fill=white,draw=black] (4,1.25) node[scale=0.75,fill=white]{1.5 V};
\filldraw[fill=white,draw=black] (4,0.625) node[scale=0.75,fill=white]{0.5 V};
\filldraw[fill=white,draw=black] (4,0) node[scale=0.75,fill=white]{0.0 V};
\filldraw[fill=white,draw=black] (6,3) node[scale=0.75,fill=white]{5.0 V};
\filldraw[fill=white,draw=black] (6,2.375) node[scale=0.75,fill=white]{2.4 V};
\filldraw[fill=white,draw=black] (6,1.75) node[scale=0.75,fill=white]{2.0 V};
\filldraw[fill=white,draw=black] (6,1.25) node[scale=0.75,fill=white]{0.8 V};
\filldraw[fill=white,draw=black] (6,0.625) node[scale=0.75,fill=white]{0.4 V};
\filldraw[fill=white,draw=black] (6,0) node[scale=0.75,fill=white]{0.0 V};

\filldraw[fill=white,draw=black] (-4,3) node[scale=0.75,fill=white]{5.0 V};
\filldraw[fill=white,draw=black] (-4,1.75) node[scale=0.75,fill=white]{3.5 V};
\filldraw[fill=white,draw=black] (-4,1.25) node[scale=0.75,fill=white]{1.5 V};
\filldraw[fill=white,draw=black] (-4,0) node[scale=0.75,fill=white]{0.0 V};
\filldraw[fill=white,draw=black] (-6,3) node[scale=0.75,fill=white]{5.0 V};
\filldraw[fill=white,draw=black] (-6,1.75) node[scale=0.75,fill=white]{2.0 V};
\filldraw[fill=white,draw=black] (-6,1.25) node[scale=0.75,fill=white]{0.8 V};
\filldraw[fill=white,draw=black] (-6,0) node[scale=0.75,fill=white]{0.0 V};
\end{tikzpicture}
\caption{Werking van ruismarge bij CMOS en TTL.}
\figlab{noiseMargin}
\end{figure}
\paragraph{Onlogische Spanningen}
Wanneer een transistor omschakelt, betekent dit niet dat van het ene moment op het andere er een andere spanning op de uitgang komt te staan, Deze overgang is een continue proces. Dit betekent dus dat op een zeker moment aan de uitgang de spanning in het ongedefinieerde gebied komt te liggen. Ideaal zou uiteraard zijn dat bij een continue verandering aan de ingang er toch een discrete verandering aan de uitgang plaatsvindt. Dit is echter fysisch onmogelijk. De ontwerper dient dus rekening te houden met dit overgangsverschijnsel. Dit vormt bovendien een groot probleem omdat op dat moment beide transistoren\footnote{PMOS en NMOS.} halfopen zijn, en er dus stroom door de componenten vloeit. Verder weet men ook niet altijd welke spanning er aan de uitgang zal staan. De \termen{Transferfunctie} is immers afhankelijk van zowel het productieproces als omgevingsfactoren zoals warmte,... \figref{transferFunctionMOS} toont verschillende realistische transferfuncties samen met de ideale transferfunctie.
\begin{figure}[hbt]
\centering
\subfigure[Transferfuncties bij een inverter]{
\begin{tikzpicture}
\def\sxy{0.625};
\fill[gray!50] (0,\sxy*2) rectangle ++(1,\sxy*1.5);
\fill[gray!50] (2,0) rectangle ++(1.5,\sxy);
\draw (0,0) node[anchor=north east,scale=0.66]{$V_{\mbox{\small{SS}}}$};
\foreach\x/\t in {1/IL,2/IH,3.5/DD} {
  \draw (\x,0) node[anchor=north,scale=0.66]{$V_{\mbox{\small{\t}}}$};
  \draw (0,\sxy*\x) node[anchor=east,scale=0.66]{$V_{\mbox{\small{\t}}}$};
}
\draw (1.5,0) node[anchor=north,scale=0.66]{$V_{\mbox{\small{T}}}$};
\draw[thick,->] (-0.1,0) -- (4,0);
\draw[thick,->] (0,-0.1) -- (0,2.5);
\draw[thick,dashed,black!80] (2,2) -- (2.25,2) node[anchor=west,scale=0.75]{Ideaal};
\draw[thick,black!80] (2,1.65) -- (2.25,1.65) node[anchor=west,scale=0.75]{Realistisch};
\draw[thick,dashed,black!80] (0,\sxy*3.4) -| (1.5,0.1*\sxy) -- (3.5,0.1*\sxy);
\draw[thick,black!80,variable=\x,domain=0:3.5,smooth] plot (\x,{2.7*\sxy/(1.0+exp(4*\x-6.5))+0.4*\sxy});
\draw[thick,black!80,variable=\x,domain=0:3.5,smooth] plot (\x,{2.7*\sxy/(1.0+exp(4*\x-6))+0.4*\sxy});
\draw[thick,black!80,variable=\x,domain=0:3.5,smooth] plot (\x,{2.7*\sxy/(1.0+exp(4*\x-5.5))+0.4*\sxy});
\end{tikzpicture}
\figlab{transferFunctionMOS}
}
\subfigure[Schmitt-Trigger transferfunctie]{
\begin{tikzpicture}[circuit logic US]
\def\sxy{0.625};
\fill[gray!50] (0,\sxy*2) rectangle ++(1,\sxy*1.5);
\fill[gray!50] (2,0) rectangle ++(1.5,\sxy);
\draw (0,0) node[anchor=north east,scale=0.66]{$V_{\mbox{\small{SS}}}$};
\foreach\x/\t in {1/IL,2/IH,3.5/DD} {
  \draw (\x,0) node[anchor=north,scale=0.66]{$V_{\mbox{\small{\t}}}$};
  \draw (0,\sxy*\x) node[anchor=east,scale=0.66]{$V_{\mbox{\small{\t}}}$};
}
\draw[thick,->] (-0.1,0) -- (4,0);
\draw[thick,->] (0,-0.1) -- (0,2.5);
\draw[thick,black!80,variable=\x,domain=0:3.5,smooth] plot (\x,{2.7*\sxy/(1.0+exp(4*\x-6.5))+0.4*\sxy});
\draw[thick,black!80,variable=\x,domain=0:3.5,smooth] plot (\x,{2.7*\sxy/(1.0+exp(4*\x-4.5))+0.4*\sxy});
\draw[thick,black!80,->] (1.625-0.05,1.75*\sxy+2.7*\sxy*0.05) -- (1.625,1.75*\sxy);
\draw[thick,black!80,->] (1.125+0.05,1.75*\sxy-2.7*\sxy*0.05) -- (1.125,1.75*\sxy);
\node[not gate,scale=0.75] (N) at (2.5,1.75) {};
\draw (N.input) -- ++(-0.25,0);
\draw (N.output) -- ++(0.25,0);
\begin{scope}[xshift=2.5 cm,yshift=1.75 cm,scale=0.6]
\draw (-0.1875,-0.125) -| (0.0625,0.125) -- (0.1875,0.125);
\draw (-0.0625,-0.125) |- (0.0625,0.125);
\end{scope}
\end{tikzpicture}
\figlab{transferFunctionSchmittTrigger}
}
\caption{Transferfuncties en Schmitt-Trigger ingangen.}
\figlab{transferSchmittTrigger}
\end{figure}
\paragraph{Schmitt-trigger ingangen} Dit effect is vooral nadelig voor bijvoorbeeld in- en uitvoer. Deze signalen veranderen doorgaans traag, en bovendien niet strikt stijgend of dalend. Dit heeft tot gevolg dat indien het ingangssignaal van een 0 naar een 1 moet gaan, het meestal verschillende malen van signaal verandert. Een oplossing hiervoor zijn \termen{Schmitt-trigger ingangen}, dit zijn componenten met een \termen{hysteresis}. Dit betekent dat de overgangsspanning bij een overgang van laag naar hoog, hoger is dan de overgangsspanning van hoog naar laag. Deze transferfunctie staat op \figref{transferFunctionSchmittTrigger}, samen met het symbool voor een Schmitt-Trigger ingang: een NOT poort, met een miniatuur van de inverse transferfunctie.
\subsection{Dynamisch gedrag}
\begin{figure}[hbt]
\centering
\begin{tikzpicture}[circuit logic US,american resistors]
\node[not gate] (N1) at (-1,0) {};
\node[not gate] (N2) at (2,0) {};
\draw (N1.output) -- (N2.input);
\draw (N1.input) -- ++(-0.5,0);
\draw (N2.output) -- ++(0.5,0);
\draw (-4,0) node[anchor=west]{Schakeling};
\draw (-4,-2) node[anchor=west]{Laag$\Rightarrow$Hoog};
\draw (-4,-4) node[anchor=west]{Hoog$\Rightarrow$Laag};
\draw (0,-2) -- (2,-2) to [resistor,label=\small{$R_{IH}$}] (2,-3) -- (0.5,-3) to [capacitor,label=\small{$C$}] (0.5,-2) -- (-1,-2) to [resistor,label=\small{$R_{OH}$}] (-1,-1) node[anchor=south,scale=0.75]{$V_{\mbox{\small{DD}}}$};
\begin{scope}[xshift=4.5 cm, yshift=-3 cm]
\draw[thick,->] (0,-0.1) -- (0,1.75) node[anchor=north west,scale=0.75]{$V$};
\draw[thick,->] (-0.1,0) -- (2,0) node[anchor=south east,scale=0.75]{$t$};
\draw[thick,dashed,black!80] (0,0.1) -- (0.5,0.1) |- (2,1.4);
\draw[thick,black!80] (0,0.1) -- (0.5,0.1);
\draw[variable=\x,domain=0.5:2,smooth,black!80,thick] plot (\x,{1.4-1.3*exp(-2*\x+1)});
\end{scope}
\draw (0.5,-4) -- (2,-4) to [resistor,label=\small{$R_{IL}$}] (2,-5) -- (0.5,-5) to [capacitor,label=\small{$C$}] (0.5,-4);
\draw (0.5,-5) -- (-1,-5) to [resistor,label=\small{$R_{OL}$}] (-1,-4) -- (0.5,-4);
\begin{scope}[xshift=4.5 cm, yshift=-5 cm]
\draw[thick,->] (0,-0.1) -- (0,1.75) node[anchor=north west,scale=0.75]{$V$};
\draw[thick,->] (-0.1,0) -- (2,0) node[anchor=south east,scale=0.75]{$t$};
\draw[thick,dashed,black!80] (0,1.4) -- (0.5,1.4) |- (2,0.1);
\draw[thick,black!80] (0,1.4) -- (0.5,1.4);
\draw[variable=\x,domain=0.5:2,smooth,black!80,thick] plot (\x,{1.3*exp(-2*\x+1)+0.1});
\end{scope}
\end{tikzpicture}
\caption{Dynamisch gedrag bij twee sequenti\"ele NOT poorten.}
\figlab{dynamicBehavior}
\end{figure}
Op simulaties zullen we meestal uitgaan van dynamisch gedrag. Dit betekent dus dat we veronderstellen dat indien we bijvoorbeeld de spanning aan de ingang van een poort naar een andere spanning brengen, de uitgang na enige vertraging van de oude naar de nieuwe spanning omschakelt. Dit zullen we illustreren met een voorbeeld zoals op \figref{dynamicBehavior} waar we twee NOT-poorten na elkaar plaatsen. We kunnen het gedrag van dit circuit nabootsen bij overgangen, door een lijn als een capaciteit te zien, verder zien we een gesloten transistor als een weerstand, open transistoren zijn open lijnen en worden dus niet beschouwd. We zullen eerst uit de beschreven modellen de spanning in functie van de tijd plaatsen.
\paragraph{Hoog naar laag}
Bij een overgang van hoog naar laag aan de ingang, zal de spanning op de kabel tussen de twee poorten stijgen. Zoals we op de grafiek zien, zal dit exponentieel gebeuren. We zullen eerst de doelspanning $V_{\infty}$ bepalen, en vervolgens de \termen{tijdsconstante $\tau$}. De doelspanning kunnen we eenvoudig afleiden uit de stroom, indien de condensator volledig opgeladen is, stroomt alle stroom uitsluitend door de twee weerstanden. Hierdoor kunnen we de stroomsterkte afleiden. De spanning over de condensator kunnen we dan berekenen, omdat deze gelijk is aan de spanning op de tweede weerstand $R_{IH}$:
\begin{equation}
I_{\infty}=\displaystyle\frac{V_{DD}}{R_{IH}+R_{OH}}\Rightarrow V_{\infty}=R_{IH}\cdot I_{\infty}=\displaystyle\frac{R_{IH}\cdot V_{DD}}{R_{IH}+R_{OH}}
\label{eqn:vInfty}
\end{equation}
We leiden vervolgens de tijdsconstante af door de het systeem als een RC-keten te modelleren. In dat geval moeten we de twee weerstanden als parallel beschouwen. De tijdsconstante is dan de vervangweerstand vermenigvuldigd met de capaciteit van de condensator:
\begin{equation}
R_{\mbox{subs.}}=\displaystyle\frac{R_{IH}\cdot R_{OH}}{R_{IH}+R_{OH}}\Rightarrow \tau=R_{\mbox{subs.}}\cdot C=\displaystyle\frac{R_{IH}\cdot R_{OH}\cdot C}{R_{IH}+R_{OH}}
\label{eqn:tauHL}
\end{equation}
Voor het opladen van een condensator geldt volgende exponenti\"ele functie:
\begin{equation}
V\left(t\right)=V_{\infty}\cdot\left(1-e^{-t/\tau}\right)
\end{equation}
\paragraph{Laag naar hoog}
We berekenen de vergelijking van laag naar hoog volledig analoog. Aangezien er geen bron meer op de schakeling staat, is de schakeling een zuivere RC-keten. De condensator zal dus ontladen, de eindspanning is dus $V_{\infty}=0\mbox{ V}$. We berekenen de tijdsconstante dan ook opnieuw met behulp van de vervangweerstand:
\begin{equation}
R_{\mbox{subs.}}=\displaystyle\frac{R_{IL}\cdot R_{OL}}{R_{IL}+R_{OL}}\Rightarrow \tau=R_{\mbox{subs.}}\cdot C=\displaystyle\frac{R_{IL}\cdot R_{OL}\cdot C}{R_{IL}+R_{OL}}
\label{eqn:tauLH}
\end{equation}
Bij een RC-keten geldt voor het ontladen van een condensator vervolgens deze exponenti\"ele functie:
\begin{equation}
V\left(t\right)=V_0\cdot e^{-t/\tau}
\end{equation}
\paragraph{Tijdsconstante minimaliseren}
We zien dus dat bij overgangen van laag naar hoog en hoog naar laag, we te maken hebben met exponentieel tijdsgedrag. Deze constante wordt bepaald door de capaciteit, en door de in- en uitgangsimpedanties. Deze parameters stellen we in met de volgende vier doelen:
\begin{equation}
\left\{\begin{array}{l|ll}
\tau\ssearrow&\tau=R_{\mbox{\small{subs.}}}\cdot C&\mbox{lage tijdsconstante}\\
P_{\mbox{\small{stat.}}}??\ssearrow&P_{\mbox{\small{stat.}}}=V^2/\left(R_I+R_O\right)=V^2/R_{\mbox{\small{tot.}}}&\mbox{minimaal statisch vermogenverbruik}\\
P_{\mbox{\small{dyn.}}}??\ssearrow&P_{\mbox{\small{dyn.}}}=V^2/R_O&\mbox{minimaal dynamisch vermogenverbruik}\\
V_{\infty}\approx V_{DD}&V_{\infty}=R_I\cdot V_{DD}/\left(R_I+R_O\right)&\mbox{uitgangsspanning dicht bij de voedingsspanning}\\
\end{array}\right.
\label{eqn:PVTsummary}
\end{equation}
Hiervoor streven we naar bepaalde waardes voor de verschillende parameters:
\begin{itemize}
 \item \termen{Ingangsimpedantie $R_I$}: We streven naar een zo hoog mogelijke ingangsimpedantie, immers stelt vergelijking (\ref{eqn:vInfty}) dat indien $R_I\gg R_O$, $V_{\infty}\approx V_{DD}$. Bovendien is het statisch vermogenverbruik lager (zie vergelijking~\ref{eqn:PVTsummary}).
 \item \termen{Uitgangsimpedantie $R_O$}: We proberen de uitgangsimpedantie zo laag mogelijk te houden. Dit haalt de tijdsconstante naar beneden, waardoor de chip sneller schakelt (zie vergelijking (\ref{eqn:tauHL}) en (\ref{eqn:tauLH})). Een nadelig bijeffect is een hoger dynamisch vermogenverbruik (zie vergelijking \ref{eqn:PVTsummary}).
 \item \termen{Capaciteit $C$}: De capaciteit probeert men zo laag mogelijk te houden. Immers verhoogt een hoge capaciteit rechtstreeks de vertraging (zie vergelijking (\ref{eqn:PVTsummary})). Hoge capaciteiten zijn kenmerkend voor lange verbindingen, vandaar dat men deze meestal op chips tot een minimum beperkt.
\end{itemize}
Al deze effecten zorgen ervoor dat een CMOS-implementatie betere karakteristieken heeft dan een NMOS-poort: NMOS heeft een groot statisch vermogenverbruik. Om dit tegen te gaan, kunnen we een sterke weerstand $R$ in de schakeling plaatsen. Dit leidt echter tot een grote uitgangsimpedantie vermits $R\approx R_O$, hierdoor wordt de stijgtijd van de schakeling groter, en dus bijgevolg de algemene vertraging.??%TODO: recheck
\paragraph{Stijg- en daaltijd}
\begin{figure}[hbt]
\centering
\begin{tikzpicture}
\def \yt{2.5};
\def \yts{2.325};
\def \yb{-3.5};
\def \ybs{-3.325};
\begin{scope}[yscale=2]
\draw[dotted,thick] (0,0.9) node[anchor=east]{$90\%$} -- ++(10,0) node[anchor=west]{$V_{IH}$};
\draw[dotted,thick] (0,0.5) node[anchor=east]{$50\%$} -- ++(10,0) node[anchor=west]{$V_T$};
\draw[dotted,thick] (0,0.1) node[anchor=east]{$10\%$} -- ++(10,0) node[anchor=west]{$V_{IL}$};
\draw plot[thick,samples=7,black!80,variable=\x,domain=0:1,smooth] (\x,{0.05+0.025*rand-0.0125}) -- plot[thick,samples=14,black!80,variable=\x,domain=1:3,smooth] (\x,{0.05+0.45*\x-0.45+0.025*rand-0.0125}) -- plot[thick,samples=21,black!80,variable=\x,domain=3:6,smooth] (\x,{0.95+0.025*rand-0.0125}) -- plot[thick,samples=7,black!80,variable=\x,domain=6:7,smooth] (\x,{0.95-0.9*\x+5.4+0.025*rand-0.0125}) -- plot[thick,samples=21,black!80,variable=\x,domain=7:10,smooth] (\x,{0.05+0.025*rand-0.0125});
\coordinate (IA) at (1.1111,0.1);
\coordinate (IMA) at (2,0.5);
\coordinate (IB) at (2.8889,0.9);
\coordinate (IC) at (6.0556,0.9);
\coordinate (IMB) at (6.5,0.5);
\coordinate (ID) at (6.9444,0.1);
\end{scope}
\draw (IA) -- (IA |- 0,\yt);
\draw (IB) -- (IB |- 0,\yt);
\draw (IC) -- (IC |- 0,\yt);
\draw (ID) -- (ID |- 0,\yt);
\draw[<->] (IA |- 0,\yts) to node[midway,above]{Stijgtijd $t_r$} (IB |- 0,\yts);
\draw[<->] (IC |- 0,\yts) to node[midway,above]{Daaltijd $t_f$} (ID |- 0,\yts);
\begin{scope}[yscale=2,yshift=-1.5cm]
\draw[dotted,thick] (0,0.9) node[anchor=east]{$90\%$} -- ++(10,0) node[anchor=west]{$V_{OH}$};
\draw[dotted,thick] (0,0.5) node[anchor=east]{$50\%$} -- ++(10,0) node[anchor=west]{$V_T$};
\draw[dotted,thick] (0,0.1) node[anchor=east]{$10\%$} -- ++(10,0) node[anchor=west]{$V_{OL}$};
\draw plot[thick,samples=7,black!80,variable=\x,domain=0:1,smooth] (\x,{0.95+0.025*rand-0.0125}) -- plot[thick,samples=14,black!80,variable=\x,domain=1.5:3.5,smooth] (\x,{0.95-0.45*\x+1.5*0.45+0.025*rand-0.0125}) -- plot[thick,samples=21,black!80,variable=\x,domain=3.5:6.35,smooth] (\x,{0.05+0.025*rand-0.0125}) -- plot[thick,samples=14,black!80,variable=\x,domain=6.35:8.35,smooth] (\x,{0.05+0.45*\x-0.45*6.35+0.025*rand-0.0125}) -- plot[thick,samples=11,black!80,variable=\x,domain=8.35:10,smooth] (\x,{0.95+0.025*rand-0.0125});
\coordinate (OMA) at (2.5,0.5);
\coordinate (OMB) at (7.35,0.5);
\end{scope}
\draw (IMA) -- (IMA |- 0,\yb);
\draw (OMA) -- (OMA |- 0,\yb);
\draw (IMB) -- (IMB |- 0,\yb);
\draw (OMB) -- (OMB |- 0,\yb);
\draw[<->] (IMA |- 0,\ybs) to node[midway,below]{$t_{pHL}$} (OMA |- 0,\ybs);
\draw[<->] (IMB |- 0,\ybs) to node[midway,below]{$t_{pLH}$} (OMB |- 0,\ybs);
\end{tikzpicture}
\caption{Het dynamisch gedrag van een NOT-poort.}
\figlab{dynamicBehaviorNotGate}
\end{figure}
We hebben het reeds uitvoerig gehad over het dynamisch karakter van een elektronische schakeling. Naast het feit dat het enige tijd duurt alvorens een spanning die op een draad wordt aangelegd daadwerkelijk dit spanningsniveau bereikt, heeft ook een poort een dynamisch karakter. Een interessante eigenschap is dat de stijg- en daaltijd invloed hebben op de vertraging van de poort. We zullen deze drie grootheden nu formaliseren:
\begin{itemize}
 \item \termen{Stijgtijd $t_r$} ofwel \termen{rise time}: de tijd die de spanning nodig heeft om van een lage spanning naar een hoge spanning te stijgen. Men neemt hiervoor de 10\% en 90\% grenzen tussen de basisspanning en de topspanning\footnote{In sommige publicaties wordt 80\% en 20\% gebruikt.}.
 \item \termen{Daaltijd $t_f$} ofwel \termen{fall time}: de tijd die de spanning nodig heeft om van een hoge spanning naar een lage spanning te dalen. Men neemt hiervoor de 90\% en 10\% grenzen tussen de basisspanning en de topspanning.
 \item \termen{Vertragingstijd $t_p$} ofwel \termen{propagation delay}: De tijd die de poort nodig heeft tussen de ingangspanning die het midden bereikt tussen de basisspanning en de topspanning, en de uitgang die deze 50\% grens bereikt. Vermits de stijg- en daaltijd ook bepalen hoe snel een signaal naar 50\% stijgt of daalt, hebben deze parameters ook invloed op de vertragingstijd. We berekenen de vertraging door het gemiddelde te nemen tussen de vertraging tijdens het dalen $t_{pHL}$ en het stijgen $t_{pLH}$:
\begin{equation}
\mbox{Vertragingstijd $t_p$}=\displaystyle\frac{t_{pHL}+t_{pLH}}{2}
\end{equation}
\end{itemize}
Dit principe wordt ge\"illustreerd op \figref{dynamicBehaviorNotGate}. We zien hier twee situaties: in de eerste situatie is de stijgtijd van de ingang gelijk aan de daaltijd van de uitgang. De poort heeft weliswaar een vertraging, maar deze is vrij beperkt. In het tweede geval is de stijgtijd van de uitgang dubbel zolang als de daaltijd van de ingang. Hoewel de poort sneller reageert op het veranderende signaal, is de vertraging groter. Vermits de stijg- en daaltijden be\"invloed worden door de parasitaire capaciteit\footnote{De capaciteit die wordt veroorzaakt door de draad zelf.}, wordt dus ook de vertraging be\"invloed door de aard van de draden. De vertraging bij korte draden is aanzienlijk kleiner dan deze van lange draden.
%??%TODO
%De ingangsimpedantie is hierbij omgekeerd evenredig met de \termen{Stijgtijd}. De uitgangsimpedantie is evenredig met de \termen{Daaltijd}. Deze 
\subsection{Vermogenverbruik}
Een van de belangrijkste ontwerpbeperkingen bij grote schakelingen is het \termen{vermogenverbruik}. Het vermogenverbruik is dan ook een parameter die zich negatief uit op twee verschillende domeinen:
\begin{itemize}
 \item Levering: De verbruikte energie heeft een kostprijs, bovendien uit het vermogenverbruik zich ook in een kortere levensduur van eventuele batterijen.
 \item Gedissipeerd: het vermogenverbruik leidt tot warmteontwikkeling. Deze warmte moet afgevoerd worden, wat ook vermogenverbruik met zich meebrengt.
\end{itemize}
We beschouwen twee vormen van vermogenverbruik:
\begin{itemize}
 \item \termen{Statisch vermogenverbruik}: Dit is het vermogen die continu verbruikt wordt, ook indien de schakeling niets doet. We zagen reeds dat dit bij NMOS implementaties het geval is, omdat we bij het sluiten van een NMOS-transistor, een lekstroom genereren van de source naar de drain. Bij NMOS-poorten mogen we dan ook uitgaan van een vermogenverbruik van $1\mbox{ mW}$. Dit betekent dus voor een gemiddelde chip dat dit makkelijk oploopt in tot $1\mbox{ kW}$ wat totaal onaanvaardbaar is. Bij CMOS hebben we eenvoudigweg geen lekstroom en dus ook geen statisch vermogenverbruik.
 \item \termen{Dynamisch vermogenverbruik}: Dit is het vermogen die verbruikt wordt op het moment dat een poort omschakelt. Hierdoor wordt de parasitaire capaciteit $C$ op- of ontladen. In \cite[3.12]{brown2004fundamentals} wordt hiervoor volgende formule gegeven:
\begin{equation}
P_{\mbox{\small{dyn.}}}=C\times f\times V^2
\end{equation}
Met chipoppervlakte $C$, een klokfrequentie $f$ en voedingsspanning $V$. We kunnen deze formule als volgt verklaren: Vermits het chipoppervlakte hoofdzakelijk draden bevat is dit een goede schatting voor de totale lengte van de draden, verder bepaalt de klokfrequentie hoeveel maal we per seconde deze draden moeten op- en ontladen. Het vermogen ten slotte om een capaciteit $C$ op te laden is evenredig met het kwadraat van de spanning. In de praktijk werken de meeste schakelingen met een klokfrequentie in de grootorde van $100\mbox{ MHz}$. Verder schakelen uiteraard niet alle draden telkens om, een realistische schatting ligt bij de 20\%. Indien we ook de oppervlakte van een poort in rekening brengen komen we uit op een ordegrootte van $35\mbox{ nW}$ per inverter. We kunnen dus in de grootorde van $300\ 000\ 000$ invertoren plaatsen voor een vermogenverbruik van $1\mbox{ W}$. In de praktijk is vermogenverbruik dan ook een van de belangrijkste beperkingen om rekening mee te houden bij het ontwerpen van digitale schakelingen, vandaar dat er in deze cursus ook een grote nadruk ligt op het minimaliseren van digitale schakelingen.
??%TODO%In de praktijk komt dit ongeveer overeen met $35\mbox{ nW}$ per inverter in een poort. Hierbij maken we een assumptie dat per klokcyclus, 20\% van de poorten omschakelt, en we een klokfrequentie rond de $100\mbox{ MHz}$ gebruiken.
\end{itemize}
\subsection{``1'' en ``0'' doorgeven}
Een oplettend lezer zal zich misschien de vraag gesteld hebben, waarom we een buffer zoals op \figref{bufferCmos} op pagina \pageref{fig:bufferCmos} niet implementeren door de NMOS en PMOS transistor om te wisselen. Dit zou ons immers twee transistoren besparen, en bovendien zou de poort effici\"enter werken, zoals op \figref{badBufferCmos}. Deze methode kunnen we echter niet consistent hanteren. Voor de verklaring moeten we terug naar een belangrijk detail bij de werking van transistoren in sectie \ref{ss:nmosPmosWork}. Een NMOS transistor geleidt immers alleen maar indien de spanning tussen de basis en de collector $V_{GS}$, groter is dan de transferspanning $V_{T}$. Indien de NMOS transistor dus gesloten is, zal de uitgangsspanning $V_{T}$ lager zijn dan de ingangsspanning. Dit geeft misschien bij \'e\'en transistor geen noemenswaardige problemen, maar bij een sequentie van transistoren, zal de spanning uiteindelijk onder het grensniveau vallen. Samenvattend kunnen we zeggen dat NMOS transistoren hoge spanningen niet goed doorgeven, en dus een slechte pull-up zijn. Analoog voor PMOS geldt dezelfde redenering, maar dan toegepast op lage spanningen. Een PMOS transistor is dus een slechte pull-down. Deze spanningsverschillen zijn ook gevisualiseerd op \figref{badBufferCmos}.
\begin{figure}[hbt]
\centering
\subfigure[Buffer met inverterende poorten]{\begin{tikzpicture}[circuit logic US]
\coordinate (F0) at (-3,0);
\coordinate (I) at (-2,0);
\draw (F0) node[anchor=east,scale=0.75]{$x$} -- (I);
\node [pmoso] (P1) at (-1,0.75) {};
\draw (I) |- (P1.gate);
\draw[->] (P1.source) -- ++(0,0.5) node[anchor=west,scale=0.75]{$V_{\mbox{\tiny DD}}$};
\node [nmosc] (N1) at (-1,-0.75) {};
\draw (I) |- (N1.gate);
\draw (N1.source) node [ground] {};
\draw (N1.drain) -- (P1.drain);
\node [pmosc] (P2) at (1,0.75) {};
\coordinate (F1) at (-1,0);
\coordinate (O) at (0,0);
\draw (F1) -- (O);
\draw[->] (P2.source) -- ++(0,0.5) node[anchor=west,scale=0.75]{$V_{\mbox{\tiny DD}}$};
\draw (O) |- (P2.gate);
\node [nmoso] (N2) at (1,-0.75) {};
\draw (O) |- (N2.gate);
\draw (N2.source) node [ground] {};
\draw (N2.drain) -- (P2.drain);
\coordinate (F2) at (1,0);
\draw (F2) -- ++(1,0) node[anchor=west,scale=0.75]{$f$};
\end{tikzpicture}
\figlab{goodBufferCmos}
}
\subfigure[Fout buffer]{\begin{tikzpicture}[circuit logic US]
\coordinate (F0) at (-3,0);
\coordinate (I) at (-2,0);
\draw (F0) node[anchor=east,scale=0.75]{$x$} -- (I);
\node [nmosc] (N1) at (-1,0.75) {};
\draw (I) |- (N1.gate);
\draw[->] (N1.drain) -- ++(0,0.5) node[anchor=west,scale=0.75]{$V_{\mbox{\tiny DD}}$};
\node [pmoso] (P1) at (-1,-0.75) {};
\draw (I) |- (P1.gate);
\draw (P1.drain) node [ground] {};
\draw (P1.source) -- (N1.source);
\coordinate (F1) at (-1,0);
\coordinate (O) at (0,0);
\draw[->,black!80,dashed] (-1.625,0 |- N1.gate) arc (180:270:0.625);
\draw[black!80] (-0.625*0.607-1,0.75-0.625*0.607) node[scale=0.75,anchor=north east]{$V_T$};
\draw (F1) -- (O) node[anchor=west,scale=0.75]{$f$};
\end{tikzpicture}
\figlab{badBufferCmos}
}
\caption{Buffer ge\"implementeerd met omgekeerde transistoren: NMOS is een slechte pull-up.??}%TODO: betere naam zoeken.
\figlab{badNmosPmos}
\end{figure}
\subsection{Fan-in en fan-out}
Enkele belangrijke eigenschappen van een poort zijn de \termen{fan-in} en \termen{fan-out}. De fan-in is het aantal ingangen die de poort in kwestie heeft. Deze eigenschap is eigen aan het type poort\footnote{Bijvoorbeeld een 3-nand heeft een fan-in van 3.}. De fan-out is het aantal ingangen van andere poorten die de poort kan aansturen, het aantal draden die uit de poort in kwestie komt zeg maar. Deze eigenschappen bepalen in grote mate de vertraging van een circuit. Indien we immers een groot aantal ingangen hebben, kan het een tijdje duren vooraleer de stroom doorheen de transistoren van de andere ingangen stroomt, als \'e\'en van de transistoren omschakelt. Verder zullen we ook de spanning moeten opdrijven bij een groot aantal transistoren die we in serie schakelen. Elke transistor heeft immers indien gesloten nog steeds een kleine weerstand. Indien de stroom door een serie transistoren moet vloeien, betekent dit dat de uitgangsspanning van de poort teveel gereduceerd zou zijn. Een hogere spanning leidt weer tot hogere vermogens wat we juist willen tegengaan. Bijgevolg willen we de fan-in zo laag mogelijk houden. Daarom komen in de realiteit poorten met een groot\footnote{Met groot bedoelen we meer dan 5 ingangen.} aantal ingangen nooit voor, indien men een poort met een groot aantal ingangen wil realiseren zal men deze meestal met een sequentie van eenvoudige poorten realiseren. Ook de fan-out moeten we onder controle houden. Elke poort heeft immers maar \'e\'en uitgang. Door deze uitgang moet alle stroom naar de ingangen van andere poorten stromen. Vermits de stroom over een draad beperkt is, kunnen we spreken over een \termen{maximale stroomsterkte $I_{\mbox{\small{Omax}}}$}. Deze stroomsterkte bepaalt dan weer hoe snel we de parasitaire capaciteit kunnen op- en ontladen:
\begin{equation}
I_{\mbox{\small{Omax}}}=C\cdot\displaystyle\frac{\partial V}{\partial t}
\end{equation}
Vermits de snelheid waarmee deze capaciteit op- en ontlaadt op zijn beurt weer invloed heeft op de vertraging, heeft het ook invloed op de \termen{maximale frequentie $f_{\mbox{\small{max}}}$} van de elektronische schakeling:
\begin{equation}
f_{\mbox{\small{max}}}=\displaystyle\frac{1}{\Delta V}\cdot\displaystyle\frac{\partial V}{\partial t}=\displaystyle\frac{I_{\mbox{\small{Omax}}}}{C\cdot \Delta V}
\end{equation}
Verder zal ook de parasitaire capaciteit zelf toenemen: de uitgang die met alle ingangen verbonden is vormt \'e\'en draad. Vermits de grootte van deze draad zal afhangen van de fan-out zal dus ook de parasitaire capaciteit toenemen. We kunnen de fan-out beperken door gebruik te maken van buffers: we verdelen de uitgang van de poort onder een paar buffers die op hun beurt de ingangen van de andere poorten bevoorraden. Het nadeel van het gebruik van een buffer is dat dit buffer ook moet omschakelen, wat extra vermogenverbruik en vertragingen teweeg brengt.
\subsection{Tri-state buffer}
Een uitbreiding van een buffer is een \termen{Tri-state buffer} ofwel \termen{3-state buffer}. \figref{triStateSymbol} toont het symbool die men voor deze component gebruikt: een buffer met een derde ingang aan de zijkant van de driehoek. Een tri-state buffer heeft drie mogelijke uitgangswaarden: 0, 1 en de zogenaamde \termen{zwevende modus $Z$}. Deze laatste toestand wordt ook wel \termen{hoog impedant} genoemd. Het betekent dat de lijn losgekoppeld is van enige bron. Er staat dus zogezegd niets op de lijn. Hiertoe wordt een buffer uitgebreid met een tweede ingang: \termen{Enable $E$}. Hierdoor kan de component in drie toestanden worden gebracht: wanneer enable op 0 staat, staat de tri-state buffer in de $Z$ stand. In het geval er een hoog signaal op de enable-ingang wordt aangelegd, laat het de ingang door, deze kan uiteraard in twee standen staan. Op \figref{triStateTruth} wordt dit concept met behulp van een waarheidstabel weergegeven. Er bestaan verschillende manieren om dit component te implementeren. Een goedkope manier wordt voorgesteld op \figref{triStateImplementation} en maakt gebruik van een \termen{transmission gate}, een component die functioneert als een schakelaar.
\begin{figure}[hbt]
\centering
\subfigure[Symbool]{
\begin{tikzpicture}
\node[tris] (TS) at (0,0) {};
\draw (TS.c) -- ++(0,-0.5) node[anchor=north]{enable $e$};
\draw (TS.x) -- ++(-0.5,0) node[anchor=east]{input $x$};
\draw (TS.z) -- ++(0.5,0) node[anchor=west]{output $f$};
\end{tikzpicture}
\figlab{triStateSymbol}}
\subfigure[Tabel]{
\begin{tikzpicture}
\node (A) at (0,0) {\begin{tabular}{cc|c}
$e$&$x$&$f$\\\hline
$0$&$0$&$Z$\\
$0$&$1$&$Z$\\
$1$&$0$&$0$\\
$1$&$1$&$1$
\end{tabular}};
\end{tikzpicture}
\figlab{triStateTruth}}
\subfigure[Implementatie]{
\begin{tikzpicture}[circuit logic US]
\node[transgate] (TG) at (0,0) {};
\node[not gate,scale=0.8] (N1) at (-1,0) {};
\node[not gate,scale=0.8] (N2) at (-2,0) {};
\node[not gate,scale=0.8] (N3) at (-1,1) {};
\draw (N1.output) -- (TG.x);
\draw (N2.output) -- (N1.input);
\draw (N2.input) -- (-3,0 |- N2.input) node[anchor=east]{$x$};
\draw (N3.input) -- (-3,0 |- N3.input) node[anchor=east]{$e$};
\draw (N3.output) -| (TG.cn);
\draw (TG.z) -- ++(0.5,0) node[anchor=west]{$f$};
\pdot{-2.6,1};
\draw (-2.6,1) |- (0,-1) -- (TG.c);
\end{tikzpicture}
\figlab{triStateImplementation}}
\subfigure[Transmission Gate]{
\begin{tikzpicture}
\node[transgate] (TG) at (0,0) {};
\draw (TG.x) -- ++(-0.5,0) node[anchor=east]{$x$};
\draw (TG.z) -- ++(0.5,0) node[anchor=west]{$f$};
\draw (TG.cn) -- ++(0,0.5) node[anchor=south]{$\bar{s}$};
\draw (TG.c) -- ++(0,-0.5) node[anchor=north]{$s$};
\draw (1.75,0) node{$\equiv$};
\begin{scope}[xshift=4 cm]
\node[pmosc,rotate=-90] (P) at (0,0.5) {};
\node[nmosc,rotate=90] (N) at (0,-0.5) {};
\coordinate (X) at (-0.75,0);
\coordinate (F) at (0.75,0);
\draw (P.source) -| (F) |- (N.source);
\draw (P.drain) -| (X) |- (N.drain);
\draw (X) -- ++(-0.5,0) node[anchor=east]{$x$};
\draw (F) -- ++(0.5,0) node[anchor=west]{$f$};
\draw (P.gate) -- ++(0,0.5) node[anchor=south]{$\bar{s}$};
\draw (N.gate) -- ++(0,-0.5) node[anchor=north]{$s$};
\pdot{X};\pdot{F};
\end{scope}
\end{tikzpicture}
\figlab{triStateTransmission}}
\subfigure[Bus]{
\begin{tikzpicture}[circuit logic US]
\node[tris] (T1) at (-0.75,0.5) {};
\node[tris] (T2) at (-0.75,-0.5) {};
\node[buffer gate] (B1) at (0.75,1) {};
\node[buffer gate] (B2) at (0.75,0) {};
\node[buffer gate] (B3) at (0.75,-1) {};
\draw (B1.input) -| (0,0) |- (B3.input);
\draw (0,0) -- (B2.input);\draw (0,0.5) -- (T1.z);\draw (0,-0.5) -- (T2.z);
\draw (T1.x) -- (T1.x -| -1.5,0) node[anchor=east]{$x_1$};
\draw (T1.c) |- (-1.25,0) node[anchor=east]{$e_1$};
\draw (T2.x) -- (T2.x -| -1.5,0) node[anchor=east]{$x_2$};
\draw (T2.c) |- (-1.25,-1) node[anchor=east]{$e_2$};
\draw (B1.output) -- ++(0.5,0) node[anchor=west]{$f_1$};
\draw (B2.output) -- ++(0.5,0) node[anchor=west]{$f_2$};
\draw (B3.output) -- ++(0.5,0) node[anchor=west]{$f_3$};
\pdot{0,0};\pdot{0,-0.5};\pdot{0,0.5};
\end{tikzpicture}
\figlab{triStateBus}}
\caption{Tri-state buffer.}
\end{figure}
De transmission gate zelf bestaat zoals te zien op \figref{triStateTransmission} uit twee transistoren. Indien $s=0$ zijn beide schakelaars open, en staat er dus geen signaal op uitgang $f$, in dat geval is de uitgang dus hoog impedant. Bij $s=1$ zijn beide transistoren gesloten, en wordt het signaal die aangelegd wordt op $x$ verder gepropageerd. De schakeling op \ref{fig:triStateImplementation} bevat extra componenten om het binnenkomende signaal opnieuw voldoende sterk te maken, dezelfde argumenten als met fan-in en fan-out gelden hier immers.
\paragraph{Bus}Tri-state buffers worden vaak gebruikt om bijvoorbeeld over een lijn gegevens van verschillende bronnen over te brengen. We hebben reeds in \ref{term:kortsluiting} gezien dat het verbinden van uitgangen tot kortsluiting leidt. Hierdoor zouden we echter genoodzaakt zijn om voor elke uitgang een aparte draad te voorzien. Verbindingen kunnen een groot gedeelte van het chipoppervlak beslaan, vandaar dat we het aantal draden ook tot een minimum proberen te beperken. We kunnen dit probleem oplossen met een zogenaamde \termen{bus}. Bij een bus wordt een draad gedeeld tussen verschillende bronnen $x_i$. Elk van die bronnen stuurt een tri-state buffer aan. Verder zorgt men er voor dat er slechts \'e\'en tri-state buffer tegelijk actief is. Omdat de andere tri-state buffers op dat moment hoog-impedant zijn, is er geen gevaar voor kortsluiting. \figref{triStateBus} geeft een implementatie van dit principe. In subsectie \ref{ss:multiplexer} zullen we verder een component tegenkomen die dit principe met een algemenere techniek realiseert: de multiplexer. In dat geval moeten de andere bronnen een $Z$ op de lijn zetten.
% \section{CAD-ontwerp in de praktijk}
% \subsection{Ingave}
% \subsection{Synthese}
% \subsection{Fysisch}
% \subsection{Chip}
%TODO: cad
\part{Combinatorische en Sequenti\"ele Schakelingen}
\chapter{Combinatorische Schakelingen (Schakelingen zonder geheugen)}
\label{ch:combinatoric}
\chplab{combinatoric}
\chapterquote{Zelfs al zou er niets nieuws geschapen worden, dan is er nog altijd een nieuwe combinatie.}{Henry Ford, Amerikaans automobielfabrikant (1863-1947)}
\begin{chapterintro}
Nu we de basis van het bouwen van digitale circuits onder de knie hebben, en de fysische beperkingen hiervan kennen, wordt het tijd om ook schakelingen te ontwikkelen. \termen{Combinatorische Schakelingen} zijn schakelingen waarbij een bitvector aan uitgangen uitsluitend bepaald wordt aan de hand van een bitvector aan ingangen. We kunnen dus stellen dat het circuit een functie $\vec{f}(\vec{x})$ berekent. In sectie \ref{s:synthese} werd reeds een manier voorgesteld om tot een canonieke vorm te komen. In sectie \ref{s:minimalisatie} zullen we technieken zien om deze implementaties verder te minimaliseren. Verder zullen we in secties \ref{s:rekenkundig} en \ref{s:andereBasis} de implementatie van enkele populaire combinatorische schakelingen zien.% Tot slot zullen we ook combinatorische schakelingen in VHDL bouwen in sectie \ref{s:combinatorischVHDL}.
\end{chapterintro}
\minitoc[n]
\section{Minimaliseren van logische functies}
\label{s:minimalisatie}
\subsection{Waarom minimaliseren}
We willen schakelingen ontwerpen voor de worteltrekking en 7 segment display in een minimale implementatie. Streven naar een minimalisatie is een algemeen probleem en is vergelijkbaar met optimalisatie in de informatica. Minimalisatie levert niet enkel snellere doorvoer op. Hieronder sommen we de meest courante voordelen op, samen met hun metriek:
\begin{itemize}
\item Minimaliseren van de \termen{kostprijs}. Afhankelijk van de realisatie hanteren we hiervoor 2 metrieken: Voor CMOS verfijnen we hiervoor de oorspronkelijke formule van de kostprijs uit vergelijking \ref{eqn:kosten}. En bekomen:
\begin{equation}
\mbox{kostprijs}=\displaystyle\sum_{\tiny\begin{array}{c}\mbox{alle}\\\mbox{poorten}\end{array}}{\mbox{kostprijs}\left(\mbox{poort}\right)}
\label{eqn:kostenCmos}
\end{equation}
De kostprijs van een poort wordt dan bepaald met volgende formule:
\begin{equation}
\mbox{kostprijs}\left(\mbox{poort}\right)=\left\{\begin{array}{lcl}
\mbox{fan-in}&\mbox{if}&\mbox{poort}\in\left\{\mbox{INV},\mbox{NAND},\mbox{NOR},\mbox{AOI},\mbox{OAI}\right\}\\
\mbox{fan-in}+1&\mbox{if}&\mbox{poort}\in\left\{\mbox{AND},\mbox{OR}\right\}\\
\end{array}\right.
\label{eqn:kostenCmosPoort}
\end{equation}
We zien dus dat alle \termen{inverterende poorten} relatief 1 goedkoper zijn dat de \termen{niet-inverterende poorten}. Bij een FPGA bepalen we de kostprijs aan de hand van het aantal logische cellen:
\begin{equation}
\mbox{kostprijs}=\#\mbox{LB's}
\end{equation}
Uiteraard dient hierbij de kanttekening gemaakt te worden, dat niet voor elk aantal logische cellen, er een FPGA bestaat. Indien er voldoende logische cellen op de FPGA aanwezig zijn, is de kostprijs dan ook van minder belang. We zullen immers toch dezelfde FPGA gebruiken.
\item Snelheid: We maximaliseren de snelheid door de maximale vertraging te minimaliseren. Deze vertraging is afhankelijk van de poorten in het \termen{Kritische Pad}. Het kritische pad is een pad van de ingang naar de uitgang met de grootste vertraging. Deze vertraging is afhankelijk van twee parameters:
\begin{itemize}
\item Vertraging van de poorten: een poort heeft tijd nodig om bij verandering van de ingang ook de uitgang te veranderen. Deze vertraging is afhankelijk van de technologie en evenredig met de fan-in. We zullen deze benaderen met volgende formule:
\begin{equation}
\mbox{vertraging}\left(\mbox{poort}\right)=\left\{\begin{array}{lcl}
0.6+0.4\cdot\mbox{fan-in}&\mbox{if}&\mbox{poort}\in\left\{\mbox{INV},\mbox{NAND},\mbox{NOR},\mbox{AOI},\mbox{OAI}\right\}\\
1.6+0.4\cdot\mbox{fan-in}&\mbox{if}&\mbox{poort}\in\left\{\mbox{AND},\mbox{OR}\right\}\\
\end{array}\right.
\label{eqn:speedPoort}
\end{equation}
Deze formule toont dus dat opnieuw de fan-in een belangrijke factor is, en dat niet-inverterende poorten opnieuw een nadeel hebben tegenover inverterende poorten.
\item (capacitieve) belasting: dit hangt hoofdzakelijk af van de geometrische implementatie van de printplaat. Deze vertraging is dan ook zeer moeilijk te berekenen en wordt niet beschouwd.
\end{itemize}
Bijgevolg berekenen we de vertraging als volgt:
\begin{equation}
\mbox{vertraging}=\displaystyle\sum_{\tiny\begin{array}{c}\mbox{kritisch}\\\mbox{pad}\end{array}}{\mbox{vertraging}\left(\mbox{poort}\right)}
\label{eqn:speed}
\end{equation}
\end{itemize}
\paragraph{Hoe minimaliseren?}Om te minimaliseren hebben we een methode nodig, deze methode manipuleert de logische uitdrukking van $\vec{f}(\vec{x})$ tot een uitdrukking die equivalent maar voordeliger is, rekening houdend met de metrieken van vergelijkingen \ref{eqn:kostenCmos} en \ref{eqn:speed}. Het probleem is dat er geen methodes bestaan die ons een reeks manipulaties voorstellen, waardoor we altijd tot de meest optimale implementatie komen. We moeten dus bijgevolg methodes bedenken waardoor we in staat zijn tot een redelijk optimale oplossingen te komen. Hiervoor zullen we methodes gebruiken zoals Karnaugh-kaarten. Computers bieden bovendien de mogelijkheid een groot aantal implementaties te simuleren. Bij een groot probleem volstaat deze rekenkracht echter ook niet om tot de beste implementatie te komen\footnote{Wat is \"uberhaupt de beste oplossing, in veel gevallen zal de ene metriek verbeteren ten koste van de tweede.}.
\subsection{Karnaugh-kaarten}
\termen{Karnaugh-kaarten} of \termen{K-kaarten} proberen het aantal nutteloze ingangen tot een minimum te beperken. Het is een visueel hulpmiddel dat gebruik maakt van een vermogen waar mensen goed in zijn: het herkennen en begrijpen van patronen. De basis van een Karnaugh-kaart is dan ook de waarheidstabel. In een waarheidstabel kunnen we vaak door bepaalde rijen te beschouwen verbanden zien.
%Zo zien we in de waarheidtabel op figuur \ref{fig:sevenSegmentDisplay} dat $e=0$ als $w=1$. Uiteraard blijft er onduidelijkheid waarom $e=0$ als $\left(w,x,y,z\right)=\left(0,1,0,0\right)$. Toch kunnen we met dergelijke patronen de expressie al behoorlijk optimaliseren. Zo kunnen we voor $e$ al een expressie maken van de vorm $e=(\mbox{NOT }w)\mbox{ AND }\varphi$ met $\varphi$ een nog onbekende logische expressie. We weten immers dat indien $w=1$ er een 0 aan de ingang van de AND poort verschijnt. Een AND met aan minstens \'e\'en ingang een 0 is per definitie 0. We hebben dus het probleem kunnen reduceren met het zoeken naar een specifiek patroon.
\paragraph{N-kubus}In feite zijn Karnaugh-kaarten niets anders dan waarheidstabellen waarbij we het aantal dimensies verhogen. Een zogenoemde \termen{$N$-kubus}. We stellen immers een bepaalde ingangstoestand voor als een knooppunt op een kubus. Figuur \ref{fig:nCube} toont een N-kubus voor de dimensies 1 tot 4.
\begin{figure}[htb]
\centering
\subfigure[1D]{\importtikz{kubus1d}}
\subfigure[2D]{\importtikz{kubus2d}}
\subfigure[3D]{\importtikz{kubus3d}}
\subfigure[4D]{\importtikz{kubus4d}}
\caption{N-kubus voor dimensies 1 tot 4.}
\figlab{nCube}
\end{figure}
Hierbij is elke knoop van de kubus een bitvector van waarden aan de ingang. We kunnen dan vervolgens op deze knooppunten de waarde die we aan de uitgang verwachten plaatsen. De $N$-kubus toont ook de buren van deze toestanden. Dit zijn toestanden waarbij exact \'e\'en bit aan de ingang veranderd is. Buren zijn cruciaal, indien de uitgang niet verandert tussen twee buren, kunnen we zeggen dat de veranderde ingangsbit irrelevant is voor de uitgang.
\paragraph{Karnaugh-Kaarten}Karnaugh-kaarten zijn in feite niets anders dan 2D projecties van de $N$-kubus. Uiteraard is deze omvorming tot 2 dimensies helemaal niet moeilijk: het is gewoon de $N$-kubus zelf. Vanaf dimensies hoger dan 2 wordt het moeilijker. Figuur \ref{fig:nCubeKarnaugh}
\importtikzfigure{nCubeKarnaugh}{Van $N$-kubus naar Karnaugh-kaart.}
toont de overgang van een 3-kubus naar de respectievelijke Karnaugh-kaart. Bij een $N$-kubus heeft elke toestand $N$ buren. Nochtans zien we op de figuur dat de linkse en rechtse toestanden slechts 2 buren hebben in plaats van 3. We moeten dan ook op een Karnaugh-kaart modulo rekenen. De linkse buur van het meest linkse veld is het rechtse veld. Verder omvat elke cel in de Karnaugh-kaart een bepaalde configuratie aan de ingang. Aan de rand van de Karnaugh-kaart staan de variabelen, en een lijn. De cellen die deze lijn omvatten zijn de cellen waar deze specifieke variabele 1 is. Logischerwijs zijn de cellen die niet omvat worden, de cellen waar deze variabele 0 is. Figuur \ref{fig:karnaughKaarten}
\begin{figure}[hbt]
\centering
\subfigure[1D]{\importtikz{karnaugh1d}}
\subfigure[2D]{\importtikz{karnaugh2d}}
\subfigure[3D]{\importtikz{karnaugh3d}}
\subfigure[4D]{\importtikz{karnaugh4d}}
\subfigure[5D gespiegeld]{\importtikz{karnaughKaarten5DMirror}}
\subfigure[5D gekopieerd]{\importtikz{karnaughKaarten5DCopy}}
\caption{Karnaugh-kaarten voor verschillende dimensies met binaire waarden.}
\figlab{karnaughKaarten}
\end{figure}
toont Karnaugh-kaarten voor een verschillend aantal variabelen. In de cellen staat de decimale waarde van de ingang die deze cel vertegenwoordigt. Hierbij gebruiken we de variabelen in de volgende volgorde: $\left(v,w,x,y,z\right)$. We kunnen in principe blijven uitbreiden. Vanaf 6 dimensies wordt het echter moeilijk om nog patronen te herkennen. Karnaugh-kaarten hebben bijgevolg maar een beperkt vermogen. Vanaf 5 dimensies groepeert men meestal cellen in groepen van $4\times 4$. Men gebruikt hierbij 2 varianten: ofwel spiegelt men \'e\'en van de twee tabellen zoals op figuur \ref{fig:karnaughKaarten5DMirror}, ofwel zijn de tabellen exacte kopies (op enkele variabelen na) zoals op figuur \ref{fig:karnaughKaarten5DCopy}. Het spiegelen van variabelen is intu\"itiever omdat dit consistent is met lagere dimensies.
\subsubsection{Optimaliseren met behulp van Karnaugh-kaarten}
\paragraph{Terminologie} Alvorens we aan het optimalisatiewerk kunnen beginnen, hebben we nood aan enige terminologie. Een \termen{implicant} is een productterm waarvoor de functie 1 is. Deze definitie heeft nauwe banden met de 1-minterm, het verschil is echter dat bij een implicant niet alle variabelen aanwezig moeten zijn. Zo zien we op figuur \ref{fig:karnaughKaartenImplicanten} verschillende implicanten met een verschillend aantal variabelen. Een \termen{priemimplicant} is een implicant die geen onderdeel is van een andere implicant met strikt minder variabelen. De priemimplicanten van figuur \ref{fig:karnaughKaartenImplicanten} worden weergegeven op figuur \ref{fig:karnaughKaartenPriemimplicanten}. Een verdere uitbreiding is de \termen{essenti\"ele priemimplicant}, dit is een priemimplicant die minstens \'e\'en 1-minterm omvat die niet in een andere priemimplicant verweven zit. Tenslotte defini\"eren we de \termen{dekking} of \termen{cover} als een verzameling van implicanten die in alle mogelijkheden voorziet waar de functie 1 is.
\begin{figure}[hbt]
\centering
\subfigure[Implicanten]{\importtikz{karnaughKaartenImplicanten}}
\subfigure[Priemimplicanten]{\importtikz{karnaughKaartenPriemimplicanten}}
\subfigure[Essenti\"ele priemimplicanten]{\importtikz{karnaughKaartenEssentielePriemimplicanten}}
\caption{Terminologie van een Karnaugh-kaart.}
\figlab{karnaughKaartTerminologie}
\end{figure}
\paragraph{Stappenplan}
We minimaliseren een functie met behulp van een Karnaugh-kaart in 4 stappen, deze stappen zullen we in de volgende paragrafen besproken:
\begin{enumerate}
 \item Maak de Karnaugh-kaart.
 \item Bepaal alle priemimplicanten.
 \item Bepaal alle essenti\"ele priemimplicanten.
 \item Zoek de minimale dekking.
\end{enumerate}
Daarna zullen we nog drie speciale gevallen bestuderen.
\paragraph{}
We bestuderen deze methode aan de hand van een voorbeeld. We zullen een schakeling synthetiseren die de functies $f$ en $g$ berekent. Functie $f$ geeft 1 terug bij de getallen 0, 1, 3, 7, 5, 8, 10, 11, 14 en 15, en 0 in de andere gevallen, $g$ is waar  als de afgeronde vierkantswortel van het getal even is.
\paragraph{Stap 1: maak de Karnaugh-kaart}
In de eerste stap bouwen we een Karnaugh-kaart op, op basis van de gegeven functie. We tekenen een Karnaugh-kaart met het juiste aantal ingangsvariabelen (zie figuur \ref{fig:karnaughKaarten}) en vullen vervolgens de uitgangswaarden voor \'e\'en bepaalde uitgang in op de respectievelijke plaatsen. Er dient dus per binaire uitgang zo'n kaart gemaakt te worden. Op figuur \ref{fig:karnaughKaartenVoorbeeldGetekend}
\importtikzfigure{karnaughKaartenVoorbeeldGetekend}{Ingevulde Karnaugh-kaarten voor de uitgangen van het leidend voorbeeld.}
%tonen we de Karnaugh-kaarten voor de eerste 3 uitgangen van het leidend voorbeeld ($a$, $b$ en $c$). De overige uitgangen worden als oefening aan de lezer overgelaten. De oplossing is te vinden op figuur \ref{fig:apxKKaartenFill} op pagina \pageref{fig:apxKKaartenFill}.
staan de Karnaugh-kaarten voor de functies $f$ en $g$. De variabelen $x_1$, $x_2$, $x_3$ en $x_4$ zijn de binaire voorstelling van het invoergetal.
\paragraph{Stap 2: bepaal alle priemimplicanten} In de volgende stap bepalen we alle priemimplicanten van de Karnaugh-kaart. Visueel is een implicant niets anders dan een rechthoek waarbij zowel de lengte en breedte een lengte hebben van machten van twee. Deze rechthoeken vallen uiteraard ook onder de modulo-regel op een Karnaugh-kaart. Deze priemimplicanten kunnen we dan ook vinden door vanuit een cel waar de uitgangswaarde 1 is, telkens ofwel de lengte ofwel de breedte te verdubbelen, uiteraard mogen er wel geen nullen onder de rechthoek vallen. Indien er verschillende uitbreidingen mogelijk zijn, dienen al de uitbreidingen gevolgd te worden. Figuur \ref{fig:karnaughKaartenVoorbeeldPriemimplicanten} toont de priemimplicanten voor de twee uitgangen van het leidend voorbeeld.
\importtikzfigure{karnaughKaartenVoorbeeldPriemimplicanten}{Karnaugh-kaarten met priemimplicanten van het leidend voorbeeld.}
\paragraph{Stap 3: Bepaal alle essenti\"ele priemimplicanten} Nadat we de priemimplicanten bepaald hebben, zullen we uit deze verzameling de essenti\"ele priemimplicanten halen. Deze stap is dan ook heel eenvoudig: als een priemimplicant \'e\'en of meer cellen omvat die geen enkele andere priemimplicant omvat is het een essenti\"ele priemimplicant. Deze implicaten zullen sowieso al tot de resulterende functie behoren. Op figuur \ref{fig:karnaughKaartenVoorbeeldEssentielePriemimplicanten} staan de essenti\"ele priemimplicanten voor $f$ en $g$. We zien duidelijk dat in beide gevallen de priemimplicanten onvoldoende zijn om de volledige functie te beschrijven daar er nog enen niet niet gedekt worden.
\importtikzfigure{karnaughKaartenVoorbeeldEssentielePriemimplicanten}{Karnaugh-kaarten met essenti\"ele priemimplicanten van het leidend voorbeeld.}
\paragraph{Stap 4: Zoek de minimale dekking}
De essenti\"ele priemimplicanten zijn de goedkoopste manier om de cellen die ze dekken te implementeren, we zien echter dat dit in de meeste gevallen onvoldoende is om de volledige functie te beschrijven. We dienen nog extra priemimplicanten toe te voegen om tot volledige dekking te komen. In het ideale geval doen we dit door alle mogelijke toevoegingen van priemimplicanten na te gaan. Dit is echter een erg arbeidsintensief proces. Men lost dit probleem dan ook meestal op met een ``\termen{gulzige strategie}'' ofwel ``\termen{greedy algorithm}''. Hierbij beschouwen we initieel de set van essenti\"ele priemgetallen, per iteratie voegen we de priemimplicant toe die het meeste aantal cellen dekt die tot dan toe ongedekt bleven. We stoppen op het moment dat de set van implicanten de volledige functie dekt. We illustreren dit proces op figuur \ref{fig:karnaughKaartenVoorbeeldGreedySearch} waarbij we elke iteratiestap tonen.
\importtikzfigure{karnaughKaartenVoorbeeldGreedySearch}{Werking van het greedy algoritme bij het leidend voorbeeld.}
\paragraph{Synthese}
Elk van de priemimplicanten die we geselecteerd hebben stelt het product voor van enkele variabelen. We implementeren de functie door de som te nemen van deze priemimplicanten. Voor het voorbeeld wordt dit dus:
\begin{equation}
\begin{array}{ll}
\left\{
\begin{array}{l}
f=x_2'x_3'x_4'+x_1'x_4+x_1x_3\\
g=x_1'x_3'x_4'+x_1'x_2'x_3x_4+x_2x_3x_4'+x_2x_3'x_4+x_1x_2x_3
\end{array}\right.&\mbox{(Leidend voorbeeld)}
\end{array}
\end{equation}
\paragraph{Uitbreiding: Dambordpatroon}
Een patroon die men vaak tegenkomt in Karnaugh-kaarten is het \termen{dambordpatroon}. Dit dambordpatroon hoeft niet noodzakelijk uit vierkanten te bestaan, rechthoek zijn ook mogelijk. Indien we dit patroon met de klassieke Karnaugh-methode implementeren bekomen we een groot aantal priemimplicanten wat leidt tot kostelijke implementaties, we kunnen in dat geval gebruik maken van XOR-operaties die we achter elkaar schakelen. Een aaneenschakeling van XOR-operaties heeft een grotere vertraging maar heeft een grote invloed op de kostprijs. Figuur \ref{fig:dambordpatronen} toont enkele dambordpatronen en hun implementatie met XOR-logica.
\importtikzfigure{dambordpatronen}{Voorbeelden van dambordpatronen in Karnaugh-kaarten.}
\paragraph{Uitbreiding: Don't cares}
\label{par:dontcare}
In sommige gevallen dienen we slechts een beperkte set van invoer-configuraties te beschouwen. Stel bijvoorbeeld dat we een digitale display implementeren die getallen van 0 tot en met 9 voorstelt. In dat geval hebben we 4 ingangen nodig. Maar we zullen bijvoorbeeld nooit de ingang $1011_2=11_{10}$ tegenkomen. Er is echter wel een plaats gereserveerd op de Karnaugh-kaart voor deze configuratie. In dat geval maken we gebruik van de zogenaamde \termen{don't care}. Dit wordt genoteerd met een horizontale streep\footnote{Engels: dash.}, een ``X'' of ``d''. Een don't care is geen speciale vorm van uitvoer. We kunnen enkel nullen of enen teruggeven. Een don't care wordt enkel gebruikt om aan te geven dat we vrij mogen kiezen of de uitvoer een 0 of 1 wordt. Uiteraard proberen we een keuze te maken die de implementatie goedkoper maakt. We kunnen tot betere implementaties komen door een don't care als een 1 te zien indien dit de priemimplicanten kan vergroten. Op die manier bereiken deze priemgetallen immers een groter gebied waardoor ze minder variabelen bevatten. Op figuur \ref{fig:sevenDigitDisplay} geven we de Karnaugh-kaart van led $A$ en $B$ bij een \termen{seven-segment display}, samen met de priemimplicanten die we bekomen na het toewijzen van de don't cares. In appendix ?? staan de Karnaugh-kaarten van de andere leds. Deze kaarten zijn een goede oefening om het volledige proces te leren.
\importtikzfigure{sevenDigitDisplay}{Karnaugh-kaart met don't cares van led $A$ en $B$ van een seven-segment display.}
\paragraph{Uitbreiding: Meerdere uitgangen}
Tot dusver hebben we steeds aangenomen dat we de functie voor \'e\'en signaaluitgang optimaliseren, uit de voorbeelden die we beschouwd hebben blijkt echter dat een component verschillende uitgangen moet uitrekenen (de seven-segment display). Daar we de functies implementeren met AND-OR logica is het mogelijk dat de AND-poorten van \'e\'en uitgang ook nuttig kunnen zijn voor de uitgang van een andere uitgang. Dit leidt misschien niet tot de goedkoopste schakeling per uitgang maar globaal kunnen we eventueel kosten besparen. We zullen hieronder een procedure bespreken die gebruik maakt van de priemimplicanten, er is echter geen garantie dat deze de globale goedkoopste schakeling realiseert. Soms is het zelfs goedkoper om met niet-priemimplicanten te werken. Met trail-and-error kunnen we dus soms tot nog goedkopere implementaties komen. Volgende procedure bekomt echter meestal een goed resultaat:
\begin{enumerate}
 \item We realiseren eerst bij elke uitgang de essenti\"ele priemimplicanten.
 \item Selecteer vervolgens priemimplicanten die essenti\"ele priemimplicanten zijn bij een andere uitgang. Bij deze keuze is het ook belangrijk om de priemimplicant te selecteren die in het kleinste aantal functies voorkomt, dit doen we om de fan-out laag te houden waardoor we minder vertraging induceren. Merk op dat deze operatie ons niets kost: we hebben immers deze implicanten al gerealiseerd.
 \item De overige priemimplicaten realiseren we per uitvoer zoals op de klassieke manier. Indien een priemimplicant in verschillende uitgangen voorkomt, kunnen we deze eventueel bevoordelen.
\end{enumerate}
\paragraph{Duale vorm}
Tot dusver hebben we telkens met behulp van Karnaugh-kaarten een AND-OR implementatie gerealiseerd. Zoals we al vaak zijn tegengekomen hebben quasi alle logische methodes een duale vorm. Ook de Karnaugh-kaarten kunnen we gebruiken om een minimale OR-AND implementatie te realiseren. In tegenstelling tot de AND-OR vorm draait alles hier rond nullen en niet rond enen. De priemimplicaten zijn hierbij gerelateerd aan 0-maxtermen: functies die overal 1 teruggeven behalve op een bepaald patroon. Verder werkt deze methode volledig analoog: we bepalen eerst de essenti\"ele priemimplicanten en voegen vervolgens andere priemimplicanten toe. We synthetiseren vervolgens de schakeling door een AND tussen alle gekozen priemimplicanten te plaatsen. Op figuur \ref{fig:karnaughKaartenVoorbeeldDualeVorm} voeren we deze methode uit op de $f$-functie van het leidend voorbeeld.
\importtikzfigure{karnaughKaartenVoorbeeldDualeVorm}{Duale methode met Karnaugh-kaarten.}
\subsection{Quine-McCluskey}
Een alternatieve methode voor Karnaugh-kaarten is het \termen{Quine-McCluskey algoritme}. Dit algoritme wordt gebruikt in CAD-pakketten voor booleaanse optimalisatie en werkt op basis van tabellen. Het algoritme zoekt ook naar priemimplicanten en essenti\"ele priemimplicanten om een functie te optimaliseren en is dus het tabel-equivalent van de methode met de Karnaugh-kaarten. Het algoritme werkt in exponenti\"ele tijd, namelijk \bigoh{3^n} met $n$ het aantal variabelen. In de meeste gevallen is het aantal variabelen te groot om deze functie te optimaliseren, in dat geval wordt er gewerkt met de Espresso heuristic logic minimizer.
\subsection{Realisatie in meer dan 2 lagen}
Een Karnaugh-kaart laat toe tot een sterke implementatie te komen met twee lagen (een AND- en OR-laag). Zoals we echter in de volgende secties zullen zien, zullen complexe schakelingen bij twee lagen toch een hoge kost met zich meebrengen. Daarom is het soms aangewezen om de logica in meer lagen te implementeren. Dit veroorzaakt tragere schakelingen maar aan een goedkopere kostprijs. Hieronder geven we enkele technieken:
\begin{itemize}
 \item Specificatie: in heel wat gevallen gaat de specificatie van het component reeds gepaard met een expliciete implementatie. Bijvoorbeeld ``1 indien $x$ en ofwel $y$ ofwel $z$ en $t$'' kunnen we dan rechtstreeks implementeren als: $f=x\wedge\left(y\oplus\left(z\wedge t\right)\right)$.
 \item \termen{Factoranalyse}: We kunnen een expressie ook \termen{algebra\"isch manipuleren} met de wetten uit sectie \ref{s:booleaanseAlgebra}. Factoranalyse wordt ook gebruikt wanneer we een schakeling dienen te implementeren met beperkte fan-in: stel dat we enkel NAND-poorten met 2 ingangen ter beschikking hebben. Bij realisaties met beperkte fan-in moet men altijd proberen deze te implementeren in een boomstructuur. Indien we dus $f=x+y+z+t$ moeten implementeren converteren we dit naar $f=\left(x+y\right)+\left(z+t\right)$ en niet naar $f=x+\left(y+\left(z+t\right)\right)$. De boomstructuur laat toe schakelingen te realiseren die een vertraging van \bigoh{\log n} hebben tegenover de lineare implementatie met een vertraging van \bigoh{n}.
 \item \termen{Functionele ontbinding}: Soms zijn we ook in staat om een functie op te delen in verschillende deelfuncties. In subsectie \ref{sss:fulladder} zullen we bijvoorbeeld een volledige opteller beschouwen. In plaats van een optelling van drie bits rechtstreeks te implementeren kunnen we twee optellingen van twee bits realiseren. In het algemeen betekent dit dat we de functie $\vec{f}\left(\vec{x}\right)$ soms kunnen herschrijven als $\vec{h}\left(\vec{g}\left(\vec{x}\right),\vec{x}\right)$ waarbij $\vec{g}$ en $\vec{h}$ meestal eenvoudiger en goedkoper zijn.
\end{itemize}
Geen enkele van deze methodes levert altijd een betere resultaat het probleem moet dan ook opgelost worden in ``\termen{trial-and-error}'' stijl.
\subsection{Welke methode kiezen?}
Samen met de methodes uit subsectie \ref{ss:canoniekestandaardrealisatie} hebben we nu volgende methodes om een schakeling te synthetiseren:
\begin{itemize}
 \item Canonieke Sum-of-Products
 \item Canonieke Product-of-Sums
 \item Minimale Sum-of-Products (Karnaugh-kaarten)
 \item Minimale Product-of-Sums (Karnaugh-kaarten)
 \item Meerlagenlogica
\end{itemize}
\paragraph{}Verder kunnen we ook vrij kiezen tussen AND-OR en NAND-NAND in het geval van Sum-of-Products, en voor OR-AND en NOR-NOR bij Product-of-Sums. Het is altijd voordeliger om voor NAND-NAND en NOR-NOR te kiezen. Deze schakelingen zijn altijd goedkoper en sneller. Het aantal poorten en de structuur blijft immers gelijk en uit vergelijkingen (\ref{eqn:kostenCmosPoort}) en (\ref{eqn:speedPoort}) blijkt duidelijk dat dit een betere keuze is. Een andere mogelijkheid is om de Minimale AND-OR implementatie om te zetten naar een AND-OR-Invert en een OR-AND implementatie naar zijn OR-AND-Invert equivalent.
\paragraph{}Ook kunnen we in het algemeen bewijzen dat de implementatie met Karnaugh-kaarten altijd goedkoper en sneller is. Immers in het slechtste geval komt dit neer op dezelfde implementatie als de canonieke sum-of-products. In de meeste gevallen zal meerlagenlogica verder een grotere vertraging induceren dan de implementatie met Karnaugh-kaarten, dit is echter niet algemeen en bovendien kan meerlagenlogica gepaard gaan met grote kostenbesparingen. Veel rekenkundige schakelingen die we in de volgende sectie zullen tegenkomen maken dan ook gebruik van meerdere functionele lagen.
\paragraph{}Tot slot is het niet altijd belangrijk om tot de meest optimale implementatie te komen. Indien we bijvoorbeeld de logica op een FPGA programmeren hebben we per functie een logic block ter beschikking. Het aantal poorten in dit blok staat al vast. Indien we dus onder dat aantal blijven levert het ons niks op om de functie verder te minimaliseren. We hebben immers toch reeds voor deze poorten betaald. Deze realisatie van de schakeling naar de beschikbare elementen (poorten, logic blocks,...) wordt dan ook de ``\termen{technology mapping}'' genoemd.
\paragraph{}Om de verschillende implementaties te illustreren zullen we tot slot een schakeling implementeren in de verschillende vormen van logica. De Karnaugh-kaart en de implementaties staan op figuur \ref{fig:differentImplementationsSSD}. Een samenvatting van deze implementaties in termen van kosten en vertraging staan in tabel \ref{tbl:differentImplementationsSSD}.
\begin{figure}[hbt]
\centering
\subfigure[Karnaugh-kaart]{\importtikz{karnaughKaartSopPos}}
\subfigure[Canonieke SOP (AND-OR)]{\importtikz{canonicSop}}
\subfigure[Minimale SOP]{\importtikz{minimalSop}}
\subfigure[Canonieke POS (OR-AND)]{\importtikz{canonicPos}}
\subfigure[Minimale POS]{\importtikz{minimalPos}}
\caption{Verschillende implementaties van dezelfde logische functie.}
\figlab{differentImplementationsSSD}
\end{figure}
\begin{table}[hbt]
\centering
\begin{tabular}{l|l|rr|rr}
Modus&Implementatie&Kosten&Relatief&Vertraging&Relatief\\\hline
\multirow{2}{*}{Canonieke SOP}&AND-OR&47&100\%&8.6&100\%\\
&NAND-NAND&39&83\%&6.6&77\%\\\hline
\multirow{2}{*}{Canonieke POS}&OR-AND&59&126\%&9.4&109\%\\
&NOR-NOR&49&104\%&7.4&86\%\\\hline
\multirow{2}{*}{Minimale SOP}&AND-OR&16&34\%&6.6&77\%\\
&NAND-NAND&12&26\%&4.6&53\%\\\hline
\multirow{2}{*}{Minimale POS}&OR-AND&17&36\%&6.2&72\%\\
&NOR-NOR&13&28\%&4.2&49\%
\end{tabular}
\caption{Samenvatting van de verschillende implementaties.}
\tbllab{differentImplementationsSSD}
\end{table}
\section{Rekenkundige basisschakelingen}
\label{s:rekenkundig}
In deze sectie defini\"eren we enkele belangrijke schakelingen voor rekenkundige bewerkingen. We hebben het dan over optellen, aftrekken vermenigvuldigen, delen, modulo rekenen en logische berekeningen. Alvorens we echter met getallen kunnen rekenen, moeten we een manier bedenken om getallen voor te stellen met binaire signalen. Doorheen deze sectie zullen we de voorstelling van getallen regelmatig veranderen om extra functionaliteit toe te voegen.
\subsection{Getallen voorstellen}
In de wereld wordt bij het voorstellen van getallen meestal het Arabisch getalsysteem gehanteerd. Hierbij stellen we een getal voor door een reeks cijfers. Een cijfer op een bepaalde plaats heeft een gewicht dat $r$ keer groter is, dan het volgende cijfer, met $r$ als de \termen{radix} van het getalstelsel. Om tot een eenduidige voorstelling van elk getal te komen wordt de verzameling mogelijke cijfers beperkt tot $r$ elementen. Een getal $D_r$ wordt dus voorgesteld als:
\begin{equation}
D_r=d_{m-1}d_{m-2}\ldots d_0,d_{-1}\ldots d_{-n}=\displaystyle\sum_{i=-n}^{m-1}{r^i\times d_i}
\label{eqn:numberRepresentation}
\end{equation}
Hierbij zijn $d_i$ de cijfers van het getal. Wereldwijd gebruikt men het \termen{decimale stelsel}, andere populaire stelsel zijn het \termen{binair} ($r=2$), \termen{octaal} ($r=8$) en \termen{hexadecimaal} ($r=16$) stelsel. Het binaire stelsel heeft logischerwijs twee mogelijke cijfers. Deze kunnen we voorstellen door $0$ of $1$ op een lijn te zetten. Indien we meer cijfers nodig hebben, zullen we eenvoudigweg meer lijnen voorzien, die elk een cijfer van het getal voorstellen. In de informatica en elektronica maakt men theoretisch vaak gebruik van andere getalstelsel. Dit komt omdat binaire getallen niet bepaald compact zijn\footnote{Een binair getal bestaat uit $3.3$ keer het aantal cijfers van zijn decimale tegenhanger.}. Immers is het omzetten van binaire getallen naar hexadecimale getallen niets anders dan het groeperen van cijfers. Indien men getallen noteert met een andere radix dan de decimale, wordt de radix in decimale notatie als subscript toegevoegd. Zo is $\mbox{2A}_{16}$ het equivalent van $42$.
\subsection{Radix-conversie}
\subsubsection{$r_1\rightarrow r_2$ omzetting met $r_1=r_2^p$}
Een speciaal geval van omzetting treedt op indien de bronradix een macht is van de doel radix. In dat geval kunnen we de omzetting eenvoudig doen door alfabetomzetting. Immers betekent dit dat we elk broncijfer kunnen omzetten naar $p$ doelcijfers. Hierbij dient de sequentie doelcijfers het wiskundig equivalent te zijn van de broncijfers. Een concreet voorbeeld is het omzetten van een hexadecimaal getal in het binair stelsel ($r_1=16=2^4=r_2^4$). Tabel \ref{tbl:radixConversionHexOctBin}
\begin{table}[hbt]
\centering
\subtable[Hex $\leftrightarrow$ Binair]{
\begin{tabular}{r|r||r|r||r|r||r|r}
Hex&Bin&Hex&Bin&Hex&Bin&Hex&Bin\\\hline
0&0000&4&0100&8&1000&C&1100\\
1&0001&5&0101&9&1001&D&1101\\
2&0010&6&0110&A&1010&E&1110\\
3&0011&7&0111&B&1011&F&1111\\
\end{tabular}
}
\subtable[Octaal $\leftrightarrow$ Binair]{
\begin{tabular}{r|r||r|r}
Oct&Bin&Oct&Bin\\\hline
0&000&4&100\\
1&001&5&101\\
2&010&6&110\\
3&011&7&111\\
\end{tabular}
}
\caption{Radix-conversie van hexadecimaal en octaal naar binair.}
\tbllab{radixConversionHexOctBin}
\end{table}
toont de omzetting van hexadecimale en octale cijfers naar binaire cijfers. Bij wijzen van voorbeeld zetten we $\mbox{B4F}_{16}$ om naar het binaire stelsel\footnote{De verticale strepen ($|$) dienen uitsluitend om educatieve doeleinden, en zijn niet verplicht.}:
\begin{equation}
\mbox{B4F}_{16}=1011|0100|1111=101101001111_2
\end{equation}
\subsubsection{$r_1\rightarrow r_2$ omzetting met $r_1^q=r_2$}
In de omgekeerde situatie willen we een getal van een radix $r_1$ omzetten naar een macht van deze radix. De oplossing ligt dan ook voor de hand: we vertalen groepjes van $q$ broncijfers naar 1 doelcijfer. Een belangrijke opmerking is hoe we de voorstelling met de bronradix onderverdelen in groepen: dit doen we vanaf de komma voor het gehele gedeelte naar links, voor het kommagedeelte naar rechts. Aan de uiteinden van de voorstelling kunnen er soms onvoldoende cijfers aanwezig zijn. In dat geval dienen nullen toegevoegd te worden. Bij wijze van voorbeeld zetten we $1011011111.10001_2$ om naar zijn octale equivalent (zie hiervoor tabel \ref{tbl:radixConversionHexOctBin}):
\begin{equation}
1011011111.10001_2=1|011|011|111.100|01_2=\underline0\underline01|011|011|111.100|01\underline0_2=1337.42_8
\end{equation}
\subsubsection{$r_1\rightarrow r_2$ omzetting met $r_1^q=r_2^p$}
Een logisch gevolg van de voorgaande omzettingen, is dat we ook de mogelijkheid hebben om getallen makkelijk tussen 2 radixen om te zetten indien ze een gemeenschappelijke macht hebben. In dat geval laten we de conversie verlopen langs een derde radix: $r'$. $r'$ vormt een basis waarvan de bron- en doelradix machten zijn. Er geldt dan ook: $r_1=r'^p$ en $r'^q=r_2$. Een concreet voorbeeld is het binaire stelsel die als een tolk kan functioneren tussen het hexadecimale en octale stelsel. Dit illustreren we door $157255_8$ om te zetten naar het hexadecimaal stelsel:
\begin{equation}
\begin{array}{ll}
\mbox{Oct$\rightarrow$Bin}&157255_8=001|101|111|010|101|101_2=1101111010101101_2\\
\mbox{Bin$\rightarrow$Hex}&1101111010101101_2=1101|1110|1010|1101_2=\mbox{DEAD}_{16}
\end{array}
\end{equation}
\subsubsection{Algemene omzetting}
De vorige methodes werkten enkel onder een fundamentele aanname: de ene radix moest een macht van een andere zijn. In de praktijk is dit meestal niet zo, zo willen we vaak decimale getallen omzetten naar hun binair equivalent, of omgekeerd. In deze subsubsectie behandelen we kort een methode om in het algemeen een radixomzetting uit te voeren. Deze omzettingen zijn relatief arbeidintensief, en vereisen bovendien een getalstel waarin we simpele rekenkundige operaties kunnen uitvoeren. Voor mensen is dit over het algemeen het decimaal stelsel, computers werken doorgaans met het binair stelsel\footnote{Dit is niet altijd zo, in de Sovjet-Unie waren in de jaren '50 ternaire computers populair.}. In deze methode zetten we eerst de representatie om naar de representatie waarop we kunnen rekenen. Indien we bijvoorbeeld $\mbox{8C989}_{16}$ willen omzetten naar een radix 36, zullen we dit getal eerst omzetten naar het decimale stelsel, dit doen we door de waarde van de cijfers te vermenigvuldigen met het gewicht van hun positie en deze vervolgens op te tellen:
\begin{equation}
\mbox{8C989}_{16}=8\cdot 16^4+12\cdot 16^3+9\cdot 16^2+8\cdot 16^1+9\cdot 16^0=575881
\end{equation}
Vervolgens bepalen we iteratief elk cijfer, hierbij beginnen we bij het laatste cijfer, dit vinden we door het getal modulo de radix te bereken: $575881\mod36=25=\mbox{P}_{36}$. Hiermee weten we al het laatste cijfers. We trekken vervolgens de bekomen waarde af van het getal, en delen het door de radix. Dit resultaat kunnen we zo verder iteratief manipuleren, om de andere cijfers te berekenen. Indien we 0 uitkomen stopt het algoritme en hebben we de equivalente representatie gevonden. Tabel \ref{tbl:radixConversionExample} illustreert dit principe.
\begin{table}[hbt]
\centering
\begin{tabular}{r|r|l|l|r}
Stap&Getal&Modulo&Volgende getal&Cijfer\\\hline
$1$&$575881$&$575881\mod36=25$&$\left(575881-25\right)/36=15996$&$25=\mbox{P}_{36}$\\
$2$&$15996$&$15996\mod36=12$&$\left(15996-12\right)/36=44$&$12=\mbox{C}_{36}$\\
$3$&$444$&$444\mod36=12$&$\left(444-12\right)/36=12$&$12=\mbox{C}_{36}$\\
$4$&$12$&$12\mod36=12$&$\left(12-12\right)/36=0$&$12=\mbox{C}_{36}$\\
$5$&$0$&$-$&$-$&$-$
\end{tabular}
\caption{Voorbeeld van algemene radix-omzetting.}
\tbllab{radixConversionExample}
\end{table}
We kunnen dus concluderen dat: $\mbox{8C989}_{16}=\mbox{CCCP}_{36}$.
\subsection{Optellen}
\label{ss:add}
Hoe tellen we nu twee getallen op? In het decimaal stelsel tellen we twee getallen op door middel van cijferen. Hieronder geven we een illustrerend voorbeeld waarbij we $1425+1917$ uitrekenen:
\begin{equation}
\begin{array}{l|lcccc}
\mbox{overdracht $c$}&&1&0&1&\\
x&&1&4&2&5\\
y&+&1&9&1&7\\\hline
\mbox{som $s$}&&3&3&4&2
\end{array}
\end{equation}
Om getallen in het decimaal stelsel te kunnen optellen gebruiken we een repetitieve structuur, waarbij we 200 basisregels moeten onthouden. Zo'n basisregel $f\left(c_i,x_i,y_i\right)=\left(s_i,c_{i+1}\right)$ is een functie die de eventuele \termen{overdracht} of \termen{carry}, en de cijfers van de twee getallen in een bepaalde kolom omzet naar de overdracht van de volgende kolom en de som van deze kolom. Een voorbeeld van zo'n basisregel is $f\left(0,4,9\right)=\left(3,1\right)$. Deze regel wordt in ons voorbeeld gebruikt in de derde kolom vanaf rechts\footnote{We nummeren de kolommen vanaf rechts, analoog aan de getalvoorstelling van vergelijking (\ref{eqn:numberRepresentation})}. Binair optellen gebeurt volledig analoog, we hebben opnieuw met een repetitieve structuur te maken, alleen dienen we nu slechts 8 verschillende regels te onthouden.
\subsubsection{Half adder}
In deze paragraaf synthetiseren we een \termen{half adder (HA)}, die de laatste bit van de twee getallen optelt. Bij het laatste getal is de overdracht sowieso 0, dus daarmee hoeven we geen rekening te houden. We dienen een functie te ontwikkelen die de overdracht van de volgende kolom berekent, en de som. We stellen een waarheidstabel en de bijbehorende Karnaugh-kaarten op in figuur \ref{fig:halfAdder}.
\begin{figure}[hbt]
\centering
\subfigure[Waarheidstabel]{\importtikz{halfadder-truthtable}}
\subfigure[Karnaugh-kaarten]{\importtikz{halfadder-karnaugh}}
\subfigure[Interface]{\importtikz{halfadder-interface}}
\subfigure[Mogelijke implementatie]{\importtikz{halfadder-implementation}}
\caption{Half adder (HA).}
\figlab{halfAdder}
\end{figure}
Deze combinatorische functies kunnen we verwezenlijken met een AND en XOR poort.
\subsubsection{Full adder}
\label{sss:fulladder}
Met een Half adder kunnen we echter geen optellingen uitrekenen. Hiervoor dienen we een component te ontwikkelen die meer functionaliteit biedt: de \termen{full adder (FA)}. Een full adder bevat slechts \'e\'en ingang extra: de overdracht van de vorige opteller $c_i$. Een full adder telt op door een XOR operatie toe te passen op de drie ingangen: $x_i$, $y_i$ en $c_i$, verder is er sprake van overdracht indien twee of meer ingangen 1 zijn. Dit kunnen we eenvoudig implementeren met tweelagen-logica. De waarheidstabellen en Karnaugh-kaarten staan samen met een mogelijke implementatie op figuur \ref{fig:fullAdder}.
\begin{figure}[hbt]
\centering
\subfigure[Waarheidstabel]{\importtikz{fulladder-truthtable}}
\subfigure[Karnaugh-kaarten]{\importtikz{fulladder-karnaugh}}
\subfigure[Interface]{\importtikz{fulladder-interface}}
\subfigure[Mogelijke implementatie]{\importtikz{fulladder-implementation}}
\subfigure[Functionele ontbinding]{\importtikz{fulladder-functional}}
\caption{Full adder (FA).}
\figlab{fullAdder}
\end{figure}
We kunnen ook een implementatie synthetiseren met behulp van functionele ontbinding. Bij een full adder voeren we immers twee optellingen uit, de volgorde speelt hierbij geen rol. Indien minstens \'e\'en van de twee optellingen overdracht genereert, is er sprake van overdracht bij de full adder. We kunnen een full adder dus ook implementeren zoals op figuur \ref{fig:fulladder-functional}. Indien we de half adders dan implementeren zoals op figuur \ref{fig:halfadder-implementation}, zien we dat we \'e\'en poort uitsparen. Dit betalen we echter met grotere vertragingen, iets wat bij optellingen met grote getallen niet gewenst is.
\subsubsection{Ripple-carry opteller}
Met behulp van een half adder en $n$ full adders kunnen we vervolgens een opteller realiseren die twee $n+1$ bit getallen optelt. Dit doen we met behulp van een \termen{Ripple-carry opteller}. Figuur \ref{fig:rippleCarryAdder}%
\importtikzfigure{rippleCarryAdder}{Schematische voorstelling van een $n$-bit Ripple-carry opteller.}
toont hoe dit in z'n werk gaat: we tellen de laatste twee bits op met een half adder, de overige bits tellen we op met full adders. De carry uitgangen van een adder gaat naar de de ingang van de volgende adder. De allerlaatste carry uitgang $c_n$, kan men gebruiken als een \termen{overflow} uitgang. Indien we immers twee $n$-bit getallen met elkaar optellen, kunnen sommige resultaten $n+1$-bit getallen vereisen om nog voorgesteld te kunnen worden. In een ander geval wordt deze uitgang gebruikt om de waarde van de $n+1$-ste bit te bepalen. In dat geval heeft de uitgang meer bits dan de operanden. Uiteraard kunnen we de half adder vervangen door een full adder met als carry ingang 0.
\paragraph{}
Omdat we meestal getallen met een groot aantal bit optellen\footnote{Op de meeste processoren is dat 32 of 64 bit.} is het interessant om het tijdsgedrag van een ripple-carry adder te bekijken. Het kritisch pad gaat logischerwijs van een ingang van de half adder tot de laatste full adder. Indien we de implementaties van de full adder met functionele ontbinding beschouwen blijkt het kritische pad $x_0\rightarrow c_n$ te zijn. In dit geval moet het signaal doorheen 1 XOR, $n$ AND en $n$ OR poorten. Elk van deze poorten heeft twee ingangen, de vertraging is dus bijgevolg:
\begin{equation}
\mbox{vertraging}=2.4n+2.4n+3.2=4.8n+3.2
\end{equation}
Het is mogelijk dat we deze vertraging verder kunnen reduceren, een probleem is echter dat de vertraging een orde $n$ blijft. Voor berekeningen met grote getallen zijn ripple-carry adders dan ook onaanvaardbaar.
\subsubsection{Carry-lookahead opteller}
Een \termen{Carry-lookahead opteller (CLA)} is een component ter vervanging van een full adder. Het is de bedoeling om met behulp van dit component rechtstreeks $c_i$ te berekenen als een functie $c_i\left(c_0,x_0,x_1,\ldots,x_{i-1},y_0,y_1,\ldots,y_{i-1}\right)$. De Carry-lookahead opteller heeft dezelfde ingangen als een full adder: $(x_i,y_i,c_i)$, maar meer uitgangen $(c_{i+1},g_i,p_i,s_i)$. Deze uitgangssignalen hebben volgende functie:
\begin{itemize}
 \item \termen{carry-generate $g_i$}: 1 indien er bij het optellen van $x_i$ en $y_i$ overdracht gegenereerd wordt. Bijgevolg is $g_i=x_iy_i$.
 \item \termen{carry-propagate $p_i$}: 1 indien bij de optelling de overdracht verder gepropageerd zal worden. Bijgevolg $p_i=x_iy_i'+x_i'y_i$.
 \item som $s_i$: het resultaat van de optelling: $s_i=c_ip_i'+c_i'p_i$.
\end{itemize}
De opsomming van de uitgangen geeft dan ook al meteen een mogelijke implementatie; de waarheidstabellen, interface en een mogelijke implementatie staan op figuur~\ref{fig:carryLookaheadAdder}.
\begin{figure}[hbt]
\centering
\subfigure[Waarheidstabel]{\importtikz{cla-truthtable}}
\subfigure[Karnaugh-kaarten]{\importtikz{cla-karnaugh}}
\subfigure[Interface]{\importtikz{cla-interface}}
\subfigure[Mogelijke implementatie]{\importtikz{cla-implementation}}
\caption{Carry-Lookahead Opteller (CLA).}
\figlab{carryLookaheadAdder}
\end{figure}
Een belangrijke opmerking bij dit component is dat we de overdracht $c_{i+1}$ vervolgens kunnen berekenen als $c_{i+1}=g_i+c_ip_i$. De overdacht hangt dus niet rechtstreeks af van $x_i$ of $y_i$, deze eigenschap is belangrijk voor de synthese van een ander component: de CLA-generator.
\subsubsection{CLA-generator}
Om een opteller te realiseren moeten we, net zoals bij de Ripple-carry opteller, alleen nog de glue voorzien tussen de verschillende carry-lookahead optellers. Bij deze optellers is het verhaal echter gecompliceerder: we hebben een extra component nodig om de CLA elementen met elkaar te verbinden: de \termen{CLA-generator}. Een $n$-bit CLA-generator is een component met ingangen $\left(c_0,g_0,\ldots g_{n-1},p_0,\ldots,p_{n-1}\right)$. De component heeft tot doel om de overdracht (carry) te berekenen met een zo klein mogelijk kritisch pad, logischerwijs heeft de component dan ook de uitgangen: $\left(c_1,\ldots c_n,g_{0,n-1},p_{0,n-1}\right)$. De laatste twee uitgangen zullen we later bespreken. Op figuur \ref{fig:clag-structure} zien we hoe we een CLA-generator aansluiten op de Carry-lookahead optellers.
\begin{figure}[hbt]
\centering
\subfigure[Structuur van de CLA-generator.]{\importtikz{clag-structure}}
\subfigure[Cascade van CLA-generators met $n=9$, $k=3$.]{\importtikz{clag-cascade}}
\caption{CLA-generator.}
\figlab{CLAGenerator}
\end{figure}
Eerst zullen we een methode ontwikkelen om een overdracht $c_i$ te berekenen. Een overdracht kan maar op twee manieren tot stand komen: ofwel wordt deze gegenereerd, ofwel wordt deze gepropageerd. Dit formaliseren we als:
\begin{equation}
c_i=g_{0,i-1}+p_{0,i-1}\cdot c_0
\end{equation}
Een oplettende lezer zal gemerkt hebben dat we hierbij $g$ en $p$ twee indices geven. De reden hiervoor is, dat de overdracht overal kan gegenereerd of gepropageerd worden. We moeten de twee indices dus als de definitie van een bereik zien. De overdacht kan uitsluitend tot de $i$-de bit gepropageerd worden, indien alle bits voor de $i$-de bit propageren, bijgevolg kunnen we stellen:
\begin{equation}
p_{i,j}=\displaystyle\prod_{k=i}^j{p_k}
\end{equation}
Tenslotte wordt er overdacht gegenereerd indien de vorige bit een overdacht genereerde, of een willekeurige bit ervoor, die deze dan tot de $i$-de bit weet te propageren. Dit formaliseren we als:
\begin{equation}
g_{i,j}=g_j+\displaystyle\sum_{k=i}^{j-1}{g_kp_{k+1,j}}=g_j+\displaystyle\sum_{k=i}^{j-1}{\left(g_k\cdot\displaystyle\prod_{l=k+1}^j{p_l}\right)}
\end{equation}
De volledige berekening voor het bepalen van overdacht $c_i$ wordt dan:
\begin{equation}
c_i=g_{i-1}+\displaystyle\sum_{k=0}^{i-2}{\left(g_k\cdot\displaystyle\prod_{l=k+1}^{i-1}{p_l}\right)}+c_0\cdot\displaystyle\prod_{k=0}^{i-1}{p_k}
\end{equation}
De berekening voor $c_i$ kost dan ook een $i+1$-input OR poort, en $i+1$ AND poorten, elk van deze poorten heeft een ander aantal ingangen vari\"erend van 2 tot $i+1$. We kunnen nu opnieuw het kritische pad analyseren. Dit pad is $x_0\rightarrow c_n$. Hierbij gaat het signaal doorheen 1 XOR poort, 1 $n+1$-input AND poort en 1 $n+1$ input OR poort. De totale vertraging is dus:
\begin{equation}
\mbox{vertraging}=3.2+2\left(2+0.4n\right)=0.8n+7.2
\label{eqn:delayCLAGenerator}
\end{equation}
Toegegeven dat de orde van de vertraging nog steeds linear is, maar de factor voor $n$ is met de CLA-generator sterk gereduceerd. We kunnen echter ook een cascade van CLA-generators maken. Indien we twee $n$-bit getallen willen optellen, en we groeperen elke $k$ componenten tot een niveau hoger, hebben we $\left\lceil\log_kn\right\rceil$ niveaus nodig. We kunnen in dat geval vergelijking (\ref{eqn:delayCLAGenerator}) gebruiken om de vertraging te berekenen:
\begin{equation}
\mbox{vertraging}=3.2+\left(2\left\lceil\log_kn\right\rceil-1\right)\times\left(4+0.8k\right)
\end{equation}
Hiermee kunnen we de vertraging reduceren tot een logaritmische orde. Figuur \ref{fig:clag-cascade} toont hoe we een dergelijke cascade kunnen bouwen. Met deze cascade wordt ook meteen de reden van de twee extra ingangen duidelijk.
\subsection{Negatieve getallen}
Om getallen te kunnen aftrekken, zullen we het getalbegrip uitbreiden, zodat deze ook negatieve getallen kunnen voorstellen. Negatieve getallen voorstellen kan op verschillende manieren. In onderstaande subsubsecties zullen we de meest populaire bespreken. Hierbij moeten we in gedachten houden dat we eventueel de bovenstaande opteller zullen moeten uitbreiden om ook negatieve getallen te kunnen optellen.
\subsubsection{`Sign-Magnitude'-voorstelling}
Het arabische getalstelsel lost dit probleem op door middel van een teken. Een getal bestaat dan uit twee delen: een teken (+ of -) en een grootte. Bij positieve getallen wordt dit teken zelfs meestal niet geplaatst. Deze voorstelling is gekend als de `\termen{sign-magnitude}' voorstelling. Aangezien een getal ofwel positief of negatief is, kunnen we bij een binaire voorstelling een extra bit bij het getal plaatsen. Deze bit is $0$ bij een positief getal, en $1$ bij een negatief getal. In dat geval wordt dus $\underline{0}1100_2=\underline{+}12_{10}$ en $\underline{1}1100_2=\underline{-}12_{10}$. Een getal met $n$ bits kan dus alle gehele waarden aannemen tussen $-2^{n-1}+1$ en $2^{n-1}-1$. Er zijn echter enkele problemen met deze voorstelling. Allereerst is er nu sprake van twee voorstellingen van $0$: $-0$ en $+0$. Dit leidt tot extra complexiteit bij bijvoorbeeld het vergelijken van twee getallen. Verder wordt optellen en aftrekken van twee getallen gecompliceerd. Er zijn veel testen nodig alvorens we de waardes van de getallen kunnen optellen of aftrekken, wat tot grote vertragingen leidt.
\subsubsection{Complement-voorstellingen}
De meeste voorstellingen van negatieve getallen zijn gebaseerd op complement voorstellingen. Er bestaan twee soorten complement voorstellingen voor een getal $D$ met $m$ cijfers en radix $r$:
\begin{itemize}
 \item \termen{cijfer-complement} of \termen{diminished-radix complement} $D'$: hierbij wordt elk cijfer $i$ vervangen door zijn complement $r-i-1$. Bijvoorbeeld:
 \begin{itemize}
    \item Het $9$-complement van $1337_{10}$ is $8662_{10}$.
    \item Het $1$-complement van $01001011\ 01010011_2$ is $10110100\ 10101100_2$.
 \end{itemize}
 Het \termen{1-complement} wordt soms gebruikt voor het voorstellen van negatieve getallen in het binaire stelsel.
 \item \termen{radix-complement} $D^*=r^m-D$. Bijvoorbeeld:
 \begin{itemize}
    \item Het radix-complement van $1337_{10}$ is $8663_{10}$.
    \item Het radix-complement van $01001011\ 01010011_2$ is $10110100\ 10101101_2$.
 \end{itemize}
 Voor binaire getallen is dit het \termen{2-complement}. Dit is veruit de meest populaire voorstellingswijze.
\end{itemize}
Een belangrijke eigenschap die ook al blijkt uit de voorbeelden is dat $D^*=D'+1$. Deze eigenschap is interessant, omdat het ons helpt snel het 2-complement te berekenen. Immers is het 1-complement niets anders dan het toepassen van een NOT operatie op alle bits.
\subsubsection{Twee-complement voorstelling}
Omdat de 2-complement voorstelling vrij populair is, zullen we deze verder bespreken. Stel we beschikken over $n$ cijfers in een getalstelsel met radix $r$. In dat geval kunnen we $r^n$ in principe niet voorstellen. Logisch gezien stellen we dit voor door $r^n\equiv 0$. Immers kunnen we stellen dat in dat geval het $n+1$-de cijfer een 1 is, en de andere cijfers 0. Aangezien $D^*=r^n-D$ geldt: $D^*=-D$. Het gevolg is dat we de traditionele optelling kunnen gebruiken, indien we 2-complement getallen met elkaar optellen. Of deze getallen nu een teken bevatten of niet, is een kwestie van interpretatie, de opteller is hiervan niet afhankelijk. Bovendien kunnen we relatief eenvoudig de negatie van een getal berekenen: we passen op elke bit een NOT operatie toe, en tellen daar vervolgens 1 bij op.
\paragraph{}
Een ander voordeel van de 2-complement voorstelling, is dat er slechts \'e\'en voorstelling van het getal 0 is. Indien we immers het negatief van 0 berekenen bekomen we:
\begin{equation}
D=0000_2\Rightarrow -D\equiv D^*=1111_2+0001_2=\underline{1}0000_2=0000_2
\end{equation}
Tot slot heeft een getal met $n$ bits een bereik van $-2^{n-1}$ tot $2^{n-1}-1$, wat exact \'e\'en element groter is dan de 1-complement tegenhanger\footnote{Dit is uiteraard de plaats die vrijkomt omdat de 2-complement voorstelling slechts \'e\'en voorstelling van 0 heeft.}.
\subsubsection{Betekenis van binaire getallen}
Bij wijze van voorbeeld om het verschil tussen de verschillende voorstellingen duidelijk te maken, zetten we de betekenis van elke $4$-bit getal naast elkaar in tabel \ref{tbl:binaryMeaningSigned}.
\begin{table}[hbt]
\centering
\begin{tabular}{r|rrrr}
Binair&Unsigned&Sign-Magnitude&1-Complement&2-Complement\\\hline
$0000$&$+0$&$+0$&$+0$&$+0$\\
$0001$&$+1$&$+1$&$+1$&$+1$\\
$0010$&$+2$&$+2$&$+2$&$+2$\\
$0011$&$+3$&$+3$&$+3$&$+3$\\
$0100$&$+4$&$+4$&$+4$&$+4$\\
$0101$&$+5$&$+5$&$+5$&$+5$\\
$0110$&$+6$&$+6$&$+6$&$+6$\\
$0111$&$+7$&$+7$&$+7$&$+7$\\
$1000$&$+8$&$-0$&$-7$&$-8$\\
$1001$&$+9$&$-1$&$-6$&$-7$\\
$1010$&$+10$&$-2$&$-5$&$-6$\\
$1011$&$+11$&$-3$&$-4$&$-5$\\
$1100$&$+12$&$-4$&$-3$&$-4$\\
$1101$&$+13$&$-5$&$-2$&$-3$\\
$1110$&$+14$&$-6$&$-1$&$-2$\\
$1111$&$+15$&$-7$&$-0$&$-1$
\end{tabular}
\caption{Betekenis van de binaire getallen.}
\tbllab{binaryMeaningSigned}
\end{table}
\subsection{Optellen en aftrekken}
Nu we de belangrijkste voorstellingen van negatieve getallen besproken hebben, zullen we voor ieder van deze voorstellingen een schema uitwerken hoe we getallen kunnen optellen en aftrekken. Deze schema's worden weergegeven in figuur \ref{fig:addSubNegSchematic}.
\begin{figure}[hbt]
\centering
\subfigure[Sign-magnitude]{\importtikz{signedmagnitude}}
\subfigure[1-complement]{\importtikz{onecomplement}}
\subfigure[2-complement]{\importtikz{twocomplement}}
\caption{Optelling en aftrekking van gehele getallen.}
\figlab{addSubNegSchematic}
\end{figure}
\subsubsection{Signed-magnitude voorstelling}
Bij een signed-magnitude voorstelling moeten eerst heel wat testen uitgevoerd worden alvorens we getallen kunnen optellen of aftrekken. In eerste instantie moeten we bij het aftrekken van twee getallen het teken van de tweede operand veranderen, om daarna de bewerking als een optelling verder te verwerken. Bij deze optelling moeten we eerst testen of beide getallen hetzelfde teken hebben. Indien dit niet het geval is, bepaalt het getal met de grootste magnitude $m$ het teken van het getal, en moeten de magnitudes van elkaar afgetrokken worden, in het andere geval is er sprake van een gewone optelling. Al deze testen zorgen voor complexe hardware die bovendien ook nog traag werkt, dit maakt signed-magnitude tot een weinig populaire voorstelling. Figuur \ref{fig:signedmagnitude} geeft de werkwijze schematisch weer.
\subsubsection{1-complement voorstelling}
Optellen en aftrekken met de 1-complement voorstelling is heel wat eenvoudiger. Bij een aftrekking berekenen we ook eerst de negatie van de tweede operand, wat een NOT operatie is. Vervolgens kunnen we de twee getallen eenvoudig optellen met onze eerder ge\"implementeerde opteller. Deze optelling volstaat soms echter niet, met het volgende voorbeelden kunnen we illustreren wat er kan fout lopen:
\begin{equation}
\begin{array}{rl|rl|rl}
\begin{array}{rr}
&0101\\
+&1010\\\hline
\textcolor{gray}{0}&1111
\end{array}
&
\begin{array}{rr}
&+5\\
+&-5\\\hline
&-0
\end{array}
&
\begin{array}{rr}
&0110\\
+&1101\\\hline
\textcolor{gray}{1}&0011
\end{array}
&
\begin{array}{rr}
&+6\\
+&-2\\\hline
&3
\end{array}
&
\begin{array}{rr}
&1011\\
+&0011\\\hline
\textcolor{gray}{0}&1110
\end{array}
&
\begin{array}{rr}
&-4\\
+&+3\\\hline
&-1
\end{array}
\end{array}
\label{eqn:oneComplementFaultExample}
\end{equation}
Indien we twee getallen optellen met een tegengesteld teken, zodat de uitkomst positief is, zien we dat het resultaat altijd 1 lager uitkomt dan het juiste resultaat. Dit komt omdat de 1-complement voorstelling twee representaties heeft voor 0. Zoals we zien in het eerste voorbeeld, komen we altijd -0 uit bij het optellen van twee tegengestelde getallen. Bijgevolg zal een resultaat van 1 uitkomen op de voorstelling die 1 hoger is dan $-0$: $+0$. Indien de carry $c_n$ dus 1 is, moeten we nog 1 optellen bij het resultaat. Dit kunnen we toepassen op de voorbeelden in vergelijking (\ref{eqn:oneComplementFaultExample}):
\begin{equation}
\begin{array}{rl|rl|rl}
\begin{array}{rr}
&0101\\
+&1010\\\hline
\underline{0}&1111\\
+&000\underline{0}\\\hline
&1111
\end{array}
&
\begin{array}{rr}
&+5\\
+&-5\\\hline
&-0\\
+&+0\\\hline
&-0
\end{array}
&
\begin{array}{rr}
&0110\\
+&1101\\\hline
\underline{1}&0011\\
+&000\underline{1}\\\hline
&00100
\end{array}
&
\begin{array}{rr}
&+6\\
+&-2\\\hline
&+3\\
+&+1\\\hline
&+4
\end{array}
&
\begin{array}{rr}
&1011\\
+&0011\\\hline
\underline{0}&1110\\
+&000\underline{0}\\\hline
&1110
\end{array}
&
\begin{array}{rr}
&-4\\
+&+3\\\hline
&-1\\
+&+0\\\hline
&-1
\end{array}
\end{array}
\label{eqn:oneComplementCorrectExample}
\end{equation}
Het grote nadeel bij deze implementatie is dat we de optelling-aftrekking dus in twee tijden moeten uitvoeren: eerst tellen we ze op en bepalen we de hoogste carry $c_n$, vervolgens tellen we deze carry nog eens bij het resultaat op. Dit veroorzaakt dus een verdubbeling van de vertraging. Figuur \ref{fig:onecomplement} geeft het hele proces schematisch weer.
\subsubsection{2-complement voorstelling}
De 2-complement voorstelling is de beste voor de implementatie van een opteller-aftrekker. Immers kunnen we een optelling eenvoudigweg uitvoeren zoals we die reeds hebben beschreven in subsectie \ref{ss:add}. Bij een optelling dienen we dus eenvoudigweg de twee operatoren optellen: $B_r=B_1+B_2$. Bij een aftrekking moeten we dan eenvoudigweg de negatie van de tweede operand berekenen. De berekening wordt in dat geval $B_r=B_1+B_2'+1$. Deze optelling lijkt misschien tot hetzelfde probleem te leiden als bij de 1-complement voorstelling. Maar we kunnen eenvoudigweg de carry van de laagste bit $c_0$ op 1 zetten. We kunnen vervolgens $B_2'$ berekenen met behulp van XOR poorten. Een XOR poort is dan ook een geprogrammeerde NOT poort. Indien \'e\'en van de ingangen van de XOR poorten 1 is, zal de andere de negatie zijn van de andere ingang. Indien de ingang 0 is, laat de XOR poort de andere ingang door. We kunnen dus een opteller-aftrekker realiseren zoals op figuur \ref{fig:addsub-twocomplement-implementation}.
\begin{figure}[hbt]
\centering
\subfigure[Interface]{\importtikz{addsub-twocomplement-interface}}
\subfigure[Mogelijke implementatie]{\importtikz{addsub-twocomplement-implementation}}
\caption{Opteller-aftrekker voor 2-complement getallen.}
\figlab{adderSubTwoComplement}
\end{figure}
Indien $s=0$ tellen we de twee getallen met elkaar op, anders trekken we de twee getallen van elkaar af. Door een XOR poort op de laatste twee carry-uitgangen $c_{n-1}$ en $c_n$ te plaatsen, kunnen we een controle op overflow inbouwen. Deze component wordt meestal samengevat tot een interface zoals op figuur \ref{fig:addsub-twocomplement-interface}\footnote{Bemerk op de figuur dat variabelen met hoofdletters een rij van in- of uitgangen voorstellen.}. Een schematische werkwijze van optellen en aftrekken met 2-complement voorstelling staat op figuur \ref{fig:twocomplement}. De 2-complement voorstelling is dan ook populair omdat deze weinig bewerkingen en testen vraagt om sommen en verschillen te berekenen. Bovendien is er weinig extra hardware nodig om een opteller om te bouwen.
\subsection{Arithmetic-Logic Unit (ALU)}
\ssclab{alu}
\importtikzfigure{alu-structure}{Schematisch implementatie van een arithmetic-logic unit (ALU).}
Hoewel veel processoren in staat zijn om sommen en verschillen te berekenen zal men meestal geen directe opteller of opteller-aftrekker vinden. Meestal gebruikt men hiervoor een \termen{Arithmetic-Logic Unit (ALU)}. Een ALU is een component die, gebaseerd op een opteller, allerlei instructies\footnote{Welke instructies is niet echt gespecificeerd. Er bestaan dan ook boeken over het ontwerpen van een goede ALU.} kan uitvoeren. In ons geval beschouwen we 4 rekenkundige (optellen, aftrekken, \termen{increment} en \termen{decrement}) en 4 logische bewerkingen (AND, OR, NOT en identiteit). We bouwen een ALU op ongeveer dezelfde manier zoals we een opteller-aftrekker bouwden uit een opteller. In plaats van XOR poorten gebruiken we een nieuw component: een \termen{Arithmetic-Logic Extender (ALE)}. Een ander component -- de CIG -- berekent welke ingang aan de carry $c_0$ moet worden gegeven. In het algemeen heeft een ALU dus een structuur zoals op figuur \ref{fig:alu-structure} waarbij de ALE en CIG vrij ge\"implementeerd kunnen worden.
\subsubsection{Instructieset}
\label{sss:aLUInstructionSet}
Alvorens we in staat zijn een ALU te maken moeten we een instructieset defini\"eren. Onze instructieset heeft drie\footnote{Omdat we 8 opdrachten gedefinieerd hebben, hebben we een $\log_28=3$ bit instructiewoord nodig.} ingangssignalen die bepalen welke opdracht er moet worden uitgevoerd. Bovendien zijn de 8 opdrachten in te delen in 2 groepen van 4: aritmetisch en logisch. Daarom noemen we de eerste bit van het instructiewoord $m$ voor mode\footnote{0=logisch, 1=aritmetisch.}. Verder wijzen we dan elke opdracht toe aan een bepaald instructiewoord zoals in tabel \ref{tbl:aLUInstructionSet}.
\begin{table}[hbt]
\centering
\begin{tabular}{ccc|c|ccc|l}
$m$&$i_1$&$i_0$&$F$&$X$&$Y$&$c_0$&\\\hline
0&0&0&$A'$&$A'$&0&0&NOT\\
0&0&1&$A\mbox{ AND }B$&$A\mbox{ AND }B$&0&0&AND\\
0&1&0&$A$&$A$&0&0&Identiteit\\
0&1&1&$A\mbox{ OR }B$&$A\mbox{ OR }B$&0&0&OR\\
1&0&0&$A-1$&$A$&-1&0&Decrement\\
1&0&1&$A+B$&$A$&$B$&0&Optelling\\
1&1&0&$A-B$&$A$&$B'$&1&Aftrekking\\
1&1&1&$A+1$&$A$&$0$&1&Increment\\
\end{tabular}
\caption{Instructieset van een typische arithmetic-logic unit (ALU).}
\tbllab{aLUInstructionSet}
\end{table}
Gebaseerd op deze instructieset moeten we een ALE en CIG implementeren. De ALE implementeert een functie die $\left(A,B\right)$-waardes afbeeldt op $\left(X,Y\right)$-waardes. Deze laatste waardes dienen als invoer  voor de optellers. De CIG ten slotte geeft gebaseerd op het instructiewoord een waarde voor $c_0$. Deze functies staan ook in de instructietabel.
\subsubsection{Synthese van de ALE en CIG}
\begin{figure}[hbt]
\centering
\subfigure[Karnaugh-kaarten]{\importtikz{ale-cig-karnaugh}}
\subfigure[Implementatie ALE]{\importtikz{ale-implementation}}
\subfigure[Implementatie CIG]{\importtikz{cig-implementation}}
\caption{Synthese van de ALE en CIG}
\end{figure}
Eenmaal we de instructieset hebben, dienen we enkel nog de ALE en CIG te implementeren. Een ALE is een component met als ingangen het instructiewoord en van elke operand een bit. In ons geval wordt dit dus $\left(m,i_1,i_0,a_j,b_j\right)$. Als uitgangen hebben we de ingangen van \'e\'en van de optellers $\left(x_j,y_j\right)$. We zijn dus in staat om een waarheidstabel en Karnaugh-kaart op te stellen van de ALE. Zoals op figuur \ref{fig:ale-cig-karnaugh}. Vervolgens is het enkel een kwestie van implementatie. Op basis van de Karnaugh-kaaren hebben we op figuur \ref{fig:ale-implementation} een AND-OR implementatie gesynthetiseerd.
\paragraph{}
De CIG is enkel afhankelijk van het instructiewoord. Met de instructieset op tabel \ref{tbl:aLUInstructionSet} is dit $\left(m,i_1,i_0\right)$. Een CIG heeft slechts \'e\'en uitgang: carry $c_0$. Op basis van deze instructieset staat op figuur \ref{fig:ale-cig-karnaugh} de bijbehorende Karnaugh-kaart. Deze blijkt een eenvoudige AND-operatie te zijn zoals op figuur \ref{fig:cig-implementation}.
\subsection{Vermenigvuldigen}
\begin{figure}[htb]
\centering
\subfigure[$1\times 1$-bit]{\importtikz{mul1x1}}
\subfigure[$3\times 2$-bit]{\importtikz{mul3x2}}
\subfigure[$n\times m$-bit]{\importtikz{mulmxn}}
\caption{Parallelle vermenigvuldigers.}
\figlab{parallelMultipliers}
\end{figure}
In deze subsectie bouwen we een vermenigvuldiger gebaseerd op de manier hoe men met behulp van cijferen twee getallen vermenigvuldigt. We zullen hierbij enkel natuurlijke getallen beschouwen, later breiden we dit uit naar gehele getallen. Vergelijking (\ref{eqn:multiplyNormal}) toont de vermenigvuldiging van twee decimale getallen:
\begin{equation}
\begin{array}{lcccccc}
&&&1&4&2&5\\
\times&&&&3&6&5\\\hline
&&&7&1&2&5\\
&&8&5&5&0&\\
+&4&2&7&5&&\\\hline
&5&2&0&1&2&5
\end{array}
\label{eqn:multiplyNormal}
\end{equation}
We kunnen dus twee getallen vermenigvuldigen door het eerste getal telkens te vermenigvuldigen met \'e\'en van de cijfers van het onderste getal en na deze op de juiste manier te hebben uitgelijnd op te tellen. Deze methode werkt ook op binaire getallen zoals blijkt uit vergelijking (\ref{eqn:multiplyBinary}):
\begin{equation}
\begin{array}{lcccccc}
&&&1&0&1&1\\
\times&&&&1&0&1\\\hline
&&&1&0&1&1\\
&&0&0&0&0&\\
+&1&0&1&1&&\\\hline
&1&1&0&1&1&1
\end{array}
\label{eqn:multiplyBinary}
\end{equation}
Een groot voordeel van vermenigvuldigingen met binaire getallen is dat een cijfer slechts 1 of 0 kan zijn. Indien het 0 is tellen we niets op bij het resultaat, anders tellen we het eerste getal op, die voldoende naar links is opgeschoven. Deze realisatie noemt men een \termen{parallelle vermenigvuldiger}. Figuur \ref{fig:parallelMultipliers} toont enkele voorbeelden van parallelle vermenigvuldigers. Zo zien we het speciale geval van een $1\times 1$-bit, wat neerkomt op een AND-poort. De $3\times 2$-bit vermenigvuldiger toont een speciale realisatie van het cascaderende effect. Tot slot implementeren we ook het algemeen geval van een $m\times n$-vermenigvuldiger. Naast parallelle vermenigvuldigers bestaan er ook andere implementaties die sneller werken en goedkoper te realiseren zijn. De kostprijs om een getal met $m$ bits te vermenigvuldigen met een getal van $n$ bits is immers $\bigoh{nm}$. Met een sequenti\"ele schakeling kunnen we de kosten sterk reduceren en zelfs de vermenigvuldiging versnellen. Vermenigvuldigen met een macht van 2 kunnen we bovendien berekenen met schuifoperators die in subsectie \ref{ss:shiftoperators} aan bod komen.
\subsubsection{2-complement vermenigvuldiger}
Het uitbreiden van deze vermenigvuldiger naar gehele getallen met 2-complement voorstelling is geen sinecure. In de praktijk zal men dan ook eerst de getallen omzetten naar sign-magnitude voorstelling alvorens we de twee getallen vermenigvuldigen. In dat geval kunnen we eenvoudigweg de waardes van de twee getallen vermenigvuldigen en passen we een XOR operatie toe op de tekenbit van de getallen. Het resultaat zetten we vervolgens weer om naar 2-complement voorstelling. Er bestaat uiteraard wel een vermenigvuldiger voor een 2-complement voorstelling. Deze zullen we enkel op hoog niveau beschrijven door middel van een voorbeeld. We rekenen hierbij $-2\times -3$ uit op 4-bit voorstelling en het resultaat stellen we voor op een 8-bit voorstellingen zoals in onderstaande vergelijking:
\begin{equation}
\begin{array}{l|r}
 \begin{array}{lr}
&-2\\
\times&-3\\\hline
&+6
\end{array}&\begin{array}{lr}
&1110\\
\times&1101\\\hline
&00000110
\end{array}
\end{array}
\end{equation}
De oplossing bestaat eruit om $-2$ te zien in zijn binaire vorm als $-8+4+2=-2$. Elk van de bits dient afzonderlijk vermenigvuldigt te worden, en vervolgens opgeteld te worden zoals in onderstaande vermenigvuldiging:
\begin{equation}
\begin{array}{lcr}
\begin{array}{lrcr}
&2&\times&-3\\
+&4&\times&-3\\
+&-8&\times&-3\\\hline
&-2&\times&-3
\end{array}&\begin{array}{c}
\Rightarrow\\
\Rightarrow\\
\Rightarrow\\
\Rightarrow
\end{array}&\begin{array}{lr}
&1111\ 1010\\
+&1111\ 0100\\
+&0001\ 1000\\\hline
&0000\ 0110
\end{array}
\end{array}
\end{equation}
\subsection{Andere courante bewerkingen}
\subsubsection{Delen}
Analoog aan vermenigvuldigen kunnen we ook een \termen{deling} in een binaire voorstelling gebaseerd op cijferen. We geven een voorbeeld van een binaire deling maar vermelden geen details over het realiseren van een schakeling:
\begin{equation}
\begin{array}{r|l}
1011\ 1010&1110\\\hline
\underline{111\ 0}\textcolor{white}{000}&1101\\
100\ 1010&\\
\underline{11\ 10}\textcolor{white}{00}&\\
1\ 0010&\\
\underline{0\ 000}\textcolor{white}{0}&\\
1\ 0010&\\
\underline{11\ 10}&\\
100&
\end{array}
\end{equation}
Door dus herhaaldelijk het resterende deeltal te vergelijken met een verschuifde deler en indien deze groter is, deze af te trekken, kunnen we de rest en de het quoti\"ent berekenen. Het aantal cycli is dus ook tevens het aantal bits van het quoti\"ent. Deze manier van implementatie is dan ook de meest populaire, en wordt vaak gebruikt. Indien we delen door een macht van 2, kunnen we het getal berekenen met een schuifoperatie (zie subsectie \ref{ss:shiftoperators}).
\subsubsection{Modulo rekenen}
Een andere populaire berekening is \termen{modulo} (ook wel bekend als \termen{$\mod$}). Modulo-rekening is sterk gerelateerd aan, maar niet equivalent aan \termen{remainder}-rekenen (ofwel \termen{$\rem$}). We defini\"eren de remainder als:
\begin{equation}
a\rem b\equiv a-\left\lfloor a/b\right\rfloor\times b
\end{equation}
De modulo-operatie daarentegen houdt rekening met het teken van zowel $a$ als $b$:
\begin{equation}
a\mod b\equiv \left\{\begin{array}{lcl}
a\rem b&\ifun&a\cdot b>0\\
\left(a\rem b\right)+b&\ifun&a\cdot b<0
\end{array}\right.
\end{equation}
Net als bij andere operaties, heeft ook de modulo-operatie een speciaal geval bij een macht van 2: in dat geval kunnen we een AND operatie uitvoeren op het getal en de macht gedecrementeerd. Bijvoorbeeld:
\begin{equation}
1653\mod 16=1653\mbox{ AND }15=5
\end{equation}
\subsection{Vaste komma getallen}
We zullen het getal verder uitbreiden naar getallen met vaste komma in deze subsectie en getallen met vlottende komma in subsectie \ref{ss:floatingPoints}. Ten slotte zullen we nog enkele andere voorstellingen van gegevens bekijken in subsectie \ref{ss:otherRepresentations}.
\paragraph{}
\termen{Vaste komma voorstelling} (ofwel \termen{fixed point}) lost het probleem van het zetten van een komma op door gewoon een bit af te spreken waarna men een komma plaatst, deze positie staat vast. Bijgevolg is er geen voorstelling van de komma zelf nodig. Indien we dus bijvoorbeeld een 8-bit getal beschouwen, kunnen we de eerste vier bit voorstellen als het \termen{geheel deel}, de overige vier bit behoren dan tot het \termen{fractionele gedeelte}. Dit formaliseren we met de notatie \termen{$\fix\left<i,f\right>$} waarbij $i$ het aantal bits voorstellen behorend tot het gehele deel, en $f$ het aantal bits tot het fractionele gedeelte. Zo stelt $1110010$ in $\fix\left<4,3\right>$, $14.25$ voor. De eerste vier bits stellen immers $14$ voor, de laatste drie stellen $2$ voor, deze delen we vervolgens door $2^f$.
\subsubsection{Aantal bits voor fouteloze voorstelling}
In de vorige secties hebben we al aandacht besteed aan het aantal bits die we moeten reserveren om de resultaten van bewerkingen zonder verlies te kunnen blijven voorstellen. Bij een optelling van 2 vaste komma getallen moet het resultaat voorgesteld worden op een vaste komma voorstelling waarbij we het geheel en fractioneelgedeelte voorstellen met het maximum aantal bits van de operanden. Bovendien dienen we \'e\'en bit toe te voegen aan het geheel gedeelte, ofwel formeler:
\begin{equation}
\fix\left<i_1,f_1\right>+\fix\left<i_2,f_2\right>=\fix\left<\max\left(i_1,i_2\right)+1,\max\left(f_1,f_2\right)\right>
\end{equation}
Wat we verder kunnen veralgemenen tot:
\begin{equation}
\displaystyle\sum_{k=1}^{n}{\fix\left<i_k,f_k\right>}=\fix\left<\left\lceil\log_2n\right\rceil+\max_{k=1}^{n}{i_k},\max_{k=1}^{n}{f_k}\right>
\end{equation}
Bij vermenigvuldigingen moeten we het aantal bits optellen van zowel het gehele als fractionele gedeelte:
\begin{equation}
\fix\left<i_1,f_1\right>\times\fix\left<i_2,f_2\right>=\fix\left<i_1+i_2,f_1+f_2\right>
\end{equation}
Of algemener:
\begin{equation}
\displaystyle\prod_{k=1}^{n}{\fix\left<i_k,f_k\right>}=\fix\left<\displaystyle\sum_{k=1}^{n}{i_k},\displaystyle\sum_{k=1}^{n}{f_k}\right>
\end{equation}
\subsection{Vlottende komma getallen}
\label{ss:floatingPoints}
Het kan voorkomen dat we echter niet weten tot welke grootorde het getal in kwestie zal behoren. In dat geval zouden we vaste getallen met een groot aantal bits moeten gebruiken, waarbij bovendien in de meeste gevallen slechts een kleine hoeveelheid bits nuttig blijkt. In deze gevallen gebruiken we \termen{vlottende komma voorstelling} (ook wel \termen{zwevendekommagetal}, \termen{drijvendekommagetal} of \termen{floating-point number} genoemd). Vlottende komma getallen worden voorgesteld door een sequentie van bits die in drie groepen worden onderverdeeld:
\begin{itemize}
 \item \termen{sign-bit $s$}: 1-bit die het teken van het getal bepaalt. De vlottende kommavoorstelling is dus vergelijkbaar met de signed-magnitude voorstelling.
 \item \termen{exponent $E$}: een getal die bepaalt met hoeveel plaatsen het getal moet worden opgeschoven. Dit deel wordt voorgesteld in `\termen{excess}'-formaat.
 \item \termen{mantisse $M$}: het getal onafhankelijk van het schuiven van plaatsen. De voorstelling van de mantisse zelf is $\fix\left<1,m-1\right>$.
\end{itemize}
Het getal dat we hierbij willen voorstellen is gelijk aan:
\begin{equation}
N=\left(-1\right)^s\times\mbox{mantisse}\times r^{\mbox{\tiny{exponent}}}
\end{equation}
met radix $r$. Net als bij een vaste komma getal noteren we een familie van vlottende komma getallen als \termen{$\float\left<m,e\right>$} met $m$ het aantal bits van de mantisse en $e$ het aantal bits van de exponent. Verder onderscheiden we nog twee families van vlottende komma voorstellingen:
\begin{itemize}
 \item \termen{Genormaliseerde vlottende komma voorstelling}: hierbij leggen we extra constraint betreffende de mantisse: $1\leq\mbox{mantisse}<r$. Omdat in dat geval de eerste bit dus altijd gelijk is aan 1, behoren enkel de bits na de komma tot de voorstelling. In dat geval denken we er de 1 dus bij.
 \item \termen{Niet-genormaliseerde vlottende komma voorstelling}: hierbij is deze bovengenoemde constraint dus niet van toepassing. Niet-genormaliseerde vlottende komma getallen kunnen meer getallen voorstellen dan hun tegenhanger, maar hebben een groot nadeel: men kan een getal op verschillende manieren voorstellen: eenzelfde getal kan op verschillende manieren voorgesteld worden.
\end{itemize}
\subsubsection{Underflow en overflow}
Naast overflow introduceert de vlottende kommavoorstelling nog een bijkomend fenomeen: ``\termen{underflow}''. Underflow treed op op het moment dat een getal niet meer voorgesteld kan worden omdat de absolute waarde te klein geworden is. Figuur \ref{fig:floatingpoint-underflow} geeft dit fenomeen schematisch weer van een \texttt{Single} (zie volgende paragraaf).
\importtikzfigure{floatingpoint-underflow}{Underflow van een vlottende komma voorstelling.}
\subsubsection{IEEE-formaat voor vlottende komma getallen}
Nu we de vlottende komma voorstelling theoretisch beschreven hebben, zullen we het formaat gedefinieerd door IEEE bespreken. De floating point wordt beschreven in IEEE 754-1985\footnote{De opvolger van deze standaard is IEEE 754-2008\cite{5976968}. Hierbij worden ook halve en viervoudige precisie gedefinieerd.}\cite{30711}. Hierbij maakt met een onderscheid tussen vlottende komma getallen met \termen{enkele precisie} (32-bit en beter bekend onder de term \termen{\texttt{float}} of \termen{\texttt{Single}}) en \termen{dubbele precisie} (64-bit en beter bekend als \termen{\texttt{double}}). Tabel \ref{tbl:iEEEFloatingPointFormat} toont de verdeling van de beschikbare bits onder de mantisse en exponent.
\begin{table}[hbt]
\centering
\subtable[Bit-indeling]{\begin{tabular}{l|crrr}
Precisie&grootte&$e$&$m$&$B$\\\hline
\texttt{float}&32-bit&8&23&127\\
\texttt{double}&64-bit&11&52&1023\\
\end{tabular}
\tbllab{iEEEFloatingPointFormat}}
\subtable[Getalvoorstelling]{\begin{tabular}{l|ccc}
&$E=0$&$0<E<2^e-1$&$E=2^e-1$\\\hline
$M=0$&$0$&$\left(-1\right)^s\times\underline{1}.M\times2^{E-B}$&$\left(-1\right)^s\times\infty$\\
$M\neq 0$&$\left(-1\right)^s\times\underline{0}.M\times2^{1-B}$&$\left(-1\right)^s\times\underline{1}.M\times2^{E-B}$&NaN\\
\end{tabular}
\tbllab{iEEEFloatingNumberRepresentation}}
\caption{IEEE 754-1985 Floating Point.}
\tbllab{iEEEFloatingPoint}
\end{table}
Elk van deze formaten heeft uiteraard radix $r=2$. IEEE 754-1985 vermeldt ook hoe een vlottende komma moet worden uitgelezen. Tabel \ref{tbl:iEEEFloatingNumberRepresentation} geeft hierbij weer hoe we het getal moeten interpreteren afhankelijk van de waarde van de exponent $E$ en mantisse $M$. Bovendien wordt bij de IEEE 754 een extra variabele ingevoerd: \termen{excess-bias $B$}. Tot op heden is $B=2^{e-1}-1$, en is de range van de exponent dus van $-2^{e-1}+2$ tot $2^{e-1}-1$. Eventueel zou men later kunnen afwijken en bijvoorbeeld meer positieve exponenten dan negatieve kunnen toelaten. Verder definieert het formaat nog enkele speciale variabelen:
\begin{itemize}
 \item Nul (0): dit getal kan eigenlijk niet weergegeven worden met ons eerdere definitie van een genormaliseerde vlottende komma. De getalexperts bij Intel bedachten dan ook de regel dat indien $E=0$, we het getal in niet genormaliseerde getalvoorstelling zien.
 \item \termen{Negatief oneindig $-\infty$}/\termen{Positief oneindig $+\infty$}: Indien een operatie tot een overflow leidt, zal het resultaat worden voorgesteld als oneindig. Oneindig gedraagt zich ongeveer hetzelfde als zijn wiskundig equivalent.
 \item \termen{Not a Number (NaN)}: In tegenstelling tot gehele getallen die bij bijvoorbeeld een deling door 0 een fout\footnote{En een interrupt bij de meeste processoren.} veroorzaken, zullen ongeldige bewerkingen zoals delen door 0, en het verschil van oneindig en oneindig resulteren in Not a Number.
\end{itemize}
In tabel \ref{tbl:iEEEFloatingNumberRepresentation} zien we ten slotte ook dat enkele bits onderlijnd worden. Dit zijn de zogenaamde \termen{verborgen bits}. Deze bits zijn geen onderdeel van de mantisse, maar dienen we er wel denkbeeldig aan toe te voegen om het getal te kunnen interpreteren.
\subsubsection{Rekenen met vlottende komma}
In deze subsubsectie bespreken we kort hoe we kunnen rekenen met getallen met vlottende komma. De details kan men vinden in \cite[\S4]{hyde2004write} maar enkel een korte introductie behoort tot dit vak.
\paragraph{Optelling}
Het optellen van twee vlottende komma getallen verloopt in drie stappen:
\begin{enumerate}
 \item Eerst denormaliseren we de twee getallen door de exponenten gelijk te maken. Deze stap is geen sinecure omdat we bij dit proces zoveel mogelijk informatie willen blijven behouden (Indien we dus enkel bij \'e\'en van de twee getallen de exponent aanpassen, lopen we kans dat zijn mantisse alle informatie verliest).
 \item Vervolgens tellen we de twee mantisses met elkaar op.
 \item Het resultaat dienen we vervolgens weer te normaliseren.
\end{enumerate}
IEEE 754 definieert het formaat van een vlottende komma, de wiskundige operaties zelf behoren hier echter niet bij. Intel heeft bijvoorbeeld gepatenteerde technologie waarbij het getal eerst wordt omgezet naar een equivalent met grotere mantisse om meer precisie te garanderen. In de praktijk komt het er dus op neer dat IEEE 754 niet specificeert wat het resultaat moet zijn na een wiskundige bewerking.
\paragraph{Vermenigvuldiging}
Vermenigvuldigen is bij vlottende komma simpeler dan optellen:
\begin{enumerate}
 \item We vermenigvuldigen de mantisse zoals we dit hebben gedaan bij vermenigvuldigen van natuurlijke getallen. We passen een XOR-operatie toe op de tekenbit van de twee getallen, en we tellen de exponenten met elkaar op\footnote{Optellen bij het excess-formaat verloopt anders, we dienen immers \'e\'enmaal de excess-bias van de som af te trekken.}.
 \item We normaliseren het resultaat. Dit betekent dat ofwel de exponent hetzelfde blijft, ofwel met \'e\'en moet worden opgehoogd.
\end{enumerate}
\subsection{Andere voorstellingen van gegevens}
\label{ss:otherRepresentations}
Naast de hierboven beschreven voorstellingen voor natuurlijke-, gehele- en kommagetallen. Bestaan er nog andere voorstelling. We stellen er in deze subsectie twee voor: \termen{Binary Coded Decimal (BCD)} is een alternatief formaat om natuurlijke getallen voor te stellen. Dit kan eventueel uitgebreid worden tot gehele- en kommagetallen. De \termen{American Standard Code for Information Interchange (ASCII)} is een formaat dat hoofdzakelijk bedoelt is om tekst op te slaan.
\subsubsection{Binary Coded Decimal (BCD)}
In de vorige subsecties hebben we geen aandacht besteed aan het omzetten van een getal naar tekst. Deze omzetting is vrij arbeidsintensief en kost dus ook veel hardware. Bovendien stuiten we op nog een probleem: heel wat decimale kommagetallen kunnen niet voorgesteld worden in het binair getalstelsel. Zo bestaat er geen enkel binaire vorm die $0.3$ voorstelt. In dat geval moeten we het getal zo goed mogelijk benaderen\footnote{Het is echter niet omdat het binair stelsel er niet in slaagt om alle decimale getallen voor te stellen dat het minder accuraat is, integendeel: het binair stelsel is accurater.}. Een oplossing voor deze twee problemen is het Binary Coded Decimal (BCD) systeem. We kunnen elke decimaal cijfer voorstellen met 4-bit. We kunnen dus een decimaal getal voorstellen, door voor ieder cijfers zijn binair equivalent te gebruiken. Dit betekent dus dat we een decimaal getal met $n$ cijfers voorstellen met $4n$ bit. Elk van deze groepen bits kan dus alleen waardes van $0$ tot $9$ aannemen. Bijvoorbeeld: $1425_{10}=0101\_1001\_0001_2=0001\_0100\_0010\_0101_{\mbox{\tiny{BCD}}}$. Een groot nadeel van BCD voorstelling is dat bewerkingen heel wat complexe worden. Tabel \ref{tbl:bCDConversion} toont de omzetting van een decimaal cijfer naar het BCD equivalent.
\begin{table}[hbt]
\centering
\begin{tabular}{c|c}
Decimaal&BCD\\\hline
0&0000\\
1&0001\\
2&0010\\
3&0011\\
4&0100\\
5&0101\\
6&0110\\
7&0111\\
8&1000\\
9&1001
\end{tabular}
\caption{Decimale cijfers en hun BCD equivalent.}
\tbllab{bCDConversion}
\end{table}
\paragraph{Bewerkingen met BCD}
\begin{figure}[hbt]
\centering
\subfigure[Schematische voorstelling]{\importtikz{bcd-schematic}}
\subfigure[Karnaugh-kaart]{
\begin{tikzpicture}
\kkaarte{0}{0}{$c_o$}{$c$/$z_3$/$z_2$/$z_1$/$z_0$}{0/0/0/0/0/0/0/0/0/0/1/1/1/1/1/1/1/1/1/-/-/-/-/-/-/-/-/-/-/-/-/-}
\end{tikzpicture}
\figlab{bCDAdderKarnaugh}}
\subfigure[Implementatie vergelijker]{
\begin{tikzpicture}[circuit logic US]
\draw (-1,-1) rectangle (1,1);
\node[or gate,rotate=-90,inputs={normal,normal,normal},scale=0.75] (o0) at (-0.33,-0.5) {};
\node[and gate,rotate=180,scale=0.75] (a0) at (0.33,0) {};
\node[and gate,rotate=180,scale=0.75] (a1) at (0.33,0.5) {};
\draw (o0.input 1) |- (a0.output);
\draw (o0.input 2) |- (a1.output);
\draw (o0.input 3) -- (o0.input 3 |- 0,1.25) node[scale=0.75,anchor=south]{$c$};
\draw (1,-0.5) -- (1.25,-0.5) node[scale=0.75,anchor=west]{$z_0$};
\draw (a0.input 1) -- (a0.input 1 -| 1.25,0) node[scale=0.75,anchor=west]{$z_1$};
\draw (a1.input 1) -- (a1.input 1 -| 1.25,0) node[scale=0.75,anchor=west]{$z_2$};
\draw (a1.input 2) -- (a1.input 2 -| 1.25,0) node[scale=0.75,anchor=west]{$z_3$};
\draw (a0.input 2) -| (a1.input 2 -| 0.875,0);
\pdot{a1.input 2 -| 0.875,0}
\draw[->] (o0.output) -- (o0.output |- 0,-1.25) node[scale=0.75,anchor=north]{$c_o$};
\end{tikzpicture}
\figlab{bCDAdderComparator}}
\caption{Mogelijke implementatie van een BCD opteller.}
\figlab{bCDAdder}
\end{figure}
Ter illustratie zullen we tonen hoe we een opteller kunnen realiseren voor het BCD formaat. Hierbij streven we niet zozeer naar een minimale implementatie, maar tonen we dat een optelling in ieder geval complexer is dan bij zijn binair equivalent. Bij een optelling dienen we immers elke decimaal afzonderlijk op te tellen. Net zoals een half- en full adder bit per bit optelden, dienen we nu per decimaal een 4-bit opteller te realiseren. Indien het resultaat van deze optelling echter groter is dan 9, dienen we dit resultaat verder aan te passen, zodat het resultaat 10 minder is, en we de overdacht (carry) doorgeven. Voor dit eerste hebben we nogmaals een 4-bit opteller nodig. Verder hebben we een component nodig die controleert of het eerste resultaat groter is dan 9. Dit component heeft niet alleen een carry als uitgang, maar moet ook invoer genereren voor de tweede opteller. Indien het eerste resultaat immers groter is dan 9, tellen we er nog eens 6 bij op. Dit betekent dat bij een tussenresultaat van 10, de tweede opteller tot 16 komt wat dus gebaseerd op de laatste 4 bits in 0 resulteert. Figuur \ref{fig:bCDAdderSchematic} toont schematisch hoe we twee decimale cijfers kunnen optellen. Indien we deze structuur voor ieder cijfer herhalen, realiseren we een BCD opteller. De vergelijker synthetiseren we met behulp van Karnaugh-kaarten zoals op figuur \ref{fig:bCDAdderKarnaugh}. Merk op dat het resultaat van de eerst opteller hoogstens 18 is, en we dus voor de andere waarden don't cares kunnen gebruiken. Een mogelijke synthese van deze vergelijker staat op figuur \ref{fig:bCDAdderComparator}. Tot slot merken we op dat we bij het bepalen van het negatieve getal van een BCD niet het 2-complement moeten berekenen, maar het 10-complement. Ook deze stap vereist extra hardware.
\subsubsection{American Standard Code for Information Interchange (ASCII)}
\begin{table}[hbt]
\centering
\begin{tabular}{c|cccccccc}
&000&001&010&011&100&101&110&111\\\hline
0000	&\verb+NUL+	&\verb+DLE+	&\verb+SP+	&\verb+0+	&\verb+@+	&\verb+P+	&\verb+`+	&\verb+p+\\
0001	&\verb+SOH+	&\verb+DC1+	&\verb+!+	&\verb+1+	&\verb+A+	&\verb+Q+	&\verb+a+	&\verb+q+\\
0010	&\verb+STX+	&\verb+DC2+	&\verb+"+	&\verb+2+	&\verb+B+	&\verb+R+	&\verb+b+	&\verb+r+\\
0011	&\verb+ETX+	&\verb+DC3+	&\verb+#+	&\verb+3+	&\verb+C+	&\verb+S+	&\verb+c+	&\verb+s+\\

0100	&\verb+EOT+	&\verb+DC4+	&\verb+$+	&\verb+4+	&\verb+D+	&\verb+T+	&\verb+d+	&\verb+t+\\
0101	&\verb+ENQ+	&\verb+NAK+	&\verb+%+	&\verb+5+	&\verb+E+	&\verb+U+	&\verb+e+	&\verb+u+\\
0110	&\verb+ACK+	&\verb+SYN+	&\verb+&+	&\verb+6+	&\verb+F+	&\verb+V+	&\verb+f+	&\verb+v+\\
0111	&\verb+BEL+	&\verb+ETB+	&\verb+'+	&\verb+7+	&\verb+G+	&\verb+W+	&\verb+g+	&\verb+w+\\

1000	&\verb+BS+	&\verb+CAN+	&\verb+(+	&\verb+8+	&\verb+H+	&\verb+X+	&\verb+h+	&\verb+x+\\
1001	&\verb+HT+	&\verb+EM+	&\verb+)+	&\verb+9+	&\verb+I+	&\verb+Y+	&\verb+i+	&\verb+y+\\
1010	&\verb+LF+	&\verb+SUB+	&\verb+*+	&\verb+:+	&\verb+J+	&\verb+Z+	&\verb+j+	&\verb+z+\\
1011	&\verb+VT+	&\verb+ESC+	&\verb/+/	&\verb+;+	&\verb+K+	&\verb+[+	&\verb+k+	&\verb+{+\\

1100	&\verb+FF+	&\verb+FS+	&\verb+,+	&\verb+<+	&\verb+L+	&\verb+\+	&\verb+l+	&\verb+|+\\
1101	&\verb+CR+	&\verb+GS+	&\verb+-+	&\verb+=+	&\verb+M+	&\verb+]+	&\verb+m+	&\verb+}+\\
1110	&\verb+SO+	&\verb+RS+	&\verb+.+	&\verb+>+	&\verb+N+	&\verb+^+	&\verb+n+	&\verb+~+\\
1111	&\verb+SI+	&\verb+US+	&\verb+/+	&\verb+?+	&\verb+O+	&\verb+_+	&\verb+o+	&\verb+DEL+
\end{tabular}
\caption{ASCII standaard.}
\tbllab{aSCIIStandard}
\end{table}
Een standaard om tekst voor te stellen is de American Standard Code for Information Interchange (ASCII). ASCII reserveert 7 bit per karakter. Deze 7 bit zorgen voor 128 mogelijke karakters. De karakterset bestaat uit de Romaanse letters, Arabische cijfers, diverse leestekens en enkele functiesymbolen. De toewijzing van deze symbolen vertoont enige logica om bijvoorbeeld kleine letters naar hoofdletters om te zetten, en binaire getallen in hun ASCII equivalent. Tabel \ref{tbl:aSCIIStandard} toont de ASCII-karakters met hun bijbehorende binaire code. ASCII bevat geen ondersteuning voor cyrillisch, Arabisch,... Unicode is een standaard die 8, 16 of 32 bit per karakter reserveert om dit probleem op te lossen. Anno 2011 is $11\%$ van de binaire waarden toegewezen.
\section{Andere basisschakelingen}
\label{s:andereBasis}
Naast de rekenkundige schakelingen in sectie \ref{s:rekenkundig} zullen we in schema's ook geregeld enkele andere bouwstenen terugvinden. In deze sectie zullen we de meest voornaamste bespreken: In subsectie \ref{ss:multiplexer} bespreken we de multiplexer. Daarna bespreken we in \ref{ss:decoder} en \ref{ss:demultiplexer} twee verwante bouwstenen: de decoder en demultiplexer. Het omgekeerde van de decoder, de encoder behandelen we in \ref{ss:encoder}. We eindigen met de vergelijker in \ref{ss:comparator} en de reeds aangehaalde schuifoperatie in subsectie \ref{ss:shiftoperators}.
\subsection{Multiplexer}
\label{ss:multiplexer}
Een \termen{multiplexer}, \termen{selector} of \termen{MUX} is een component die bij $n$ \termen{selectie-ingangen $s_i$} en $2^n$ \termen{data-ingangen $d_i$}, de data-ingang met index $S$ op de uitgang zet. Hierbij is $S$ de waarde die voorgesteld wordt door de selectie-ingangen. In een blokschema wordt dit component voorgesteld door een trapezium, waarbij de data-ingangen aan de lange zijde staan, de uitgangen aan de korte zijde, en de selectie-ingangen aan \'e\'en van de schuine zijden zoals op figuur \ref{fig:multiplexerInterface}. Deze figuur toont een 4-naar-1 MUX. Eventueel worden aan de andere kant ook uitgangen toegevoegd, deze zijn identiek aan de selectie-ingangen aan de ene kant en dienen enkel om het blokschema overzichtelijker te maken. Figuren \ref{fig:multiplexerTruthTable} en \ref{fig:multiplexerSchema} tonen respectievelijk de eerder informeel besproken waarheidstabel en een mogelijke implementatie van dit component.
\begin{figure}[hbt]
\centering
\subfigure[Interface]{
  \begin{tikzpicture}
  \node[mux4to1] (I) at (0,0) {};
  \draw[<-] (I.selin0) -- (I.selin0 -| -1,0) node[scale=0.75,anchor=east]{$s_0$};
  \draw[<-] (I.selin1) -- (I.selin1 -| -1,0) node[scale=0.75,anchor=east]{$s_1$};
  \draw[->] (I.selout0) -- (I.selout0 -| 1,0) node[scale=0.75,anchor=west]{$s_0$};
  \draw[->] (I.selout1) -- (I.selout1 -| 1,0) node[scale=0.75,anchor=west]{$s_1$};
  \draw[->] (I.output) -- ++(0,-0.25) node[scale=0.75,anchor=north]{$f$};
  \draw[<-] (I.data0) -- ++(0,0.25) node[scale=0.75,anchor=south]{$d_0$};
  \draw[<-] (I.data1) -- ++(0,0.25) node[scale=0.75,anchor=south]{$d_1$};
  \draw[<-] (I.data2) -- ++(0,0.25) node[scale=0.75,anchor=south]{$d_2$};
  \draw[<-] (I.data3) -- ++(0,0.25) node[scale=0.75,anchor=south]{$d_3$};
  \end{tikzpicture}
  \figlab{multiplexerInterface}
}
\subfigure[Waarheidstabel] {
\begin{tikzpicture}
\draw (0,0) node[scale=0.95]{\begin{tabular}{cc|c}
$s_0$&$s_1$&$f$\\\hline
0&0&$d_0$\\
0&1&$d_1$\\
1&0&$d_2$\\
1&1&$d_3$
\end{tabular}};
\end{tikzpicture}
\figlab{multiplexerTruthTable}}
\subfigure[Mogelijke Implementatie]{
%\begin{tikzpicture}[circuit logic US,scale=1,xscale=0.8]
%\draw[dashed] (-3.5,2.25) -- (3.5,2.25) -- (2.25,-0.25) -- (-2.25,-0.25) -- cycle;
%\node[or gate, inputs={normal,normal,normal,normal},scale=0.7,rotate=-90] (O) at (0,0.25) {};
%\foreach\d/\dga/\dgb in {0/inverted/inverted,1/inverted/normal,2/normal/inverted,3/normal/normal} {
%  \node[and gate, inputs={\dga,\dgb,normal},scale=0.7,rotate=-90] (A\d) at (2.25-1.5*\d,1.25) {};
%  \draw (A\d.input 3) -- (A\d.input 3 |- 0,2.25);
%  \draw (A\d.input 1) -- (A\d.input 1 |- 0,1.75);
%  \draw (A\d.input 2) -- (A\d.input 2 |- 0,2);
%  \draw[dashed] (A\d.input 3 |- 0,2.25) -- ++(0,0.25) node[anchor=south]{$d_\d$};
%}
%\foreach\s/\g in {0/1,1/2} {
%  \draw[dashed] (-3-1/6-\s/6,1.75+0.25*\s) -- (-3.75,1.75+0.25*\s) node[anchor=east]{$s_\s$};
%  \draw (-3-1/6-\s/6,1.75+0.25*\s) -- (A0.input \g |- 0,1.75+0.25*\s);
%}
%\foreach\d/\g/\y in {0/1/0,1/2/1,2/3/1,3/4/0} {
%  \draw (A\d.output) -- ++(0,-0.25+\y*0.1) -| (O.input \g);
%}
%\draw (O.output) -- (O.output |- 0,-0.25);
%\draw[dashed] (O.output |- 0,-0.25) -- (O.output |- 0,-0.5) node[anchor=north]{$f$};
%\end{tikzpicture}
\begin{tikzpicture}[circuit logic US,scale=4]
\draw[dashed] (-0.5,-0.3) -- (-0.8,0.3) -- (0.8,0.3) -- (0.5,-0.3) -- cycle;
\node[or gate,scale=0.2*0.85,rotate=-90,inputs={normal,normal,normal,normal}] (O) at (0,-0.15) {};
\node[and gate,scale=0.2,rotate=-90,inputs={inverted,normal,inverted}] (A0) at (0.45,0.09) {};
\node[and gate,scale=0.2,rotate=-90,inputs={normal,normal,inverted}] (A1) at (0.16,0.09) {};
\node[and gate,scale=0.2,rotate=-90,inputs={inverted,normal,normal}] (A2) at (-0.16,0.09) {};
\node[and gate,scale=0.2,rotate=-90,inputs={normal,normal,normal}] (A3) at (-0.48,0.09) {};
\coordinate (Fdl) at (0,-0.3);
\coordinate (Sia) at (-0.6,-0.1);
\coordinate (Sib) at (-0.7,0.1);
\coordinate (Soa) at (0.6,-0.1);
\coordinate (Sob) at (0.7,0.1);
\coordinate (Dia) at (A0.output |- 0,0.3);
\coordinate (Dib) at (A1.output |- 0,0.3);
\coordinate (Dic) at (A2.output |- 0,0.3);
\coordinate (Did) at (A3.output |- 0,0.3);
\draw (Sib) |- (Sob |- 0,0.265) -- (Sob);
\draw (Sia) |- (Soa |- 0,0.225) -- (Soa);
\draw (Dia) -- (A0.input 2);
\draw (Dib) -- (A1.input 2);
\draw (Dic) -- (A2.input 2);
\draw (Did) -- (A3.input 2);
\draw[dashed] (Fdl) -- ++(0,-0.1) node[scale=0.75,anchor=north]{$f$};
\draw[dashed] (Dia) -- ++(0,0.1) node[scale=0.75,anchor=south]{$d_0$};
\draw[dashed] (Dib) -- ++(0,0.1) node[scale=0.75,anchor=south]{$d_1$};
\draw[dashed] (Dic) -- ++(0,0.1) node[scale=0.75,anchor=south]{$d_2$};
\draw[dashed] (Did) -- ++(0,0.1) node[scale=0.75,anchor=south]{$d_3$};
\draw (A0.input 1) -- (A0.input 1 |- 0,0.225);
\draw (A1.input 1) -- (A1.input 1 |- 0,0.225);
\draw (A2.input 1) -- (A2.input 1 |- 0,0.225);
\draw (A3.input 1) -- (A3.input 1 |- 0,0.225);
\draw (A0.input 3) -- (A0.input 3 |- 0,0.265);
\draw (A1.input 3) -- (A1.input 3 |- 0,0.265);
\draw (A2.input 3) -- (A2.input 3 |- 0,0.265);
\draw (A3.input 3) -- (A3.input 3 |- 0,0.265);
\draw[dashed] (Sib) -- (-0.9,0 |- Sib) node[scale=0.75,anchor=east]{$s_1$};
\draw[dashed] (Sia) -- (-0.9,0 |- Sia) node[scale=0.75,anchor=east]{$s_0$};
\draw[dashed] (Sob) -- (0.9,0 |- Sib) node[scale=0.75,anchor=west]{$s_1$};
\draw[dashed] (Soa) -- (0.9,0 |- Sia) node[scale=0.75,anchor=west]{$s_0$};
\draw (A3.output) -- (A3.output |- 0,-0.03) -| (O.input 4);
\draw (A2.output) -- (A2.output |- 0,-0.01) -| (O.input 3);
\draw (A1.output) -- (A1.output |- 0,-0.01) -| (O.input 2);
\draw (A0.output) -- (A0.output |- 0,-0.03) -| (O.input 1);
\draw (Fdl) -- (O.output);
\begin{scope}[scale=0.25]
\pdot{A0.input 1 |- 0,0.9}
\pdot{A1.input 1 |- 0,0.9}
\pdot{A2.input 1 |- 0,0.9}
\pdot{A3.input 1 |- 0,0.9}
\pdot{A0.input 3 |- 0,1.06}
\pdot{A1.input 3 |- 0,1.06}
\pdot{A2.input 3 |- 0,1.06}
\pdot{A3.input 3 |- 0,1.06}
\end{scope}
\end{tikzpicture}
\figlab{multiplexerSchema}}
\subfigure[Cascade]{
\begin{tikzpicture}[circuit logic US]
\node[mux4to1] (A) at (6,0) {};
\draw (A.selin1) -- (A.selin1 -| 1.5,0) node[scale=0.65,anchor=east]{$s_3$};
\draw (A.selin0) -- (A.selin0 -| 1.5,0) node[scale=0.65,anchor=east]{$s_2$};
\draw (A.output) -- ++(0,-0.25) node[scale=0.65,anchor=north]{$f$};
\foreach \x/\s in {0/0,1/1,2/1,3/0} {
  \node[mux4to1] (B\x) at (9-2*\x,1) {};
  \draw (B\x.output) -- ++(0,-0.25+0.125*\s) -| (A.data\x);
}
\foreach \x/\g/\i in {0/0/0,0/1/1,0/2/2,0/3/3,1/0/4,1/1/5,1/2/6,1/3/7,2/0/8,2/1/9,2/2/10,2/3/11,3/0/12,3/1/13,3/2/14,3/3/15} {
  \draw (B\x.data\g) -- ++(0,0.25) node[scale=0.65,anchor=south]{$d_{\i}$};
}
\foreach \x/\xa in {3/2,2/1,1/0} {
  \draw (B\x.selout0) -- (B\xa.selin0);
  \draw (B\x.selout1) -- (B\xa.selin1);
}
\draw (B3.selin1) -- (B3.selin1 -| 1.5,0) node[scale=0.65,anchor=east]{$s_1$};
\draw (B3.selin0) -- (B3.selin0 -| 1.5,0) node[scale=0.65,anchor=east]{$s_0$};
\end{tikzpicture}
\figlab{multiplexerCascade}}
\caption{Multiplexer.}
\figlab{multiplexer}
\end{figure}
\subsubsection{Cascade}
We kunnen uiteraard multiplexers met een verschillende parameter $n$ bouwen. Het probleem is echter dat deze waarden erg kunnen vari\"eren, wat weinig interessant is voor massaproductie. Daarom zal men veelal met een \termen{cascade} werken. Figuur \ref{fig:multiplexerCascade} toont een 16-naar-1 MUX gebouwd met behulp van 5 4-naar-1 multiplexers. We kunnen dit uiteraard veralgemenen: Indien we een $2^{n\times m}$-naar-1 multiplexer willen bouwen, kunnen we dit met $n$ niveaus van $2^m$-naar-1 multiplexers. Het aantal multiplexers die we in dat geval nodig hebben is:
\begin{equation}
\mbox{aantal multiplexers}=\displaystyle\frac{2^{n\times m}-1}{2^m-1}
\end{equation}
\subsection{Decoder}
\label{ss:decoder}
Een andere schakeling die sterk gerelateerd is aan een multiplexer is een \termen{decoder}. Een decoder beschikt over een \termen{enable-ingang $e$}, en $n$ \termen{adres-ingangen $a_i$}. De uitvoer bestaat uit $2^n$ \termen{selectie-uitgangen $s_i$}. Een decoder zal indien er een 1 aangelegd wordt op de enable-ingang $e$, een 1 op de uitgang zetten met index $A$. Hierbij staat $A$ voor de binaire waarde die voorgesteld wordt door de adres-ingangen $a_i$. In een blokschema wordt een decoder voorgesteld zoals op figuur \ref{fig:decoderInterface} door een rechthoek. In dit geval een 2-naar-4 decoder. Bovenaan staat de adres-ingangen $a_i$, aan de zijkant de enable-ingang $e$, en onderaan de uitgangen $s_i$.
\begin{figure}[hbt]
\centering
\subfigure[Interface]{
\begin{tikzpicture}
  \node[decoder2to4] (I) at (0,0) {Decoder};
  \draw[<-] (I.enable) -- ++(-0.25,0) node[scale=0.75,anchor=east]{$e$};
  \draw[->] (I.s0) -- ++(0,-0.25) node[scale=0.75,anchor=north]{$s_0$};
  \draw[->] (I.s1) -- ++(0,-0.25) node[scale=0.75,anchor=north]{$s_1$};
  \draw[->] (I.s2) -- ++(0,-0.25) node[scale=0.75,anchor=north]{$s_2$};
  \draw[->] (I.s3) -- ++(0,-0.25) node[scale=0.75,anchor=north]{$s_3$};
  \draw[<-] (I.a0) -- ++(0,0.25) node[scale=0.75,anchor=south]{$a_0$};
  \draw[<-] (I.a1) -- ++(0,0.25) node[scale=0.75,anchor=south]{$a_1$};
\end{tikzpicture}
\figlab{decoderInterface}
}
\subfigure[Waarheidstabel] {
\begin{tikzpicture}
\draw (0,0) node[scale=0.75]{\begin{tabular}{ccc|cccc}
$e$&$a_0$&$a_1$&$s_0$&$s_1$&$s_2$&$s_3$\\\hline
0&-&-&0&0&0&0\\
1&0&0&1&0&0&0\\
1&0&1&0&1&0&0\\
1&1&0&0&0&1&0\\
1&1&1&0&0&0&1\\
\end{tabular}};
\end{tikzpicture}
\figlab{decoderTruthTable}}
\subfigure[Mogelijke implementatie]{
\begin{tikzpicture}[circuit logic US,scale=3]
\draw[dashed] (-0.8,-0.3) rectangle (0.8,0.3);
\draw[black!30] (0,0) node[scale=3]{Decoder};
\coordinate (Ei) at (-0.8,0);
\coordinate (Aia) at (0.26666,0.3);
\coordinate (Aib) at (-0.26666,0.3);
\coordinate (Soa) at (0.48,-0.3);
\coordinate (Sob) at (0.16,-0.3);
\coordinate (Soc) at (-0.16,-0.3);
\coordinate (Sod) at (-0.48,-0.3);
\node[and gate,scale=0.27,rotate=-90,inputs={normal,inverted,inverted},anchor=output] (A0) at (Soa |- 0,-0.25) {};
\node[and gate,scale=0.27,rotate=-90,inputs={normal,normal,inverted},anchor=output] (A1) at (Sob |- 0,-0.25) {};
\node[and gate,scale=0.27,rotate=-90,inputs={normal,inverted,normal},anchor=output] (A2) at (Soc |- 0,-0.25) {};
\node[and gate,scale=0.27,rotate=-90,inputs={normal,normal,normal},anchor=output] (A3) at (Sod |- 0,-0.25) {};
\draw (Ei) -| (-0.7,0.25) -| (A0.input 1);
\draw (A1.input 1 |- 0,0.25) -- (A1.input 1);
\draw (A2.input 1 |- 0,0.25) -- (A2.input 1);
\draw (A3.input 1 |- 0,0.25) -- (A3.input 1);
\draw (A1.input 2 |- 0,0.2) -- (A1.input 2);
\draw (A2.input 2 |- 0,0.2) -- (A2.input 2);
\draw (A1.input 3 |- 0,0.15) -- (A1.input 3);
\draw (A2.input 3 |- 0,0.15) -- (A2.input 3);
\draw (Aia) -- (Aia |- 0,0.2);
\draw (Aib) -- (Aib |- 0,0.15);
\draw (A3.input 2) -- (A3.input 2 |- 0,0.2) -| (A0.input 2);
\draw (A3.input 3) -- (A3.input 3 |- 0,0.15) -| (A0.input 3);
\draw (A0.output) -- (Soa);
\draw (A1.output) -- (Sob);
\draw (A2.output) -- (Soc);
\draw (A3.output) -- (Sod);
\draw[dashed] (Soa) -- ++(0,-0.1) node[scale=0.75,anchor=north]{$s_0$};
\draw[dashed] (Sob) -- ++(0,-0.1) node[scale=0.75,anchor=north]{$s_1$};
\draw[dashed] (Soc) -- ++(0,-0.1) node[scale=0.75,anchor=north]{$s_2$};
\draw[dashed] (Sod) -- ++(0,-0.1) node[scale=0.75,anchor=north]{$s_3$};
\draw[dashed] (Aia) -- ++(0,0.1) node[scale=0.75,anchor=south]{$a_0$};
\draw[dashed] (Aib) -- ++(0,0.1) node[scale=0.75,anchor=south]{$a_1$};
\draw[dashed] (Ei) -- ++(-0.1,0) node[scale=0.75,anchor=east]{$e$};
\begin{scope}[scale=0.333]
\pdot{A1.input 1 |- 0,0.75}
\pdot{A2.input 1 |- 0,0.75}
\pdot{A3.input 1 |- 0,0.75}
\pdot{A1.input 2 |- 0,0.6}
\pdot{A2.input 2 |- 0,0.6}
\pdot{A1.input 3 |- 0,0.45}
\pdot{A2.input 3 |- 0,0.45}
\pdot{Aia |- 0,0.6}
\pdot{Aib |- 0,0.45}
\end{scope}
\end{tikzpicture}
\figlab{decoderSchema}}
\subfigure[Cascade]{
\begin{tikzpicture}[circuit logic US]
\node[decoder2to4] (A) at (0,0) {Decoder};
\draw (A.a0) -- ++(0,0.25) node[anchor=south,scale=0.75]{$a_2$};
\draw (A.a1) -- ++(0,0.25) node[anchor=south,scale=0.75]{$a_3$};
\draw (A.enable) -- ++(-0.25,0) node[anchor=east,scale=0.75]{$e$};
\foreach\x/\y in {0/1,1/0,2/0,3/1} {
  \node[decoder2to4] (B\x) at (3-2*\x,-1.5) {Decoder};
  \draw (B\x.a0) -- ++(0,0.125) node[anchor=south,scale=0.75]{$a_0$};
  \draw (B\x.a1) -- ++(0,0.125) node[anchor=south,scale=0.75]{$a_1$};
  \draw (B\x.enable) -- ++(-0.2,0) |- (A.s\x |- 0,-0.7+0.2*\y) -- (A.s\x);
}
\foreach \x/\y/\z in {0/0/0,0/1/1,0/2/2,0/3/3,1/0/4,1/1/5,1/2/6,1/3/7,2/0/8,2/1/9,2/2/10,2/3/11,3/0/12,3/1/13,3/2/14,3/3/15} {
  \draw (B\x.s\y) -- ++(0,-0.25) node[anchor=north,scale=0.75]{$s_{\z}$};
}
\end{tikzpicture}
\figlab{decoderCascade}}
\caption{Decoder.}
\figlab{decoder}
\end{figure}
Op figuur \ref{fig:decoderTruthTable} staat de waarheidstabel voor dit component. Figuur \ref{fig:decoderSchema} toont een mogelijke implementatie. Decoders worden hoofdzakelijk gebruikt voor het decoderen van adressen.
\subsubsection{Cascade}
Net als bij multiplexer kunnen we in plaats van een heel arsenaal aan decoders aan te bieden door middel van een cascade een nieuwe decoder bouwen, zoals op figuur \ref{fig:decoderCascade}. Hier bouwen we een 4-naar-16 decoder met 5 2-naar-4 decoders. In het algemeen kunnen we een $nm$-naar-$2^{nm}$ decoder bouwen met $n$ niveaus van $m$-naar-$2^m$ decoders. In totaal hebben we dus volgend aantal decoders nodig:
\begin{equation}
\mbox{aantal decoders}=\displaystyle\frac{2^{n\times m}-1}{2^m-1}
\end{equation}
\subsubsection{Alternatieve implementatie voor Multiplexers}
Decoders worden ook gebruikt voor de synthese van bijvoorbeeld multiplexers. Figuur \ref{fig:decoderMultiplexerAnd} toont een manier om op basis van AND-poorten en een decoder een multiplexer te bouwen. We kunnen echter de OR-poort weglaten en de AND-poorten vervangen door 3-state buffers zoals op figuur \ref{fig:decoderMultiplexerTriState}. In dat geval hebben we een schakeling gerealiseerd die we een \termen{bus} noemen. Bussen zijn multiplexers waarbij we toelaten dat de ingangen gedistribueerd zijn over de verschillende delen van de schakeling. Elk van deze ingangen dienen we eenvoudigweg een uitgang van de decoder toe toe te wijzen, en de uitgang met een 3-state buffer verbinden met de bus. Bussen hebben verschillende voordelen:
\begin{itemize}
 \item We kunnen makkelijk het aantal ingangen uitbreiden. We dienen enkel over een decoder met voldoende grote $n$ te beschikken en een tri-state buffer per ingang.
 \item Indien we de traditionele Multiplexer gebruiken zal bij een groot aantal ingangen de fan-in van de OR-poort toenemen. Bovendien moeten alle ingangen van de multiplexer dicht bij elkaar staan.
 \item 3-state buffers zijn meestal gratis op een FPGA. Elk logisch blok (LB) heeft immers minstens \'e\'en uitgang langs een 3-state buffer die verbonden is met een lange lijn.
\end{itemize}
\begin{figure}[hbt]
\centering
\subfigure[Muliplexer]{
\begin{tikzpicture}[circuit logic US,scale=4]
\coordinate (Fdl) at (0,-0.3);
\coordinate (Sia) at (-0.6,-0.1);
\coordinate (Sib) at (-0.7,0.1);
\draw[dashed] (-0.5,-0.3) -- (-0.8,0.3) -- (0.8,0.3) -- (0.5,-0.3) -- cycle;
\node[decoder2to4,scale=0.75,rotate=90] (D) at (-0.50,0) {Decoder};
\draw (D.enable) -- ++(0,-0.05) node[anchor=north,scale=0.75]{$1$};
\draw (D.a0) -| (Sia);
\draw (D.a1) -- ++(-0.025,0) |- (Sib);
\node[or gate,scale=0.2*0.85,rotate=-90,inputs={normal,normal,normal,normal}] (O) at (0,-0.15) {};
\node[and gate,scale=0.2,rotate=-90,inputs={normal,normal}] (A0) at (0.24,0.09) {};
\node[and gate,scale=0.2,rotate=-90,inputs={normal,normal}] (A1) at (0.08,0.09) {};
\node[and gate,scale=0.2,rotate=-90,inputs={normal,normal}] (A2) at (-0.08,0.09) {};
\node[and gate,scale=0.2,rotate=-90,inputs={normal,normal}] (A3) at (-0.24,0.09) {};
\coordinate (Dia) at (A0.input 1 |- 0,0.3);
\coordinate (Dib) at (A1.input 1 |- 0,0.3);
\coordinate (Dic) at (A2.input 1 |- 0,0.3);
\coordinate (Did) at (A3.input 1 |- 0,0.3);
\draw (Dia) -- (A0.input 1);
\draw (Dib) -- (A1.input 1);
\draw (Dic) -- (A2.input 1);
\draw (Did) -- (A3.input 1);
\draw (A0.input 2) |- (-0.41,0.27) |- (D.s0);
\draw (A1.input 2) |- (-0.38,0.24) |- (D.s1);
\draw (A2.input 2) |- (-0.35,0.21) |- (D.s2);
\draw (A3.input 2) |- (-0.32,0.18) |- (D.s3);
\draw[dashed] (Fdl) -- ++(0,-0.1) node[scale=0.75,anchor=north]{$f$};
\draw[dashed] (Dia) -- ++(0,0.1) node[scale=0.75,anchor=south]{$d_0$};
\draw[dashed] (Dib) -- ++(0,0.1) node[scale=0.75,anchor=south]{$d_1$};
\draw[dashed] (Dic) -- ++(0,0.1) node[scale=0.75,anchor=south]{$d_2$};
\draw[dashed] (Did) -- ++(0,0.1) node[scale=0.75,anchor=south]{$d_3$};
\draw[dashed] (Sib) -- (-0.9,0 |- Sib) node[scale=0.75,anchor=east]{$s_1$};
\draw[dashed] (Sia) -- (-0.9,0 |- Sia) node[scale=0.75,anchor=east]{$s_0$};
\draw (A3.output) -- (A3.output |- 0,-0.03) -| (O.input 4);
\draw (A2.output) -- (A2.output |- 0,-0.01) -| (O.input 3);
\draw (A1.output) -- (A1.output |- 0,-0.01) -| (O.input 2);
\draw (A0.output) -- (A0.output |- 0,-0.03) -| (O.input 1);
\draw (Fdl) -- (O.output);
\end{tikzpicture}
\figlab{decoderMultiplexerAnd}}
\subfigure[Bus]{
\begin{tikzpicture}[circuit logic US,scale=4]
\coordinate (Fdl) at (0,-0.3);
\coordinate (Sia) at (-0.6,-0.1);
\coordinate (Sib) at (-0.7,0.1);
\foreach \x in {0,1,2,3} {
  \fill[black!30] (0.18-0.16*\x,0.1) rectangle (0.30-0.16*\x,-0.27);
}
\draw[dashed] (-0.5,-0.3) -- (-0.8,0.3) -- (0.8,0.3) -- (0.5,-0.3) -- cycle;
\node[decoder2to4,scale=0.75,rotate=90] (D) at (-0.50,0) {Decoder};
\draw (D.enable) -- ++(0,-0.05) node[anchor=north,scale=0.75]{$1$};
\draw (D.a0) -| (Sia);
\draw (D.a1) -- ++(-0.025,0) |- (Sib);
\node[tris,scale=0.75,rotate=-90] (T0) at (0.24,0) {};
\node[tris,scale=0.75,rotate=-90] (T1) at (0.08,0) {};
\node[tris,scale=0.75,rotate=-90] (T2) at (-0.08,0) {};
\node[tris,scale=0.75,rotate=-90] (T3) at (-0.24,0) {};
\draw (T0.z) -- (T0.z |- 0,-0.25);
\draw (T1.z) -- (T1.z |- 0,-0.25);
\draw (T2.z) -- (T2.z |- 0,-0.25);
\draw (T3.z) -- (T3.z |- 0,-0.25);
\draw (Fdl |- 0,-0.25) -- (Fdl);
\draw[thick] (T0.z |- 0,-0.25) node[anchor=west,scale=0.75]{bus} -- (T3.z |- 0,-0.25);
\coordinate (Dia) at (T0.z |- 0,0.3);
\coordinate (Dib) at (T1.z |- 0,0.3);
\coordinate (Dic) at (T2.z |- 0,0.3);
\coordinate (Did) at (T3.z |- 0,0.3);
\draw (Dia) -- (T0.x);
\draw (Dib) -- (T1.x);
\draw (Dic) -- (T2.x);
\draw (Did) -- (T3.x);
\draw (T3.c) -- (T3.c -| -0.32,0) |- (D.s3);
\foreach \x in {0,1,2} {
  \draw (T\x.c) -| (0.16-0.16*\x,0.25-0.03*\x) -| (D.s\x -| -0.41+0.03*\x,0) -- (D.s\x);
}
\draw[dashed] (Fdl) -- ++(0,-0.1) node[scale=0.75,anchor=north]{$f$};
\draw[dashed] (Dia) -- ++(0,0.1) node[scale=0.75,anchor=south]{$d_0$};
\draw[dashed] (Dib) -- ++(0,0.1) node[scale=0.75,anchor=south]{$d_1$};
\draw[dashed] (Dic) -- ++(0,0.1) node[scale=0.75,anchor=south]{$d_2$};
\draw[dashed] (Did) -- ++(0,0.1) node[scale=0.75,anchor=south]{$d_3$};
\draw[dashed] (Sib) -- (-0.9,0 |- Sib) node[scale=0.75,anchor=east]{$s_1$};
\draw[dashed] (Sia) -- (-0.9,0 |- Sia) node[scale=0.75,anchor=east]{$s_0$};
\end{tikzpicture}
\figlab{decoderMultiplexerTriState}}
\caption{Multiplexer en bus gesynthetiseerd met decoders.}
\figlab{decoderMultiplexer}
\end{figure}
\subsection{Demultiplexer}
\label{ss:demultiplexer}
Het inverse van een $2^n$-naar-1 multiplexer is een 1-naar-$2^n$ \termen{demultiplexer} (ook wel \termen{demux} genoemd). Dit component bestaat dan ook logischerwijs uit 1 \termen{data-ingang $d$} en $n$ \termen{selectie-ingangen $a_i$}. Het component beschikt verder over $2^n$ uitgangen $s_i$. Logischerwijs zetten we de waarde van de data-ingang $d$ op de uitgang met de index die wordt voorgesteld door de selectie-ingangen. Een aandachtige lezer zal misschien al opgemerkt hebben dat dit probleem in wezen niet veel verschilt van de constructie van een decoder. Sterker nog, we hoeven niets aan te passen, het is alleen een kwestie van een andere interface. Figuur \ref{fig:demultiplexer} toont de interface van een demultiplexer en de equivalentie met een decoder. In tegenstelling tot de decoder zetten we de selectie-ingangen aan de zijkant en de data-ingang aan de bovenkant.
\begin{figure}[hbt]
\centering
\begin{tikzpicture}
\node[demux1to4] (DM) at (0,0) {Demux};
\draw (DM.a0) -- ++(-0.25,0) node[scale=0.75,anchor=east]{$a_0$};
\draw (DM.a1) -- ++(-0.25,0) node[scale=0.75,anchor=east]{$a_1$};
\draw (DM.s0) -- ++(0,-0.25) node[scale=0.75,anchor=north]{$s_0$};
\draw (DM.s1) -- ++(0,-0.25) node[scale=0.75,anchor=north]{$s_1$};
\draw (DM.s2) -- ++(0,-0.25) node[scale=0.75,anchor=north]{$s_2$};
\draw (DM.s3) -- ++(0,-0.25) node[scale=0.75,anchor=north]{$s_3$};
\draw (DM.data) -- ++(0,0.25) node[scale=0.75,anchor=south]{$d$};
\node (E) at (2,0) {$\equiv$};
\node[decoder2to4] (DC) at (4,0) {Decoder};
\draw (DC.a0) -- ++(0,0.25) node[scale=0.75,anchor=south]{$a_0$};
\draw (DC.a1) -- ++(0,0.25) node[scale=0.75,anchor=south]{$a_1$};
\draw (DC.s0) -- ++(0,-0.25) node[scale=0.75,anchor=north]{$s_0$};
\draw (DC.s1) -- ++(0,-0.25) node[scale=0.75,anchor=north]{$s_1$};
\draw (DC.s2) -- ++(0,-0.25) node[scale=0.75,anchor=north]{$s_2$};
\draw (DC.s3) -- ++(0,-0.25) node[scale=0.75,anchor=north]{$s_3$};
\draw (DC.enable) -- ++(-0.25,0) node[scale=0.75,anchor=east]{$d$};
\end{tikzpicture}
\caption{Demultiplexer}
\figlab{demultiplexer}
\end{figure}
\paragraph{}
Demultiplexers worden maar zeer zelden gebruikt, kun doel is immers om data op een bepaalde lijn te plaatsen, terwijl men op de andere lijnen een 0 aanlegt. In de praktijk is het aanleggen van de data op alle lijnen meestal geen probleem. In dat geval kunnen we dus de demultiplexer eenvoudigweg vervangen door een draad die de data-ingang met alle uitgangen verbindt.
\subsection{Encoder}
\label{ss:encoder}
Ook de decoder heeft een invers: een $2^n$-naar-$n$ \termen{encoder}. Een encoder bevat $2^n$ \termen{data-ingangen $d_i$}. Het is de bedoeling dat de encoder afhankelijk van de lijn waarop een 1 staat de index weergeeft op de \termen{selectie-uitgangen $s_i$}. Verder bevat de encoder ook nog een extra uitgang: de \termen{any-uitgang $a$}, op deze uitgang wordt 1 aangelegd indien \'e\'en van de data-ingangen een 1 vertoont, anders wordt er een 0 op de any-uitgang aangelegd. We kunnen dit gedrag samenvatten in een waarheidstabel zoals in tabel \ref{tbl:truthTableEncoderNormal} voor een 4-naar-2 encoder.
\begin{table}[hbt]
\centering
\subtable[Encoder]{
\begin{tabular}{cccc|ccc}
$d_3$&$d_2$&$d_1$&$d_0$&$a$&$f_1$&$f_0$\\\hline
0&0&0&0&0&-&-\\
0&0&0&1&1&0&0\\
0&0&1&0&1&0&1\\
0&1&0&0&1&1&0\\
1&0&0&0&1&1&1\\
\end{tabular}
\tbllab{truthTableEncoderNormal}
}
\subtable[Prioriteitsencoder]{
\begin{tabular}{cccc|ccc}
$d_3$&$d_2$&$d_1$&$d_0$&$a$&$f_1$&$f_0$\\\hline
0&0&0&0&0&-&-\\
0&0&0&1&1&0&0\\
0&0&1&-&1&0&1\\
0&1&-&-&1&1&0\\
1&-&-&-&1&1&1\\
\end{tabular}
\tbllab{truthTablePriorityEncoder}
}
\caption{Waarheidtabellen van een encoder en prioriteitsencoder.}
\end{table}
Merk op dat de tabel niet alle strikt mogelijke ingangen toont. We nemen aan dat aan de ingang enkel geldige toestanden verschijnen. Indien dit niet zo is, staat het in principe vrij om elke uitgang aan te leggen. De ontbrekende rijen bevatten dus don't cares op alle uitgangen. Op basis van deze tabel kunnen we eventueel Karnaugh-kaarten maken en een implementatie voorstellen:
\begin{equation}
\left\{
\begin{array}{l}
a=d_3+d_2+d_1+d_0\\
f_1=d_3+d_1\\
f_0=d_3+d_2
\end{array}
\right.
\end{equation}
We kunnen deze schakeling dus eenvoudig realiseren met \'e\'en OR-poort per uitgang.
\paragraph{Prioriteitsencoder}
Een variant van de encoder is de \termen{prioriteitsencoder}. Een prioriteitsencoder biedt een antwoord op de ongeldige ingangen van de encoder. Hierbij telt niet d\'e index van de data-ingang waarop een 1 wordt aangelegd. Maar de hoogste index van alle datalijnen met 1. Tabel \ref{tbl:truthTablePriorityEncoder} formaliseert dit. Deze schakeling kunnen we als volgt implementeren:
\begin{equation}
\left\{
\begin{array}{l}
a=d_3+d_2+d_1+d_0\\
f_1=d_3+d_2'd_1\\
f_0=d_3+d_2
\end{array}
\right.
\end{equation}
Wat overeenkomt met \'e\'en extra AND- en NOT-poort. De interface voor beide schakelingen staat beschreven op figuur \ref{fig:encoderInterface}.
\begin{figure}[hbt]
\centering
\subfigure[Interfaces]{
\begin{tikzpicture}
\node[encoder4to2] (E) at (0,0) {Encoder};
\draw[<-] (E.d0) -- ++(0,0.25) node[anchor=south,scale=0.75]{$d_0$};
\draw[<-] (E.d1) -- ++(0,0.25) node[anchor=south,scale=0.75]{$d_1$};
\draw[<-] (E.d2) -- ++(0,0.25) node[anchor=south,scale=0.75]{$d_2$};
\draw[<-] (E.d3) -- ++(0,0.25) node[anchor=south,scale=0.75]{$d_3$};
\draw[->] (E.f0) -- ++(0,-0.25) node[anchor=north,scale=0.75]{$f_0$};
\draw[->] (E.f1) -- ++(0,-0.25) node[anchor=north,scale=0.75]{$f_1$};
\draw[->] (E.any) -- ++(-0.25,0) node[anchor=east,scale=0.75]{$a$};
\node[encoder4to2,text width=1.6 cm] (PE) at (3,0) {Prioriteits-encoder};
\draw[<-] (PE.d0) -- ++(0,0.25) node[anchor=south,scale=0.75]{$d_0$};
\draw[<-] (PE.d1) -- ++(0,0.25) node[anchor=south,scale=0.75]{$d_1$};
\draw[<-] (PE.d2) -- ++(0,0.25) node[anchor=south,scale=0.75]{$d_2$};
\draw[<-] (PE.d3) -- ++(0,0.25) node[anchor=south,scale=0.75]{$d_3$};
\draw[->] (PE.f0) -- ++(0,-0.25) node[anchor=north,scale=0.75]{$f_0$};
\draw[->] (PE.f1) -- ++(0,-0.25) node[anchor=north,scale=0.75]{$f_1$};
\draw[->] (PE.any) -- ++(-0.25,0) node[anchor=east,scale=0.75]{$a$};
\end{tikzpicture}
\figlab{encoderInterface}
}
\subfigure[Cascade]{
\begin{tikzpicture}
\foreach \x in {0,1,2,3} {
  \node[encoder4to2,text width=1.6 cm] (Ea\x) at (-3*\x,0) {Prioriteits-encoder};
}
\node[encoder4to2,text width=1.6 cm] (Eb3) at (-9,-3) {Prioriteits-encoder};
\node[mux4to1] (Mb2) at (-6,-3) {};
\node[mux4to1] (Mb1) at (-3,-3) {};
\foreach \x/\y/\z in {0/0/0,1/1/1,2/2/1,3/1/0} {
  \draw (Ea\x.any) -| (-3*\x-1.4,-2.4+0.1*\y) -| (Eb3.d\x);
  \draw (Ea\x.f1) -- (Ea\x.f1 |- 0,-1.6+0.1*\z) -| (Mb2.data\x);
  \draw (Ea\x.f0) -- (Ea\x.f0 |- 0,-1+0.1*\z) -| (Mb1.data\x);
}
\draw[<-] (Ea0.d0) -- ++(0,0.25) node[scale=0.75,anchor=south]{$d_0$};
\draw[<-] (Ea0.d1) -- ++(0,0.25) node[scale=0.75,anchor=south]{$d_1$};
\draw[<-] (Ea0.d2) -- ++(0,0.25) node[scale=0.75,anchor=south]{$d_2$};
\draw[<-] (Ea0.d3) -- ++(0,0.25) node[scale=0.75,anchor=south]{$d_3$};
\draw[<-] (Ea1.d0) -- ++(0,0.25) node[scale=0.75,anchor=south]{$d_4$};
\draw[<-] (Ea1.d1) -- ++(0,0.25) node[scale=0.75,anchor=south]{$d_5$};
\draw[<-] (Ea1.d2) -- ++(0,0.25) node[scale=0.75,anchor=south]{$d_6$};
\draw[<-] (Ea1.d3) -- ++(0,0.25) node[scale=0.75,anchor=south]{$d_7$};
\draw[<-] (Ea2.d0) -- ++(0,0.25) node[scale=0.75,anchor=south]{$d_8$};
\draw[<-] (Ea2.d1) -- ++(0,0.25) node[scale=0.75,anchor=south]{$d_9$};
\draw[<-] (Ea2.d2) -- ++(0,0.25) node[scale=0.75,anchor=south]{$d_{10}$};
\draw[<-] (Ea2.d3) -- ++(0,0.25) node[scale=0.75,anchor=south]{$d_{11}$};
\draw[<-] (Ea3.d0) -- ++(0,0.25) node[scale=0.75,anchor=south]{$d_{12}$};
\draw[<-] (Ea3.d1) -- ++(0,0.25) node[scale=0.75,anchor=south]{$d_{13}$};
\draw[<-] (Ea3.d2) -- ++(0,0.25) node[scale=0.75,anchor=south]{$d_{14}$};
\draw[<-] (Ea3.d3) -- ++(0,0.25) node[scale=0.75,anchor=south]{$d_{15}$};
\draw[->] (Eb3.any) -- ++(-0.25,0) node[scale=0.75,anchor=east]{$a$};
\draw[->] (Eb3.f1) -- (Eb3.f1 |- 0,-4.5) node[scale=0.75,anchor=north]{$f_3$};
\draw[->] (Eb3.f0) -- (Eb3.f0 |- 0,-4.5) node[scale=0.75,anchor=north]{$f_2$};
\draw[->] (Mb2.output) -- (Mb2.output |- 0,-4.5) node[scale=0.75,anchor=north]{$f_1$};
\draw[->] (Mb1.output) -- (Mb1.output |- 0,-4.5) node[scale=0.75,anchor=north]{$f_0$};
\draw (Eb3.f1 |- 0,-3.8) -- (-4.5,-3.8) |- (Mb1.selin1);
\draw (-7.5,-3.8) |- (Mb2.selin1);
\draw (-7.3,-4) |- (Mb2.selin0);
\draw (Eb3.f0 |- 0,-4) -- (-4.3,-4) |- (Mb1.selin0);
\pdot{Eb3.f1 |- 0,-3.8};
\pdot{-7.5,-3.8};
\pdot{-7.3,-4};
\pdot{Eb3.f0 |- 0,-4};
\end{tikzpicture}
\figlab{encoderCascade}
}
\caption{Encoder en Prioriteitsencoder.}
\end{figure}
De interface is opnieuw een rechthoek. De data-ingangen staan bovenaan. De selectie-uitgangen onderaan, en de any-uitgang aan de linkerkant.
\subsubsection{Prioriteitsencoders cascaderen}
Ook bij prioriteitsencoders gaan we even in het op het cascaderend karakter. Ook encoders laten zich eenvoudig cascaderen, mits we ook extra multiplexers gebruiken. In figuur \ref{fig:encoderCascade} bouwen we een 16-naar-4 prioriteitsencoder met 4 4-naar-2 prioriteitsencoders en 2 4-naar-1 multiplexers.
\subsection{Vergelijker}
\label{ss:comparator}
Een \termen{vergelijker} of \termen{comparator} is een component die in staat is om uitspraken te doen over hoe twee getallen zich tot elkaar verhouden. Meestal is een vergelijker in staat om een uitspraak te doen over 4 mogelijke verhoudingen van de getallen $X$ en $Y$:
\begin{equation}
\left\{\begin{array}{c}
X<Y\\
X>Y\\
X=Y\\
X\neq Y
\end{array}\right.
\end{equation}
Op basis van de eerste twee relaties, kunnen we uitspraken doen over de laatste twee. Op basis van deze vaststelling synthetiseren we een comparator. Een comparator bevat 4 ingangen ($x_0$ ofwel $g_{\mbox{\tiny{in}}}$, $y_0$ ofwel $l_{\mbox{\tiny{in}}}$, $x_1$ en $y_1$). De twee uitgangen ($g$ en $l$) geven respectievelijk weer of het getal $X>Y$ en $X<Y$. Figuur \ref{fig:comparatorInterface} toont de interface van een vergelijker.
\begin{figure}[hbt]
\centering
\subfigure[Interface]{
\begin{tikzpicture}
\node[comp] (C) at (0,0) {Comp};
\draw[<-] (C.x0) -- ++(0.25,0) node[anchor=west,scale=0.75]{$x_0/g_{\mbox{\tiny{in}}}$};
\draw[<-] (C.y0) -- ++(0.25,0) node[anchor=west,scale=0.75]{$y_0/l_{\mbox{\tiny{in}}}$};
\draw[<-] (C.x1) -- ++(0,0.25) node[anchor=south,scale=0.75]{$x_1$};
\draw[<-] (C.y1) -- ++(0,0.25) node[anchor=south,scale=0.75]{$y_1$};
\draw[->] (C.g) -- ++(-0.25,0) node[anchor=east,scale=0.75]{$g$};
\draw[->] (C.l) -- ++(-0.25,0) node[anchor=east,scale=0.75]{$l$};
\end{tikzpicture}
\figlab{comparatorInterface}
}
\subfigure[Karnaugh-kaarten]{
\begin{tikzpicture}
\kkaartd{0}{0}{$g$}{$x_1$/$y_1$/$x_0$/$y_0$}{0/0/1/0/0/0/0/0/1/1/1/1/0/0/1/0};
\kkaartd{3}{0}{$l$}{$x_1$/$y_1$/$x_0$/$y_0$}{0/1/0/0/1/1/1/1/0/0/0/0/0/1/0/0};
\end{tikzpicture}
\figlab{comparatorKarnaugh}
}
\subfigure[Lineare cascade]{
\begin{tikzpicture}[scale=0.75]
\node[comp,scale=0.75,draw=white] (C0) at (0,0) {};
\foreach \x in {1,...,7} {
  \node[comp,scale=0.75] (C\x) at (-1.5*\x,0) {Comp};
  \draw[<-] (C\x.x1) -- ++(0,0.25) node[anchor=south,scale=0.75]{$x_{\x}$};
  \draw[<-] (C\x.y1) -- ++(0,0.25) node[anchor=south,scale=0.75]{$y_{\x}$};
}
\foreach \x/\y in {1/2,2/3,3/4,4/5,5/6,6/7} {
  \draw (C\x.g) -- (C\y.x0);
  \draw (C\x.l) -- (C\y.y0);
}
\draw[<-] (C1.x0) -| (C0.x1) -- ++(0,0.25) node[anchor=south,scale=0.75]{$x_0$};
\draw[<-] (C1.y0) -| (C0.y1) -- ++(0,0.25) node[anchor=south,scale=0.75]{$y_0$};
\draw[->] (C7.g) -- ++(-0.25,0) node[anchor=east,scale=0.75]{$g$};
\draw[->] (C7.l) -- ++(-0.25,0) node[anchor=east,scale=0.75]{$l$};
\end{tikzpicture}
\figlab{comparatorCascadeLinear}
}
\subfigure[Hi\"erarchische cascade]{
\begin{tikzpicture}[scale=0.75]
\foreach \x in {1,3,5,7} {
  \node[comp,scale=0.75] (C\x) at (-1.5*\x,0) {Comp};
  \draw[<-] (C\x.x1) -- ++(0,0.25) node[anchor=south,scale=0.75]{$x_{\x}$};
  \draw[<-] (C\x.y1) -- ++(0,0.25) node[anchor=south,scale=0.75]{$y_{\x}$};
}
\foreach \x/\y in {0/1,2/3,4/5,6/7} {
  \node[comp,scale=0.75,draw=white] (C\x) at (-1.5*\x,0) {};
  \draw[<-] (C\y.x0) -| (C\x.x1) -- ++(0,0.25) node[anchor=south,scale=0.75]{$x_{\x}$};
  \draw[<-] (C\y.y0) -| (C\x.y1) -- ++(0,0.25) node[anchor=south,scale=0.75]{$y_{\x}$};
}
\foreach \x/\y in {3/1,7/5} {
  \node[comp,scale=0.75] (Cb\x) at (-1.5*\x,-2) {Comp};
  \draw (C\x.g) -| ++(-0.4,-1.2) -| (Cb\x.x1);
  \draw (C\x.l) -| ++(-0.2,-0.6) -| (Cb\x.y1);
  \draw (C\y.g) -- ++(-0.4,0) |- (Cb\x.x0);
  \draw (C\y.l) -- ++(-0.2,0) |- (Cb\x.y0);
}
\foreach \x/\y in {7/3} {
  \node[comp,scale=0.75] (Cc\x) at (-1.5*\x,-4) {Comp};
  \draw (Cb\x.g) -| ++(-0.4,-1.2) -| (Cc\x.x1);
  \draw (Cb\x.l) -| ++(-0.2,-0.6) -| (Cc\x.y1);
  \draw (Cb\y.g) -- ++(-0.4,0) |- (Cc\x.x0);
  \draw (Cb\y.l) -- ++(-0.2,0) |- (Cc\x.y0);
}
\draw[->] (Cc7.g) -- ++(-0.25,0) node[anchor=east,scale=0.75]{$g$};
\draw[->] (Cc7.l) -- ++(-0.25,0) node[anchor=east,scale=0.75]{$l$};
\end{tikzpicture}
\figlab{comparatorCascadeHierarchical}
}
\caption{Vergelijker}
\end{figure}
Op basis van de gedragsbeschrijving kunnen we een waarheidstabel en Karnaugh-kaarten opstellen zoals op figuur \ref{fig:comparatorKarnaugh}. Op basis van deze kaarten synthetiseren we volgende formules\footnote{Merk de dualiteit op: beide formules hebben dezelfde variabelen, alleen zijn alle atomen ge\"inverteerd.}:
\begin{equation}
\left\{\begin{array}{l}
g=x_1y_1'+x_0x_1y_0'+x_0y_0'y_1'\\
l=x_1'y_1+x_0'x_1'y_0+x_0'y_0y_1
\end{array}\right.
\end{equation}
\subsubsection{Vergelijkers cascaderen}
Hoe vergelijken we nu twee getallen die uit meer dan 2-bit bestaan? Net als bij een ripple-carry opteller, kunnen we door verschillende vergelijkers aan elkaar te schakelen, een vergelijker met een groter aantal bits bouwen. In dat geval bekomen we een structuur zoals op figuur \ref{fig:comparatorCascadeLinear}. Hiermee komen we echter tot hetzelfde probleem als bij een ripple-carry opteller: de vertraging schaalt linear met het aantal bits. We kunnen echter de vergelijkers ook in een hi\"erarchische structuur kunnen we de vertraging beperken tot een logaritmische orde zoals op figuur \ref{fig:comparatorCascadeHierarchical}.
\subsubsection{Speciale gevallen}
\paragraph{Testen op gelijkheid}
We kunnen twee getallen altijd vergelijken met een vergelijker zoals eerder beschreven. Soms willen we echter twee getallen testen op specifieke eigenschappen. Bijvoorbeeld of twee getallen gelijk zijn. Dit kunnen we natuurlijk realiseren met een vergelijker, maar dit introduceert extra hardware en vertraging. In dat geval kunnen we gebruik maken van een rij van XNOR-poorten waarbij elke poort \'e\'en bit van het ene en het andere getal als invoer krijgt. De uitgangen van al deze XNOR-poorten laten we vervolgens door een OR-poort gaan. De uitvoer van deze OR poort toont ons dan of $X$ gelijk is aan $Y$. Figuur \ref{fig:comparatorEquality} toont een realisatie van dit concept voor 2 $n$-bit getallen.
\begin{figure}[hbt]
\centering
\subfigure[Gelijkheid]{
\begin{tikzpicture}[circuit logic US]
\node[and gate,inputs={normal,normal,normal,normal,normal,normal,normal},scale=0.5,rotate=-90] (O) at (-3,-1.5) {};
\node[scale=0.75] (P) at (-5,0) {$\ldots$};
\draw (O.output) -- ++(0,-0.25) node[scale=0.75,anchor=north]{$X=Y$};
\foreach \x in {0,1,2,3,4,6} {
  \coordinate (y\x) at (0.25-\x,0.75);
  \coordinate (x\x) at (-0.25-\x,0.75);
}
\foreach \x/\y/\z/\t in {0/1/3/0,1/2/2/1,2/3/1/2,3/4/0/3,4/5/1/4,6/7/3/n-1} {
  \node[xnor gate,rotate=-90] (XN\x) at (-\x,0) {};
  \draw (XN\x.input 1) -- ++(0,0.2) -| (y\x) node[scale=0.75,anchor=south]{$y_{\t}$};
  \draw (XN\x.input 2) -- ++(0,0.2) -| (x\x) node[scale=0.75,anchor=south]{$x_{\t}$};
  \draw (XN\x.output) -- ++(0,-0.1*\z) -| (O.input \y);
}
\end{tikzpicture}
\figlab{comparatorEquality}}
\subfigure[Testen met constanten]{
\begin{tikzpicture}[circuit logic US]
\node[nor gate,inputs={normal,normal,normal,normal,normal}] (O1) at (0,0) {};
\draw (O1.output) -- ++(0.25,0) node[scale=0.75,anchor=west]{$X=0$};
\node[and gate,inputs={normal,normal,normal,normal,normal}] (O2) at (0,-1.5) {};
\draw (O2.output) -- ++(0.25,0) node[scale=0.75,anchor=west]{$X=2^n-1$};
\node[not gate] (O3) at (0,-3) {};
\draw (O3.input) -- ++(-0.25,0) node[scale=0.75,anchor=east]{$x_0$};
\draw (O3.output) -- ++(0.25,0) node[scale=0.75,anchor=west]{$\mbox{even}\left(X\right)$};
\foreach \x/\t in {5/$x_0$,4/$x_1$,3/$x_2$,2/$\ldots$,1/$x_{n-1}$} {
  \draw (O1.input \x) -- ++(-0.25,0) node[scale=0.75,anchor=east]{\t};
  \draw (O2.input \x) -- ++(-0.25,0) node[scale=0.75,anchor=east]{\t};
}
\node[or gate,inputs={normal,normal,normal,normal,normal}] (O4) at (4,0) {};
\draw (O4.output) -- ++(0.25,0) node[scale=0.75,anchor=west]{$X\geq 2^k$};
\node[nand gate,inputs={normal,normal,normal,normal,normal}] (O5) at (4,-1.5) {};
\draw (O5.output) -- ++(0.25,0) node[scale=0.75,anchor=west]{$X<2^n-2^k$};
\draw (O5.output |- 0,-3) -- (O5.input 1 |- 0,-3);
\draw (O5.input 1 |- 0,-3) -- ++(-0.25,0) node[scale=0.75,anchor=east]{$x_0$};
\draw (O5.output |- 0,-3) -- ++(0.25,0) node[scale=0.75,anchor=west]{$\mbox{oneven}\left(X\right)$};
\foreach \x/\t in {5/$x_k$,4/$x_{k+1}$,3/$x_{k+2}$,2/$\ldots$,1/$x_{n-1}$} {
  \draw (O4.input \x) -- ++(-0.25,0) node[scale=0.75,anchor=east]{\t};
  \draw (O5.input \x) -- ++(-0.25,0) node[scale=0.75,anchor=east]{\t};
}
\end{tikzpicture}
\figlab{comparatorConstants}}
\caption{Speciale gevallen van vergelijkers}
\end{figure}
\paragraph{Vergelijken met constanten}
We willen een getal niet altijd vergelijken met andere getal, maar soms met een constante. In dat geval zouden we natuurlijk ook van de vergelijker kunnen gebruikmaken, en de $Y$ zelf samenstellen. Aangezien $Y$ echter op voorhand gekend is spreekt het voor zichzelf dat we de implementatie echter grondig kunnen verbeteren. Bovendien kunnen we ook testen ontwerpen die we niet kunnen uitvoeren met een vergelijker\footnote{Bijvoorbeeld deelbaarheid door 2.}. Dergelijke componenten bouwen getuigd verder ook van enig inzicht. Alle speciale gevallen beschouwen is moeilijk. Op figuur \ref{fig:comparatorConstants} geven we enkele voorbeelden.
\subsection{Schuifoperator}
\ssclab{shiftoperators}
\label{ss:shiftoperators}
Een laatste set van operaties die vaak gebruikt wordt zijn \termen{schuifoperaties}. Vele processoren implementeren schuifoperaties en bijvoorbeeld de ARM processor laat toe om tijdens elke operatie de operand over enkele plaatsen te schuiven. Een eerste probleem met schuifoperaties is dat er verschillende vormen van schuifoperaties bestaan. In het algemeen houdt een schuifoperatie in dat de waarde van de bit die eerst op plaats $i$ stond, nu op een plaats $i+m$ staat. Of formeler: $Y$ is het resultaat van een schuifoperatie van $X$ over $m$ plaatsen naar links als
\begin{equation}
\forall i\in\NN:i\in\left[0,n\right]\wedge i+m\in\left[0,n\right]\Rightarrow y_{i+m}=x_i
\end{equation}
Het probleem is wat we doen met de bits die buiten de grenzen vallen, en met wat we de nieuwe plaatsen opvullen. Voor dit probleem bestaan drie populaire oplossingen:
\begin{itemize}
 \item Bij het \termen{schuiven} negeren we de bits die buiten de grenzen vallen. Er bestaan twee vormen van schuiven die vari\"eren in wat we op de vrijgekomen plaatsen zetten:
 \begin{itemize}
   \item \termen{Logisch schuiven}: indien we logisch schuiven hebben we een extra parameter nodig, namelijk welke waarde we op de nieuwe plaatsen zetten. Soms is dit zelfs een rij van bits. We maken dus de invoer groot genoeg om geen vrije plaatsen meer over te houden.
   \item \termen{Aritmetisch schuiven}: hierbij willen we eigenlijk een wiskundige functie realiseren namelijk vermenigvuldigen of delen met een macht van 2. Hoe we dus aritmetisch schuiven hangt vooral af van de getalvoorstelling. Indien we naar links schuiven betekent dit meestal dat we de nieuwe plaatsen vullen met nullen. Indien we naar rechts schuiven zal bij een 2-complement voorstelling de hoogste bit (MSB) van het originele getal ingevoegd worden. Bij een ``unsigned'' voorstelling vullen we de vrije plaatsen ook met nullen op. Aritmetisch schuiven is vrij populair in programmeertalen. Talen die behoren tot de \verb|C/C++/C#/Java| familie introduceren daarom 2 functies: voor links (\verb+<<+) en rechts (\verb+>>+) aritmetisch schuiven deze worden gedefinieerd als:
\begin{equation}
\left\{\begin{array}{l}
X\verb+<<+M\equiv X\times 2^M\\
X\verb+>>+M\equiv X\div 2^M
\end{array}\right.
\end{equation}
 \end{itemize}
 \item Bij het \termen{roteren} vangen we de bits op die er langs \'e\'en kant afvallen en plaatsen we deze op de vrijgekomen plaatsen aan de andere kant.
\end{itemize}
\subsubsection{Implementatie van schuifoperaties}
In deze subsubsectie zullen we twee schakelingen realiseren. De eerste is een component die alle schuifoperaties kan realiseren op 4-bit getallen. Maar slechts over 1 plaats naar links of rechts. Het tweede component is 8-bit \termen{barrel left rotator}. Dit component roteert 8-bit getallen naar links over een variabel aantal plaatsen. Merk op dat het bij een rotatie het eigenlijk niet uitmaakt in welke richting we roteren: een $n$-bit getal $m$ plaatsen naar links roteren is net hetzelfde als het getal $n-m$ plaatsen naar rechts roteren. Dit tweede component illustreert verder hoe we schuifoperaties effici\"ent over meerdere posities kunnen uitvoeren.
\paragraph{Schuifoperaties over 1 bit}
Indien we een component maken die meerdere operaties kan uitvoeren moeten we altijd eerst een instructieset defini\"eren, zoals in subsubsectie \ref{sss:aLUInstructionSet}. De instructieset dient zowel een onderscheid te maken tussen schuiven of roteren, aritmetisch of logisch\footnote{Enkel in geval van schuiven is dit relevant.} en links of rechts. Verder willen we ook een bit voorzien die aangeeft of er \"uberhaupt een schuifoperatie moet worden uitgevoerd. We zullen dus een 4-bit instructieset gebruiken. Tabel \ref{tbl:shiftInstructionSet} toont de betekenis van elke bit.
\begin{table}[hbt]
\centering
\begin{tabular}{c|cc}
Signaal&0&1\\\hline
$s_3$&geen schuifoperatie&schuifoperatie\\
$s_2$&links&rechts\\
$s_1$&schuiven&roteren\\
$s_0$&aritmetisch&logisch
\end{tabular}
\caption{Instructieset voor de schuifoperaties over 1 bit.}
\tbllab{shiftInstructionSet}
\end{table}
Op basis van deze instructieset kunnen we nu een component bouwen. Voor de berekening van de uitgangsbits gebruiken we multiplexers. Immers kan elke uitgang $y_i$ maar drie mogelijke uitgangen hebben: $x_{i-1}$, $x_i$ en $x_{i+1}$. Bij de uitgangen aan de rand dienen we alleen een andere interpretatie voor sommige $x_j$ te vinden. Welke van deze drie ingangen we kiezen hangt verder alleen af van twee bits uit het instructiewoord: $s_3$ en $s_2$. In het geval dat $s_3=0$ geldt $y_i=x_i$. Bijgevolg zetten we $x_i$ zowel op de $d_0$ en $d_1$ ingang van de multiplexer voor $y_i$. Indien $s_3=1$ en $s_2=0$ schuiven we naar links. We zetten dus $x_{i+1}$ en $x_{i-1}$ respectievelijk op de $d_2$ en $d_3$ ingangen. Vervolgens dienen we nog de randgevallen op te lossen. Deze randgevallen beslaan enkel $x_{i+1}$ voor $y_3$ en  $x_{i-1}$ voor $y_0$. Deze waarden zullen we respectievelijk noteren als $x_4$ en $x_{-1}$ en vallen dus buiten de grenzen van de ingangen. In geval van rotatie geldt:
\begin{equation}
\begin{array}{ll}
\left\{\begin{array}{l}
x_{-1}=x_3\\
x_4=x_0
\end{array}\right.&\mbox{(rotatie)}
\end{array}
\end{equation}
Dit dwingen we af met twee 2-naar-1 multiplexers. Deze multiplexers hebben als schakelelement instructiebit $s_1$. We dienen nu enkel nog het geval te behandelen waarin we schuiven. Indien we logisch schuiven, schuiven we de bits $L_{\mbox{\tiny{in}}}$ en $R_{\mbox{\tiny{in}}}$ in. Bij arithmetisch zullen we aan de rechterzijde een 0 inschuiven. Aan de linkerkant is dit ook het geval tenzij we schuiven met een 2-complement voorstelling. In dat laatste geval bepaalt de hoogste bit immers ook het teken van het getal. In dat geval moeten we de waarde van de hoogste bit dus nogmaals inschuiven. We kunnen bovenstaande beschrijvingen formaliseren tot volgende formules:
\begin{equation}
\begin{array}{ll}
\left\{\begin{array}{l}
x_{-1}=R_{\mbox{\tiny{in}}}\\
x_4=L_{\mbox{\tiny{in}}}
\end{array}\right.&\mbox{(schuiven, aritmetisch)}\\\\\left\{\begin{array}{l}
x_{-1}=0\\
x_4=0
\end{array}\right.&\mbox{(schuiven, logisch, unsigned)}\\\\\left\{\begin{array}{l}
x_{-1}=0\\
x_4=x_3
\end{array}\right.&\mbox{(schuiven, logisch, 2-complement)}
\end{array}
\end{equation}
Deze logica kunnen we vervolgens implementeren met OR-AND-poorten. Het resultaat van deze volledige implementatie staat op figuur \ref{fig:shiftImplementation}.
\begin{figure}[hbt]
\centering
\begin{tikzpicture}[circuit logic US]
\def\xxi{4};
\foreach \x in {0,...,3} {
  \node[mux4to1] (M\x) at (-2*\x,0) {};
  \draw (M\x.output) -- ++(0,-0.25) node[scale=0.75,anchor=north]{$y_{\x}$};
  \draw (M\x.data1) |- (M\x.data0 |- 0,0.5);
  \pdot{M\x.data0 |- 0,0.5};
  \draw (M\x.data0) -- (M\x.data0 |- 0,\xxi) node[scale=0.75,anchor=south]{$x_{\x}$};
}
\node[mux2to1] (MR) at (1.5,1.5) {};
\node[mux2to1] (ML) at (-7,1.5) {};
\draw (ML.selout0) -- (MR.selin0);
\draw (ML.selin0) -- (-8.5,0 |- ML.selin0) node[scale=0.75,anchor=east]{$s_1$};
\draw (M3.selin0) -- (-8.5,0 |- M3.selin0) node[scale=0.75,anchor=east]{$s_2$};
\draw (M3.selin1) -- (-8.5,0 |- M3.selin1) node[scale=0.75,anchor=east]{$s_3$};
\foreach \x/\y/\z in {0/1/1,1/2/0,2/3/1} {
  \draw (M\x.selin0) -- (M\y.selout0);
  \draw (M\x.selin1) -- (M\y.selout1);
  \draw (M\y.data0 |- 0,0.75) -| (M\x.data2);
  \pdot{M\y.data0 |- 0,0.75};
  \draw (M\x.data0 |- 0,1+0.25*\z) -| (M\y.data3);
  \pdot{M\x.data0 |- 0,1+0.25*\z};
}
\draw (ML.output) -- (ML.output |- 0,1) -| (M3.data2);
\draw (MR.output) -- (MR.output |- 0,1) -| (M0.data3);
\draw (ML.data1) |- (M0.data0 |- 0,1.75);
\pdot{M0.data0 |- 0,1.75};
\draw (MR.data1) |- (M3.data0 |- 0,2);
\pdot{M3.data0 |- 0,2};
\node[and gate,rotate=-90,anchor=output,scale=0.75] (AR) at (MR.data0 |- 0,2) {};
\draw (AR.output) -- (MR.data0);
\draw (AR.input 1) -- (AR.input 1 |- 0,\xxi) node[anchor=south,scale=0.75]{$R_{\mbox{\tiny{in}}}$};
\node[or gate,rotate=-90,anchor=output,scale=0.75] (OL) at (ML.data0 |- 0,1.875) {};
\draw (OL.output) -- (ML.data0);
\node[and gate,rotate=-90,anchor=north,scale=0.75] (ALa) at (OL.input 2 |- 0,3.1) {};
\node[and gate,rotate=-90,inputs={normal,normal,inverted},anchor=south,scale=0.75] (ALb) at (OL.input 1 |- 0,3.1) {};
\draw (ALa.output) -- ++(0,-0.125) -| (OL.input 2);
\draw (ALb.output) -- ++(0,-0.125) -| (OL.input 1);
\draw (ALa.input 2) |- (AR.input 2 |- 0,3.75) -- (AR.input 2);
\draw (ALa.input 2 |- 0,3.75) -- (-8.5,3.75) node[scale=0.75,anchor=east]{$s_0$};
\pdot{ALa.input 2 |- 0,3.75};
\draw (ALb.input 3 |- 0,3.75) -- (ALb.input 3);
\pdot{ALb.input 3 |- 0,3.75};
\draw (ALb.input 1) |- (M3.data0 |- 0,3.5);
\pdot{M3.data0 |- 0,3.5};
\draw (ALa.input 1) -- (ALa.input 1 |- 0,\xxi) node[anchor=south,scale=0.75]{$L_{\mbox{\tiny{in}}}$};
\draw (ALb.input 2) -- (ALb.input 2 |- 0,\xxi) node[anchor=west,scale=0.75,rotate=90]{2-comp};
\end{tikzpicture}
\caption{Implementatie van een schuifoperator over 1 bit.}
\figlab{shiftImplementation}
\end{figure}
\paragraph{8-bit barrel left rotator}
In de vorige paragraaf hebben we een schuifoperator gebouwd die 1-bit kan schuiven. Door verschillende van deze schuifoperatoren na elkaar te plaatsen kunnen we schuiven over meerdere plaatsen. Toch is dit niet erg praktisch: om $m$ plaatsen te schuiven zouden we $m$ van deze schuifoperatoren na elkaar moeten plaatsen, wat grote vertragingen zou impliceren. In deze paragraaf zullen we een rotator bouwen die de vertraging beperkt tot $\log_2 m$, met $m$ het maximaal aantal plaatsen dat kan opgeschoven worden. Het concept is echter ook toepasbaar op algemene schuifoperaties. Alleen is de implementatie te complex voor pedagogische doeleinden. Figuur \ref{fig:hBitBarrelLeftRotatorImplementation} geeft het concept mooi weer.
\begin{figure}[hbt]
\centering
\begin{tikzpicture}
\foreach \y in {0,1,2} {
  \foreach \x in {0,...,7} {
    \node[mux2to1] (M\y\x) at (-1.5*\x,1.5*\y) {};
  }
  \foreach \a/\b in {0/1,1/2,2/3,3/4,4/5,5/6,6/7} {
   \draw (M\y\b.selout0) -- (M\y\a.selin0);
  }
  \draw (M\y7.selin0) -- (M\y7.selin0 -| -11.5,0) node[scale=0.75,anchor=east]{$s_{\y}$};
}
\foreach \y/\yi in {0/1,1/2} {
  \foreach \x in {0,...,7} {
    \draw (M\yi\x.output) -- (M\y\x.data0);
    \pdot{$(M\yi\x.output)!0.25!(M\y\x.data0)$};
  }
}
\foreach \x/\xa/\xb/\xc in {0/4/2/1,1/5/3/2,2/6/4/3,3/7/5/4,4/0/6/5,5/1/7/6,6/2/0/7,7/3/1/0} {
  \draw (M2\x.data0) -- (M2\x.data0 |- 0,4.5) node[scale=0.75,anchor=south]{$x_{\x}$};
  \pdot{M2\x.data0 |- 0,4.25};
  \draw (M0\x.output) -- (M0\x.output |- 0,-0.5) node[scale=0.75,anchor=north]{$y_{\x}$};
  \draw (M2\x.data0 |- 0,4.25) -- (M2\xa.data1);
  \draw ($(M2\x.output)!0.25!(M1\x.data0)$) -- (M1\xb.data1);
  \draw ($(M1\x.output)!0.25!(M0\x.data0)$) -- (M0\xc.data1);
}
\end{tikzpicture}
\caption{Implementatie van een 8-bit barrel left rotator.}
\figlab{hBitBarrelLeftRotatorImplementation}
\end{figure}
Deze barrel left rotator is gebaseerd op het additief principe. Het additief principe stelt dat indien we een getal willen schuiven of roteren over $m$ bits, we dit ook kunnen verwezenlijken door eerst te schuiven/roteren over $m_1$ bits en vervolgens over $m_2$ bits met $m=m_1+m_2$. Indien we onze rotator bouwen volgens een structuur waarbij elke niveau $i$ een rotatie uitvoert over $2^i$ plaatsen kunnen we elke rotatie-operatie uitvoeren. Op figuur \ref{fig:hBitBarrelLeftRotatorImplementation} zien we dat het niveau $s_2$, vier plaatsen naar links roteert. Het volgende niveau twee en het laatste \'e\'en. Elk van deze niveaus heeft dezelfde vertraging. Indien we dus een $n$-bit barrel left rotator willen bouwen die maximaal $m$ plaatsen kan opschuiven moeten we dus $\left\lceil\log_2m\right\rceil$ niveaus synthetiseren. Een niveau $i$ roteert $2^i$ plaatsen voor $\forall i=0,1,\ldots,\left\lceil\log_2m\right\rceil-1$. Dit concept is eenvoudig te veralgemenen naar een volledige schuifoperator. Merk op dat het aantal niveaus dus theoretisch niet afhangt van het aantal bit in het getal. Op de meeste processoren neemt men echter $m=n$. Een andere mooie eigenschap is dat de volgorde van de niveaus niet relevant is.
\chapter{Sequenti\"ele Schakelingen (Schakelingen met geheugen)}
\label{ch:SeqComp}
\chapterquote{We denken dat sommige mensen intelligent zijn, terwijl ze alleen maar een goed geheugen hebben.}{Fr\'ed\'eric Dard, Frans humoristisch schrijver (1921-2000)}
\begin{chapterintro}
??
\end{chapterintro}
\minitoc[n]
\section{Terminologie}
Een \termen{Sequenti\"ele schakeling} is een schakeling waarbij niet alleen de ingangen $X$ van belang zijn, maar ook de toestand van deze schakeling. De \termen{toestand $S$} wordt bepaald door de ingang, en door de vorige toestand $S_{\mbox{\tiny{prev.}}}$. In de wiskunde wordt een dergelijke constructie beschreven als een \termen{eindige-toestanden machine} of \termen{finite state machine (FSM)}. Om dit te verwezenlijken hebben we een \termen{geheugencomponent} nodig, een component die de toestand bijhoudt. Dit kunnen we bijvoorbeeld verwezenlijken met een condensator. In DRAM geheugens wordt dergelijke implementatie gebruikt. Het probleem met een condensator is dat de lading na verloop van tijd verloren gaat, wat tot dynamische logica leidt. In DRAM wordt dit opgelost door de condensatoren in kwestie terug op te laden. Een alternatief zijn componenten met \termen{positieve terugkoppeling}, doorgaans zijn deze bekend als \termen{flipflop} of \termen{register}. In sectie \ref{s:memoryBlocks} zullen we verschillende van deze bouwblokken bespreken.
\subsection{Classificatie van sequenti\"ele schakelingen}
\label{ss:classificationSequential}
Doorgaans kunnen we sequenti\"ele schakelingen onderverdelen volgens twee classificatiesystemen: gebaseerd op de uitgangsfunctie $F$, en de synchroniciteit van de schakeling. Wat betreft de uitgangsfunctie beschouwen we twee types:
\begin{itemize}
 \item \termen{Toestandsgebonden sequenti\"ele schakelingen}: In dit geval hangt de uitgangsfunctie enkel af van de toestand op dat moment. In dat geval is de uitgangsfunctie $F\left(S\right)$. Dit concept is ook bekend als de \termen{Moore machine} ofwel \termen{Moore-FSM}. Bij een Moore machine is de invloed van de ingang dus met vertraging te zien, vermits de schakeling eerst van toestand moet veranderen.
 \item \termen{Inputgebonden sequenti\"ele schakelingen}: Hierbij wordt de signatuur van de uitgangsfunctie uitgebreid. De uitgang is zowel afhankelijk van de toestand $S$ als van de ingang $X$. Dit betekent dat indien de ingang verandert, maar de toestand niet de uitvoer $F\left(S,X\right)$ ook verandert. Dit concept wordt ook wel de \termen{Mealy machine} ofwel \termen{Mealy-FSM} genoemd. Elke Moore machine is dus ook een Mealy machine waar de invoer geen onderdeel uitmaakt van de logica die de uitgang berekent.
\end{itemize}
Verder maken we ook een onderscheid in schakelingen inzake synchroniciteit, ook hier beschouwen we twee soorten:
\begin{itemize}
 \item \termen{Asynchrone sequenti\"ele schakelingen}: Hierbij verandert uitgang $F$ en toestand $S$ wanneer de ingang $X$ verandert. We zullen asynchrone schakelingen kort bespreken in sectie~\ref{s:asynchroneSequence}, bovendien zullen we veronderstellen dat slechts \'e\'en bit tegelijk verandert aan de ingang.
 \item \termen{Synchrone sequenti\"ele schakelingen}: Hierbij veranderen de uitgang $F$ en toestand $S$ enkel op het moment de zogenoemde \termen{klokingang} verandert.
Veruit de meeste sequenti\"ele schakelingen worden gebouwd op basis van een klok. In sectie \ref{s:synchroneSequence} zullen we synchrone schakelingen en hun synthese uitgebreid bespreken.
\end{itemize}
\subsection{Terminologie van het kloksignaal}
Bij synchrone schakelingen maken we gebruik van een klokingang. Op deze klokingang staat periodiek een 0 gevolgd door een 1. Om dit kloksignaal te beschrijven maken we gebruik van de volgende terminologie:
\begin{itemize}
 \item \termen{Klokperiode}: de tijd tussen twee opeenvolgende klokovergangen van 0 naar 1.
 \item \termen{Klokfrequentie}: het aantal klokperiodes per seconden of formeler:
\begin{equation}
\mbox{klokfrequentie}=\displaystyle\frac{1}{\mbox{klokperiode}}
\end{equation}
 \item ``\termen{Duty cycle}'': fragment van de klokcyclus die dat de klok op 1 staat.
\begin{equation}
\mbox{duty cycle}=\displaystyle\frac{\mbox{tijd klok is 1}}{\mbox{klokperiode}}
\end{equation}
De klok staat dus niet per definitie even lang op 0 als op 1.
 \item \termen{Stijgende flank} (ook wel \termen{rising edge} genoemd): de klokovergang waarbij de klok van 0 naar 1 gaat.
 \item \termen{Dalende flank} (ook wel \termen{falling edge} genoemd): de klokovergang waarbij de klok van 1 naar 0 gaat.
\end{itemize}
\section{Bouwblokken}
\label{s:memory}
\label{s:memoryBlocks}
In deze sectie zullen we de bouwblokken van geheugen ontwikkelen. Hierbij zullen we in subsectie \ref{ss:flipflop} eerst het basisblok ontwikkelen om \'e\'en bit te onthouden: de flipflop. Door flipflops te groeperen kunnen we een groter geheugen ontwikkelen. Dergelijke geheugens worden registers genoemd. Registers kunnen meestal maar een beperkt aantal flipflops tegelijk aanspreken. Registers bespreken we in subsectie \ref{ss:registers}. Een andere populaire toepassing van geheugens is een teller. We ontwikkelen een tellerschakeling in subsectie \ref{ss:counters}.
\subsection{De flipflop}
\label{ss:flipflop}
\ssclab{flipflop}
Bij de synthese van een flipflop zullen we een eerste positief feedback component introduceren: de \termen{set-reset latch}. Vervolgens zullen we deze latch uitbreiden en verschillende varianten bespreken. Een probleem met latches is het zogenaamde \termen{transparantie-probleem}. Dit probleem zullen we oplossen door extra logica met deze latch te verbinden wat resulteert in een \termen{flipflop}. Er zijn verschillende varianten van flipflops om verschillende toepassingen te ondersteunen, we eindigen met een beknopt overzicht van de verschillende flipflops.
\subsubsection{De latch}
\begin{figure}[hbt]
\centering
\importtikzsubfigure{setResetLatch-nor}{Implementatie met NOR-poorten.}
\importtikzsubfigure{setResetLatch-nand}{Implementatie met NAND-poorten.}
\caption{Set-reset latch.}
\figlab{setResetLatch}
\end{figure}
\figref{setResetLatch} toont de NOR- en NAND-implementatie van de zogenaamde set-reset latch (ofwel \termen{SR-latch}). Dit component werkt met positieve terugkoppeling waarbij de uitgangen dus ook een deel van de ingangen vormen. We onthouden de uitvoer door bepaalde waarden aan de ingang aan te leggen, die resulteren in het opnieuw bekomen van dezelfde uitvoer. Bij de NOR-implementatie is dit $\left(S,R\right)=\left(0,0\right)$. De NAND-implementatie is in feite de volledig duale vorm en werkt volledig equivalent, alleen zijn $S$ en $R$ hier actief lage signalen (zie \ref{s:negativeLogic}). We zouden ook negatieve logica aan $Q$ en $Q_n$ moeten toekennen maar omdat altijd geldt $Q_n=Q'$ kunnen we ook eenvoudigweg de twee uitgangen omdraaien en bekomen we positieve logica aan de uitgang. De stabiele invoer van de NAND-implementatie is dan ook $\left(S^*,R^*\right)=\left(1,1\right)$. Indien we \'e\'en van de ingangen laten afwijken van de stabiele toestand, komen we in een onstabiele toestand. Deze stabiele toestand kan maar op \'e\'en manier terug stabiel worden. Op deze manier kunnen we een nieuwe waarde toekennen aan de latch. Deze waarde zal ook verder gelden nadat de ingangen weer stabiel zijn. Hierdoor kent men ook de termen \termen{set} (1 in het geheugen) en \termen{reset} (0 in het geheugen) toe aan de ingangen. Indien beide ingangen afwijken van de stabiele ingangen is het gedrag niet meer bepaald. In dat geval zal de uiteindelijke waarde afhangen van de implementatie van de latch en de vertraging van de poorten. Bij identieke vertragingen leidt dit tot een oscillerend effect wat ook te zien is op de tijdsgrafieken op figuur \ref{fig:setResetLatch}. In een minder ideaal systeem zal de snelste poort\footnote{Uiteraard zijn de twee poorten in theorie even traag, in de praktijk zullen er altijd kleine verschillen zijn.} de uiteindelijke vertraging bepalen. Dit gedrag noemt men een ``\termen{race}'' (zie \ref{term:race}).
\paragraph{Geklokte SR-latch}
Bij de SR-latch is er geen sprake van een klokingang. We kunnen immers ook geheugens zonder klok gebruiken in bijvoorbeeld asynchrone schakelingen. Door extra hardware voor de SR-latch te plaatsen, kunnen we een \termen{geklokte SR-latch} bouwen. \figref{clockedSRLatch} toont hoe we dit kunnen realiseren. Zolang het kloksignaal op 0 staat zal het geheugen zijn waarde behouden, indien het kloksignaal op 1 staat gelden dezelfde regels als bij een SR-latch.
\begin{figure}[hbt]
\centering
\importtikzsubfigure{clockedSRLatch-gates}{Met AND- en NOR-poorten.}
\importtikzsubfigure{clockedSRLatch-nand}{Met NAND-poorten.}
\importtikzsubfigure{clockedSRLatch-trans}{Overgangstabel.}
\caption{Geklokte SR-latch.}
\figlab{clockedSRLatch}
\end{figure}
\paragraph{Geklokte D-latch}
Indien we elke klokflank een nieuwe waarde in de latch willen opslaan loont het meestal de moeite om de geklokte SR-latch om te vormen tot een \termen{geklokte D-latch}. Een geklokte D-latch zoals op figuur \ref{fig:clockedDLatch} bouwt meestal extra logica rond een geklokte SR-latch die met de \termen{data-ingang $D$} de $S$ en $R$ ingang aanstuurt. D-latches zijn populair bij schakelingen waarbij bij iedere klokflank een nieuwe waarde wordt ingelezen. Soms wordt dan echter ook een SR-latch gebruikt omdat dit de logica rond deze geheugenmodules soms kan vereenvoudigen.
\begin{figure}[hbt]
\centering
\importtikzsubfigure{clockedDLatch-impl}{Implementatie.}
\importtikzsubfigure{clockedDLatch-trans}{Overgangstabel.}
\caption{Geklokte D-latch.}
\figlab{clockedDLatch}
\end{figure}
\paragraph{Set-up- en houdtijd}
Aan de hand van de implementatie van de geklokte D-latch op figuur \ref{fig:clockedDLatch} zullen we de vertragingen van verschillende signalen berekenen:
\begin{equation}
\begin{array}{ccl}
D\rightarrow Q&:&\left\{\begin{array}{lll}
t_{HL}=1.4+1.4=2.8&\ifun&D=0\wedge\mbox{Clk}=1\\
t_{LH}=1+1.4+1.4+1.4=5.2&\ifun&D=1\wedge\mbox{Clk}=1
\end{array}\right.\\\\
D\rightarrow Q_n&:&\left\{\begin{array}{lll}
t_{HL}=1+1.4+1.4=3.8&\ifun&D=0\wedge\mbox{Clk}=1\\
t_{LH}=1.4+1.4+1.4=4.2&\ifun&D=1\wedge\mbox{Clk}=1
\end{array}\right.\\\\
\mbox{Clk}\rightarrow Q&:&\left\{\begin{array}{lll}
t_{HL}=1.4+1.4+1.4=4.2&\ifun&D=0\wedge\mbox{Clk}=1\\
t_{LH}=1.4+1.4=2.8&\ifun&D=1\wedge\mbox{Clk}=1
\end{array}\right.\\\\
\mbox{Clk}\rightarrow Q_n&:&\left\{\begin{array}{lll}
t_{HL}=1.4+1.4+1.4=4.2&\ifun&D=0\wedge\mbox{Clk}=1\\
t_{LH}=1.4+1.4=2.8&\ifun&D=1\wedge\mbox{Clk}=1
\end{array}\right.
\end{array}
\label{eqn:dLatchDelays}
\end{equation}
We kunnen dergelijke vertragingen grafisch weergeven zoals op de tijdsgrafieken op figuur \ref{fig:timebehavDLatch}.
\begin{figure}[hbt]
\centering
\subfigure[Set-up-tijd??]{%TODO: ongeldige toestand op figuur zetten
\begin{tikzpicture}
\timebehav{0.00}{0.00}{6.70}{7}{0.80}{0/$Qn$,1/$Q$,2/$R*$,3/$S*$,4/$Dn$,5/$Clk$,6/$D$}{0/0.00/1/5.20/0, 0/5.20/0/8.00/0, 1/0.00/0/3.80/1, 1/3.80/1/8.00/1, 2/0.00/0/3.40/1, 2/3.40/1/8.00/1, 3/0.00/1/2.40/0, 3/2.40/0/5.40/1, 3/5.40/1/8.00/1, 4/0.00/1/2.00/0, 4/2.00/0/8.00/0, 5/0.00/1/4.00/0, 5/4.00/0/8.00/0, 6/0.00/0/1.00/1, 6/1.00/1/8.00/1}{3.80/1.40/1/0, 2.40/1.40/3/1, 2.00/1.40/4/2, 1.00/1.40/6/3, 4.00/1.40/5/3, 1.00/1.00/6/4};
\timebehav{8.00}{0.00}{6.70}{7}{0.80}{0/$Qn$,1/$Q$,2/$R*$,3/$S*$,4/$Dn$,5/$Clk$,6/$D$}{0/0.00/1/5.20/0, 0/5.20/0/5.70/1, 0/5.70/1/8.00/0, 0/8.00/0/8.00/0, 1/0.00/0/3.80/1, 1/3.80/1/4.30/0, 1/4.30/0/6.60/1, 1/6.60/1/7.10/0, 1/7.10/0/8.00/0, 2/0.00/0/2.90/1, 2/2.90/1/8.00/1, 3/0.00/1/2.40/0, 3/2.40/0/2.90/1, 3/2.90/1/8.00/1, 4/0.00/1/2.00/0, 4/2.00/0/8.00/0, 5/0.00/1/1.50/0, 5/1.50/0/8.00/0, 6/0.00/0/1.00/1, 6/1.00/1/8.00/1}{3.80/1.40/1/0, 4.30/1.40/1/0, 6.60/1.40/1/0, 2.40/1.40/3/1, 2.90/1.40/3/1, 5.20/1.40/0/1, 5.70/1.40/0/1, 1.50/1.40/5/2, 1.00/1.40/6/3, 1.50/1.40/5/3, 1.00/1.00/6/4};
\end{tikzpicture}
\figlab{timebehavDLatchSetup}
}
\subfigure[Houdtijd??]{%TODO: afwerken
\begin{tikzpicture}
\timebehav{0.00}{0.00}{6.70}{7}{0.80}{0/$Qn$,1/$Q$,2/$R*$,3/$S*$,4/$Dn$,5/$Clk$,6/$D$}{0/0.00/1/5.20/0, 0/5.20/0/7.70/1, 0/7.70/1/8.00/0, 0/8.00/0/8.00/0, 1/0.00/0/3.80/1, 1/3.80/1/6.30/0, 1/6.30/0/6.60/1, 1/6.60/1/8.00/1, 2/0.00/0/3.40/1, 2/3.40/1/8.00/1, 3/0.00/1/2.40/0, 3/2.40/0/4.90/1, 3/4.90/1/8.00/1, 4/0.00/1/2.00/0, 4/2.00/0/4.50/1, 4/4.50/1/8.00/1, 5/0.00/1/4.00/0, 5/4.00/0/8.00/0, 6/0.00/0/1.00/1, 6/1.00/1/3.50/0, 6/3.50/0/8.00/0}{3.80/1.40/1/0, 6.30/1.40/1/0, 6.60/1.40/1/0, 2.40/1.40/3/1, 4.90/1.40/3/1, 5.20/1.40/0/1, 2.00/1.40/4/2, 1.00/1.40/6/3, 3.50/1.40/6/3, 1.00/1.00/6/4, 3.50/1.00/6/4};
\timebehav{8.00}{0.00}{6.70}{7}{0.80}{0/$Qn$,1/$Q$,2/$R*$,3/$S*$,4/$Dn$,5/$Clk$,6/$D$}{0/0.00/1/5.20/0, 0/5.20/0/8.00/0, 1/0.00/0/3.80/1, 1/3.80/1/8.00/1, 2/0.00/0/3.40/1, 2/3.40/1/8.00/1, 3/0.00/1/2.40/0, 3/2.40/0/5.20/1, 3/5.20/1/8.00/1, 4/0.00/1/2.00/0, 4/2.00/0/4.80/1, 4/4.80/1/8.00/1, 5/0.00/1/4.00/0, 5/4.00/0/8.00/0, 6/0.00/0/1.00/1, 6/1.00/1/3.80/0, 6/3.80/0/8.00/0}{3.80/1.40/1/0, 2.40/1.40/3/1, 2.00/1.40/4/2, 1.00/1.40/6/3, 3.80/1.40/6/3, 1.00/1.00/6/4, 3.80/1.00/6/4};
\end{tikzpicture}
\figlab{timebehavDLatchHold}
}
\caption{Tijdsgrafieken van een D-latch.}
\figlab{timebehavDLatch}
\end{figure}
Deze grafische voorstelling toont twee bekende problemen die veroorzaakt worden door het niet respecteren van twee parameters:
\begin{itemize}
 \item \termen{Set-up-tijd}: De tijd alvorens de het actieve kloksignaal wordt verlaten waarin de waarde op de data-ingang niet meer mag wijzigen. Bij een geklokte $D$-latch zoals op figuur \ref{fig:clockedDLatch} is dit:
\begin{equation}
\begin{array}{cr}
t_{\mbox{\tiny{set-up}}}=t_{pHL}\left(\mbox{inverter}\right)&\mbox{(vertraging van een hoog-naar-laag signaal door een inverter)}
\end{array}
\end{equation}
\figref{timebehavDLatchSetup} toont twee senarios. Bij het eerste wordt de set-up tijd gerespecteerd, bij het tweede faalt de toekenning. Dit komt omdat bij dit scenario $S^*$ weer hoog wordt alvorens $R^*$ een hoog signaal aanlegt. Indien we de vertragingen van de poorten doorrekenen komen we uit dat de vertraging van aan de inverter een cruciale rol speelt. Dit is ook enigszins logisch: indien we een 0 aan de data-ingang $D$ aanleggen zal gedurende deze periode 1 op zowel de $S$ als $R$ van de geklokte SR-latch worden aangelegd, wat eigenlijk een ongeldige invoer is.
 \item \termen{Houdtijd}: We moeten niet alleen het signaal op tijd aanleggen voor de klok een laag signaal aanlegt. Meestal moeten we het signaal daarna nog een tijdje laten staan om te vermijden dat de latch alsnog een foute waarde aanneemt. \figref{timebehavDLatchHold} toont opnieuw twee scenarios.??%TODO: afwerken
\end{itemize}
\paragraph{Metastabiliteit}
Een ander probleem dat komt kijken bij latches is de \termen{metastabiliteit}. Dit probleem treedt op bij de twee poorten\footnote{NAND- of NOR-implementatie maakt niet uit.} van de SR-latch en dus bijgevolg alle afgeleide latches. Als we bij de NOR-implementatie $\left(S,R\right)=\left(0,0\right)$ aanleggen, of bij een NAND-implementatie $\left(S^*,R^*\right)=\left(1,1\right)$, kunnen we deze poorten modelleren als NOT-poorten zoals op figuur \ref{fig:metastabilityNotGates}.
\begin{figure}[hbt]
\centering
\subfigure[Implementatie.]{
\begin{tikzpicture}[circuit logic US,scale=1.2]
\node[not gate] (NO0) at (0,0.65) {};
\node[not gate] (NO1) at (0,-0.65) {};
\draw (NO1.output) -- (NO1.output -| 0.85,0) node[anchor=west,scale=0.75]{$x$} -- (0.85,-0.4) -- (-0.75,0.4) -- (NO0.input -| -0.75,0) -- (NO0.input);
\draw (NO0.output) -- (NO0.output -| 0.85,0) node[anchor=west,scale=0.75]{$y$} -- (0.85,0.4) -- (-0.75,-0.4) -- (NO1.input -| -0.75,0) -- (NO1.input);
\end{tikzpicture}
\figlab{metastabilityNotGates}}
\subfigure[Transfer-functies.]{
\begin{tikzpicture}
\draw[thick,->] (-0.1,0) -- (2.2,0) node[anchor=north east,scale=0.75]{$x$};
\draw[thick,->] (0,-0.1) -- (0,2.2) node[anchor=north east,scale=0.75]{$y$};
\draw [samples=100,smooth,domain=0.1:1.9,variable=\x,dashed] plot (\x,{1.75-1.5/(1+exp(-8*\x+8))});
\draw [samples=100,smooth,domain=0.1:1.9,variable=\x] plot ({1.75-1.5/(1+exp(-8*\x+8))},\x);
\pdot{1,1};
\pdot{0.2538,1.7462};
\pdot{1.7462,0.2538};
\end{tikzpicture}
\figlab{metastabilityNotFunctions}}
\subfigure[Bal-en-heuvel-analogie.]{
\begin{tikzpicture}
\draw[pattern=vertical lines] [samples=100,smooth,domain=-2:2,variable=\x] plot (\x,{cos(114.591559029*\x)}) -- (2,-1.3) -- (-2,-1.3) -- cycle;
\filldraw[fill=black!20,draw=black] (0,1.125) circle (0.125 cm);
\filldraw[fill=black!20,draw=black] (-1.570796327,-0.875) circle (0.125 cm);
\filldraw[fill=black!20,draw=black] (1.570796327,-0.875) circle (0.125 cm);
\end{tikzpicture}
\figlab{metastabilityAnalogy}}
\caption{Metastabiliteit.}
\figlab{metastability}
\end{figure}
Indien we deze schakeling logisch analyseren zien we twee mogelijke stabiele oplossingen: waarbij ofwel $x$ ofwel $y$ 1 is, en de andere 0. Deze waarden zijn ook de enige die we beschouwen indien we de NOT-poort louter als logisch component zien. We implementeren deze poorten echter met behulp van transistoren, bijgevolg behoudt de poort een zeker analoog karakter. Op de grafiek op figuur \ref{fig:metastabilityNotFunctions} geven we de transfer-functie van de twee NOT-poorten weer. De stippelijn geeft de transfer-functie van de bovenste NOT-poort weer, de volle lijn de onderste. We zien zoals verwacht de twee \termen{stabiele toestanden}. We bemerken echter ook een \termen{metastabiele toestand}. Op het moment dat op $x$ of $y$ een kleine hoeveelheid ruis wordt aangebracht zullen de poorten dit effect versterken en zal zal de schakeling in een stabiele toestand terechtkomen. Het probleem is echter dat we uiteraard niet weten hoelang dit zal duren. We gaan er echter vanuit dat de kans dat na een bepaald tijdstip er nog onvoldoende ruis is opgetreden Poisson-verdeeld is\footnote{De meeste kansverdelingen op het voorkomen van een gebeurtenis zijn Poisson-verdeeld.}. We formaliseren dus tot:
\begin{equation}
p\left(\mbox{nog in metastabiele toestand na $t$}\right)=e^{-t/\tau}
\end{equation}
De \termen{tijdsconstante $\tau$} is hierbij afhangen van twee factoren:
\begin{itemize}
 \item De hoeveelheid ruis: hoe meer ruis hoe lager de tijdsconstante.
 \item De stijlheid van de curves rond de metastabiele toestand: hoe stijler hoe lager de tijdsconstante.
\end{itemize}
Een latch kan in een metastabiele toestand komen door een zogenaamde \termen{marginale triggering}: een schending van de set-up- of houdtijd of van de minimale pulsbreedte\footnote{De tijd dat een signaal wordt aangelegd.}. In deze gevallen kan een overgang tussen twee stabiele toestanden worden onderbroken. Indien dit gebeurt op het moment dat men net voorbij de metastabiele toestand passeert treedt dit probleem op. Een latch komt dan ook bij elke overgang kortstondig in een metastabiele toestand. Ook een tijdje in een metastabiele toestand blijven is geen probleem. Zolang deze toestand niet meer actief is wanneer we het signaal gaan gebruiken zullen er geen problemen optreden. Een populaire voorstelling van metastabiliteit is de zogenaamde \termen{bal-en-heuvel-analogie}. In deze analogie beschrijven we een heuvel zoals op figuur \ref{fig:metastabilityAnalogy}. Een bal kan op deze heuvel in drie toestanden een evenwicht bereiken: twee toestanden in een dal (passief evenwicht) en een metastabiele toestand op de heuvel (actief evenwicht). Bij asynchrone circuits is metastabiliteit een veel voorkomend fenomeen. Zeker wanneer de klokfrequentie geen veelvoud is van de frequentie waarmee de ingang omwisselt. In dat geval bestaat de oplossing van het probleem eerder uit ``hoe ga ik om met metastabiliteit?'', in plaats van ``hoe los ik de metastabiliteit op?''.
\subsubsection{De flipflop}
Latches kunnen we gebruiken bij het opslaan van \'e\'en bit. Indien we een geklokte latch aan een kloksignaal hangen zijn er twee toestanden afhankelijk van het niveau van het kloksignaal:
\begin{itemize}
 \item Indien het kloksignaal hoog is $\mbox{Clk}=1$ is de latch \termen{transparant}. De latch neemt de waarde over die aan de ingang staat.
 \item Indien het kloksignaal laag is $\mbox{Clk}=0$ onthoudt de latch de laatste waarde die aan de ingang stond toen de latch transparant was.
\end{itemize}
Latches geven echter problemen op het moment we verschillende latches na elkaar willen hangen. In dat geval zullen immers alle latches transparant zijn op hetzelfde moment. Hierdoor zal de laatste latch de waarde aannemen die op de eerste latch wordt aangelegd\footnote{Bij een groot aantal latches zal het signaal door vertragingen en set-up- en houdtijd uiteraard maar door een beperkt aantal latches in \'e\'en klokflank propageren.}. Dit probleem wordt ook wel het \termen{transparantie-probleem} genoemd. Meestal willen we echter bij een sequentie van een aantal geheugencomponenten afdwingen dat de informatie door \'e\'en geheugencomponent per klokflank propageert. De flipflop is een geheugencomponent die in tegenstelling tot de latch \termen{flankgevoelig} is. Dit betekent dat de flipflop enkel transparant is op het moment dat de klok van 0 naar 1 gaat, in tegenstelling tot een latch die transparant is gedurdende de volledige periode dat de klok hoog is. Om dit te realiseren zijn er doorgaans twee methodes:
\begin{itemize}
 \item De \termen{master-slave flipflop}.
 \item De \termen{edge-triggered flipflop}.
\end{itemize}
We zullen deze twee verschillende technieken in de volgende paragrafen toelichten.
\paragraph{Master-slave flipflop}
Een master-slave flipflop maakt gebruik van twee latches die beurtelings transparant zijn. De master (eerste latch) is transparant wanneer het kloksignaal laag is, de tweede latch is transparant bij een hoog kloksignaal. Gegroepeerd vormen we dus een geheugen die de laatste waarde opslaat die aan de ingang stond op het moment dat het kloksignaal laag was, en deze verder propageert op het moment dat het kloksignaal hoog is. \figref{masterSlaveFlipflop} toont dit concept samen met een tijdsgrafiek.??
\begin{figure}[hbt]
\centering
\subfigure[Implementatie.]{
\begin{tikzpicture}[circuit logic US]
\def\cly{-1.125};
\def\nff{2};
\def\nffd{1};
\def\nffi{3};
\foreach\x in {0,...,\nff} {
  \filldraw[draw=black,dashed,fill=black!20] (4*\x-1.125,\cly-0.125) rectangle (4*\x+2.425,1.25);
  \node[cldlatchmaster,scale=0.75] (M\x) at (4*\x,0) {};
  \node[cldlatch,scale=0.75] (S\x) at (4*\x+1.75,0) {};
  \draw (M\x.Clk) -| (4*\x-1,\cly);
  \node[anchor=south] (Mt\x) at (M\x.north) {Master};
  \node[anchor=south] (St\x) at (S\x.north) {Slave};
  \pdot{4*\x-1,\cly};
  \draw (M\x.Q) -- (S\x.D);
  \draw (S\x.Clk) -| (4*\x+0.75,\cly);
}
\foreach\xd/\x in {0/1,1/2} {
  \draw (S\xd.Q) to node[midway,above,scale=0.75]{$Q_{\x}$} (M\x.D);
  \pdot{0.75+4*\xd,\cly};
}
\draw (0.75+4*\nff,\cly) -- (-1.5,\cly) node[scale=0.75,anchor=east]{Clk};
\draw (M0.D) -- (M0.D -| -1.5,0) node[scale=0.75,anchor=east]{$D$};
\draw (S\nff.Q) -- (S\nff.Q -| 4*\nff+3,0) node[scale=0.75,anchor=west]{$Q_{\nffi}$};
\end{tikzpicture}}
\subfigure[Interface]{
\begin{tikzpicture}
\node[dff] (DFF) at (0,0) {};
\end{tikzpicture}}
\caption{Master-slave flipflop.}
\figlab{masterSlaveFlipflop}
\end{figure}
\paragraph{Edge-triggered flipflop}
Een Edge-trigger flipflop maakt gebruik van een structuur die we kunnen groeperen als drie latches. Deze schakeling stelt ons in staat om om het signaal op te slaan die aan de ingang staat op het moment dat de klok van 0 naar 1 gaat. \figref{edgeTriggeredFlipflop} toont een basis en meer uitgebreide implementatie samen met een tijdsgrafiek.
\begin{figure}[hbt]
\centering
\subfigure[Basisimplementatie.]{
\begin{tikzpicture}[circuit logic US]
\def\dx{-1.75};
\def\dy{1.35};
\node[nand gate] (NA10) at (0,0.65) {};
\node[nand gate] (NA11) at (0,-0.65) {};
\draw (NA11.output -| 0.85,0) -- (0.85,-0.4) -- (-0.75,0.4) -- (NA10.input 2 -| -0.75,0) -- (NA10.input 2);
\draw (NA10.output -| 0.85,0) -- (0.85,0.4) -- (-0.75,-0.4) -- (NA11.input 1 -| -0.75,0) -- (NA11.input 1);
\draw (NA10.output) -- (NA10.output -| 1.1,0) node[anchor=west,scale=0.75]{$Q_n$};
\draw (NA11.output) -- (NA11.output -| 1.1,0) node[anchor=west,scale=0.75]{$Q$};
\pdot{NA10.output -| 0.85,0};
\pdot{NA11.output -| 0.85,0};
\node[nand gate] (NA01) at (NA10.input 1 -| \dx,0) {};
\node[nand gate,inputs={normal,normal,normal}] (NA02) at (NA11.input 2 -| \dx,0) {};
\draw (NA01.output) -- (NA10.input 1);
\draw (NA02.output) -- (NA11.input 2);
\node[nand gate] (NA00) at (\dx,2) {};
\node[nand gate] (NA03) at (\dx,-2) {};
\draw (NA01.output -| \dx+0.85,0) -- (\dx+0.85,\dy-0.3) -- (\dx-0.75,\dy+0.3) -- (NA00.input 2 -| \dx-0.75,0) -- (NA00.input 2);
\draw (NA00.output) -- (NA00.output -| \dx+0.85,0) -- (\dx+0.85,\dy+0.3) -- (\dx-0.75,\dy-0.3) -- (NA01.input 1 -| \dx-0.75,0) -- (NA01.input 1);
\draw (NA03.output) -- (NA03.output -| \dx+0.85,0) -- (\dx+0.85,-\dy-0.3) -- (\dx-0.75,-\dy+0.3) -- (NA02.input 3 -| \dx-0.75,0) -- (NA02.input 3);
\draw (NA02.output -| \dx+0.85,0) -- (\dx+0.85,-\dy+0.4) -- (\dx-0.75,-\dy-0.4) -- (NA03.input 1 -| \dx-0.75,0) -- (NA03.input 1);
\draw (NA01.output -| \dx+0.85,0) -- (\dx+0.85,0.4) -- (\dx-0.75,-0.4) -- (NA02.input 1 -| \dx-0.75,0) -- (NA02.input 1);
\draw (NA00.output -| \dx+0.85,0) node[anchor=west,scale=0.75]{$A$};
\pdot{NA01.output -| \dx+0.85,0};
\draw (NA01.output -| \dx+0.85,0) node[anchor=south west,scale=0.75]{$S^*$};
\pdot{NA02.output -| \dx+0.85,0};
\draw (NA02.output -| \dx+0.85,0) node[anchor=north west,scale=0.75]{$R^*$};
\pdot{NA02.input 3 -| \dx-0.75,0};
\draw (NA03.output -| \dx+0.85,0) node[anchor=west,scale=0.75]{$B$};
\draw (NA02.input 3 -| \dx-0.75,0) -- ++(-0.5,0) |- (NA00.input 1);
\draw (NA02.input 2) -- (NA02.input 2 -| \dx-0.75,0) -- ++(-0.25,0) |- (NA01.input 2);
\draw (\dx-1,0) -- ++(-0.5,0) node[anchor=east,scale=0.75]{Klok Clk};
\pdot{\dx-1,0};
\draw (NA03.input 2) -- (NA03.input 2 -| \dx-1.5,0) node[anchor=east,scale=0.75]{$D$};
\end{tikzpicture}
\figlab{edgeTriggeredFlipflopBasic}}
\subfigure[Uitgebreide implementatie.]{
\begin{tikzpicture}[circuit logic US]
\def\dx{-1.75};
\def\dy{1.35};
\node[nand gate,inputs={normal,normal,normal}] (NA10) at (0,0.65) {};
\node[nand gate,inputs={normal,normal,normal}] (NA11) at (0,-0.65) {};
\draw (NA11.output -| 0.85,0) -- (0.85,-0.3) -- (-0.75,0.3) -- (NA10.input 3 -| -0.75,0) -- (NA10.input 3);
\draw (NA10.output -| 0.85,0) -- (0.85,0.3) -- (-0.75,-0.3) -- (NA11.input 1 -| -0.75,0) -- (NA11.input 1);
\draw (NA10.output) -- (NA10.output -| 1.1,0) node[anchor=west,scale=0.75]{$Q_n$};
\draw (NA11.output) -- (NA11.output -| 1.1,0) node[anchor=west,scale=0.75]{$Q$};
\pdot{NA10.output -| 0.85,0};
\pdot{NA11.output -| 0.85,0};
\node[nand gate,inputs={normal,normal,normal}] (NA01) at (NA10.input 2 -| \dx,0) {};
\node[nand gate,inputs={normal,normal,normal}] (NA02) at (NA11.input 2 -| \dx,0) {};
\draw (NA01.output) -- (NA10.input 2);
\draw (NA02.output) -- (NA11.input 2);
\node[nand gate,inputs={normal,normal,normal}] (NA00) at (\dx,2) {};
\node[nand gate,inputs={normal,normal,normal}] (NA03) at (\dx,-2) {};
\draw (NA01.output -| \dx+0.85,0) -- (\dx+0.85,\dy-0.3) -- (\dx-0.75,\dy+0.3) -- (NA00.input 3 -| \dx-0.75,0) -- (NA00.input 3);
\draw (NA00.output) -- (NA00.output -| \dx+0.85,0) -- (\dx+0.85,\dy+0.3) -- (\dx-0.75,\dy-0.3) -- (NA01.input 1 -| \dx-0.75,0) -- (NA01.input 1);
\draw (NA03.output) -- (NA03.output -| \dx+0.85,0) -- (\dx+0.85,-\dy-0.3) -- (\dx-0.75,-\dy+0.3) -- (NA02.input 3 -| \dx-0.75,0) -- (NA02.input 3);
\draw (NA02.output -| \dx+0.85,0) -- (\dx+0.85,-\dy+0.3) -- (\dx-0.75,-\dy-0.3) -- (NA03.input 1 -| \dx-0.75,0) -- (NA03.input 1);
\draw (NA01.output -| \dx+0.85,0) -- (\dx+0.85,0.4) -- (\dx-0.75,-0.4) -- (NA02.input 1 -| \dx-0.75,0) -- (NA02.input 1);
\pdot{NA01.output -| \dx+0.85,0};
\pdot{NA02.output -| \dx+0.85,0};
\pdot{NA02.input 3 -| \dx-0.75,0};
\draw (NA02.input 3 -| \dx-0.75,0) -- ++(-0.5,0) |- (NA00.input 2);
\draw (NA02.input 2) -- (NA02.input 2 -| \dx-0.75,0) -- ++(-0.25,0) |- (NA01.input 2);
\draw (\dx-1,0) -- ++(-0.75,0) node[anchor=east,scale=0.75]{Klok Clk};
\pdot{\dx-1,0};
\draw (NA03.input 2) -- (NA03.input 2 -| \dx-1.75,0) node[anchor=east,scale=0.75]{$D$};
\coordinate (CLRI) at (\dx-1.75,-2.5);
\coordinate (PRI) at (\dx-1.75,2.5);
\draw (PRI) node[anchor=east,scale=0.75]{Preset $\mbox{PR}^*$} -- (PRI -| -0.75,0) |- (NA10.input 1);
\draw (CLRI) node[anchor=east,scale=0.75]{Clear $\mbox{CLR}^*$} -- (CLRI -| -0.75,0) |- (NA11.input 3);
\draw (PRI -| \dx-0.75,0) |- (NA00.input 1);
\pdot{PRI -| \dx-0.75,0};
\draw (CLRI -| \dx-1.5,0) |- (NA03.input 3);
\pdot{CLRI -| \dx-1.5,0};
\draw (\dx-1.5,0 |- NA03.input 3) |- (NA01.input 3);
\pdot{\dx-1.5,0 |- NA03.input 3};
\end{tikzpicture}
\figlab{edgeTriggeredFlipflopExtended}}
\subfigure[Tijdsgrafiek]{\begin{tikzpicture}
\timebehav{5.00}{-1.00}{8.50}{8}{0.75}{0/$Qn$,1/$Q$,2/$S^*$,3/$R^*$,4/$A$,5/$B$,6/$D$,7/Clk}{0/0.00/1/1.55/0, 0/1.55/0/3.88/1, 0/3.88/1/6.55/0, 0/6.55/0/8.88/1, 0/8.88/1/11.48/1, 1/0.00/0/1.30/1, 1/1.30/1/4.13/0, 1/4.13/0/6.30/1, 1/6.30/1/9.13/0, 1/9.13/0/11.48/0, 2/0.00/1/1.05/0, 2/1.05/0/2.30/1, 2/2.30/1/6.05/0, 2/6.05/0/7.30/1, 2/7.30/1/11.48/1, 3/0.00/1/3.63/0, 3/3.63/0/4.88/1, 3/4.88/1/8.63/0, 3/8.63/0/9.88/1, 3/9.88/1/11.13/0, 3/11.13/0/11.48/0, 4/0.00/0/0.68/1, 4/0.68/1/3.18/0, 4/3.18/0/5.38/1, 4/5.38/1/7.55/0, 4/7.55/0/11.48/0, 5/0.00/1/0.43/0, 5/0.43/0/2.93/1, 5/2.93/1/5.13/0, 5/5.13/0/6.68/1, 5/6.68/1/11.48/1, 6/0.00/0/0.18/1, 6/0.18/1/2.68/0, 6/2.68/0/3.93/1, 6/3.93/1/6.43/0, 6/6.43/0/11.48/0, 7/0.00/0/0.80/1, 7/0.80/1/2.05/0, 7/2.05/0/3.30/1, 7/3.30/1/4.55/0, 7/4.55/0/5.80/1, 7/5.80/1/7.05/0, 7/7.05/0/8.30/1, 7/8.30/1/9.55/0, 7/9.55/0/10.80/1, 7/10.80/1/11.48/1}{1.30/0.25/1/0, 3.63/0.25/3/0, 6.30/0.25/1/0, 8.63/0.25/3/0, 1.05/0.25/2/1, 3.88/0.25/0/1, 6.05/0.25/2/1, 8.88/0.25/0/1, 0.80/0.25/7/2, 2.05/0.25/7/2, 5.80/0.25/7/2, 7.05/0.25/7/2, 3.30/0.32/7/3, 4.55/0.32/7/3, 8.30/0.32/7/3, 9.55/0.32/7/3, 10.80/0.32/7/3, 0.43/0.25/5/4, 2.93/0.25/5/4, 5.13/0.25/5/4, 7.30/0.25/2/4, 0.18/0.25/6/5, 2.68/0.25/6/5, 4.88/0.25/3/5, 6.43/0.25/6/5};
\end{tikzpicture}
}
\caption{Edge-triggered flipflop.}
\figlab{edgeTriggeredFlipflop}
\end{figure}
In de meer uitgebreide versie zijn er naast de data- en klokingang ook nog twee andere ingangen beschreven:
\begin{itemize}
 \item \termen{Preset $\mbox{PR}^*$}: Indien dit signaal laag wordt, slaan we asynchroon een 1 op in de flipflop.
 \item \termen{Clear $\mbox{CLR}^*$}: Indien dit signaal laag wordt, slaan we asynchroon een 0 op in de flipflop. 
\end{itemize}
Deze twee signalen worden ook wel de \termen{asynchrone set en reset} genoemd. Ze zijn asynchroon omdat ze onafhankelijk van de toestand van de klok een waarde in het geheugen kunnen inbrengen, deze eigenschap wordt bijvoorbeeld gebruikt om bij het opkomen van de stroom in de elektronica de flipflop in een gekende toestand te brengen\footnote{Bij het opkomen van de stroom zal de flipflop of latch immers een waarde aannemen afhankelijk van de ruis en minimale verschillen in poortvertragingen.}.
\subsubsection{Types flipflops}
\label{sss:typesFlipflops}
Er bestaan analoog aan de latches ook verschillende types flipflops. Hierbij is niet de klokaansturing variabel, maar de manier hoe data ingelezen wordt. Denk bijvoorbeeld aan het verschil tussen een SR-latch en D-latch. Elk type flipflop kunnen we karakteriseren aan de hand van twee tabellen:
\begin{itemize}
 \item De \termen{karakteristieke tabel}: deze tabel gebruikt men bij de implementatie van de flipflops. Het toont aan de linkerkant de ingangen, en aan de rechterkant geeft het weer hoe er op deze ingangen wordt ingespeeld. 
 \item De \termen{excitatietabel}: deze tabel beschrijft de verschillende vormen van gedrag van de component aan de linkerkant, en aan de rechterkant hoe we dit gedrag kunnen verwezenlijken met de ingangen.
\end{itemize}
Deze twee tabellen zijn niet strikt het omgekeerde omdat de excitatietabel een kolom voorziet voor $Q$ en $Q_{\mbox{\tiny{next}}}$, terwijl de karakteristieke tabel uitsluitend \'e\'en kolom aan de rechterkant voorziet. In de volgende paragrafen zullen we de verschillende types flipflops bespreken met hun karakteristieke- en excitatietabel.
\paragraph{SR-flipflop} De \termen{SR-flipflop} ofwel \termen{set-reset flipflop} werkt volledig analoog aan een SR-latch, alleen verandert de waarde uitsluitend op het moment dat de klok van een laag signaal naar een hoog signaal gaat. De karakteristieke tabel is dan ook volledig equivalent met deze van de SR-latch op figuur \ref{fig:setResetLatch}. \figref{setResetFlipflop} toont de interface en de karakteristieke- en excitatietabel van de SR-flipflop.
\begin{figure}[hbt]
\centering
\begin{tikzpicture}
\node[srff,anchor=west] (I) at (0,0) {};
\node (KT) at (8.5,0) {$\begin{array}{cc|c}
S&R&Q_{\mbox{\tiny{next}}}\\\hline
0&0&Q\\
0&1&0\\
1&0&1\\
1&1&\mbox{N/A}
\end{array}$};
\node[anchor=east] (ET) at (14,0) {$\begin{array}{cc|cc}
Q&Q_{\mbox{\tiny{next}}}&S&R\\\hline
0&0&0&-\\
0&1&1&0\\
1&0&0&1\\
1&1&-&0\\
\end{array}$};
\node[anchor=north] (IT) at (I.south) {Symbool};
\node[anchor=north] (KTT) at (KT.south |- IT.north) {Karakteristieke tabel};
\node[anchor=north] (ETT) at (ET.south |- IT.north) {Excitatietabel};
\end{tikzpicture}
\caption{Set-reset flipflop.}
\figlab{setResetFlipflop}
\end{figure}
\paragraph{D-flipflop} Ook de \termen{data-flipflop} of \termen{D-flipflop} is conceptueel equivalent aan zijn latch tegenhanger. Op het moment dat de klok van laag naar hoog gaat, zal het signaal dat aan de data-ingang $D$ staat in het geheugen worden geladen. Hierdoor is het bouwen van schakelingen met D-flipflops meestal zeer eenvoudig. Merk dus op dat elk signaal slechts \'e\'en klokperiode bewaard wordt. \figref{dataFlipflop} toont het symbool samen met de karakteristieke- en excitatietabel.
\begin{figure}[hbt]
\centering
\begin{tikzpicture}
\node[dff,anchor=west] (I) at (0,0) {};
\node (KT) at (8.5,0) {$\begin{array}{c|c}
D&Q_{\mbox{\tiny{next}}}\\\hline
0&0\\
1&1
\end{array}$};
\node[anchor=east] (ET) at (14,0) {$\begin{array}{cc|c}
Q&Q_{\mbox{\tiny{next}}}&D\\\hline
0&0&0\\
0&1&1\\
1&0&0\\
1&1&1\\
\end{array}$};
\node[anchor=north] (IT) at (I.south) {Symbool};
\node[anchor=north] (KTT) at (KT.south |- IT.north) {Karakteristieke tabel};
\node[anchor=north] (ETT) at (ET.south |- IT.north) {Excitatietabel};
\end{tikzpicture}
\caption{Data-flipflop.}
\figlab{dataFlipflop}
\end{figure}
\paragraph{T-flipflop} Een nieuwe variant is de zogenaamde \termen{toggle-flipflop} ofwel \termen{T-flipflop}. De toggle-flipflop heeft een `toggle'-ingang $T$. Indien deze ingang bij een stijgende klokflank hoog is, zal de flipflop het omgekeerde van zijn huidige waarde opslaan, de zogenaamde `\termen{toggle}'-operatie. Indien $T=0$, blijft de opgeslagen toestand dezelfde. We kunnen een toggle-flipflop bouwen met behulp van een data-flipflop en een XOR-poort. \figref{toggleFlipflop} toont het symbool, een mogelijke implementatie en de karakteristieke- en excitatietabel van de T-flipflop.
\begin{figure}[hbt]
\centering
\begin{tikzpicture}[circuit logic US]
\node[tff,anchor=west] (I) at (0,0) {};
\begin{scope}[xshift=4.5cm]
\filldraw[draw=black,dashed,fill=black!20] (-1.55,-1.1) rectangle (1.55,1.1);
\node[dff,scale=0.7] (DFF) at (0.65,0) {};
\node[xor gate,scale=0.7] (X) at (DFF.D -| -0.65,0) {};
\draw (X.output) -- (DFF.D);
\draw (X.input 1) -- ++(-0.2,0) |- (1.35,0.9) |- (DFF.Q);
\pdot{DFF.Q -| 1.35,0};
\draw (DFF.Q -| 1.35,0) -- (DFF.Q -| 1.75,0) node[scale=0.75,anchor=west]{$Q$};
\draw (DFF.Qn) -- (DFF.Qn -| 1.75,0) node[scale=0.75,anchor=west]{$Q_n$};
\draw (DFF.Clk) -- (DFF.Clk -| -1.75,0) node[scale=0.75,anchor=east]{Clk};
\draw (X.input 2) -- (X.input 2 -| -1.75,0) node[scale=0.75,anchor=east]{$T$};
\end{scope}
\node (KT) at (8.5,0) {$\begin{array}{c|c}
T&Q_{\mbox{\tiny{next}}}\\\hline
0&Q\\
1&Q'\\
\end{array}$};
\node[anchor=east] (ET) at (14,0) {$\begin{array}{cc|c}
Q&Q_{\mbox{\tiny{next}}}&T\\\hline
0&0&0\\
0&1&1\\
1&0&1\\
1&1&0\\
\end{array}$};
\node[anchor=north] (IT) at (I.south) {Symbool};
\node[anchor=north] (IMT) at (4.5,0 |- IT.north) {Implementatie};
\node[anchor=north] (KTT) at (KT.south |- IT.north) {Karakteristieke tabel};
\node[anchor=north] (ETT) at (ET.south |- IT.north) {Excitatietabel};
\end{tikzpicture}
\caption{Toggle-flipflop.}
\figlab{toggleFlipflop}
\end{figure}
\paragraph{JK-flipflop} De \termen{JK-flipflop} of voluit \termen{Jack Kilby-flipflop}\footnote{Genoemd naar Jack Kilby (1923-2005), Amerikaans natuurkundige en nobelprijswinnaar.} is een combinatie van de set-reset flipflop en de toggle-flipflop. Bij een SR-flipflop komen we immers in een ongeldige toestand indien we aan de ingangen $\left(S,R\right)=\left(1,1\right)$. De JK-flipflop lost dit probleem op door in dat geval een toggle-operatie uit te voeren op het geheugenelement, zoals te zien op de karakteristieke tabel op figuur \ref{fig:jackKilbyFlipflop}. We kunnen een JK-flipflop realiseren met behulp van een SR-flipflop waarbij:
\begin{equation}
\left\{\begin{array}{l}
S=J\cdot Q'\\
R=K\cdot Q
\end{array}\right.
\end{equation}
JK-flipflops worden vooral gebruikt om goedkopere schakelingen te synthetiseren. Deze eigenschap kunnen we afleiden uit de vele don't cares in de excitatietabel.
\begin{figure}[hbt]
\centering
\begin{tikzpicture}[circuit logic US]
\node[jkff,anchor=west] (I) at (0,0) {};
\begin{scope}[xshift=4.5cm]
\filldraw[draw=black,dashed,fill=black!20] (-1.55,-1.1) rectangle (1.55,1.1);
\node[srff,scale=0.7] (SRFF) at (0.65,0) {};
\node[and gate,scale=0.7] (A0) at (SRFF.S -| -0.65,0) {};
\node[and gate,scale=0.7] (A1) at (SRFF.R -| -0.65,0) {};
\draw (A0.output) -- (SRFF.S);
\draw (A1.output) -- (SRFF.R);
\draw (A0.input 2) -- ++(-0.2,0) |- (1.35,-0.9) |- (SRFF.Qn);
\draw (A1.input 1) -- ++(-0.4,0) |- (1.35,0.9) |- (SRFF.Q);
\pdot{SRFF.Q -| 1.35,0};
\pdot{SRFF.Qn -| 1.35,0};
\draw (SRFF.Q -| 1.35,0) -- (SRFF.Q -| 1.75,0) node[scale=0.75,anchor=west]{$Q$};
\draw (SRFF.Qn) -- (SRFF.Qn -| 1.75,0) node[scale=0.75,anchor=west]{$Q_n$};
\draw (SRFF.Clk) -- (SRFF.Clk -| -1.75,0) node[scale=0.75,anchor=east]{Clk};
\draw (A0.input 1) -- (A0.input 1 -| -1.75,0) node[scale=0.75,anchor=east]{$J$};
\draw (A1.input 2) -- (A1.input 2 -| -1.75,0) node[scale=0.75,anchor=east]{$K$};
\end{scope}
\node (KT) at (8.5,0) {$\begin{array}{cc|c}
J&K&Q_{\mbox{\tiny{next}}}\\\hline
0&0&Q\\
0&1&0\\
1&0&1\\
1&1&Q'
\end{array}$};
\node[anchor=east] (ET) at (14,0) {$\begin{array}{cc|cc}
Q&Q_{\mbox{\tiny{next}}}&J&K\\\hline
0&0&0&-\\
0&1&1&-\\
1&0&-&1\\
1&1&-&0\\
\end{array}$};
\node[anchor=north] (IT) at (I.south) {Symbool};
\node[anchor=north] (IMT) at (4.5,0 |- IT.north) {Implementatie};
\node[anchor=north] (KTT) at (KT.south |- IT.north) {Karakteristieke tabel};
\node[anchor=north] (ETT) at (ET.south |- IT.north) {Excitatietabel};
\end{tikzpicture}
\caption{Jack-Kilby flipflop.}
\figlab{jackKilbyFlipflop}
\end{figure}
\paragraph{Overzicht} We bundelen vervolgens al de latches en flipflops in figuur \ref{fig:latchFlipflopOverview}. Indien een cel leeg is bij dit overzicht is er geen realisatie mogelijk. Zo kunnen we onmogelijk een toggle-latch bouwen: deze zou immers bij een actief kloksignaal continue inverteren, waardoor het uiteindelijke resultaat onvoorspelbaar is. We kunnen elke flipflop verrijken met de assynchrone preset en clear ingangen. Verder bestaan er ook nog varianten van deze componenten waarbij we eerst een negatie toepassen op de ingang. In dat geval plaatsen we een cirkel bij deze ingang. Indien we een negatie toepassen op de klokingang bij een flipflop zal de flipflop dus de waarde memoriseren bij een falling edge, in plaats van een rising edge.
\begin{figure}[hbt]
\centering
\begin{tikzpicture}[decoration={brace}]
\def\dx{3};
\def\dy{-2.5};
\def\ya{0.5*\dy+0.25};
\def\xa{0.5*\dx-0.25};
\node (T0) at (\dx,\ya) {\textbf{Set-Reset (SR)}};
\node (T1) at (2*\dx,\ya) {\textbf{Data (D)}};
\node (T2) at (3*\dx,\ya) {\textbf{Toggle (T)}};
\node (T3) at (4*\dx,\ya) {\textbf{Jack Kilby (JK)}};
\node[rotate=90] (C0) at (\xa,\dy) {\textbf{Ongeklokt}};
\node[rotate=90] (C1) at (\xa,2*\dy) {\textbf{Geklokt}};
\node[rotate=90] (C2) at (\xa,3*\dy) {\textbf{Flankgeklokt}};
\foreach \i in {1,2,3} {
  \draw[thin,dashed] (0.5*\dx+\i*\dx,0 |- T0.north) -- (0.5*\dx+\i*\dx,3.5*\dy);
}
\foreach \i in {1,2} {
  \draw[thin,dashed] (0,0.5*\dy+\i*\dy -| C0.north) -- (4.5*\dx,0.5*\dy+\i*\dy);
}
\draw[very thick] (T0.south -| C0.south) -- (T0.north -| C0.south) -- (4.5*\dx,0 |- T0.north) -- (4.5*\dx,3.5*\dy) -- (0,3.5*\dy -| C2.north) -- (T0.south -| C0.north) -- (T0.south -| C0.south);
\draw[very thick] (4.5*\dx,0 |- T0.south) -- (T0.south -| C0.south) -- (0,3.5*\dy -| C0.south);
\draw [decorate,thick] (4.5*\dx+0.25,0 |- T0.south) -- (4.5*\dx+0.25,2.5*\dy);
\node[rotate=-90,anchor=south] (G0) at (4.5*\dx+0.5,1.5*\dy) {\textbf{Latch}};
\draw [decorate,thick] (4.5*\dx+0.25,2.5*\dy) -- (4.5*\dx+0.25,3.5*\dy);
\node[rotate=-90,anchor=south] (G1) at (4.5*\dx+0.5,3*\dy) {\textbf{Flipflop}};
\foreach\i/\j/\dj/\t/\tx in {1/1/-1/srlatch/NOR,1/1/1/srlatchnand/NAND, 2/1/0/clsrlatch/,2/2/0/cldlatch/, 3/1/-1/srff/Basis,3/1/1/srffx/Uitgebreid,3/2/-1/dff/Basis,3/2/1/dffx/Uitgebreid,3/3/-1/tff/Basis,3/3/1/tffx/Uitgebreid,3/4/-1/jkff/Basis,3/4/1/jkffx/Uitgebreid} {
  \node[scale=0.75,\t] (P) at (\j*\dx+\dj*0.22*\dx,\i*\dy) {};
  \draw(P.south) node[anchor=north,scale=0.75]{\tx};
}
\end{tikzpicture}
\caption{Overzicht van de interfaces van geheugencomponenten.}
\figlab{latchFlipflopOverview}
\end{figure}
\subsection{Registers}
\label{ss:registers}
\paragraph{Register}Een flipflop kan \'e\'en bit opslaan voor \'e\'en of meerdere klokflanken. Meestal willen we echter meerdere bits opslaan om bijvoorbeeld een getal voor te stellen. Dit kunnen we doen door middel van verschillende flipflops die elk een individuele bit opslaan. Een \termen{register} is in dat opzicht eigenlijk ook een $n$-flipflop. Omdat we vaak meerdere bits opslaan defini\"eren we dus het registercomponent. Een register neemt enige functionaliteit weg van de flipflops: zo kunnen we niet beslissen dat een individuele flipflop een bit aan de ingang opslaat, en de andere flipflops niet. Desalniettemin maken ze schema's eenvoudiger en worden registers op chipniveau verkocht. Verder biedt een dergelijk chip naast geheugenopslag ook extra functies aan: bijvoorbeeld schuifregisters en tellers. Een register heeft tradioneel ingangen voor de klok $\mbox{Clk}$, data $D_1, D_2,\ldots, D_n$, load $\mbox{LD}$ en assynchrone preset $\mbox{Pr}^*$ en clear $\mbox{Clr}^*$. Als uitgangen heeft het minstens de data-uitgangen $Q_1, Q_2, \ldots, Q_n$. Enkel indien de load hoog is bij een rising edge zal de waarde die aan de data-ingangen staat opgeslagen worden. Indien dit niet zo is, wordt de vorige waarde behouden. \figref{register} toont de interface van een register, samen met een mogelijke implementatie. We werken hier met D-flipflops en multiplexers om te kiezen tussen het laden van een nieuwe waarde en een vorige waarde. Meestal wordt de clear en preset in negatieve logica ge\"implementeerd\footnote{Vandaar de asterisk (*) bij $\mbox{Clr}^*$ en $\mbox{Pr}^*$.}.
\begin{figure}[hbt]
\centering
\subfigure[Interface]{\begin{tikzpicture}
\node[regd,scale=0.85] (R) at (0,0) {};
\draw (R.Q0) -- ++(0,-0.4) node[scale=0.75,anchor=north]{$Q_0$};
\draw (R.Q1) -- ++(0,-0.4) node[scale=0.75,anchor=north]{$Q_1$};
\draw (R.Q2) -- ++(0,-0.4) node[scale=0.75,anchor=north]{$Q_2$};
\draw (R.Q3) -- ++(0,-0.4) node[scale=0.75,anchor=north]{$Q_3$};
\draw (R.D0) -- ++(0,0.4) node[scale=0.75,anchor=south]{$D_0$};
\draw (R.D1) -- ++(0,0.4) node[scale=0.75,anchor=south]{$D_1$};
\draw (R.D2) -- ++(0,0.4) node[scale=0.75,anchor=south]{$D_2$};
\draw (R.D3) -- ++(0,0.4) node[scale=0.75,anchor=south]{$D_3$};
\draw (R.PR) -- ++(0.4,0) node[scale=0.75,anchor=west]{$\mbox{Pr}^*$};
\draw (R.CLR) -- ++(0.4,0) node[scale=0.75,anchor=west]{$\mbox{Clr}^*$};
\draw (R.LD) -- ++(-0.4,0) node[scale=0.75,anchor=east]{LD};
\draw (R.Clk) -- ++(-0.4,0) node[scale=0.75,anchor=east]{Clk};
\end{tikzpicture}}
\subfigure[Implementatie]{\begin{tikzpicture}
\filldraw[dashed,fill=black!20] (-1.5,-1.25) rectangle (5.5,1.65);
\foreach \i/\j in {0/3,1/2,2/1,3/0} {
  \node[scale=0.75,mux2to1] (M\i) at (1.5*\i-0.6,1.2) {};
  \node[scale=0.5,dffxn] (D\i) at (1.5*\i,0) {};
  \draw (M\i.output) |- (D\i.D);
  \draw (M\i.data1) --++ (0,0.5) node[scale=0.75,anchor=south]{$D_\j$};
  \coordinate (outQ\i) at (D\i.Q -| 1.5*\i+0.6,0);
  \coordinate (clkQ\i) at (1.5*\i-0.6,-1);
  \draw (clkQ\i) |- (D\i.Clk);
  \pdot{outQ\i};
  \draw (D\i.Q) -| (1.5*\i+0.6,-1.45) node[scale=0.75,anchor=north]{$Q_\j$};
  \draw (D\i.PR) -- (D\i.PR |- 0,0.8);
  \draw (D\i.CLR) -- (D\i.CLR |- 0,-0.8);
}
\foreach \i/\ii in {0/1,1/2,2/3} {
  \draw (M\i.data0) -- ++(0,0.1) -| (1.5*\i+0.2,1) -| (outQ\i);
  \draw (M\i.selout0) -- (M\ii.selin0);
  \pdot{D\ii.CLR |- 0,-0.8};
  \pdot{D\ii.PR |- 0,0.8};
  \pdot{clkQ\i};
}
\draw (M3.data0) -- ++(0,0.1) -| (outQ3);
\draw (M0.selin0) -- (M0.selin0 -| -1.75,0) node[scale=0.75,anchor=east]{LD};
\draw (clkQ3) -- (clkQ3 -| -1.75,0) node[scale=0.75,anchor=east]{Clk};
\draw (D0.PR |- 0,0.8) -- (5.75,0.8) node[scale=0.75,anchor=west]{$\mbox{Pr}^*$};
\draw (D0.CLR |- 0,-0.8) -- (5.75,-0.8) node[scale=0.75,anchor=west]{$\mbox{Clr}^*$};
\end{tikzpicture}}
\caption{Interface en implementatie van een 4-bit register.}
\figlab{register}
\end{figure}
\paragraph{Schuifregister}\label{s:schuifregisters}We hebben reeds vermeld dat men de meeste registers uitrust met extra functionaliteit. Een concreet voorbeeld hiervan is het \termen{schuifregister}. Indien we met normale registers werken kunnen we per klokflank tussen twee acties kiezen: een nieuwe waarde inladen (``load''), of de oude waarde behouden. Een schuifregisters voegt daar een functionaliteit aan toe: de oude waarde \'e\'en plaats naar rechts schuiven, en deze waarde bij de rising edge inladen. Hiervoor bevat een schuifregister een extra ingang $\mbox{Shft}$ die indien actief, de waarde van het register opschuift. Sommige schuifregisters bevatten bovendien extra ingangen om te bepalen of er naar links of rechts geschoven moet worden. $\mbox{SerIn}$ is een andere ingang, deze bepaalt welke bit er op de vrijgekomen plaats komt te staan. \figref{shiftRegister} toont de implementatie van een schuifregister die de data naar rechts opschuift.
\begin{figure}[hbt]
\centering
\subfigure[Implementatie]{\begin{tikzpicture}[scale=1.25]
\def\sc{1.25};
\filldraw[dashed,fill=black!20] (-1.5,-1.25) rectangle (5.5,2.15);
\foreach \i/\j in {0/3,1/2,2/1,3/0} {
  \node[scale=0.75*\sc,mux2to1] (M\i) at (1.5*\i-0.6,1.2) {};
  \node[scale=0.75*\sc,mux2to1] (N\i) at (1.5*\i-0.2,1.7) {};
  \draw (N\i.output) -- (N\i.output |- 0,1.45) -| (M\i.data0);
  \node[scale=0.5*\sc,dffxn] (D\i) at (1.5*\i,0) {};
  \draw (M\i.output) |- (D\i.D);
  \draw (M\i.data1) --++ (0,1) node[scale=0.75*\sc,anchor=south]{$D_\j$};
  \coordinate (outQ\i) at (D\i.Q -| 1.5*\i+0.6,0);
  \coordinate (clkQ\i) at (1.5*\i-0.6,-1);
  \draw (clkQ\i) |- (D\i.Clk);
  \pdot{outQ\i};
  \draw (D\i.Q) -| (1.5*\i+0.6,-1.45) node[scale=0.75*\sc,anchor=north]{$Q_\j$};
  \draw (D\i.PR) -- (D\i.PR |- 0,0.8);
  \draw (D\i.CLR) -- (D\i.CLR |- 0,-0.8);
}
\foreach \i/\ii in {0/1,1/2,2/3} {
  \coordinate (datbN\i) at (1.5*\i+0.4,1.95);
  \pdot{datbN\i};
  \draw (N\i.data0) |- (datbN\i) -- (1.5*\i+0.4,1) -| (outQ\i);
  \draw (N\ii.data1) |- (datbN\i);
  \draw (M\i.selout0) -- (M\ii.selin0);
  \draw (N\i.selout0) -- (N\ii.selin0);
  \pdot{D\ii.CLR |- 0,-0.8};
  \pdot{D\ii.PR |- 0,0.8};
  \pdot{clkQ\i};
}
\draw (N3.data0) -- ++(0,0.1) -| (outQ3);
\draw (M0.selin0) -- (M0.selin0 -| -1.75,0) node[scale=0.75*\sc,anchor=east]{LD};
\draw (N0.selin0) -- (N0.selin0 -| -1.75,0) node[scale=0.75*\sc,anchor=east]{Shft};
\draw (N0.data1) |- (-1.75,1.95) node[scale=0.75*\sc,anchor=east]{SerIn};
\draw (clkQ3) -- (clkQ3 -| -1.75,0) node[scale=0.75*\sc,anchor=east]{Clk};
\draw (D0.PR |- 0,0.8) -- (5.75,0.8) node[scale=0.75*\sc,anchor=west]{$\mbox{Pr}^*$};
\draw (D0.CLR |- 0,-0.8) -- (5.75,-0.8) node[scale=0.75*\sc,anchor=west]{$\mbox{Clr}^*$};
\end{tikzpicture}}
\caption{Implementatie van een 4-bit schuifregister.}
\figlab{shiftRegister}
\end{figure}
Schuifregisters worden vooral gebruikt om data serieel te maken: er wordt een getal ingeladen in het register, en vervolgens kunnen we per klokflank de laatste bit doorsturen. Aangezien het getal telkens verder opschuift sturen we zo na $n$-klokcycli een $n$-bit getal door. Dit kan handig zijn indien de kostprijs van meerdere draden te hoog is.
\subsection{Tellers}
\label{ss:counters}
Een andere functionaliteit die men vaak combineert met registers is tellen. Een \termen{teller} maakt het mogelijk om de waarde tijdens een klokflank met \'e\'en op te hogen: \termen{increment}. De meeste tellers laten echter ook toe om naar beneden te tellen: \termen{decrement}. De meeste tellers laten ook toe om een waarde in te laden en deze dan vervolgens in de volgende klokcycli te verhogen of verlagen. Indien de teller op de maximale of minimale voor te stellen waarde komt, treedt na de volgende cyclus een ``\termen{wrap-around}'' op: het getal voor het kleinste getal is het grootste en vice versa. Modulo-tellers wachten niet op de maximaal voor te stellen waarde alvorens deze wrap-around toe te passen. Zo telt een BCD-teller enkel tussen 0 en 9. We zouden een teller kunnen implementeren aan de hand van een register en een opteller\footnote{Analoog aan het schuifregister die uit een schuifoperatie en register bestaat.}. Er bestaan echter goedkopere manieren om tellers te implementeren.
\paragraph{}
Een teller die enkel naar boven telt wordt een \termen{up-counter} genoemd. Analoog wordt een teller die enkel naar beneden telt een \termen{down-counter} genoemd. Een teller die in beide richtingen kan tellen is een \termen{bidirectionial counter} ofwel \termen{bidirectionele teller}. Tellers introduceren ook een aantal nieuwe ingangen:
\begin{itemize}
 \item \termen{Counter Enabled $\mbox{CE}_{\mbox{\tiny{in}}}$}: enkel indien dit signaal hoog is, wordt het tellen uitgevoerd. In het andere geval blijft de waarde uit de vorige klokcyclus behouden.
 \item \termen{Down-Up $D/U^*$}: deze ingang bepaalt de richting waarin er geteld wordt. Uit het symbool kunnen we afleiden dat indien het signaal laag is we naar boven tellen, indien het signaal hoog is tellen we naar beneden.
\end{itemize}
Een uitgang die we regelmatig in een teller zien terugkomen in \termen{Counter Enabled $\mbox{CE}_{\mbox{\tiny{out}}}$}. Deze uitgang is hoog op de klokflank waarbij de teller een wrap-around uitvoert. Indien we dan een cascade van tellers bouwen zal de teller die erna geschakeld is bij een wrap-around opgehoogd worden. Op die manier kunnen we bijvoorbeeld met 3 4-bit tellers een 12-bit teller bouwen. Deze uitgang wordt soms ook de ``\termen{Ripple Carry Output (RCO)}'' genoemd.
\begin{figure}[hbt]
\centering
\begin{tikzpicture}[circuit logic US]
\def\sc{0.8};
\filldraw[dashed,fill=black!20] (-1.7*\sc,-2*\sc) rectangle (3*3*\sc+2.5*\sc,1.6*\sc);
\node[anchor=east,scale=\sc] (CE) at (-2*\sc,1.4*\sc) {$\mbox{CE}_{\mbox{\tiny{in}}}$};
\node[anchor=west,scale=\sc] (CEO) at (12*\sc,1.4*\sc) {$\mbox{CE}_{\mbox{\tiny{out}}}$};
\node[anchor=east,scale=\sc] (Clk) at (-2*\sc,-1.4*\sc) {Clk};
\node[anchor=east,scale=\sc] (Clr) at (-2*\sc,-1.7*\sc) {$\mbox{Clr}^*$};
\foreach\i in {0,1,2,3} {
  \node[tffxn,scale=\sc] (T\i) at (3*\sc*\i,0) {};
  \coordinate(Q\i) at (3*\sc*\i+\sc,-2.2*\sc);
  \coordinate(TT\i) at (3*\sc*\i-\sc,1.4*\sc);
  \coordinate (C\i) at (T\i.CLR |- Clr);
  \draw (TT\i) |- (T\i.T);
  \pdot{TT\i};
  \draw (T\i.CLR) -- (C\i);
  \draw (T\i.Q) -| (Q\i) node[anchor=north,scale=\sc]{$Q_\i$};
}
\foreach \i/\ii in {0/1,1/2,2/3} {
  \pdot{C\i};
  \draw (T\i.Qn) -- (T\ii.Clk);
}
\draw (CEO) -- (CE);
\draw (C3) -- (Clr);
\draw (Clk) -- (Clk -| TT0) |- (T0.Clk);
\end{tikzpicture}
\caption{Asynchrone 4-bit teller}
\figlab{asynchroneCounter}
\end{figure}
\paragraph{Asynchrone teller}
Bij een \termen{asynchrone teller}, \termen{asynchronous counter} ofwel \termen{ripple counter} wordt de verandering van de bittoestand van \'e\'en van de bits gebruikt als klokingang voor de volgende bit. Dit leidt tot minimaal hardwaregebruik, maar heeft hetzelfde probleem als een ripple adder: het signaal dient te propageren door de bits\footnote{Hiervan komt overigens de notie van asynchroniciteit: niet alle bits veranderen op hetzelfde moment van waarde.}, wat leidt tot grote vertragingen als het om veel bits gaat. Vooral indien we logica koppelen aan hoge bits kunnen deze vertragingen nefast zijn en leiden tot een lage performantie. We kunnen een down-counter realiseren door de $Q_n$-uitgang aan de toggle-ingang van de volgende flipflop te koppelen (en niet de $Q$-uitgang). Eventueel kunnen we deze schakeling dus zelfs uitbreiden door een multiplexer te plaatsen tussen de flipflops die selecteert tussen $Q$ en $Q_n$ en zo de gebruiker laat kiezen in welke richting de teller werkt. Het nadeel is dat ook deze multiplexer een vertraging induceert. Een eenvoudige implementatie van een asynchrone teller staat op figuur \ref{fig:asynchroneCounter}.
\begin{figure}[hbt]
\centering
\begin{tikzpicture}[circuit logic US]
\def\sc{0.8};
\filldraw[dashed,fill=black!20] (-1.7*\sc,-2*\sc) rectangle (3*3*\sc+2.5*\sc,1.6*\sc);
\node[anchor=east,scale=\sc] (CE) at (-2*\sc,1.4*\sc) {$\mbox{CE}_{\mbox{\tiny{in}}}$};
\node[anchor=west,scale=\sc] (CEO) at (12*\sc,1.4*\sc) {$\mbox{CE}_{\mbox{\tiny{out}}}$};
\node[anchor=east,scale=\sc] (Clk) at (-2*\sc,-1.4*\sc) {Clk};
\node[anchor=east,scale=\sc] (Clr) at (-2*\sc,-1.7*\sc) {$\mbox{Clr}^*$};
\foreach\i in {0,1,2,3} {
  \node[tffxn,scale=\sc] (T\i) at (3*\sc*\i,0) {};
  \coordinate(Q\i) at (3*\sc*\i+\sc,-2.2*\sc);
  \coordinate(TT\i) at (3*\sc*\i-\sc,1.4*\sc);
  \coordinate(Cl\i) at (3*\sc*\i-\sc,-1.4*\sc);
  \coordinate (C\i) at (T\i.CLR |- Clr);
  \draw (T\i.CLR) -- (C\i);
  \draw (T\i.Q) -| (Q\i) node[anchor=north,scale=\sc]{$Q_\i$};
  \draw (T\i.Clk) -| (Cl\i);
  \node[and gate,rotate=-90,scale=\sc] (A\i) at (3*\sc*\i+1.5*\sc,0.5*\sc) {};
  \draw (A\i.input 2) -- ++(0,0.2*\sc) -| (T\i.Q -| Q\i);
  \pdot{T\i.Q -| Q\i};
}
\draw (TT0) |- (T0.T);
\foreach \i/\ii in {0/1,1/2,2/3} {
  \draw (T\ii.T) -| (TT\ii |- 0,-0.2*\sc) -| (A\i.output);
  \draw (T\ii.T -| TT\ii) -- (TT\ii |- CE) -| (A\ii.input 1);
  \pdot{C\i};
  \pdot{Cl\i};
  \pdot{T\ii.T -| TT\ii};
}
\draw (CEO) -- ++(-1.5*\sc,0) |- (A3.output |- 0,-0.2*\sc) -- (A3.output);
\draw (A0.input 1) |- (CE);
\draw (C3) -- (Clr);
\draw (Clk) -- (Cl3);
\end{tikzpicture}
\caption{Synchrone 4-bit teller}
\figlab{synchroneCounter}
\end{figure}
\paragraph{Synchrone teller}
Een \termen{synchrone teller} laat alle bits wel tegelijk van waarde veranderen. Dit doen we door de volgende toestand reeds vooraf uit te rekenen en bij het kloksignaal deze toestand op te slaan in de flipflops. Hoe we deze toestand berekenen staat ons in principe vrij net als de keuze van het type flipflop. \figref{synchroneCounter} toont een implementatie met T-flipflops. Merk op dat ondanks het feit dat de nieuwe toestand berekend wordt voor de klokflank de teller daarom geen vertraging kan teweeg brengen: het berekenen van de volgende toestand van een teller met een grote woordlengte vraagt immers ook veel tijd (en kan dus het kritieke pad worden). Het voordeel van een synchrone teller is dat we deze berekening parallel doen met andere berekeningen die van de teller afhangen. Doordat de volgende toestand ook meteen beschikbaar is zullen berekeningen die afhangen van de teller ook onmiddellijk kunnen worden uitgerekend.
\begin{figure}[hbt]
\centering
\subfigure[Algemene implementatie]{\begin{tikzpicture}[circuit logic US]
\def\sc{0.8};
\def\ya{2.8*\sc};
\def\yb{\ya-0.6*\sc};
\filldraw[dashed,fill=black!20] (-1.7*\sc,-2*\sc) rectangle (3*3*\sc+2.5*\sc,3.6*\sc);
\node[anchor=east,scale=\sc] (CE) at (-2*\sc,2.7*\sc) {$\mbox{CE}_{\mbox{\tiny{in}}}$};
\node[anchor=west,scale=\sc] (CEO) at (12*\sc,2.7*\sc) {$\mbox{CE}_{\mbox{\tiny{out}}}$};
\node[anchor=east,scale=\sc] (DU) at (-2*\sc,3.4*\sc) {$\mbox{D}/\mbox{U}^*$};
\node[anchor=east,scale=\sc] (LD) at (-2*\sc,1.4*\sc) {LD};
\node[anchor=east,scale=\sc] (Clk) at (-2*\sc,-1.4*\sc) {Clk};
\node[anchor=east,scale=\sc] (Clr) at (-2*\sc,-1.7*\sc) {$\mbox{Clr}^*$};
\foreach\i in {0,1,2,3} {
  \node[dffxn,scale=\sc] (T\i) at (3*\sc*\i,0) {};
  \coordinate(TT\i) at (3*\sc*\i-\sc,1.4*\sc);
  \node[mux2to1,scale=\sc] (M\i) at (TT\i) {};
  \coordinate(Q\i) at (3*\sc*\i+\sc,-2.2*\sc);
  \coordinate(Cl\i) at (3*\sc*\i-\sc,-1.4*\sc);
  \coordinate (C\i) at (T\i.CLR |- Clr);
  \draw (T\i.CLR) -- (C\i);
  \draw (M\i.output) |- (T\i.D);
  \draw (M\i.data1) -- (M\i.data1 |- 0,3.9*\sc) node[anchor=south,scale=\sc]{$D_\i$};
  \draw (T\i.Q) -| (Q\i) node[anchor=north,scale=\sc]{$Q_\i$};
  \pdot{T\i.Q -| Q\i};
  \draw (T\i.Clk) -| (Cl\i);
  \node[counterdir,anchor=Ei,scale=\sc] (CD\i) at (CE -| 3*\sc*\i-0.5*\sc,0) {};
  \draw (CD\i.Di) -| (M\i.data0);
  \draw (T\i.Q -| Q\i) -- (Q\i |- M\i.north) -| (CD\i.Qi);
  \coordinate (DU\i) at (CD\i.Dir |- DU);
  \draw (CD\i.Dir) -- (DU\i);
  %\node[and gate,scale=0.75*\sc] (A\i) at (3*\sc*\i+2.5*\sc,\ya) {};
  %\node[xor gate,rotate=180,scale=0.75*\sc] (Xa\i) at (3*\sc*\i-0.35*\sc,\yb) {};
  %\node[xor gate,scale=0.75*\sc] (Xb\i) at (3*\sc*\i+\sc,\yb) {};
  %\coordinate (Xm\i) at (3*\sc*\i+0.325*\sc,0 |- Xb\i.input 2);
  %\coordinate (DU\i) at (3*\sc*\i+0.55*\sc,0 |- DU);
  %\coordinate (CE\i) at (3*\sc*\i+0.15*\sc,0 |- CE);
  %\pdot{Xm\i};
  %\pdot{CE\i};
  %\draw (Xm\i) -- (Xm\i |- M\i.data0) -| (T\i.Q -| Q\i);
  %\draw (Xa\i.output) -| (M\i.data0);
  %\draw (Xa\i.input 1) -- (Xb\i.input 2);
  %\draw (Xb\i.input 1) -| (DU\i);
  %\draw (Xa\i.input 2) -| (CE\i);
  %\draw (Xb\i.output) -- ++(0.2*\sc,0) |- (A\i.input 2);
}
\foreach \i/\ii in {0/1,1/2,2/3} {
  \draw (M\i.selout0) -- (M\ii.selin0);
  \draw (CD\i.Eii) -- (CD\ii.Ei);
  \pdot{DU\i};
  %\draw (A\i.output) -- (A\i.output -| 3*\sc*\ii+0.325*\sc,0) |- (A\ii.input 1);
}
%\draw (CE) -- (CE -| 0.325*\sc,0) |- (A0.input 1);
\draw (CE) -- (CD0.Ei);
\draw (CEO) -- (CD3.Eii);
\draw (M0.selin0) -- (LD);
\draw (DU3) -- (DU);
\draw (C3) -- (Clr);
\draw (Clk) -- (Cl3);
\end{tikzpicture}
\figlab{loadableSynchroneCounter}}
\subfigure[Interface]{\begin{tikzpicture}[circuit logic US]
\node[counterdbit] (C) at (0,0) {};
\draw (C.Q0) -- ++(0,-0.4) node[scale=0.75,anchor=north]{$Q_0$};
\draw (C.Q1) -- ++(0,-0.4) node[scale=0.75,anchor=north]{$Q_1$};
\draw (C.Q2) -- ++(0,-0.4) node[scale=0.75,anchor=north]{$Q_2$};
\draw (C.Q3) -- ++(0,-0.4) node[scale=0.75,anchor=north]{$Q_3$};
\draw (C.D0) -- ++(0,0.4) node[scale=0.75,anchor=south]{$D_0$};
\draw (C.D1) -- ++(0,0.4) node[scale=0.75,anchor=south]{$D_1$};
\draw (C.D2) -- ++(0,0.4) node[scale=0.75,anchor=south]{$D_2$};
\draw (C.D3) -- ++(0,0.4) node[scale=0.75,anchor=south]{$D_3$};
\draw (C.CEIN) -- (-2.025,0 |- C.CEIN) node[scale=0.75,anchor=east]{$\mbox{CE}_{\mbox{\tiny{in}}}$};
\draw (C.CEOUT) -- ++(0.4,0) node[scale=0.75,anchor=west]{$\mbox{CE}_{\mbox{\tiny{out}}}$};
\draw (C.CLR) -- (-2.125,0 |- C.CLR) node[scale=0.75,anchor=east]{$\mbox{Clr}^*$};
\draw (C.LD) -- (-2.125,0 |- C.LD) node[scale=0.75,anchor=east]{LD};
\draw (C.DU) -- (-2.125,0 |- C.DU) node[scale=0.75,anchor=east]{D/U$^*$};
\draw (C.Clk) -- (-2.125,0 |- C.Clk) node[scale=0.75,anchor=east]{Clk};
\end{tikzpicture}
\figlab{counterInterface}}
\subfigure[Hulpcomponent]{\begin{tikzpicture}[circuit logic US]
\def\sc{0.8};
\filldraw[dashed,fill=black!20] (-2.75*\sc,-1.25*\sc) rectangle (2.75*\sc,1.25*\sc);
\node[anchor=south,scale=\sc] (Dir) at (0,1.5*\sc) {Dir};
\node[anchor=north,scale=\sc] (Qi) at (0,-1.5*\sc) {$Q_i$};
\node[anchor=east,scale=\sc] (Ei) at (-3*\sc,0.5*\sc) {$E_i$};
\node[anchor=east,scale=\sc] (Di) at (-3*\sc,-0.5*\sc) {$D_i$};
\node[xor gate,scale=\sc,rotate=180] (Xa) at (-1.25*\sc,-0.5*\sc) {};
\node[xor gate,scale=\sc] (Xb) at (0.75*\sc,-0.5*\sc) {};
\node[and gate,scale=\sc,anchor=input 1] (A) at (1.75*\sc,0.5*\sc) {};
\node[anchor=west,scale=\sc] (Eii) at (3*\sc,0 |- A.output) {$E_{i+1}$};
\draw (Xa.input 1) -- (Xb.input 2);
\draw (Xb.input 1) -| (Dir);
\draw (Xa.output) -- (Di);
\draw (Ei) -- (A.input 1);
\draw (Xb.output) -- ++(0.25*\sc,0) |- (A.input 2);
\draw (Xa.input 2) -| (Ei -| -0.5*\sc,0);
\draw (A.output) -- (Eii);
\draw (Qi) -- (Qi |- Xa.input 1);
\pdot{Qi |- Xa.input 1};
\pdot{Ei -| -0.5*\sc,0};
\end{tikzpicture}
\figlab{loadableSynchroneCounterLambda}}
\caption{Parallel-laadbare bidirectionele 4-bit teller.}
\figlab{loadableSynchroneCounterTotal}
\end{figure}
\paragraph{Parallel-laadbare bidirectionele teller}
Een speciaal geval van een synchrone teller is een \termen{parallel-laadbare bidirectionele teller}. Een implementatie van zo'n teller staat op figuur \ref{fig:loadableSynchroneCounter}. Deze teller is \termen{parallel laadbaar} wanneer we de waarde van de teller kunnen zetten op een ingegeven waarde in \'e\'en klokflank\footnote{Het schuifregister of figuur \ref{fig:shiftRegister} kan bijvoorbeeld ook sequentieel geladen worden, waarbij we per klokflank een bit van het getal inschuiven.}. Bij het laden wordt de LD ingang hoog gezet. In het andere geval dienen we elke bit te berekenen. Hiervoor wordt op figuur \ref{fig:loadableSynchroneCounter} een naamloze component ingevoerd. De nieuwe waarde van een bit wordt ge\"inverteerd indien er geteld wordt op deze bit. In het andere geval blijft de bit onveranderd. Een bit wordt opgeteld indien de vorige bit geteld wordt en de bit \'e\'en is bij een telling naar boven (analoog aan een overdracht (``carry'') dus), of nul en een telling naar beneden (een lening (``borrow'') dus). Bij het berekenen van de eerste bit $Q_0$ gebruiken we logischerwijs de counter enabled CE ingang. \figref{loadableSynchroneCounterLambda} toont een implementatie voor een dergelijk component. De teller die we hiermee ontwikkelen verenigt de meeste functionaliteiten die we in tellers kunnen terugvinden. \figref{counterInterface} toont een interface die vaak gebruik wordt voor tellers. Indien een functionaliteit niet wordt aangeboden wordt de vermelding eenvoudigweg weggelaten.
\paragraph{Modulo-teller}Tot nu toe hebben we enkel teller geconstrueerd die tellen van 0 tot het maximaal voor te stellen getal. Bij een $n$-bit teller is dit dus $2^n-1$. In de praktijk komt het geregeld voor dat we tellen tot een bepaalde waarde om daarna terug bij 0 te beginnen. Stel bijvoorbeeld dat we een teller maken die het aantal minuten bijhoudt. Indien dit getal boven de 60 gaat komt het aantal minuten terug op 0, en dient het aantal uren verhoogd te worden. Ook controllers die elektronica aansturen dienen soms een beperkt aantal toestanden met de regelmaat van de klok te herhalen, zelden zijn dit aantal toestanden een macht van twee. Voor dergelijke problemen kunnen we een \termen{Modulo-teller} gebruiken. Een modulo-teller bestaat traditioneel uit een teller met daarrond extra logica die indien de maximale waarde wordt vastgesteld, in de volgende klokflank een 0 in de teller laadt. We kunnen deze teller veralgemenen door ook het tellen naar beneden toe te laten, in dat geval dient de maximale waarde in de klokcyclus na 0 ingeladen te worden. \figref{moduloCounter} toont een algemeen geval van een 4-bit teller.
\begin{figure}[hbt]
\centering
\subfigure[Algemeen geval]{
\begin{tikzpicture}[circuit logic US]
\node[counterdbit] (C) at (0,0) {};%1,392625
\draw (C.CLR) -- (-4,0 |- C.CLR) node[scale=0.75,anchor=east]{Clr$^*$};
\draw (C.Clk) -- (-4,0 |- C.Clk) node[scale=0.75,anchor=east]{Clk};
\draw (C.DU) -- (-4,0 |- C.DU) node[scale=0.75,anchor=east]{D/U$^*$};
\draw (C.CEIN) -- (-4,0 |- C.CEIN) node[scale=0.75,anchor=east]{$\mbox{CE}_{\mbox{\tiny{in}}}$};
\draw (C.Q0) -- ++(0,-3) node[scale=0.75,anchor=north]{$Q_0$};
\draw (C.Q1) -- ++(0,-3) node[scale=0.75,anchor=north]{$Q_1$};
\draw (C.Q2) -- ++(0,-3) node[scale=0.75,anchor=north]{$Q_2$};
\draw (C.Q3) -- ++(0,-3) node[scale=0.75,anchor=north]{$Q_3$};
\node[or gate,rotate=90,scale=0.75] (O) at (-3.5,-1.25) {};
\node[mux8to4,xscale=1.450651042,yscale=0.75] (M) at (0,2) {};%0.96
\node[and gate,rotate=180,scale=0.75,inputs={normal,normal,inverted,inverted,inverted,inverted}] (A1) at (-2.75,-2) {};
\draw (A1.output) -| (O.input 2);
\coordinate (DUT) at (C.DU -| -2.25,0);
\pdot{DUT};
\coordinate (CET) at (C.CEIN -| -2.5,0);
\pdot{CET};
\coordinate (LDT) at (C.LD -| -2,0);
\pdot{LDT};
\draw (LDT) |- (2,1.25) |- (C.CEOUT -| 2.3,0) node[scale=0.75,anchor=west]{$\mbox{CE}_{\mbox{\tiny{out}}}$};
\draw (DUT) |- (M.selin0);
\draw (A1.output |- 0,-2.75) rectangle (A1.input 1 |- 0,-3.75);
\draw (A1.output |- 0,-3.25) -| (O.input 1);
\foreach \x in {0,1,2,3} {
  \draw (M.output\x) -- (C.D\x);
  \draw (M.data0\x) -- ++(0,0.3) node[scale=0.75,anchor=south]{$0$};
  \draw (M.data1\x) -- ++(0,0.3);
}
\draw (-1.160520834,2.75) node[scale=0.75,anchor=south]{MAX};
\foreach\x/\y in {3/3,2/4,1/5,0/6} {
  \draw (A1.input \y) -- (A1.input \y -| C.Q\x);
  \draw (A1.input 1 |- 0,-3.75+\y/7) -- (0,-3.75+\y/7 -| C.Q\x);
  \pdot{0,-3.75+\y/7 -| C.Q\x};
  \pdot{A1.input \y -| C.Q\x};
}
\draw (A1 |- 0,-3.25) node[scale=0.75,text width=0.75 cm]{= MAX};
\draw (A1.input 2) -- (A1.input 2 -| -1.5,0);
\pdot{A1.input 2 -| -1.5,0};
\draw (A1.input 1) -- (A1.input 1 -| -1.25,0);
\pdot{A1.input 1 -| -1.25,0};
\draw (DUT) |- (-1.25,-1.25) |- (A1.input 1 |- 0,-3.75+1/7);
\draw (CET) |- (-1.5,-1.5) |- (A1.input 1 |- 0,-3.75+2/7);
\draw (O.output) |- (C.LD);
\end{tikzpicture}
\figlab{moduloCounter}}
\subfigure[BCD-teller]{
\begin{tikzpicture}[circuit logic US]
\node[counterdbit] (C) at (0,0) {};%1,392625
\draw (C.CLR) -- (-4,0 |- C.CLR) node[scale=0.75,anchor=east]{Clr$^*$};
\draw (C.Clk) -- (-4,0 |- C.Clk) node[scale=0.75,anchor=east]{Clk};
\draw (C.DU) -- (-4,0 |- C.DU) node[scale=0.75,anchor=east]{D/U$^*$};
\draw (C.CEIN) -- (-4,0 |- C.CEIN) node[scale=0.75,anchor=east]{$\mbox{CE}_{\mbox{\tiny{in}}}$};
\draw (C.Q0) -- ++(0,-3) node[scale=0.75,anchor=north]{$Q_0$};
\draw (C.Q1) -- ++(0,-3) node[scale=0.75,anchor=north]{$Q_1$};
\draw (C.Q2) -- ++(0,-3) node[scale=0.75,anchor=north]{$Q_2$};
\draw (C.Q3) -- ++(0,-3) node[scale=0.75,anchor=north]{$Q_3$};
\node[or gate,rotate=90,scale=0.75] (O) at (-3.5,-1.25) {};
\node[mux8to4,xscale=1.450651042,yscale=0.75] (M) at (0,2) {};%0.96
\node[and gate,rotate=180,scale=0.75,inputs={normal,normal,inverted,inverted,inverted,inverted}] (A1) at (-2.75,-2) {};
\node[and gate,rotate=180,scale=0.75,inputs={normal,normal,normal,inverted,inverted,normal}] (A2) at (-2.75,-3.25) {};
\draw (A1.output) -| (O.input 2);
\coordinate (DUT) at (C.DU -| -2.25,0);
\pdot{DUT};
\coordinate (CET) at (C.CEIN -| -2.5,0);
\pdot{CET};
\coordinate (LDT) at (C.LD -| -2,0);
\pdot{LDT};
\draw (LDT) |- (2,1.25) |- (C.CEOUT -| 2.3,0) node[scale=0.75,anchor=west]{$\mbox{CE}_{\mbox{\tiny{out}}}$};
\draw (DUT) |- (M.selin0);
\draw (A1.output |- 0,-3.25) -| (O.input 1);
\foreach \x/\y in {0/1,1/0,2/0,3/1} {
  \draw (M.output\x) -- (C.D\x);
  \draw (M.data0\x) -- ++(0,0.3) node[scale=0.75,anchor=south]{$0$};
  \draw (M.data1\x) -- ++(0,0.3) node[scale=0.75,anchor=south]{$\y$};
}
\foreach\x/\y in {3/3,2/4,1/5,0/6} {
  \draw (A1.input \y) -- (A1.input \y -| C.Q\x);
  \draw (A2.input \y) -- (A2.input \y -| C.Q\x);
  \pdot{A1.input \y -| C.Q\x};
  \pdot{A2.input \y -| C.Q\x};
}
\draw (A1.input 2) -- (A1.input 2 -| -1.5,0);
\pdot{A1.input 2 -| -1.5,0};
\draw (A1.input 1) -- (A1.input 1 -| -1.25,0);
\pdot{A1.input 1 -| -1.25,0};
\draw (DUT) |- (-1.25,-1.25) |- (A2.input 1);
\draw (CET) |- (-1.5,-1.5) |- (A2.input 2);
\draw (O.output) |- (C.LD);
\end{tikzpicture}
\figlab{bcdCounter}}
\caption{4-bit modulo-tellers.}
\end{figure}
We tellen hierbij van 0 tot en met MAX. Uiteraard moet MAX in dit geval wel voor te stellen zijn met 4-bit. Voor een $n$-bit modulo-counter geldt dus steeds: MAX$\leq 2^n$. Verder dienen we \'e\'en component afhankelijk van MAX te synthetiseren. Deze component vergelijkt de waarde die op de $Q_i$-uitgangen staat met MAX, indien deze aan elkaar gelijk zijn en $\mbox{D/U}^*$ en $\mbox{CE}_{\mbox{\tiny{in}}}$ hoog zijn, dient een 1 aan de uitgang van deze component te verschijnen, in alle andere gevallen een 0. Voor eender welke MAX is dit te realiseren met een AND-poort\footnote{Waarom?}. Verder zien we op de figuur ook een variant van een multiplexer. Deze multiplexer stelt eigenlijk 4 multiplexers voor die elk een bit uit het linkse en een bit uit het rechtse vak als invoer hebben, en die selecteren met dezelfde ingang, namelijk: $\mbox{D/U}^*$. Vanaf de volgende sectie zullen we vaak met registers met een groot aantal bits werken. In dat geval besparen ontwerpers zich het tekenen van de verschillende parallelle lijnen door \'e\'en lijn te tekenen. Verder wordt er dus gebruik gemaakt van de notatie van de multiplexers om aan te duiden dat elke bit van deze lijnen door een dergelijk component gaat.
\paragraph{BCD-teller}
Een speciaal geval van een modulo teller is een \termen{BCD-teller}. Deze teller telt van 0 tot en met 9. BCD ofwel Binary Coded Decimal werd reeds eerder besproken in subsectie \ref{ss:bcd}. BCD-tellers zijn populair bij het voorstellen van getallen die ook snel naar de gebruiker moeten worden gecommuniceerd. Dit komt omdat bij de omzetting van een getal naar het equivalent op bijvoorbeeld seven-segment displays, we cijfer per cijfer kunnen werken. Indien we het getal binair opslaan wordt deze omzetting veel complexer. Op figuur \ref{fig:bcdCounter} implementeren we dan ook een BCD-teller. Vermits 9 binair voorgesteld wordt door $1001_2$ kunnen we een AND-poort realiseren die in dat geval een 1 op de uitgang plaatst. Verder dienen we dus ook deze waarde in de multiplexer in te brengen. Een oplettende lezer zal misschien merken dat sommige van de multiplexers in dat geval zinloos worden, omdat we bijvoorbeeld moeten kiezen tussen een 0 en een 0.
\section{Synchrone schakelingen}
\label{s:synchroneSequence}
Nu we de bouwstenen hebben beschreven om een sequenti\"ele schakeling te bouwen zullen we een stappenplan ontwikkelen uit een set van specificaties een sequenti\"ele schakeling te ontwikkelen. In deze sectie gaan we ervan uit dat deze schakeling synchroon is. Dit betekent dat er sprake is van een klok die de schakeling aanstuurt. Het stappenplan kan opgedeeld worden in volgende stappen:
\begin{enumerate}
 \item Het opstellen van een toestandsdiagram uit de specificaties.
 \item Het minimaliseren van het aantal toestanden door F-gelijke toestanden te bepalen.
 \item Het implementeren van de toestanden in een geheugen
 \begin{enumerate}
  \item Het coderen van deze toestanden in geheugenelementen.
  \item Het kiezen van het type flipflop die de toestanden bijhoudt.
 \end{enumerate}
 \item Het implementeren van de combinatorische logica
 \begin{enumerate}
  \item Implementeren van logica die de volgende toestand berekent.
  \item Implementeren van logica die de uitgangen (uitvoer) berekent.
 \end{enumerate}
\end{enumerate}
\subsection{Leidende voorbeelden}
We zullen dit stappenplan doorlopen met behulp van twee voorbeelden. Het eerste is een Moore-FSM. Uit subsectie \ref{ss:classificationSequential} weten we nog dat de uitgang dus volledig bepaald wordt door de toestand. In het tweede voorbeeld behandelen we een Mealy-FSM waarbij ook de invoer een rol speelt. Het stappenplan zelf verschilt meestal op enkele punten tussen het proces voor een Moore-machine en een Mealy-machine. In dat geval zullen op de afbeeldingen aan de linkerkant de Moore-machine staan, en aan de rechterkant de Mealy-machine. Indien dit niet relevant is voor het concept zullen we telkens gebruik maken van de Mealy-machine. Deze keuze is uiteraard louter arbitrair.
\paragraph{}
We implementeren een Moore-machine die 1 op de uitvoer aanlegt als de laatste drie klokcycli afwisselend een 0, 1 en 0 zijn. We implementeren een min of meer equivalente constructie voor een Mealy-machine, hier dienen echter op de twee laatste klokcycli een 0 en 1 op de ingang aangelegd te worden, en dient er op het moment zelf een 0 op de ingang te staan. De Mealy-machine is dus de Moore-machine maar met \'e\'en klokcyclus verschoven.
\subsection{Stap 1: opstellen van het toestandsdiagram}
Het opstellen van het \termen{toestandsdiagram} is meestal de moeilijkste stap. Dit komt omdat men niet formeel kan uitdrukken hoe men uit de specificaties een toestandsdiagram opstelt. Het opstellen van een dergelijk diagram omvat echter meestal dezelfde vaardigheden als deze die bij bijvoorbeeld programmeren aan bod komen. Door middel van oefening kan men dan ook bekwaamheid verwerven.
\begin{figure}[hbt]
\centering
\subfigure[Moore-machine]{\begin{tikzpicture}[->,shorten >=1pt,auto,node distance=2cm,on grid,semithick,state/.style=state with output,every state/.style={draw=black!50,very thick,fill=black!20,scale=0.75}]
\node[state] (H) {$H$\nodepart{lower} $0$};
\node[state] (I) [right=of H] {$I$\nodepart{lower} $0$};
\node[state] (E) [above right=of I] {$E$\nodepart{lower} $0$};
\node[state] (M) [below right=of E] {$M$\nodepart{lower} $0$};
\node[state] (K) [right=of M] {$K$\nodepart{lower} $0$};
\node[state] (L) [below=of H] {$L$\nodepart{lower} $0$};
\node[state] (J) [below=of I] {$J$\nodepart{lower} $1$};
\node[state] (N) [below=of M] {$N$\nodepart{lower} $0$};
\node[state] (O) [below=of K] {$O$\nodepart{lower} $0$};
\node[state] (D) [left=of E] {$D$\nodepart{lower} $0$};
\node[state] (F) [below right=of J] {$F$\nodepart{lower} $0$};
\node[state] (G) [right=of F] {$G$\nodepart{lower} $0$};
\node[state] (B) [above=of H] {$B$\nodepart{lower} $0$};
\node[state] (C) [below=of L] {$C$\nodepart{lower} $0$};
\node[state,initial,initial text=RST, initial where=right] (A) [above=of B] {$A$\nodepart{lower} $0$};
\path (H) edge [loop above] node {0} (H)
          edge node {1} (I)
      (I) edge[swap] node {0} (J)
          edge[bend left] node {1} (K)
      (M) edge[bend left,swap] node {0} (J)
          edge node {1} (K)
      (K) edge node {0} (N)
          edge node {1} (O)
      (L) edge node {0} (H)
          edge node {1} (I)
      (J) edge node {0} (L)
          edge[bend left,swap] node {1} (M)
      (N) edge[bend left] node {0} (L)
          edge[swap] node {1} (M)
      (O) edge node {0} (N)
          edge [loop below] node {1} (O)
       (D) edge[swap] node {0} (H)
           edge node {1} (I)
       (E) edge node {0} (J)
           edge[bend left] node {1} (K)
       (F) edge[bend left] node {0} (L)
           edge node {1} (M)
       (G) edge node {0} (N)
           edge[swap] node {1} (O)
       (B) edge[bend left,swap] node {0} (D)
           edge[bend left] node {1} (E)
       (C) edge[bend right] node {0} (F)
           edge[bend right] node {1} (G)
       (A) edge node {0} (B)
           edge[bend right] node {1} (C);
\end{tikzpicture}}
\subfigure[Mealy-machine]{\begin{tikzpicture}[->,shorten >=1pt,auto,node distance=2cm,on grid,semithick,every state/.style={draw=black!50,very thick,fill=black!20,scale=0.75}]
\node[state] (D) {$D$};
\node[state] (B) [above right=of D] {$B$};
\node[state] (E) [below right=of B] {$E$};
\node[state,initial,initial text=RST, initial where=right] (A) [below right=of E]{$A$};
\node[state] (G) [below left=of A] {$G$};
\node[state] (C) [below left=of G] {$C$};
\node[state] (F) [above left=of C] {$F$};
\path (D) edge node[swap] {1/0} (E)
          edge [loop above] node {0/0} (D)
      (E) edge[bend left,swap] node {0/1} (F)
          edge node {1/0} (G)
      (F) edge node {0/0} (D)
          edge[bend left,swap] node {1/0} (E)
      (G) edge node[swap] {0/0} (F)
          edge [loop below] node {1/0} (G)
      (B) edge node {0/0} (D)
          edge node {1/0} (E)
      (C) edge node {0/0} (F)
          edge node {1/0} (G)
      (A) edge[bend right,swap] node {0/0} (B)
          edge[bend left] node {1/0} (C);
\end{tikzpicture}}
\caption{Toestandsdiagrammen van de leidende voorbeelden.}
\figlab{toestandsdiagram}
\end{figure}
\paragraph{Het toestandsdiagram} Alvorens we een toestandsdiagram kunnen opstellen zullen we eerst een toestandsdiagram formeel defini\"eren. Een toestandsdiagram bestaat uit een set van \termen{toestanden}. Een toestand duiden we aan met een ellips. Meestal benoemen we toestanden met een hoofdletter. Dit is niet verplicht maar maakt het makkelijk om naar een toestand te verwijzen. Verder zijn er ook \termen{transities}: overgangen van de ene toestand naar de andere\footnote{Een transitie kan ook naar dezelfde toestand gaan.} toestand onder een bepaalde \termen{ingangscombinatie}. Een transitie duiden we dan ook aan met behulp van een gerichte pijl tussen de twee toestanden. Voor elke mogelijke ingangscombinatie dienen we in elke toestand een transitie te voorzien. De ingangscombinatie waarbij de transitie van toepassing is noteren we bij de gerichte pijl. Het kloksignaal is in een synchrone sequenti\"ele schakeling geen onderdeel van de ingangscombinatie. Tot slot bevat een toestandsdiagram ook de \termen{initalisatie}, een gerichte pijl die naar een bepaalde toestand wijst maar niet uit een toestand komt. Het is de eerste toestand waarin de schakeling zich bevindt. Verder keert de schakeling ook naar deze toestand terug als de gebruiker een reset-operatie op de schakeling uitvoert. De \termen{uitgangscombinatie} wordt ook weergegeven op het toestandsdiagram. Deze verschilt uiteraard tussen een Moore-machine en een Mealy-machine: bij een Moore-machine worden de uitgangen bij de toestanden genoteerd, dus in de ellips. Vermits de uitgangen bij een Mealy-machine afhangen van de ingangscombinatie wordt de uitgang bij de transitiepijlen gezet. We maken hierbij een onderscheid tussen de ingangscombinatie en de uitgang door hier een slash (``/'') tussen te plaatsen. Op figuur \ref{fig:toestandsdiagram} geven we de toestandsdiagrammen van de Moore- en Mealy-machine weer.
\begin{table}[hbt]
\centering
\subtable[Moore-machine]{\small{\begin{tabular}{c|cc|c}
Toestand&0&1&Uitgang\\\hline
$A$&$B$&$C$&0\\
$B$&$D$&$E$&0\\
$C$&$F$&$G$&0\\
$D$&$H$&$I$&0\\
$E$&$J$&$K$&0\\
$F$&$L$&$M$&0\\
$G$&$N$&$O$&0\\
$H$&$H$&$I$&0\\
$I$&$J$&$K$&0\\
$J$&$L$&$M$&1\\
$K$&$N$&$O$&0\\
$L$&$H$&$I$&0\\
$M$&$J$&$K$&0\\
$N$&$L$&$M$&0\\
$O$&$N$&$O$&0
\end{tabular}}}
\subtable[Mealy-machine]{\small{\begin{tabular}{c|cc}
Toestand&0&1\\\hline
$A$&$B/0$&$C/0$\\
$B$&$D/0$&$E/0$\\
$C$&$F/0$&$G/0$\\
$D$&$D/0$&$E/0$\\
$E$&$F/1$&$G/0$\\
$F$&$D/0$&$E/0$\\
$G$&$F/0$&$G/0$
\end{tabular}}}
\caption{Toestandstabellen van de leidende voorbeelden.}
\tbllab{toestandstabel}
\end{table}
\paragraph{Toestandstabel}Een toestandsdiagram is een grafische voorstelling. Bij een groot aantal toestanden of bij bijvoorbeeld invoer van een toestandsdiagram in een computer, maken we meestal gebruik van een alternatieve voorstelling: de \termen{toestandstabel}. De rijen in de toestandstabel stellen de verschillende toestanden voor, de kolommen stellen de invoercombinatie voor. In een cel $i,j$ voor toestand $A_i$ en invoer $I_j$ plaatsen we de volgende toestand. In het geval van een Moore-machine voorzien we een extra kolom die per rij de uitgangscombinatie van deze toestand weergeeft. Bij een Mealy-machine plaatsen we in elke cel naast de volgende toestand de uitgangscombinatie voor toestand $A_i$ en invoer $I_j$. Ook hier plaatsen we een slash om de volgende toestand van de uitvoer te onderscheiden. De toestandstabellen van de leidende voorbeelden staan in tabel \ref{tbl:toestandstabel}.
\paragraph{Opstellen van een toestandsdiagram en -tabel}Zoals reeds gezegd is het opstellen van een toestandsdiagram of -tabel een kunst. Een algemene techniek die gehanteerd kan worden is het aantal klokflanken te bepalen dat we geheugen moeten voorzien.  In het geval van de voorbeelden is dit drie klokflanken voor de Moore-machine en twee klokflanken voor de Mealy-machine. Vervolgens kunnen we een toestandstabel of -diagram opstellen die de overgang naar verschillende toestanden van het geheugen voorstelt. Zo komt $A$ in beide voorbeelden overeen met de toestand waarin we starten, er is bijgevolg nog geen sprake over inhoud in het geheugen. $B$ en $C$ betekent dat er na de eerste klokflank respectievelijk een 0 en een 1 in het geheugen opgeslagen is. $D$, $E$, $F$ en $G$ spreken over de respectievelijke geheugentoestanden $00$, $01$, $10$ en $11$. Voor de Mealy-machine is dit reeds voldoende. Bij de Moore-machine introduceren we nog een extra niveau. De overgangen op dit laatste niveau houden in dat we de oudste bit vergeten en de nieuwe bit memoriseren. Hierdoor gaan alle overgangen van een toestand op het laatste niveau enkel naar toestanden op hetzelfde niveau. We kunnen machinaal eenvoudig een tabel opstellen voor een arbitraire diepte. Een groot nadeel is dat indien we $n$ klokflanken willen memoriseren, we $2^{n+1}-1$ toestanden bekomen. Na het opstellen van deze tabel dienen we enkel de specificaties nog te vertalen in de uitgangen van de toestanden in het geval van een Moore-machine, en de overgangen bij een Mealy-machine. Meestal kan men echter door logisch te redeneren voorkomen dat we dergelijke grote toestandstabellen moeten opstellen. In de volgende subsectie reduceren we het aantal toestanden tot de theoretische ondergrens. Of we de initi\"ele tabel opstellen met de hier beschreven methode of door logisch te redeneren verandert niets aan het resultaat van de volgende stap.
\subsection{Stap 2: Minimaliseren van de toestanden}
\label{ss:minimizeFSMSeq}
Nu we een toestandsdiagram opgesteld hebben kunnen we dit diagram implementeren met behulp van logica. Het loont echter meestal de moeite om eerst het toestandsdiagram te minimaliseren. Minder toestanden impliceren minder geheugencomponenten om de toestand bij te houden. Bovendien kunnen we meestal ook de achterliggende logica die de volgende toestand en de uitgangen berekent minimaliseren. Hiertoe geven we een methode die gegarandeerd het kleinste toestandsdiagram vindt die de specificaties kan implementeren. Met minimaal bedoelen we hier het aantal toestanden. Dit betekent dus niet noodzakelijk dat de schakeling die we hiermee verwezenlijken ook minimaal is.
\subsubsection{Het minimalisatiealoritme}
\paragraph{Het algemeen idee}De methode die we implementeren gaat uit van een positief idee: alle toestanden zijn vertegenwoordigers van dezelfde toestand. Uiteraard is dit niet altijd het geval. Er zijn twee omstandigheden waarin twee toestanden niet gelijk aan elkaar zijn:
\begin{enumerate}
 \item De uitgang van de toestanden of overgangen uit een toestand zijn verschillend. In dat geval kunnen we immers onmogelijk de uitgangsfunctie implementeren. Een uitgang kan immers niet tegelijk 0 of 1 zijn.
 \item De uitgang na een willekeurig aantal willekeurige invoer (configuraties bij een klokflank) is verschillend. In dat geval betekent dit dat we dus in de toekomst op het vorige probleem zullen botsen.
\end{enumerate}
\paragraph{De eerste voorwaarde}We kunnen deze twee condities eenvoudig in een algoritme implementeren. Het algoritme werkt op basis van een partitionering van de aanvankelijke toestandsruimte. Als twee toestanden tot dezelfde partitite behoren. betekent dit dat ze eigenlijk hetzelfde zijn. Zoals eerder vermeld begint de methode met een positieve ingesteldheid: alle toestanden zijn gelijk. De initi\"ele configuratie bevat dus \'e\'en partitie waar alle toestanden in zitten. Vervolgens kunnen de eerste voorwaarde toepassen. Toestanden die een verschillende uitgang opleveren kunnen onmogelijk hetzelfde zijn. Voor een Moore-machine betekent dit dus de uitgang van de toestand. In het leidend voorbeeld is de uitgang 0 of 1. We partitioneren de configuratie dus in een partitie die 0 als uitgang geeft en een partitie met 1 als uitgang. De configuratie is dan:
\begin{equation}
\begin{array}{lr}
\mbox{partitie}_{\mbox{\small{Moore,0}}}=\left\{\left\{A,B,C,D,E,F,G,H,I,K,L,M,N,O\right\},\left\{J\right\}\right\}&\mbox{(Leidend voorbeeld)}
\end{array}
\end{equation}
Merk echter op dat een schakeling ook meerdere lijnen als uitgang kan hebben. Indien de uitgang $n$ bits telt levert dit ons een configuratie op van hoogstens $2^n$ partities. In het geval een Mealy-machine leidt een toestand niet rechtstreeks tot een uitgang. Twee toestanden zijn dan gelijk indien voor elke invoer-configuratie bij deze toestand, we dezelfde uitvoerconfiguratie bekomen. In het leidend voorbeeld beschouwen we een 1-bit ingang en een 1-bit uitgang. Dit leidt dus tot hoogstens 4 partities. In ons geval zijn er maar 2 partities:
\begin{equation}
\begin{array}{lr}
\mbox{partitie}_{\mbox{\small{Mealy,0}}}=\left\{\left\{A,B,C,D,F,G\right\},\left\{E\right\}\right\}&\mbox{(Leidend voorbeeld)}
\end{array}
\end{equation}
In het algemene geval met een $m$-bit ingang en een $n$-bit uitgang bekomen we hoogstens $2^{n+m}$ partities.
\paragraph{De tweede voorwaarde}De tweede voorwaarde kunnen we ook afdwingen door middel van iteratie. Hierbij itereren we over de lengte van de invoer. Het specifieke geval is uiteraard het geval waarbij de uitvoer na \'e\'en klokcyclus reeds verschilt. Dit kunnen we herformuleren tot het volgende: Indien twee toestanden $x_0$ en $y_0$ onder een willekeurige invoer $i_0$ naar twee toestanden $x_1$ en $y_1$ gaan, en deze twee toestanden behoren niet tot dezelfde partities, dan behoren $x_0$ en $y_0$ ook niet tot dezelfde partitie. Deze uitspraak is logisch, stel dat $x_0$ en $y_0$ tot dezelfde partitie zouden behoren, dan kan het gebeuren dat we in de toestand geraken die door de partitie waar $x_0$ en $y_0$ toe behoort. Indien we daarna de invoer $i_0$ dienen te verwerken komen we in toestand die zowel $x_1$ en $x_2$ dient te vertegenwoordigen. Vermits $x_1$ en $y_1$ echter tot een andere partitie behoren is dit onmogelijk. We gebruiken deze regel om partities vervolgens verder op te delen. We gebruiken de eerste configuratie van de minimale Moore-machine als voorbeeld. Hierbij gaan van de eerste partitie de toestanden $A$, $B$, $C$, $D$, $F$, $G$, $H$, $K$, $L$, $N$ en $O$ onder invoer van $0$ en $1$ naar de eerste partitie. De toestanden $E$, $I$ en $M$ gaan onder invoer van $0$ naar de tweede partitie ($J$) en onder invoer van $1$ naar de eerste partitie. Bijgevolg splitsen we de eerste partitie op in $\left\{\left\{A,B,C,D,F,G,H,K,L,N,O\right\},\left\{E,I,M\right\}\right\}$. Vermits de tweede partitie reeds uit \'e\'en element bestaat, valt deze niet verder op te delen. De totale partitie na \'e\'en iteratie is dan gelijk aan:
\begin{equation}
\begin{array}{lr}
\mbox{partitie}_{\mbox{\small{Moore,1}}}=\left\{\left\{A,B,C,D,F,G,H,K,L,N,O\right\},\left\{E,I,M\right\},\left\{J\right\}\right\}&\mbox{(Leidend voorbeeld)}
\end{array}
\end{equation}
Het enige wat we nu moeten doen is deze stap herhalen op de nieuwe partitie $\mbox{partitie}_{\mbox{\small{Moore,1}}}$. Op die manier passen we de regel toe bij een invoerlengte van 2. De enige vraag die open blijft is wanneer we mogen stoppen. Deze vraag is eenvoudig te beantwoorden: zolang er partities bijkomen kan een volgende stap de partities verder verdelen. Indien de uitvoering van een stap geen wijzigingen aan de partities aanbrengt, zal de volgende stap dit uiteraard ook niet meer doen, en alle stappen hierna ook niet meer. In dat geval is het veilig om te stoppen. Bij de minimalisatie van de Moore-machine zullen we dan volgende partities bekomen:
\begin{equation}
\small{
\begin{array}{lr}
\left\{\begin{array}{l}
\mbox{partitie}_{\mbox{\small{Moore,0}}}=\left\{\left\{A,B,C,D,E,F,G,H,I,K,L,M,N,O\right\},\left\{J\right\}\right\}\\
\mbox{partitie}_{\mbox{\small{Moore,1}}}=\left\{\left\{A,B,C,D,F,G,H,K,L,N,O\right\},\left\{E,I,M\right\},\left\{J\right\}\right\}\\
\mbox{partitie}_{\mbox{\small{Moore,2}}}=\left\{\left\{A,C,G,K,O\right\},\left\{B,D,F,H,L,N\right\},\left\{E,I,M\right\},\left\{J\right\}\right\}\\
\mbox{partitie}_{\mbox{\small{Moore,3}}}=\left\{\left\{A,C,G,K,O\right\},\left\{B,D,F,H,L,N\right\},\left\{E,I,M\right\},\left\{J\right\}\right\}
\end{array}\right.&\mbox{(Leidend voorbeeld)}
\end{array}}
\end{equation}
Het geval van de Mealy-machine is volledig analoog: we bekomen dan:
\begin{equation}
\small{
\begin{array}{lr}
\left\{\begin{array}{l}
\mbox{partitie}_{\mbox{\small{Mealy,0}}}=\left\{\left\{A,B,C,D,F,G\right\},\left\{E\right\}\right\}\\
\mbox{partitie}_{\mbox{\small{Mealy,1}}}=\left\{\left\{A,C,G\right\},\left\{B,D,F\right\},\left\{E\right\}\right\}\\
\mbox{partitie}_{\mbox{\small{Mealy,2}}}=\left\{\left\{A,C,G\right\},\left\{B,D,F\right\},\left\{E\right\}\right\}
\end{array}\right.&\mbox{(Leidend voorbeeld)}
\end{array}}
\end{equation}
\subsubsection{Omzetting naar een toestandsdiagram en -tabel}
Nadat we het aantal toestanden geminimaliseerd hebben, dienen we alleen nog een nieuwe Moore- of Mealy-machine te bouwen op basis van de partitie van de toestanden van de originele machine. Zoals we al enkele keren vermeld hebben staat een partitie voor de toestand van de minimale machine. We dienen dus voor elke partitie een nieuwe toestand te voorzien. Dit doet men meestal door de nieuwe toestand dezelfde naam te geven als de toestand in de partitie die alfabetisch het eerst voorkomt. Maar we kunnen uiteraard ook een arbitraire naam kiezen, of een naam die de letters van alle inwendige toestanden omvat. Verder dienen we in het geval van een Moore machine aan elke toestand een uitvoerconfiguratie toe te kennen. Omdat we de eerste voorwaarde hebben afgedwongen hebben alle toestanden in een partitie dezelfde uitvoerconfiguratie, bijgevolg nemen we de configuratie van \'e\'en van de toestanden in de partitie over. Hetzelfde geldt voor de Mealy-machine. De uitgang verbonden aan de transitie uit een bepaalde nieuwe toestand is dezelfde als de uitgang van dezelfde transitie uit \'e\'en van de toestanden in de partitie. De transitiefunctie kunnen we opstellen op basis van de tweede voorwaarde: alle toestanden in partitie zullen voor eenzelfde ingangscombinatie naar eenzelfde partitie lopen. Op die manier kunnen we dus een transitie-functie opstellen over de nieuwe toestanden. De initi\"ele toestand ten slotte is de toestand verbonden aan de partitie die de oorspronkelijke initi\"ele toestand bevat. Op figuur \ref{fig:minimaalToestandsdiagram} en tabel \ref{tbl:minimaalToestandstabel} staan de nieuwe machines na minimalisatie van de leidende voorbeelden. We kunnen eenvoudig vaststellen dat minimalisatie soms tot spectaculaire reducties van het aantal toestanden kan leiden.
\begin{figure}[hbt]
\centering
\subfigure[Moore-machine]{\begin{tikzpicture}[->,shorten >=1pt,auto,node distance=2cm,on grid,semithick,state/.style=state with output,every state/.style={draw=black!50,very thick,fill=black!20,scale=0.75}]
\node[state,initial,initial text=RST, initial where=left] (A) {$A$\nodepart{lower} $0$};
\node[state] (B) [right=of A] {$B$\nodepart{lower} $0$};
\node[state] (E) [below=of B] {$E$\nodepart{lower} $0$};
\node[state] (J) [left=of E] {$J$\nodepart{lower} $1$};
\path (A) edge node {0} (B)
          edge[loop above] node {1} (A)
      (B) edge[loop above] node {0} (B)
          edge node {1} (E)
      (E) edge[swap] node {0} (J)
          edge[bend right] node {1} (A)
      (J) edge[bend left,swap] node {0} (B)
          edge[bend right] node {1} (E);
\end{tikzpicture}
\figlab{minimaalToestandsdiagramMoore}}
\subfigure[Mealy-machine]{\begin{tikzpicture}[->,shorten >=1pt,auto,node distance=2cm,on grid,semithick,every state/.style={draw=black!50,very thick,fill=black!20,scale=0.75}]
\node[state,initial,initial text=RST, initial where=left] (A) {$A$};
\node[state] (B) [right=of A] {$B$};
\node[state] (E) [below=of B] {$E$};
\path (A) edge node {0/0} (B)
          edge[loop above] node {1/0} (A)
      (B) edge[loop above] node {0/0} (B)
          edge node[swap] {1/0} (E)
      (E) edge[bend right,swap] node {0/1} (B)
          edge node {1/0} (A);
\end{tikzpicture}
\figlab{minimaalToestandsdiagramMealy}}
\caption{Geminimaliseerde toestandsdiagrammen van de leidende voorbeelden.}
\figlab{minimaalToestandsdiagram}
\end{figure}
\begin{table}[hbt]
\centering
\subtable[Moore-machine]{\small{\begin{tabular}{c|cc|c}
Toestand&0&1&Uitgang\\\hline
$A$&$B$&$A$&0\\
$B$&$B$&$E$&0\\
$E$&$J$&$A$&0\\
$J$&$B$&$E$&1
\end{tabular}}}
\subtable[Mealy-machine]{\small{\begin{tabular}{c|cc}
Toestand&0&1\\\hline
$A$&$B/0$&$A/0$\\
$B$&$B/0$&$E/0$\\
$E$&$B/1$&$A/0$
\end{tabular}}}
\caption{Geminimaliseerde toestandstabellen van de leidende voorbeelden.}
\tbllab{minimaalToestandstabel}
\end{table}
\paragraph{Een formeel algoritme}
De cursus ``Automaten en Berekenbaarheid'' van prof. Demoen\cite{aenb10} bevat een formeel algoritme voor de minimalisatie van een deterministische eindige toestandsautomaat (DFA). Met minimale veranderingen kan dit algoritme omgevormd worden tot een algoritme die Moore- en Mealy-machines minimaliseert. We beschouwen dit in \algoref{alg:minimizeFSM}.
\begin{algorithm}[hbt]
\caption{Minimaliseren van een toestandsdiagram.}\label{alg:minimizeFSM}
\begin{algorithmic}[1]
\Procedure{Minimize}{$S,I,O,\delta,f$}\Comment{Minimaliseer de machine}
\State $\mathcal{Q}\gets\Call{Minimize1}{S,I,O,f}$\Comment{Initi\"ele partitionering gebaseerd op uitvoer.}
\Repeat
\State $\mathcal{P}\gets\mathcal{Q}$
\State $\mathcal{Q}\gets\Call{Minimize2}{S,I,\delta,\mathcal{P}}$\Comment{Bereken de nieuwe partitionering op basis van de oude.}
\Until{$\mathcal{P}=\mathcal{Q}$}\Comment{Indien geen verandering is het algoritme ten einde.}
\State \textbf{return} $\mathcal{Q}$\Comment{De uiteindelijke partitie.}
\EndProcedure
\Function{Minimize1Moore}{$S,I,O,f$}
\State \textbf{return} $\left\{\left\{s|s\in S:f\left(s\right)=o\right\}|o\in O\right\}$\Comment{$I$ wordt genegeerd.}
\EndFunction
\Function{Minimize1Mealy}{$S,I,O,f$}
\State \textbf{return} $\left\{\left\{s|s\in S,\forall i\in I\wedge\forall f\left(s,i\right)=o_i\right\}|o\in O,\forall i\in I:\exists o_i\in O\right\}$
\EndFunction
\Function{Minimize2}{$S,I,\delta,\mathcal{P}$}
\State \textbf{return} $\left\{\left\{s|s\in P\wedge \forall i \in I:\delta\left(s,i\right)=P_i\right\}|P\in\mathcal{P},\forall i\in I:\exists P_i\in\mathcal{P}\right\}$
\EndFunction
\end{algorithmic}
\end{algorithm}
Het algoritme bevat de algemene \procedureref{Minimize}-procedure. Deze procedure maakt gebruik van twee functies: \procedureref{Minimize1} en \procedureref{Minimize2}. De eerste functie wijkt af tussen een Moore- en Mealy-machine. Daarom wordt deze functie opgesplitst: voor de Moore-machine gebruiken we dus \procedureref{Minimize1Moore}, de Mealy-machine maakt gebruik van \procedureref{Minimize1Mealy}. We dienen ook een formele beschrijving te geven van de variabelen in het algoritme. Het algoritme heeft als invoer 3 verzamelingen:
\begin{itemize}
 \item $S$ is de verzameling van alle toestanden.
 \item $I$ is de verzameling van alle invoerconfiguraties.
 \item $O$ is de verzameling van alle uitvoerconfiguraties.
\end{itemize}
Daarnaast heeft het algoritme ook nood aan twee functies:
\begin{itemize}
 \item Een transitiefunctie $\delta:S\times I\rightarrow S$ die de volgende toestand beschouwt na een bepaalde invoer op de ingangen te hebben waargenomen bij de klokflank.
 \item Een uitvoerfunctie $f$. Bij de Moore-machine is deze functie van de signatuur $f:S\rightarrow O$, bij een Mealy-machine is dit $f:S\times I\rightarrow O$.
\end{itemize}
Het algoritme is ge\"implementeerd in het softwarepakket die wordt meegeleverd met deze cursus. De interface en de implementatie wordt besproken in \apprefpag{software}.
\subsection{Stap 3: Implementeren van de toestanden in het geheugen}
\subsubsection{Stap 3A: Coderen van de toestanden}
\label{term:minimalBitChange}
\label{term:grayCodeCounter}
Tot nu toe hebben we toestanden altijd voorgesteld met letters. Uiteraard dienen we deze toestanden op de een of andere manier voor te stellen in de geheugencomponenten in onze schakeling. Vermits letters niet opgeslagen kunnen worden in een flipflop of een register\footnote{Uiteraard kunnen we bijvoorbeeld wel het ASCII equivalent opslaan.} zullen we de toestanden moeten omzetten naar een binaire voorstelling. Dit wordt het \termen{coderen} van de toestanden genoemd. Doorgaans lijkt dit geen ingewikkeld probleem, we kunnen eenvoudigweg elke toestand een opeenvolgend nummer geven en deze toestanden dan binair coderen. Merk echter op dat de toestand geen impliciete ordening hebben, alleen al door opeenvolgende nummers te gebruiken zijn er $n!$ mogelijke coderingen mogelijk. Er bestaan echter technieken die ervoor zorgen dat we door een intelligente codering minder hardware zullen gebruiken. Dit leidt meestal tot goedkopere en snellere schakelingen. Algemeen zijn er drie technieken die we kunnen gebruiken:
\begin{itemize}
 \item ``\termen{Straightforward codering}'': soms is de codering triviaal.
 \item De ``\termen{one-hot codering}'': hierbij stellen we $n$ toestanden voor door $n$ bits, elke toestand krijgt z'n eigen bit. Indien de toestand actief is, is deze bit hoog, de andere bits zijn dan laag.
 \item De ``\termen{minimal-bit-change}'': we zoeken een codering die telkens een minimum aan bits verandert. Hierbij gebruiken we $\left\lceil\log_2 n\right\rceil$ bits om de $n$ toestanden voor te stellen.
\end{itemize}
Uiteraard dienen we ons niet te beperken tot \'e\'en van deze implementaties. Indien het aantal uitgangen bijvoorbeeld niet voldoet om een straightforward codering toe te passen kunnen we dit bijvoorbeeld aanvullen met extra bits die een minimal-bit change gebruiken. We bespreken nu de verschillende coderingstechnieken in detail.
\paragraph{Straightforward codering}Een codering kan triviaal zijn als de toestand een duidelijke betekenis heeft. Dit is doorgaans het geval bij bijvoorbeeld tellers. Ook indien we over een Moore-machine beschikken waarbij elke toestand een andere uitgang heeft en de uitgang minstens $\left\lceil\log_2 n\right\rceil$ bits bevat, kunnen we een straightforward codering toepassen. In dat geval gebruiken we de uitgang van een toestand ook als zijn codering. Merk op dat in dat geval we geen logica nodig hebben die de toestand naar de uitgang codeert. Deze codering is echter niet altijd ideaal. We reduceren dan de logica aan de uitgang, maar meestal resulteert dit in complexere logica om de transities te implementeren. Verder veranderen er meestal heel wat bits per overgang. We herinneren ons dat een CMOS-implementatie enkel energie verbruikt wanneer deze omschakelt. Indien we dus veel flipflops van waarde moeten laten veranderen betekent dit dus een groter vermogenverbruik. Een tweede aspect is dat niet elke bit tegelijk verandert. Vermits er meerdere bits zijn kan dit problemen opleveren: er bestaat een gevaar voor \termen{glitches}, een tijdelijk foute waarde op een lijn ten gevolge van asynchroon gedrag aan de ingang. Indien het resultaat van een teller dus nog in een andere schakeling gebruikt wordt leidt dit mogelijk tot problemen.
\paragraph{One-hotcodering}We voorzien een flipflop voor elke toestand, hierdoor is het aantal flipflops \bigoh{n} in tegenstelling tot de \bigoh{\log_2n}. Bij elke toestand is er \'e\'en flipflop hoog, alle andere flipflops zijn laag. Het spreekt dus voor zich dat we deze configuratie enkel kunnen gebruiken bij een klein aantal toestanden. One-hotcoderingen hebben enkele voordelen: het ontwerp van een dergelijke schakeling is vrij eenvoudig en is dus snel te realiseren. Verder is het ideaal voor een implementatie bij een FPGA. Een one-hotcodering verbruikt doorgaans weinig vermogen aangezien bij elke overgang tussen twee verschillende toestanden er twee flipflops omschakelen. Tot slot zijn ook de combinatorische schakelingen voor de uitgangen en de transities doorgaans vrij goedkoop. Het grootste nadeel van een one-hot codering is dan ook de kostprijs voor de flipflops.
\paragraph{Minimal-bit-change}Indien de kostprijs en vermogenverbruik een belangrijke factor zijn maken we meestal gebruik van de minimal-bit-change. Hierbij proberen we ervoor te zorgen dat de som van het aantal bits die veranderen van alle overgangen minimaal is. Dit probleem is echter niet triviaal en is NP-compleet. Bij een klein aantal toestanden kunnen we alle mogelijkheden uitproberen, bij een groter aantal maakt men gebruik van programma's zoals bijvoorbeeld MUSE, JEDI, MUSTANG,... Deze programma's werken op basis van heuristieken en geven meestal een benaderende oplossing. Soms worden er ook kansen bij de overgangen betrokken om het de kosten en het vermogenverbruik verder te minimaliseren. Een typisch voorbeeld is de \termen{Gray-code teller}\footnote{Vernoemd naar Frank Gray (1887-1969), fysicus bij Bell Labs.}. Gray ontwikkelde een teller die bij elke overgang slechts \'e\'en bit veranderde. \figref{grayCodeCounter} toont het toestandsdiagram van een 2-bit en 3-bit teller. Op de bogen plaatsen we hier het aantal bits die veranderen. Het feit dat we niet voor een straightforward implementatie voor een teller kiezen kan vreemd lijken. Deze teller dient dan ook meestal niet om de binaire waarde onmiddellijk uit te lezen, maar kan bijvoorbeeld gebruikt worden als een controller die een lus vormt over een vast aantal toestanden. Over het concreet oplossen van dit probleem gaan we niet verder in.
\begin{figure}[hbt]
\centering
\subfigure[Gray-code (minimal-bit-change)]{\begin{tikzpicture}[->,shorten >=1pt,auto,node distance=2cm,on grid,semithick,every state/.style={draw=black!50,very thick,fill=black!20,scale=0.75}]
\node[state,initial,initial text=RST, initial where=left] (A) {$00$};
\node[state] (B) [right=of A] {$01$};
\node[state] (C) [below=of B] {$11$};
\node[state] (D) [left=of C] {$10$};
\begin{scope}[node distance=1.414cm]
  \node[state,initial,initial text=RST, initial where=left,above left=of A] (AA) {$000$};
  \node[state] (CC) [above right=of B] {$011$};
  \node[state] (EE) [below right=of C] {$110$};
  \node[state] (GG) [below left=of D] {$101$};
\end{scope}
\node[state] (BB) [right=of AA] {$001$};
\node[state] (DD) [below=of CC] {$010$};
\node[state] (FF) [left=of EE] {$111$};
\node[state] (HH) [above=of GG] {$100$};
\path (A) edge node{1} (B)
      (B) edge node{1} (C)
      (C) edge node{1} (D)
      (D) edge node{1} (A);
\path (AA) edge node{1} (BB)
      (BB) edge node{1} (CC)
      (CC) edge node{1} (DD)
      (DD) edge node{1} (EE)
      (EE) edge node{1} (FF)
      (FF) edge node{1} (GG)
      (GG) edge node{1} (HH)
      (HH) edge node{1} (AA);
\end{tikzpicture}}
\subfigure[Straightforward]{\begin{tikzpicture}[->,shorten >=1pt,auto,node distance=2cm,on grid,semithick,every state/.style={draw=black!50,very thick,fill=black!20,scale=0.75}]
\node[state,initial,initial text=RST, initial where=left] (A) {$00$};
\node[state] (B) [right=of A] {$01$};
\node[state] (C) [below=of B] {$10$};
\node[state] (D) [left=of C] {$11$};
\begin{scope}[node distance=1.414cm]
  \node[state,initial,initial text=RST, initial where=left,above left=of A] (AA) {$000$};
  \node[state] (CC) [above right=of B] {$010$};
  \node[state] (EE) [below right=of C] {$100$};
  \node[state] (GG) [below left=of D] {$110$};
\end{scope}
\node[state] (BB) [right=of AA] {$001$};
\node[state] (DD) [below=of CC] {$011$};
\node[state] (FF) [left=of EE] {$101$};
\node[state] (HH) [above=of GG] {$111$};
\path (A) edge node{1} (B)
      (B) edge node{2} (C)
      (C) edge node{1} (D)
      (D) edge node{2} (A);
\path (AA) edge node{1} (BB)
      (BB) edge node{2} (CC)
      (CC) edge node{1} (DD)
      (DD) edge node{3} (EE)
      (EE) edge node{1} (FF)
      (FF) edge node{2} (GG)
      (GG) edge node{1} (HH)
      (HH) edge node{3} (AA);
\end{tikzpicture}}
\caption{Een 2-bit en 3-bit Gray-code teller en zijn straightforward equivalent.}
\figlab{grayCodeCounter}
\end{figure}
\paragraph{Leidende voorbeelden}Voor de Moore-machine kiezen we voor een minimal-bit-change benadering en voor de Mealy-machine een one-hot codering. Deze keuzes zijn louter didactief. De toegewezen bitvoorstellingen worden in een toestandstabel geschreven. Voor een one-hot codering is dit vrij triviaal. We wijzen elke toestand \'e\'en bit toe. Welke bit heeft geen invloed op de schakeling. Tabel \ref{tbl:mealyBitRepresentation} stelt de Mealy-machine uit de leidende voorbeelden voor met de one-hot codering.
\begin{table}[hbt]
\centering
\begin{tabular}{c|cc}
Toestand&0&1\\\hline
\texttt{001}&\texttt{010/0}&\texttt{001/0}\\
\texttt{010}&\texttt{010/0}&\texttt{100/0}\\
\texttt{100}&\texttt{010/1}&\texttt{001/0}\\
\end{tabular}
\caption{Codering van de Mealy-machine van het leidend voorbeeld.}
\tbllab{mealyBitRepresentation}
\end{table}
Bij het zoeken naar een minimal-bit-change voor de Moore-machine maken we gebruik van een greedy algoritme. We stellen eerst een tabel op die het aantal bindingen tussen twee toestanden bevat. Vervolgens wijzen we coderingen aan toestanden toe volgens het aantal bindingen tussen de toestanden. De bindingstabel staat beschreven in tabel \ref{tbl:mooreBitRepresentationBinding}. We zien dat $E$ en $J$ een dubbele binding bevat. We kennen hier de waarde $01$ toe aan $E$ en $11$ bij $J$. Verder is $B$ gelinkt aan $E$ en daarom geven we dit de waarde $00$. $A$ krijgt ten slotte de waarde $10$.
\begin{table}[hbt]
\centering
\subtable[Bindingstabel]{
\begin{tabular}{c|cccc}
&$A$&$B$&$E$&$J$\\\hline
$A$&1&1&1&0\\
$B$&-&1&1&1\\
$E$&-&-&0&2\\
$J$&-&-&-&0\\
\end{tabular}
\tbllab{mooreBitRepresentationBinding}
}
\subtable[Coderingstabel]{\begin{tabular}{c|cc|c}
Toestand&0&1&Uitgang\\\hline
\texttt{10}&\texttt{00}&\texttt{10}&\texttt{0}\\
\texttt{00}&\texttt{00}&\texttt{01}&\texttt{0}\\
\texttt{01}&\texttt{11}&\texttt{10}&\texttt{0}\\
\texttt{11}&\texttt{00}&\texttt{01}&\texttt{1}\\
\end{tabular}}
\caption{Codering van de Moore-machine van het leidend voorbeeld.}
\tbllab{mooreBitRepresentation}
\end{table}
\subsubsection{Stap 3B: De keuze van het type flipflop}
Nadat we elke toestand kunnen voorstellen met een sequentie aan bits, dienen we flipflops te voorzien om deze toestand bij te houden. Hierbij hebben we de keuze tussen de verschillende flipflops die we in \ref{sss:typesFlipflops} besproken hebben. Elk van deze types zal een specifieke combinatorische logica vereisen met specifieke kosten en vertraging. Om de meest optimale schakeling te realiseren zullen we altijd alle types moeten uitproberen. Verder kunnen we ook voor verschillende bits een verschillend type flipflop voorzien. Hierop gaan we echter niet in. De ervaring leert ons wel dat voor verschillende toepassingen verschillende soorten flipflops een zekere voorkeur genieten. De typische toepassingen zeg maar. Een cruciale factor is ook het aantal don't cares. In het algemeen is het zo dat hoe meer don't cares de combinatorische schakeling bevat, hoe eenvoudiger te implementeren. Sommige flipflops introduceren makkelijker don't cares dan andere flipflops.
\paragraph{De kosten van een flipflop}Als we de kostprijs van de verschillende types flipflops met elkaar vergelijken merken we dat de JK-flipflop opmerkelijk duurder\footnote{Ter illustratie een JK-flipflop ge\"integreerd circuit kost \$0.116, een gelijkaardige D-flipflop kost ongeveer \$0.049.} is. Dit betekent daarom niet dat de totale kostprijs van de schakeling groter is. Immers dienen we ook een combinatorisch gedeelte te voorzien; de optelling van beide leidt tot de totale kostprijs.
\paragraph{Don't cares}Een JK-flipflop is dan duurder, maar door de typische realisatie leidt dit tot veel don't cares. Dit leidt tot een eenvoudige schakeling. Het is logisch dat een JK-flipflop nogal wat don't cares impliceert: er zijn immers enkele manieren om dezelfde waarde in een JK-flipflop te klokken. Ook een SR-flipflop laat ruimte voor don't cares in het combinatorische gedeelte. Een D-flipflop laat doorgaans weinig plaats voor don't cares. Dit komt omdat in elke klokflank een nieuwe waarde uit de ingang wordt gelezen. We dienen er dus telkens voor te zorgen dat op dit moment de juiste waarde aan de ingang staat. Een T-flipflop ten slotte heeft net hetzelfde probleem, meestal is het effect hier zelfs nog erger.
\paragraph{Eenvoud van het ontwerp}Indien we de schakeling zelf moeten realiseren moeten we ook rekening houden met het tijdsaspect en dus de eenvoud van het ontwerp. Doorgaans leidt een D-flipflop tot de eenvoudigste implementatie. Dit komt omdat we meestal geneigd zijn absoluut te denken. Verder heeft een D-flipflop ook slechts \'e\'en ingang, en is de component makkelijk te specifi\"eren: "wat we aan de ingang aanleggen staat de volgende klokflank in de flipflop". Het opstellen van een \termen{excitatietabel} is dan ook eenvoudig. Een T-flipflop is ook conceptueel eenvoudig: "indien de waarde moet omslaan, leg dan 1 aan op de ingang". Moeilijker zijn de SR- en JK-flipflop. Dit komt in de eerste plaats omdat beide componenten twee ingangen hebben. Merk dus op dat we een excitatietabel met twee ingangen moeten opstellen.
\paragraph{Toepassingen}Door jaren ervaring heeft men kennis opgebouwd welke toepassing welke flipflop vereist. In subsectie \ref{ss:counters} hebben we reeds tellers gebruikt en hebben we vaak gebruik gemaakt van T- en D-flipflops. Tellers en frequentiedelers\footnote{Een frequentiedeler is een moduloteller die de frequentie waarmee er geteld wordt deelt door het modulo-getal. Enkel wanneer zich een overflow voordoet geven we het signaal door. Hierdoor kunnen we delen van de schakeling aan een lagere klokfrequentie laten werken.} worden typisch ge\"implementeerd met T-flipflops. Meestal zal dit tot ijle excitatietabellen leiden. Een D-flipflop wordt typisch gebruikt om een waarde zeer tijdelijk te onthouden, meestal slechts enkele klokflanken. Voor complexe toepassingen waarbij de waarde van de flipflop frequent een 0 of 1 wordt zijn SR- en JK-flipflops het meest geschikt.
\paragraph{Welke flipflop kiezen?}De vorige paragrafen proberen hints te geven welke flipflops het meest geschikt zijn, deze hints zijn echter niet absoluut. Om tot de goedkoopste schakeling te komen, moeten we alle mogelijkheden uitproberen. Meestal hebben we de tijd niet alle configuraties te proberen. In dat geval zijn D-flipflops meestal de beste keuze. Ook een FPGA volgt deze redenering en bevat enkel D-flipflops.
\paragraph{Samevatting}
We vatten de vorige paragrafen samen in tabel \ref{tbl:whichFlipflop}.
\begin{table}[hbt]
\centering
\small{\begin{tabular}{|*{5}{M}}
\hline
Flipflop&JK&SR&D&T\\\hline\hline
Kostprijs&\verb/--/&\verb/++/&\verb/++/&\verb/++/\\\hline
Don't cares&\verb/++/&\verb/+/&\verb/-/&\verb/--/\\\hline
Ontwerp&\verb/--/&\verb/-/&\verb/++/&\verb/+/\\\hline
Toepassingen&\multicolumn{2}{c|}{veel veranderingen}&tijdelijke geheugens&tellers, frequentiedelers\\\hline
\end{tabular}}
\caption{Keuze van het type flipflop.}
\tbllab{whichFlipflop}
\end{table}
\subsection{Stap 4: Implementeren van de combinatorische logica}
Met alle vorige stappen hebben we beslist hoe we de specificaties van de sequenti\"ele schakeling zullen implementeren. Het enige wat we nu nog moeten doen is de schakeling zelf implementeren. Deze implementatie is op te delen in het implementeren van de logica die de volgende toestand berekent, en de logica die de uitgang bepaalt. We bespreken elk van deze onderdelen apart.
\subsubsection{Stap 4A: Logica die de volgende toestand berekent}
\paragraph{Excitatietabellen van flipflops}Het deel van de logica die de volgende toestand van het circuit berekent is bij zowel een Moore- als een Mealy-machine identiek. Het is een combinatorische schakeling die als ingangen de invoer van de component en de toestand heeft, als uitvoer heeft het de ingangen van de flipflops die de toestand bijhouden. Het aantal uitgangen en de uitgangen zelf hangen dus af van het type flipflop die we gekozen hebben. We dienen dus een tabel op te stellen die op basis van de invoer van de schakeling en de toestand aangeeft welke ingangen van welke flipflops hoog of laag moeten zijn. Deze functies noemen we de \termen{excitatiefuncties}; deze komen voort uit de \termen{excitatietabellen} van de verschillende flipflops. Deze tabellen werden reeds ge\"introduceerd in \ref{sss:typesFlipflops}. In tabel \ref{tbl:excitationTablesFlipflops} geven we eerst opnieuw een kort overzicht van de excitatietabellen van de verschillende flipflops.
\begin{table}[hbt]
\centering
\subtable[JK]{
\begin{tabular}{c|c|cc}
$Q$&$Q_{\mbox{\small{next.}}}$&$J$&$K$\\\hline
0&0&0&-\\
0&1&1&-\\
1&0&-&1\\
1&1&-&0\\
\end{tabular}
\tbllab{excitationTablesFlipflopsJK}}
\subtable[SR]{
\begin{tabular}{c|c|cc}
$Q$&$Q_{\mbox{\small{next.}}}$&$S$&$R$\\\hline
0&0&0&-\\
0&1&1&0\\
1&0&0&1\\
1&1&-&0\\
\end{tabular}
\tbllab{excitationTablesFlipflopsSR}}
\subtable[D]{
\begin{tabular}{c|c|c}
$Q$&$Q_{\mbox{\small{next.}}}$&$D$\\\hline
0&0&0\\
0&1&1\\
1&0&0\\
1&1&1\\
\end{tabular}
\tbllab{excitationTablesFlipflopsD}}
\subtable[T]{
\begin{tabular}{c|c|c}
$Q$&$Q_{\mbox{\small{next.}}}$&$T$\\\hline
0&0&0\\
0&1&1\\
1&0&1\\
1&1&0\\
\end{tabular}
\tbllab{excitationTablesFlipflopsT}}
\caption{Excitatietabellen van de verschillende flipflops}
\tbllab{excitationTablesFlipflops}
\end{table}
Op basis van deze tabellen kunnen we nu de excitatiefuncties berekenen. Deze functies bevatten traditioneel veel don't cares. Deze don't cares hebben drie oorzaken:
\begin{itemize}
 \item De don't cares die we terug vinden bij de excitatietabellen. Dit zijn don't cares in de uitgang van de functie en komen enkel voor bij implementaties met JK- en SR-flipflops.
 \item Binaire coderingen van toestanden die niet bestaan. Een concreet geval is bijvoorbeeld $011$ bij een one-hot codering. Dergelijke invoer codeert nooit naar een uitvoer.
 \item Binaire coderingen van ingangscombinaties die niet mogelijk zijn. Stel bijvoorbeeld dat we een $2$-bit ingang beschouwen en enkel configuraties $00$, $01$ en $11$ kunnen voorkomen.
\end{itemize}
Deze laatste twee zijn een vorm van invoer die niet kan voorkomen. In een tabel kunnen we de rijen weglaten, of we kunnen de rij opvullen met don´t cares aan het uitganggedeelte.
\paragraph{De transitiefunctie}We stellen de transitiefunctie op door voor elke bit die de toestand voorstelt de ingangen van deze flipflop als uitgangen van onze schakeling te zien. Hiervoor kunnen we eerst het coderingstabel wijzigen. We lineariseren het diagram en laten de uitvoer voorlopig vallen. We illustreren dit concept met de Moore-machine uit het leidende voorbeeld. In tabel \ref{tbl:stateTableMooreComb} staat de toestandstabel van deze Moore-machine beschreven. Elke bit van de toestand $F_i$ is een ingang samen met de in dit geval 1-bit ingang $I_i$ van de schakeling. Als uitgang nemen we voorlopig de bits $D_i$ die de volgende toestand voorstellen. Deze linearisatie staat in tabel \ref{tbl:stateTableMooreCombLinD}.
\begin{table}[hbt]
\centering
\subtable[Coderingstabel]{\begin{tabular}{c|cc|c}
Toestand&0&1&Uitgang\\\hline
\texttt{10}&\texttt{00}&\texttt{10}&\texttt{0}\\
\texttt{00}&\texttt{00}&\texttt{01}&\texttt{0}\\
\texttt{01}&\texttt{11}&\texttt{10}&\texttt{0}\\
\texttt{11}&\texttt{00}&\texttt{01}&\texttt{1}\\
\end{tabular}
\tbllab{stateTableMooreComb}}
\subtable[D-flipflop]{\begin{tabular}{ccc|cc}
$F_0$&$F_1$&$I_0$&$D_0$&$D_1$\\\hline
0&0&0&0&0\\
0&0&1&0&1\\
0&1&0&1&1\\
0&1&1&1&0\\
1&0&0&0&0\\
1&0&1&1&0\\
1&1&0&0&0\\
1&1&1&0&1
\end{tabular}
\tbllab{stateTableMooreCombLinD}}
\subtable[T-flipflop]{\begin{tabular}{ccc|cc}
$F_0$&$F_1$&$I_0$&$T_0$&$T_1$\\\hline
0&0&0&0&0\\
0&0&1&0&1\\
0&1&0&1&0\\
0&1&1&1&1\\
1&0&0&1&0\\
1&0&1&0&0\\
1&1&0&1&1\\
1&1&1&1&0
\end{tabular}
\tbllab{stateTableMooreCombLinT}}
\subtable[JK-flipflop]{\begin{tabular}{ccc|cccc}
$F_0$&$F_1$&$I_0$&$J_0$&$K_0$&$J_1$&$K_1$\\\hline
0&0&0&0&-&0&-\\
0&0&1&0&-&1&-\\
0&1&0&1&-&-&1\\
0&1&1&1&-&-&0\\
1&0&0&-&1&0&-\\
1&0&1&-&0&0&-\\
1&1&0&-&1&-&0\\
1&1&1&-&1&-&1
\end{tabular}
\tbllab{stateTableMooreCombLinJK}}
\subtable[SR-flipflop]{\begin{tabular}{ccc|cccc}
$F_0$&$F_1$&$I_0$&$S_0$&$R_0$&$S_1$&$R_1$\\\hline
0&0&0&0&-&0&-\\
0&0&1&0&-&1&0\\
0&1&0&1&0&0&1\\
0&1&1&1&0&-&0\\
1&0&0&0&1&0&-\\
1&0&1&-&0&0&-\\
1&1&0&0&1&-&0\\
1&1&1&0&1&0&1
\end{tabular}
\tbllab{stateTableMooreCombLinSR}}
\caption{Voorstelling van de transitiefunctie van de Moore-machine.}
\tbllab{stateTableMooreCombLin}
\end{table}
Merk op dat deze reeds de implementatie zijn als we werken met D-flipflops. In het geval we niet met D-flipflops werken dienen we nog een extra stap te beschouwen: in dat geval beschouwen we voor elke flipflop het tuple $\left(F_i,D_i\right)$. Hierbij geldt $F_i=Q$ en $D_i=Q_{\mbox{\small{next.}}}$. We dienen dan nog enkel op te zoeken welke invoer het specifieke type flipflop nodig heeft in de excitatietabellen (zie tabel \ref{tbl:excitationTablesFlipflops}). In het geval van een SR- en JK-flipflop leidt dit dus tot een verdubbeling van het aantal uitgangen. Voorbeelden hiervan voor respectievelijk de T-, JK- en SR-flipflop staan in tabellen \ref{tbl:stateTableMooreCombLinT}, \ref{tbl:stateTableMooreCombLinJK} en \ref{tbl:stateTableMooreCombLinSR}. In principe hebben we nu alle elementen om de combinatorische schakeling te realiseren. We minimaliseren eerst de functies met behulp van Karnaugh-kaarten en implementeren dan vervolgens de logica. Merk op dat deze schakelingen meervoudige uitgangen hebben (1 per D- en T-flipflop en 2 per JK- en SR-flipflop). We kunnen dus ook de logica verder optimaliseren door implicanten over de verschillende Karnaugh-kaarten samen te nemen. Op figuur \ref{fig:mooreCombImpl} geven we voor elk type flipflop de Karnaugh-kaarten en een implementatie.
\begin{figure}[hbt]
\centering
\subfigure[D-flipflops]{
\begin{tikzpicture}[circuit logic US]
\def\ta{0.09};
\def\tb{0.03};
\kkaartcmarks[1]{0}{3}{/1/0/1/1/\ta,/3/1/3/1/\ta}{};
\kkaartc{0}{3}{$D_0$}{$F_0$/$F_1$/$I_0$}{0/0/1/1/0/1/0/0};
\kkaartcmarks[1]{0}{0}{/0/1/0/1/\ta,/1/0/1/0/\ta,/2/1/2/1/\ta}{};
\kkaartc{0}{0}{$D_1$}{$F_0$/$F_1$/$I_0$}{0/1/1/0/0/0/0/1};
\def\sc{0.75};
\def\sca{0.8};
\begin{scope}[xshift=6 cm,yshift=2.5 cm,scale=\sc]
\node[dffxn,scale=\sca] (FF0) at (0,2) {$F_0$};
\node[dffxn,scale=\sca] (FF1) at (0,-2) {$F_1$};
\coordinate (F0I) at (-3.5,3.8);
\coordinate (F1I) at (-3.7,4.0);
\coordinate (F2I) at (-3.9,4.0);
\coordinate (F3I) at (-4.1,4.0);
\coordinate (F4I) at (-4.3,4.0);
\coordinate (F0O) at (1,3.8);
\coordinate (F1O) at (1.2,4.0);

\coordinate (Clk0) at (F3I |- 0,1);
\coordinate (Clk1) at (F3I |- 0,-3);
\coordinate (CLR0) at (F4I |- 0,0.5);
\coordinate (CLR1) at (F4I |- 0,-3.5);
\coordinate (PR0) at (F4I |- 0,3.5);
\coordinate (PR1) at (F4I |- 0,-0.5);
\coordinate (MID) at (FF0.CLR |- 0,0);

\draw (FF0.Q) -| (F0O) -- (F0I);
\draw (FF1.Q) -| (F1O) -- (F1I);
\draw (F0I) -- (F0I |- 0,-4);
\draw (F1I) -- (F1I |- 0,-4);
\draw (F2I) -- (F2I |- 0,-4);
\draw (F3I) -- (F3I |- 0,-4);
\draw (F4I) -- (F4I |- 0,-4);
\coordinate (I0) at (-5,0);
\coordinate (I1) at (-5,-0.5);
\coordinate (I2) at (-5,-1);
\draw (I0) node[anchor=east]{$I_0$} -- (I0 -| F2I);
\draw (I1) node[anchor=east]{Clk} -- (I1 -| F3I);
\draw (I2) node[anchor=east]{Clr$^*$} -- (I2 -| F4I);
\coordinate (ORr) at (-1.5,0);
\coordinate (ANDr) at (-2.65,0);

\node[or gate] (O0) at (FF0.D -| ORr) {};
\node[or gate,inputs={normal,normal,normal}] (O1) at (FF1.D -| ORr) {};

\node[and gate,inputs={inverted,normal}] (A00) at ([yshift=0.5 cm] ANDr |- O0.output) {};
\node[and gate,inputs={normal,inverted,normal}] (A01) at ([yshift=-0.5 cm] ANDr |- O0.output) {};

\node[and gate,inputs={inverted,inverted,normal}] (A10) at ([yshift=1 cm] ANDr |- O1.output) {};
\node[and gate,inputs={inverted,normal,inverted}] (A11) at (ANDr |- O1.output) {};
\node[and gate,inputs={normal,normal,normal}] (A12) at ([yshift=-1 cm] ANDr |- O1.output) {};
\foreach \x in {0,1,2} {
  \coordinate (L\x) at (-3.5-0.2*\x,0);
}
\end{scope}
\foreach \x in {0,1} {
  \draw (O\x.output) -- (FF\x.D);
}
\foreach \x/\y/\z/\t in {0/0/1/0.2,0/1/2/0.2,1/0/1/0.2,1/1/2/0,1/2/3/0.2} {
  \draw (O\x.input \z) -- ++(-\t,0) |- (A\x\y.output);
}
\foreach \x/\y/\z/\t in {0/0/1/0,0/0/2/1,0/1/1/0,0/1/2/1,0/1/3/2,1/0/1/0,1/0/2/1,1/0/3/2,1/1/1/0,1/1/2/1,1/1/3/2,1/2/1/0,1/2/2/1,1/2/3/2} {
  \draw (A\x\y.input \z) -- (A\x\y.input \z -| L\t);
  \pdot{A\x\y.input \z -| L\t};
}
\foreach \x/\z/\t in {} {
  \draw (O\x.input \z) -- (O\x.input \z -| L\t);
  \pdot{O\x.input \z -| L\t};
}
\foreach \x/\y/\z in {FF0/Clk0/CLR0,FF1/Clk1/CLR1} {
  \draw (\x.Clk) -- ++(-0.35,0) |- (\y);
  \pdot{\y};
}
\draw (FF1.CLR) |- (CLR1);\pdot{CLR1};
\draw (FF0.PR) |- (PR0);\pdot{PR0};
\draw (FF0.CLR) -- (FF1.PR);\pdot{MID};\draw (MID) -- ++(-1,0) node[scale=0.75,anchor=east]{$1$};
\pdot{I0 -| F2I};
\pdot{I1 -| F3I};
\pdot{I2 -| F4I};
\end{tikzpicture}
\figlab{mooreTotalImplementationD}}
\subfigure[T-flipflops]{
\begin{tikzpicture}[circuit logic US]
\def\ta{0.09};
\def\tb{0.03};
\kkaartcmarks[1]{0}{3}{/1/0/2/1/\ta,/2/0/3/0/\tb}{};
\kkaartc{0}{3}{$T_0$}{$F_0$/$F_1$/$I_0$}{0/0/1/1/1/0/1/1};
\kkaartcmarks[1]{0}{0}{/0/1/1/1/\ta,/2/0/2/0/\ta}{};
\kkaartc{0}{0}{$T_1$}{$F_0$/$F_1$/$I_0$}{0/1/0/1/0/0/1/0};
\def\sc{0.75};
\def\sca{0.8};
\begin{scope}[xshift=6 cm,yshift=2.5 cm,scale=\sc]
\node[tffxn,scale=\sca] (FF0) at (0,2) {$F_0$};
\node[tffxn,scale=\sca] (FF1) at (0,-2) {$F_1$};
\coordinate (F0I) at (-3.5,3.8);
\coordinate (F1I) at (-3.7,4.0);
\coordinate (F2I) at (-3.9,4.0);
\coordinate (F3I) at (-4.1,4.0);
\coordinate (F4I) at (-4.3,4.0);
\coordinate (F0O) at (1,3.8);
\coordinate (F1O) at (1.2,4.0);

\coordinate (Clk0) at (F3I |- 0,1);
\coordinate (Clk1) at (F3I |- 0,-3);
\coordinate (CLR0) at (F4I |- 0,0.5);
\coordinate (CLR1) at (F4I |- 0,-3.5);
\coordinate (PR0) at (F4I |- 0,3.5);
\coordinate (PR1) at (F4I |- 0,-0.5);
\coordinate (MID) at (FF0.CLR |- 0,0);

\draw (FF0.Q) -| (F0O) -- (F0I);
\draw (FF1.Q) -| (F1O) -- (F1I);
\draw (F0I) -- (F0I |- 0,-4);
\draw (F1I) -- (F1I |- 0,-4);
\draw (F2I) -- (F2I |- 0,-4);
\draw (F3I) -- (F3I |- 0,-4);
\draw (F4I) -- (F4I |- 0,-4);
\coordinate (I0) at (-5,0);
\coordinate (I1) at (-5,-0.5);
\coordinate (I2) at (-5,-1);
\draw (I0) node[anchor=east]{$I_0$} -- (I0 -| F2I);
\draw (I1) node[anchor=east]{Clk} -- (I1 -| F3I);
\draw (I2) node[anchor=east]{Clr$^*$} -- (I2 -| F4I);
\coordinate (ORr) at (-1.5,0);
\coordinate (ANDr) at (-2.65,0);

\node[or gate] (O0) at (FF0.T -| ORr) {};
\node[or gate] (O1) at (FF1.T -| ORr) {};

\node[and gate,inputs={normal,inverted}] (A00) at ([yshift=0.5 cm] ANDr |- O0.output) {};

\node[and gate,inputs={inverted,normal}] (A10) at ([yshift=0.5 cm] ANDr |- O1.output) {};
\node[and gate,inputs={normal,normal,inverted}] (A11) at ([yshift=-0.5 cm] ANDr |- O1.output) {};
\foreach \x in {0,1,2} {
  \coordinate (L\x) at (-3.5-0.2*\x,0);
}
\end{scope}
\foreach \x in {0,1} {
  \draw (O\x.output) -- (FF\x.T);
}
\foreach \x/\y/\z/\t in {0/0/1/0.2,1/0/1/0.2,1/1/2/0.2} {
  \draw (O\x.input \z) -- ++(-\t,0) |- (A\x\y.output);
}
\foreach \x/\y/\z/\t in {0/0/1/0,0/0/2/2,1/0/1/0,1/0/2/2,1/1/1/0,1/1/2/1,1/1/3/2} {
  \draw (A\x\y.input \z) -- (A\x\y.input \z -| L\t);
  \pdot{A\x\y.input \z -| L\t};
}
\foreach \x/\z/\t in {0/2/1} {
  \draw (O\x.input \z) -- (O\x.input \z -| L\t);
  \pdot{O\x.input \z -| L\t};
}
\foreach \x/\y/\z in {FF0/Clk0/CLR0,FF1/Clk1/CLR1} {
  \draw (\x.Clk) -- ++(-0.35,0) |- (\y);
  \pdot{\y};
}
\draw (FF1.CLR) |- (CLR1);\pdot{CLR1};
\draw (FF0.PR) |- (PR0);\pdot{PR0};
\draw (FF0.CLR) -- (FF1.PR);\pdot{MID};\draw (MID) -- ++(-1,0) node[scale=0.75,anchor=east]{$1$};
\pdot{I0 -| F2I};
\pdot{I1 -| F3I};
\pdot{I2 -| F4I};
\end{tikzpicture}}
\subfigure[JK-flipflops]{
\begin{tikzpicture}[circuit logic US]
\def\ta{0.09};
\def\tb{0.03};
\kkaartcmarks[0.75]{0}{3.5}{/1/0/2/1/\ta}{};
\kkaartc[0.75]{0}{3.5}{$J_0$}{$F_0$/$F_1$/$I_0$}{0/0/1/1/-/-/-/-};
\kkaartcmarks[0.75]{1.5}{3.5}{/1/0/2/1/\ta,/0/0/3/0/\tb}{};
\kkaartc[0.75]{1.5}{3.5}{$K_0$}{$F_0$/$F_1$/$I_0$}{-/-/-/-/1/0/1/1};
\kkaartcmarks[0.75]{0}{-0.5}{/0/1/1/1/\ta}{};
\kkaartc[0.75]{0}{-0.5}{$J_1$}{$F_0$/$F_1$/$I_0$}{0/1/-/-/0/0/-/-};
\kkaartcmarks[0.75]{1.5}{-0.5}{/0/0/1/0/\ta,/2/1/3/1/\ta}{};
\kkaartc[0.75]{1.5}{-0.5}{$K_1$}{$F_0$/$F_1$/$I_0$}{-/-/1/0/-/-/0/1};
\def\sc{0.75};
\def\sca{0.8};
\begin{scope}[xshift=6 cm,yshift=2.5 cm,scale=\sc]
\node[jkffxn,scale=\sca] (FF0) at (0,2) {};
\node[jkffxn,scale=\sca] (FF1) at (0,-2) {};
\coordinate (F0I) at (-3.5,3.8);
\coordinate (F1I) at (-3.7,4.0);
\coordinate (F2I) at (-3.9,4.0);
\coordinate (F3I) at (-4.1,4.0);
\coordinate (F4I) at (-4.3,4.0);
\coordinate (F0O) at (1,3.8);
\coordinate (F1O) at (1.2,4.0);

\coordinate (Clk0) at (F3I |- FF0.Clk);
\coordinate (Clk1) at (F3I |- FF1.Clk);
\coordinate (CLR0) at (F4I |- 0,0.2);
\coordinate (CLR1) at (F4I |- 0,-3.8);
\coordinate (PR0) at (F4I |- 0,3.5);
\coordinate (PR1) at (F4I |- 0,-0.5);
\coordinate (MID) at (FF0.CLR |- 0,0);

\draw (FF0.Q) -| (F0O) -- (F0I);
\draw (FF1.Q) -| (F1O) -- (F1I);
\draw (F0I) -- (F0I |- 0,-4);
\draw (F1I) -- (F1I |- 0,-4);
\draw (F2I) -- (F2I |- 0,-4);
\draw (F3I) -- (F3I |- 0,-4);
\draw (F4I) -- (F4I |- 0,-4);
\coordinate (I0) at (-5,0);
\coordinate (I1) at (-5,-0.5);
\coordinate (I2) at (-5,-1);
\draw (I0) node[anchor=east]{$I_0$} -- (I0 -| F2I);
\draw (I1) node[anchor=east]{Clk} -- (I1 -| F3I);
\draw (I2) node[anchor=east]{Clr$^*$} -- (I2 -| F4I);
\coordinate (ORr) at (-1.5,0);
\coordinate (ANDr) at (-2.65,0);

\node[or gate,inputs={normal,inverted}] (O0) at (FF0.K -| ORr) {};
\node[or gate] (O1) at (FF1.K -| ORr) {};

%\node[and gate,inputs={normal,inverted}] (A00) at ([yshift=0.5 cm] ANDr |- O0.output) {};

\node[and gate,inputs={inverted,normal}] (A00) at (ANDr |- FF1.J) {};
\node[and gate,inputs={inverted,inverted}] (A10) at (ANDr |- O1.input 1) {};
\node[and gate,inputs={normal,normal}] (A11) at ([yshift=-0.8 cm] ANDr |- O1.input 1) {};
\foreach \x in {0,1,2} {
  \coordinate (L\x) at (-3.5-0.2*\x,0);
}
\end{scope}
\foreach \x in {0,1} {
  \draw (O\x.output) -- (FF\x.K);
}
\foreach \x/\y in {00/1} {
  \draw (A\x.output) -- (FF\y.J);
}
\foreach \x/\y/\z/\t in {1/0/1/0.2,1/1/2/0.2} {
  \draw (O\x.input \z) -- ++(-\t,0) |- (A\x\y.output);
}
\foreach \x/\y/\z/\t in {0/0/1/0,0/0/2/2,1/0/1/0,1/0/2/2,1/1/1/0,1/1/2/2} {
  \draw (A\x\y.input \z) -- (A\x\y.input \z -| L\t);
  \pdot{A\x\y.input \z -| L\t};
}
\foreach \x/\z/\t in {0/1/1,0/2/2} {
  \draw (O\x.input \z) -- (O\x.input \z -| L\t);
  \pdot{O\x.input \z -| L\t};
}
\foreach \x/\y/\t in {FF0/J/1} {
  \draw (\x.\y) -- (\x.\y -| L\t);
  \pdot{\x.\y -| L\t};
}
\foreach \x/\y/\z in {FF0/Clk0/CLR0,FF1/Clk1/CLR1} {
  \draw (\x.Clk) -- ++(-0.35,0) |- (\y);
  \pdot{\y};
}
\draw (FF1.CLR) |- (CLR1);\pdot{CLR1};
\draw (FF0.PR) |- (PR0);\pdot{PR0};
\draw (FF0.CLR) -- (FF1.PR);\pdot{MID};\draw (MID) -- ++(-1,0) node[scale=0.75,anchor=east]{$1$};
\pdot{I0 -| F2I};
\pdot{I1 -| F3I};
\pdot{I2 -| F4I};
\end{tikzpicture}}
\subfigure[SR-flipflops]{
\begin{tikzpicture}[circuit logic US]
\def\ta{0.09};
\def\tb{0.03};
\kkaartcmarks[0.75]{0}{3.5}{/1/0/1/1/\ta}{};
\kkaartc[0.75]{0}{3.5}{$S_0$}{$F_0$/$F_1$/$I_0$}{0/0/1/1/0/-/0/0};
\kkaartcmarks[0.75]{1.5}{3.5}{/2/0/2/1/\ta,/2/0/3/0/\tb}{};
\kkaartc[0.75]{1.5}{3.5}{$R_0$}{$F_0$/$F_1$/$I_0$}{-/-/0/0/1/0/1/1};
\kkaartcmarks[0.75]{0}{-0.5}{/0/1/1/1/\ta}{};
\kkaartc[0.75]{0}{-0.5}{$S_1$}{$F_0$/$F_1$/$I_0$}{0/1/0/-/0/0/-/0};
\kkaartcmarks[0.75]{1.5}{-0.5}{/0/0/1/0/\ta,/2/1/3/1/\ta}{};
\kkaartc[0.75]{1.5}{-0.5}{$R_1$}{$F_0$/$F_1$/$I_0$}{-/0/1/0/-/-/0/1};
\def\sc{0.75};
\def\sca{0.8};
\begin{scope}[xshift=6 cm,yshift=2.5 cm,scale=\sc]
\node[srffxn,scale=\sca] (FF0) at (0,2) {};
\node[srffxn,scale=\sca] (FF1) at (0,-2) {};
\coordinate (F0I) at (-3.5,3.8);
\coordinate (F1I) at (-3.7,4.0);
\coordinate (F2I) at (-3.9,4.0);
\coordinate (F3I) at (-4.1,4.0);
\coordinate (F4I) at (-4.3,4.0);
\coordinate (F0O) at (1,3.8);
\coordinate (F1O) at (1.2,4.0);

\coordinate (Clk0) at (F3I |- FF0.Clk);
\coordinate (Clk1) at (F3I |- FF1.Clk);
\coordinate (CLR0) at (F4I |- 0,0.2);
\coordinate (CLR1) at (F4I |- 0,-3.8);
\coordinate (PR0) at (F4I |- 0,3.5);
\coordinate (PR1) at (F4I |- 0,-0.5);
\coordinate (MID) at (FF0.CLR |- 0,0);

\draw (FF0.Q) -| (F0O) -- (F0I);
\draw (FF1.Q) -| (F1O) -- (F1I);
\draw (F0I) -- (F0I |- 0,-4);
\draw (F1I) -- (F1I |- 0,-4);
\draw (F2I) -- (F2I |- 0,-4);
\draw (F3I) -- (F3I |- 0,-4);
\draw (F4I) -- (F4I |- 0,-4);
\coordinate (I0) at (-5,0);
\coordinate (I1) at (-5,-0.5);
\coordinate (I2) at (-5,-1);
\draw (I0) node[anchor=east]{$I_0$} -- (I0 -| F2I);
\draw (I1) node[anchor=east]{Clk} -- (I1 -| F3I);
\draw (I2) node[anchor=east]{Clr$^*$} -- (I2 -| F4I);
\coordinate (ORr) at (-1.5,0);
\coordinate (ANDr) at (-2.65,0);

\node[or gate,inputs={normal,normal}] (O0) at (FF0.R -| ORr) {};
\node[or gate] (O1) at (FF1.R -| ORr) {};

%\node[and gate,inputs={normal,inverted}] (A00) at ([yshift=0.5 cm] ANDr |- O0.output) {};

\node[and gate,inputs={normal,normal}] (A00) at (ANDr |- O0.input 1) {};
\node[and gate,inputs={normal,inverted}] (A01) at ([yshift=-0.8 cm] ANDr |- O0.input 1) {};
\node[and gate,inputs={inverted,normal}] (A20) at (ANDr |- FF0.S) {};
\node[and gate,inputs={inverted,normal}] (A30) at (ANDr |- FF1.S) {};
\node[and gate,inputs={inverted,inverted}] (A10) at (ANDr |- O1.input 1) {};
\node[and gate,inputs={normal,normal}] (A11) at ([yshift=-0.8 cm] ANDr |- O1.input 1) {};
\foreach \x in {0,1,2} {
  \coordinate (L\x) at (-3.5-0.2*\x,0);
}
\end{scope}
\foreach \x in {0,1} {
  \draw (O\x.output) -- (FF\x.R);
}
\foreach \x/\y in {30/1,20/0} {
  \draw (A\x.output) -- (FF\y.S);
}
\foreach \x/\y/\z/\t in {1/0/1/0.2,1/1/2/0.2,0/0/1/0.2,0/1/2/0.2} {
  \draw (O\x.input \z) -- ++(-\t,0) |- (A\x\y.output);
}
\foreach \x/\y/\z/\t in {2/0/1/0,2/0/2/1,3/0/1/0,3/0/2/2,1/0/1/0,1/0/2/2,1/1/1/0,1/1/2/2,0/0/1/0,0/0/2/1,0/1/1/0,0/1/2/2} {
  \draw (A\x\y.input \z) -- (A\x\y.input \z -| L\t);
  \pdot{A\x\y.input \z -| L\t};
}
\foreach \x/\z/\t in {} {
  \draw (O\x.input \z) -- (O\x.input \z -| L\t);
  \pdot{O\x.input \z -| L\t};
}
\foreach \x/\y/\t in {} {
  \draw (\x.\y) -- (\x.\y -| L\t);
  \pdot{\x.\y -| L\t};
}
\foreach \x/\y/\z in {FF0/Clk0/CLR0,FF1/Clk1/CLR1} {
  \draw (\x.Clk) -- ++(-0.35,0) |- (\y);
  \pdot{\y};
}
\draw (FF1.CLR) |- (CLR1);\pdot{CLR1};
\draw (FF0.PR) |- (PR0);\pdot{PR0};
\draw (FF0.CLR) -- (FF1.PR);\pdot{MID};\draw (MID) -- ++(-1,0) node[scale=0.75,anchor=east]{$1$};
\pdot{I0 -| F2I};
\pdot{I1 -| F3I};
\pdot{I2 -| F4I};
\end{tikzpicture}}
\caption{Implementatie van de Moore-schakeling met verschillende soorten flipflops.}
\figlab{mooreCombImpl}
\end{figure}
Naast de logica voor de volgende toestand, verbinden we ook de klokingangen van de flipflops met het globale kloksignaal, en implementeren we de clear logica. De clear zorgt ervoor dat op het moment dat we de schakeling herzetten en dus het \clrsin{} signaal van de schakeling 0 wordt, de schakeling in de eerste toestand terechtkomt. In ons geval is dat 10. Daarom verbinden we de clear-ingang met de preset ingang van de eerste flipflop en de clear-ingang van de tweede flipflop. Op de clear-ingang van de eerste flipflop en de preset-ingang van de tweede wordt steeds een hoog signaal aangelegd. Merk verder ook op we in principe geen negatieve ingangen voor toestandssignalen moeten gebruiken. In plaats van een negatie aan de ingang van de AND-poort te plaatsen, kunnen we eenvoudig gebruik maken van de $\overline{Q}$-uitgang van de flipflop. Omdat we echter hiermee het aantal lijnen voor de flipflops bijna verdubbelen, maken we er in de figuur abstractie van. Dit principe kunnen we enkel toepassen voor signalen die uit de flipflops komen. De signalen die de ingangsconfiguratie bepalen hebben wel een expliciete NOT-poort nodig.
\paragraph{Het one-hot-codering geval}Naast het geval van de Moore-machine zullen we ook de transitiefunctie van de Mealy-machine implementeren. Over deze realisatie is het ook eenvoudiger om enkele eigenschappen te formaliseren. Deze eigenschappen gaan over de kosten van de transitiefunctie v\'o\'or dat we minimalisatie van de Karnaugh-kaarten toepassen. Verder hangen de eigenschappen af van het type flipflop die we gebruiken:
\begin{itemize}
 \item Bij een D-flipflop is het aantal AND-poorten voor een flipflop equivalent met het aantal transities naar de toestand die de flipflop voorstelt. Bij een transitie\footnote{Inclusief lussen.} naar toestand $x$ dient immer de overeenkomstige flipflop op 1 gezet te worden, in de andere gevallen is de waarde van de flipflop altijd 0. We kunnen dit verder formaliseren tot: ``Het aantal enen in de $D_i$ kolom is gelijk aan het aantal transities naar toestand $i$''.
 \item Bij een T-flipflop is het aantal T-poorten gelijk met het aantal transities die naar een de bijbehorende toestand gaan, of vanuit de toestand vertrekken, die geen lussen zijn. Of eenvoudiger: ``Het aantal enen in de $T_i$-kolom is gelijk aan het aantal transitites van of naar toestand $i$ die geen lussen zijn''.
 \item Bij een SR-flipflop is het aantal AND-poorten die naar de S-ingang gaan gelijk aan het aantal transities naar de toestand, die niet uit de toestand komen. Het aantal AND-poorten die voor de R-ingang staan zijn het aantal transities die uit de toestand vertrekken, en die geen lussen zijn. Deze logica kunnen we ook omzetten naar het aantal enen in de kolommen $S_i$ en $R_i$.
 \item Bij JK-flipflops gelden dezelfde regels als bij de SR-flipflops, we dienen hier S door J en R door K te vervangen.
\end{itemize}
We kunnen deze theorie testen met de praktijk in tabel \ref{tbl:stateTableMealyCombLin}. Hierbij hebben we de tabellen voor de verschillende types flipflops naast elkaar gezet, dit is louter om de voorstelling eenvoudig te houden. Merk op dat verschillende rijen uitsluitend don't cares bevatten. Dit is het gevolg van het feit dat heel wat binaire voorstellingen van toestanden bij de invoer geen overeenkomstige toestand hebben. Meestal worden deze rijen in een tabel weggelaten, wat de tweede tabel een stuk korter en leesbaarder maakt. Uiteraard dienen we dan wel op deze plaatsen in de Karnaugh-kaarten don't cares te plaatsen.
\begin{table}[hbt]
\centering
\subtable[D- en JK-flipflops]{\small{\begin{tabular}{cccc|ccc|cccccc}
$F_0$&$F_1$&$F_2$&$I_0$&$D_0$&$D_1$&$D_2$&$J_0$&$K_0$&$J_1$&$K_1$&$J_2$&$K_2$\\\hline
0&0&0&0&-&-&-&-&-&-&-&-&-\\
0&0&0&1&-&-&-&-&-&-&-&-&-\\
0&0&1&0&0&1&0&0&-&1&-&-&1\\
0&0&1&1&0&0&1&0&-&0&-&-&0\\
0&1&0&0&0&1&0&0&-&-&0&0&-\\
0&1&0&1&1&0&0&1&-&-&1&0&-\\
0&1&1&0&-&-&-&-&-&-&-&-&-\\
0&1&1&1&-&-&-&-&-&-&-&-&-\\
1&0&0&0&0&1&0&-&1&1&-&0&-\\
1&0&0&1&0&0&1&-&1&0&-&1&-\\
1&0&1&0&-&-&-&-&-&-&-&-&-\\
1&0&1&1&-&-&-&-&-&-&-&-&-\\
1&1&0&0&-&-&-&-&-&-&-&-&-\\
1&1&0&1&-&-&-&-&-&-&-&-&-\\
1&1&1&0&-&-&-&-&-&-&-&-&-\\
1&1&1&1&-&-&-&-&-&-&-&-&-
\end{tabular}}}
\subtable[T- en SR-flipflops]{\small{\begin{tabular}{cccc|ccc|cccccc}
$F_0$&$F_1$&$F_2$&$I_0$&$T_0$&$T_1$&$T_2$&$S_0$&$R_0$&$S_1$&$R_1$&$S_2$&$R_2$\\\hline
0&0&1&0&0&1&1&0&-&1&0&0&1\\
0&0&1&1&0&0&0&0&-&0&-&-&0\\
0&1&0&0&0&0&0&0&-&-&0&0&-\\
0&1&0&1&1&1&0&1&0&0&1&0&-\\
1&0&0&0&1&1&0&0&1&1&0&0&-\\
1&0&0&1&1&0&1&0&1&0&-&1&0
\end{tabular}}}
\caption{Implementatie van de Mealy-schakeling met verschillende soorten flipflops.}
\tbllab{stateTableMealyCombLin}
\end{table}
We zien dat $D_0$ \'e\'en 1 bevat, wat overeenkomt met \'e\'en transitie naar toestand $E$ op figuur \ref{fig:minimaalToestandsdiagramMealy}, verder bevatten de kolommen voor toestand $A$ en $B$ respectievelijk 2 en 3 enen. Bij de JK-flipflop heeft toestand $A$ 1 transitie naar $A$ en 1 transitie uit $A$. We zien dat dit ook overeenkomt met het aantal enen bij $J_2$ en $K_2$. We kunnen dus besluiten dat dergelijke eigenschappen handig kunnen zijn bij het controleren van onze tabel. Een andere eigenschap is dat bij een T-flipflop er altijd twee bits veranderen. Bij elke rij zijn er dus ofwel 0 enen ofwel 2. Op figuur \ref{fig:mealyCombImpl} implementeren we de D- en JK-flipflop variant van de Mealy-machine. Het implementeren van de T- en SR-flipflop variant wordt als een oefening voor de lezer overgelaten.
\begin{figure}[hbt]
\centering
\subfigure[D-flipflop]{\begin{tikzpicture}[circuit logic US]
\def\ta{0.09};
\kkaartdmarks[0.75]{0}{0}{/1/2/2/3/\ta}{}{};
\kkaartd[0.75]{0}{0}{$D_0$}{$F_0$/$F_1$/$F_2$/$I_0$}{-/-/0/0/0/1/-/-/0/0/-/-/-/-/-/-/-/-};
\kkaartdmarks[0.75]{0}{-2.5}{/0/0/3/1/\ta}{}{};
\kkaartd[0.75]{0}{-2.5}{$D_1$}{$F_0$/$F_1$/$F_2$/$I_0$}{-/-/1/0/1/0/-/-/1/0/-/-/-/-/-/-/-/-};
\kkaartdmarks[0.75]{0}{-5}{}{/2/3/1/\ta}{};
\kkaartd[0.75]{0}{-5}{$D_2$}{$F_0$/$F_1$/$F_2$/$I_0$}{-/-/0/1/0/0/-/-/0/1/-/-/-/-/-/-/-/-};
\def\sc{0.75};
\def\sca{0.8};
\begin{scope}[xshift=5 cm,yshift=-1.75 cm,scale=\sc]
\node[dffxn,scale=\sca] (FF0) at (0,3.333) {$FF_0$};
\node[dffxn,scale=\sca] (FF1) at (0,0) {$FF_1$};
\node[dffxn,scale=\sca] (FF2) at (0,-3.333) {$FF_2$};
\def\gx{-1.75};
\node[and gate] (G0) at (\gx,0 |- FF0.D) {};
\node[not gate] (G1) at (\gx,0 |- FF1.D) {};
\node[and gate,inputs={inverted,normal}] (G2) at (\gx,0 |- FF2.D) {};
\foreach \x in {0,...,2} {
  \coordinate (LI\x) at (1+0.2*\x,5+0.2*\x);
  \coordinate (LO\x) at (-2.5-0.2*\x,5+0.2*\x);
  \coordinate (LF\x) at (LO\x |- 0,-5);
  \draw (FF\x.Q) -| (LI\x) -- (LO\x) -- (LF\x);
  \draw (G\x.output) -- (FF\x.D);
}
\foreach \x in {3,4,5} {
  \coordinate (LO\x) at (-2.5-0.2*\x,5.4);
  \coordinate (LF\x) at (LO\x |- 0,-5);
  \draw (LO\x) -- (LF\x);
}
\coordinate (FFCLR1) at (LO5 |- 0,-1.667);
\coordinate (I0) at (-4.25,4.75);
\coordinate (I1) at (-4.25,1.5);
\coordinate (I2) at (-4.25,-1.667);
\draw (I0) node[anchor=east]{$I_0$} -- (I0 -| LO3);
\draw (I1) node[anchor=east]{Clk} -- (I1 -| LO4);
\draw (I2) node[anchor=east]{Clr$^*$} -- (I2 -| LO5);
\end{scope}
\pdot{I0 -| LO3};
\pdot{I1 -| LO4};
\foreach \x/\y/\z in {0/input 1/1,0/input 2/3,1/input/3,2/input 1/1,2/input 2/3} {
  \draw (G\x.\y) -- (G\x.\y -| LO\z);
  \pdot{G\x.\y -| LO\z};
}
\foreach \x/\y/\z in {0,1,2} {
  \draw (FF\x.Clk) -- (FF\x.Clk -| LO4);
  \pdot{FF\x.Clk -| LO4};
}
\draw (FF2.CLR) |- ++(-0.5,-0.11) node[anchor=east,scale=0.75]{$1$};
\draw (FF1.CLR) -- (FF2.PR);
\draw (FF0.PR) |- ++(-0.5,0.11) node[anchor=east,scale=0.75]{$1$};
\draw (FF1.PR) |- ++(-0.5,0.11) node[anchor=east,scale=0.75]{$1$};
\coordinate (FFCLR0) at ([yshift=-0.11 cm] FF0.CLR -| LO5);
\draw (FF0.CLR) |- (FFCLR0);
\draw (FFCLR1 -| FF1.CLR) |- (FFCLR1);
\pdot{FFCLR0};
\pdot{FFCLR1};
\pdot{FFCLR1 -| FF1.CLR};
\end{tikzpicture}}
\subfigure[JK-flipflop]{\begin{tikzpicture}[circuit logic US]
\def\ta{0.09};
\kkaartdmarks[0.75]{0}{0}{/1/2/2/3/\ta}{}{};
\kkaartd[0.75]{0}{0}{$J_0$}{$F_0$/$F_1$/$F_2$/$I_0$}{-/-/0/0/0/1/-/-/-/-/-/-/-/-/-/-/-/-};
\kkaartdmarks[0.75]{2.25}{0}{/0/0/3/3/\ta}{}{};
\kkaartd[0.75]{2.25}{0}{$K_0$}{$F_0$/$F_1$/$F_2$/$I_0$}{-/-/-/-/-/-/-/-/1/1/-/-/-/-/-/-/-/-};
\kkaartdmarks[0.75]{0}{-2.5}{/0/0/3/1/\ta}{}{};
\kkaartd[0.75]{0}{-2.5}{$J_1$}{$F_0$/$F_1$/$F_2$/$I_0$}{-/-/1/0/-/-/-/-/1/0/-/-/-/-/-/-/-/-};
\kkaartdmarks[0.75]{2.25}{-2.5}{/0/2/3/3/\ta}{}{};
\kkaartd[0.75]{2.25}{-2.5}{$K_1$}{$F_0$/$F_1$/$F_2$/$I_0$}{-/-/-/-/0/1/-/-/-/-/-/-/-/-/-/-/-/-};
\kkaartdmarks[0.75]{0}{-5}{/2/2/3/3/\ta}{}{};
\kkaartd[0.75]{0}{-5}{$J_2$}{$F_0$/$F_1$/$F_2$/$I_0$}{-/-/-/-/0/0/-/-/0/1/-/-/-/-/-/-/-/-};
\kkaartdmarks[0.75]{2.25}{-5}{/0/0/3/1/\ta}{}{};
\kkaartd[0.75]{2.25}{-5}{$K_2$}{$F_0$/$F_1$/$F_2$/$I_0$}{-/-/1/0/-/-/-/-/-/-/-/-/-/-/-/-/-/-};
\def\sc{0.75};
\def\sca{0.8};
\begin{scope}[xshift=7.5 cm,yshift=-1.75 cm,scale=\sc]
\node[jkffxn,scale=\sca] (FF0) at (0,3.333) {};
\node[jkffxn,scale=\sca] (FF1) at (0,0) {};
\node[jkffxn,scale=\sca] (FF2) at (0,-3.333) {};
\coordinate (SS) at (FF1.J -| -1.125,0);
\def\gx{-2};
\node[and gate] (GJ0) at (\gx,0 |- FF0.J) {};
\node[not gate] (GJ1) at (\gx,0 |- FF1.J) {};
\node[and gate] (GJ2) at (\gx,0 |- FF2.J) {};
\foreach \x in {0,...,2} {
  \coordinate (LI\x) at (1+0.2*\x,5+0.2*\x);
  \coordinate (LO\x) at (-2.7-0.2*\x,5+0.2*\x);
  \coordinate (LF\x) at (LO\x |- 0,-5);
  \draw (FF\x.Q) -| (LI\x) -- (LO\x) -- (LF\x);
}
\foreach \x in {0,1,2} {
  \draw (GJ\x.output) -- (FF\x.J);
}
\foreach \x in {} {
  \draw (GK\x.output) -- (FF\x.K);
}
\foreach \x in {3,4,5} {
  \coordinate (LO\x) at (-2.7-0.2*\x,5.4);
  \coordinate (LF\x) at (LO\x |- 0,-5);
  \draw (LO\x) -- (LF\x);
}
\coordinate (FFCLR1) at (LO5 |- 0,-1.667);

\coordinate (I0) at (-4.25,4.75);
\coordinate (I1) at (-4.25,1.5);
\coordinate (I2) at (-4.25,-1.667);
\draw (I0) node[anchor=east]{$I_0$} -- (I0 -| LO3);
\draw (I1) node[anchor=east]{Clk} -- (I1 -| LO4);
\draw (I2) node[anchor=east]{Clr$^*$} -- (I2 -| LO5);
\end{scope}
\pdot{I0 -| LO3};
\pdot{I1 -| LO4};
\draw (FF2.K) -| (SS);
\pdot{SS};
\foreach \x/\y/\z in {0/input 1/1,0/input 2/3,1/input/3,2/input 1/0,2/input 2/3} {
  \draw (GJ\x.\y) -- (GJ\x.\y -| LO\z);
  \pdot{GJ\x.\y -| LO\z};
}
\foreach \x/\y/\z in {0,1,2} {
  \draw (FF\x.Clk) -- (FF\x.Clk -| LO4);
  \pdot{FF\x.Clk -| LO4};
}
\draw (FF1.K) -- (FF1.K -| LO3);
\pdot{FF1.K -| LO3};
\draw (FF2.CLR) |- ++(-0.5,-0.11) node[anchor=east,scale=0.75]{$1$};
\draw (FF1.CLR) -- (FF2.PR);
\draw (FF0.PR) |- ++(-0.5,0.11) node[anchor=east,scale=0.75]{$1$};
\draw (FF1.PR) |- ++(-0.5,0.11) node[anchor=east,scale=0.75]{$1$};
\draw (FF0.K) -- ++(-0.5,0) node[anchor=east,scale=0.75]{$1$};
\coordinate (FFCLR0) at ([yshift=-0.11 cm] FF0.CLR -| LO5);
\draw (FF0.CLR) |- (FFCLR0);
\draw (FFCLR1 -| FF1.CLR) |- (FFCLR1);
\pdot{FFCLR0};
\pdot{FFCLR1};
\pdot{FFCLR1 -| FF1.CLR};
\end{tikzpicture}}
\caption{Implementatie van de Mealy-schakeling met verschillende soorten flipflops.}
\figlab{mealyCombImpl}
\end{figure}
Merk op dat we hier de verbindingen naar de \clrsin{} en \prsin{} ingangen anders geconfigureerd zijn dan bij de Moore-machine. Dit komt omdat de begintoestand van de Mealy-machine 001 is, terwijl dit 10 is bij de Moore-machine.
\subsubsection{Stap 4B: Logica die de uitgang toestand berekent}
Naast de logica die de volgende toestand berekent, dienen we ook de de logica te synthetiseren die de uitgangen bepaalt. Dit is eigenlijk niets anders dan het synthetiseren van een combinatorische schakeling. Hiervoor dienen een tabel op te stellen die vanuit de coderingstabel een tabel opstelt die de functie bepaalt. Vermits de uitgang bij een Moore-machine anders bepaald wordt dan bij een Mealy-machine, zal ook de techniek om deze tabel op te stellen licht verschillen. Bij een Moore-machine wordt de uitgang enkel bepaald door de toestand. Hierdoor kunnen we de tabel opstellen door de transitiekolommen weg te laten. Bij een Mealy-machine dienen we een linearisering toe te passen. Hierbij beschouwen we niet meer de codering van de volgende toestand zoals in de vorige stap, maar uiteraard de uitgang. Vanuit die tabel stellen we dan opnieuw een combinatorische schakeling op. Merk op dat deze schakeling niet afhangt van het type flipflop. Immers hebben alle types flipflop een $Q$-uitgang. De schakeling hangt wel af van de toestandscodering. Op tabel \ref{tbl:mooreMealyTables}
\begin{table}[hbt]
\centering
\subtable[Moore coderingstabel]{\begin{tabular}{c|cc|c}
Toestand&0&1&Uitgang\\\hline
\texttt{10}&\texttt{00}&\texttt{10}&\texttt{0}\\
\texttt{00}&\texttt{00}&\texttt{01}&\texttt{0}\\
\texttt{01}&\texttt{11}&\texttt{10}&\texttt{0}\\
\texttt{11}&\texttt{00}&\texttt{01}&\texttt{1}\\
\end{tabular}}
\subtable[Moore]{\begin{tabular}{cc|c}
$F_0$&$F_1$&$O_0$\\\hline
0&0&0\\
0&1&0\\
1&0&0\\
1&1&1\\
\end{tabular}}
\subtable[Mealy coderingstabel]{\begin{tabular}{c|cc}
Toestand&0&1\\\hline
\texttt{001}&\texttt{010/0}&\texttt{001/0}\\
\texttt{010}&\texttt{010/0}&\texttt{100/0}\\
\texttt{100}&\texttt{010/1}&\texttt{001/0}\\
\end{tabular}}
\subtable[Mealy]{\begin{tabular}{cccc|c}
$F_0$&$F_1$&$F_2$&$I_0$&$O_0$\\\hline
0&0&1&0&0\\
0&0&1&1&0\\
0&1&0&0&0\\
0&1&0&1&0\\
1&0&0&0&1\\
1&0&0&1&0
\end{tabular}}
\caption{Uitgangslogica van de Moore- en Mealy-machine naast hun coderingstabellen.}
\tbllab{mooreMealyTables}
\end{table}
toont de tabellen die we opstellen voor de Moore- en Mealy-machine naast hun coderingstabellen. Daarna is het alleen nog een kwestie van de schakeling te implementeren, dit wordt meestal opnieuw gedaan met behulp van Karnaugh-kaarten. We implementeren vervolgens op figuur \ref{fig:mealyTotalImplementation} een volledige sequenti\"ele schakeling gebaseerd op de Mealy-machine met een D-flipflop. Merk op dat we in deze figuur gebruik maken van de $\overline{Q}$-uitgangen. Hoewel we in de figuur de logica implementeren met AND-OR logica zal men in de praktijk altijd opteren voor NAND- en NOR-poorten bij de implementatie. Een andere interessante vaststelling dat het geheugen van de tweede flipflop niet gebruikt wordt. We kunnen bijgevolg deze flipflop en de logica er rond weglaten. Deze extra flipflop is het gevolg van een slecht gekozen toestandscodering. Dit drukt niet alleen de kosten, maar zorgt ook voor een kleinere vertraging waardoor we de klokfrequentie kunnen opdrijven. We gaan kort in op het tijdsgedrag in subsectie \ref{ss:timeBehaviorSeqSync}.
\begin{figure}[hbt]
\centering
\begin{tikzpicture}[circuit logic US]
\def\ta{0.09};
\kkaartdmarks{2.5}{-1}{/2/0/3/1/\ta}{}{};
\kkaartd{2.5}{-1}{$O_0$}{$F_0$/$F_1$/$F_2$/$I_0$}{-/-/0/0/0/0/-/-/1/0/-/-/-/-/-/-/-/-};
\def\sc{0.75};
\def\sca{0.8};
\def\xa{-2.75};
\def\xi{-4.25};
\def\xii{-4.75};
\def\xc{-2.25};
\def\xr{-1.75};
\def\xb{2};
\def\xs{-1.25};

\begin{scope}[scale=\sc]
\node[dffxn,scale=\sca] (FF0) at (0,4) {$FF_0$};
\node[dffxn,scale=\sca] (FF1) at (0,0) {$FF_1$};
\node[dffxn,scale=\sca] (FF2) at (0,-4) {$FF_2$};
\draw (FF0.PR) |- ++(0.25,0.15) node[anchor=west]{$1$};
\draw (FF1.PR) |- ++(0.25,0.15) node[anchor=west]{$1$};
\draw (FF2.CLR) |- ++(0.25,-0.15) node[anchor=west]{$1$};
\draw (FF2.PR) -- (FF1.CLR);
\coordinate (FFM) at (FF1.CLR |- 0,-2);
\node[and gate,anchor=output] (A0) at (FF0.D -| \xa,0) {};
\node[and gate,anchor=input 1] (A1) at (FF0.Q -| -\xc,0) {};
\node[not gate,anchor=output] (N0) at (FF1.D -| \xa,0) {};
\node[and gate,anchor=output] (A2) at (FF2.D -| \xa,0) {};
\draw (A0.output) -- (FF0.D);
\draw (N0.output) -- (FF1.D);
\draw (A2.output) -- (FF2.D);
\foreach \x in {0,1,2} {
  \coordinate (Cl\x) at (FF\x.Clk -| \xc,0);
  \draw (FF\x.Clk) -- (Cl\x);
}
\coordinate (In) at (FF1.D -| \xs,0);
\coordinate (Ina) at (-\xs,2);
\draw (FF0.Q) -- (A1.input 1);
\draw (A1.output) -- ++(0.5,0) node[anchor=west]{$O_0$};
\draw (N0.input) -- (N0.input -| \xii,0) node[anchor=east]{$I_0$};
\coordinate (R0) at (\xr,2.25);
\coordinate (R2) at (\xr,0 |- FFM);
\draw (FFM) -| (R0) -- (R0 -| \xii,0) node[anchor=east]{Clr$^*$};
\draw (FF0.CLR) |- (R0);
\draw (Cl0) -- (Cl2) -- (Cl2 -| \xii,0) node[anchor=east]{Clk};
\draw (In) |- (Ina) |- (A1.input 2);
\draw (N0.input -| \xi,0) |- (A0.input 2);
\draw (N0.input -| \xi,0) |- (A2.input 1);
\coordinate (I) at (N0.input -| \xi,0);
\draw (A2.input 2) -- (A2.input 2 -| \xi,0) |- (-\xr,-6) |- (FF1.Qn);
\draw (A0.input 1) -- (A0.input 1 -| \xi,0) |- (-\xr,6) |- (FF1.Q);
\end{scope}
\foreach \x in {1,2} {
  \pdot{Cl\x};
}
\pdot{R0};
\pdot{FFM};
\pdot{In};
\pdot{I};
\end{tikzpicture}
\caption{Volledige implementatie van de Mealy-machine met D-flipflops.}
\figlab{mealyTotalImplementation}
\end{figure}
\subsection{Tijdsgedrag}
\label{ss:timeBehaviorSeqSync}
Tot slot bepalen we de maximale klokfrequentie die we bij een implementatie van een sequenti\"ele schakeling kunnen hanteren. We herinneren ons uit het hoofdstuk over combinatorische schakelingen dat de vertraging in een systeem bepaald wordt door het kritische pad, de maximale tijd die een signaal nodig heeft om zich doorheen de schakeling voort te planten. In een sequenti\"ele schakeling betekent dit dat het signaal zich doorheen de combinatorische schakeling dient te propageren die de volgende toestand bepaalt. Verder dient ook de set-tup-tijd van de geheugenmodules en de tijd die het kloksignaal nodig heeft om zich tot aan de $Q$- of $\overline{Q}$-uitgang te propageren in rekening te worden gebracht. Bij het bepalen van het kritische pad van de transitiefunctie moeten we alle paden vanuit alle ingangen (dus ook de flipflops die toestand bijhouden) beschouwen, naar alle flipflops. Of formeler:
\begin{equation}
\mbox{vertraging pad}=\mbox{logica transitie}+\mbox{set-up flipflop}+\mbox{Clk naar $Q$ of $\overline{Q}$}
\end{equation}
De vertragingen van de klok naar $Q$ en $\overline{Q}$ kunnen we berekenen vanuit de implementatie van de flipflop. De fabrikant van een flipflop zal bovendien steeds deze vertragingen in het datasheet\footnote{Elk ge\"integreerd circuit heeft een datasheet, een document opgesteld door de producent die de karakteristieken van de component formeel beschrijft.} van de component zetten. Indien de implementatie gegeven is, kunnen we de vertraging berekenen. Indien we een master-slave D-flipflop gebruiken dienen we enkel rekening te houden met de propagatietijd van de klok naar $Q$ of $\overline{Q}$ van een D-latch. We hebben deze vertragingen reeds uitgerekend in vergelijking (\ref{eqn:dLatchDelays}) op pagina \pageref{eqn:dLatchDelays}. Eenmaal we de vertraging van het kritisch pad hebben berekend kunnen we de maximale klokfrequentie bepalen. Dit doen we door het inverse te berekenen van deze vertraging:
\begin{equation}
f_{\mbox{\small{max.}}}=\displaystyle\frac{1}{\mbox{vertraging kritisch pad}}
\end{equation}
Merk wel op dat we bij het berekenen van de vertraging geen eenheid gegeven hebben. Deze vertragingen die we tot hier toe hebben berekend dienen uitsluitend om verschillende implementaties met elkaar te vergelijken. In de praktijk zal de technologie waarmee we de schakeling realiseren een tijds\'e\'enheid impliceren waarmee we die vertraging kunnen vermenigvuldigen.
\subsubsection{Leidend voorbeeld}
\paragraph{Moore-machine}
We berekenen de vertraging van de implementatie van de Moore-machine die beschreven staat op figuur \ref{fig:mooreTotalImplementationD}. In totaal kunnen we 10 verschillende paden onderscheiden: in de schakeling staan twee flipflops. Het signaal van deze flipflops vormt een ingang bij alle 5 de AND-poorten. Merk op dat we de inverter aan de AND-poort niet moeten meetellen in onze berekening. We dienen in dat geval de $\overline{Q}$-uitgang te beschouwen. In de onderstaande vergelijking berekenen we de vertraging van alle paden. Het subscript van de vertraging wijst op de bron- en doel-flipflop, en de AND-poort. Deze poorten nummeren we van 0 tot 4 van boven naar beneden:
\begin{equation}
\begin{array}{ll}
\left\{\begin{array}{l}
t_{\small{FF_0\rightarrow FF_0,0}}=\mbox{Clk}\rightarrow \overline{Q}+\mbox{set-up D}+\mbox{2-AND}+\mbox{2-OR}=4.2+1+2.4+2.4=10.0\\
t_{\small{FF_0\rightarrow FF_0,1}}=\mbox{Clk}\rightarrow Q+\mbox{set-up D}+\mbox{3-AND}+\mbox{2-OR}=4.2+1+2.8+2.4=10.4\\
t_{\small{FF_0\rightarrow FF_1,2}}=\mbox{Clk}\rightarrow \overline{Q}+\mbox{set-up D}+\mbox{3-AND}+\mbox{3-OR}=4.2+1+2.8+2.8=10.8\\
t_{\small{FF_0\rightarrow FF_1,3}}=\mbox{Clk}\rightarrow \overline{Q}+\mbox{set-up D}+\mbox{3-AND}+\mbox{3-OR}=4.2+1+2.8+2.8=10.8\\
t_{\small{FF_0\rightarrow FF_1,4}}=\mbox{Clk}\rightarrow Q+\mbox{set-up D}+\mbox{3-AND}+\mbox{3-OR}=4.2+1+2.8+2.8=10.8\\

t_{\small{FF_1\rightarrow FF_0,0}}=\mbox{Clk}\rightarrow Q+\mbox{set-up D}+\mbox{2-AND}+\mbox{2-OR}=4.2+1+2.4+2.4=10.0\\
t_{\small{FF_1\rightarrow FF_0,1}}=\mbox{Clk}\rightarrow \overline{Q}+\mbox{set-up D}+\mbox{3-AND}+\mbox{2-OR}=4.2+1+2.8+2.4=10.4\\
t_{\small{FF_1\rightarrow FF_1,2}}=\mbox{Clk}\rightarrow \overline{Q}+\mbox{set-up D}+\mbox{3-AND}+\mbox{3-OR}=4.2+1+2.8+2.8=10.8\\
t_{\small{FF_1\rightarrow FF_1,3}}=\mbox{Clk}\rightarrow Q+\mbox{set-up D}+\mbox{3-AND}+\mbox{3-OR}=4.2+1+2.8+2.8=10.8\\
t_{\small{FF_1\rightarrow FF_1,4}}=\mbox{Clk}\rightarrow Q+\mbox{set-up D}+\mbox{3-AND}+\mbox{3-OR}=4.2+1+2.8+2.8=10.8
\end{array}\right.&\mbox{(Voorbeeld)}
\end{array}
\end{equation}
Het maximale pad heeft dus een vertraging van $10.8$. Indien we een referentievertraging van $1\mbox{ ns}$ nemen, is de maximale frequentie $f_{\mbox{\small{max.}}}=1/10.8\mbox{ ns}\approx 93\mbox{ MHz}$. Indien we dezelfde schakeling met NAND-NAND logica zouden hebben gebouwd, zou de vertraging van elk pad gereduceerd worden met 2. In dat geval zou de maximale klokfrequentie $f_{\mbox{\small{max.}}}=1/8.8\mbox{ ns}\approx 114\mbox{ MHz}$.
\paragraph{Mealy-machine}
Voor de Mealy-machine gebruiken we de implementatie op figuur \ref{fig:mealyTotalImplementation}. In deze schakeling dienen we twee paden te beschouwen: het eerste loopt van $FF_1$ naar $FF_0$, vertrekt vanuit $Q$ en gaat door \'e\'en 2-AND-poort. Het andere pad ligt tussen $FF_1$ en $FF_2$. Ook hierbij passeren we een 2-AND-poort, merk echter op dat het signaal vertrekt vanuit $\overline{Q}$. De vertraging van dit pad is dus gelijk aan:
\begin{equation}
\begin{array}{ll}
\left\{\begin{array}{l}
t_{\small{FF_1\rightarrow FF_0}}=\mbox{Clk}\rightarrow Q+\mbox{set-up D}+\mbox{2-AND}=4.2+1+2.4=7.6\\
t_{\small{FF_1\rightarrow FF_2}}=\mbox{Clk}\rightarrow \overline{Q}+\mbox{set-up D}+\mbox{2-AND}=4.2+1+2.4=7.6
\end{array}\right.&\mbox{(Voorbeeld)}
\end{array}
\end{equation}
Beide paden hebben hier dus dezelfde vertraging. Als we een referentievertraging van $1\mbox{ ns}$ nemen, is de maximale frequentie $f_{\mbox{\small{max.}}}=1/7.6\mbox{ ns}\approx 131\mbox{ MHz}$.
\section{Asynchrone schakelingen}
\label{s:asynchroneSequence}
Naast schakelingen waar het ritme bepaald wordt door een klok, bestaan er ook schakelingen die een vorm van geheugen bezitten, maar van toestand veranderen doordat het ingangssignaal verandert. We hebben in dit hoofdstuk reeds dergelijke schakelingen ge\"implementeerd. In subsectie \ref{ss:flipflop} hebben we een latch geconstrueerd. Een latch heeft in principe geen klokingang, maar houdt wel een toestand bij. Op het moment dat we een ingang aan een SR-latch aanpasen, zal de latch soms in een andere toestand terechtkomen. Verder bezit de component ook duidelijk een geheugen vermits als de ingang $\left(S,R\right)=\left(0,0\right)$, de vorige toestand behouden blijft. De situatie van de latch illustreert hoe asynchrone schakelingen worden gerealiseerd: We hebben een aantal ingangssignalen, die door logica worden verwerkt in een soort geheugen. Dit geheugen vertraagt een terugkoppeling, en houdt de toestand bij. Het terugkoppelen betekent dat de uitgangen van een logische schakeling als een deel van de invoer fungeren die door die logica verwerkt wordt. In deze cursus zullen we uitsluitend asynchrone sequenti\"ele schakelingen beschouwen die voldoen aan een belangrijke voorwaarde: de \termen{fundamentele modus}, ofwel de ``\termen{fundamental mode restriction}''. Deze beperking bestaat uit twee delen:
\begin{itemize}
 \item Er mag hooguit \'e\'en ingangssignaal tegelijk veranderen. Een voorbeeld die dit concept duidelijk illustreert is de SR-latch: indien we $S$ en $R$ ingang tegelijk van 1 naar 0 omschakelen, komt de component in een oscillerend toestand komt.
 \item Een ingangsverandering mag slechts optreden als alle effecten van de vorige verandering uitgestorven zijn. Ook dit principe kennen we al impliciet. Zo dienen we de set-up tijd van een flipflop te respecteren om te voorkomen dat de flipflop in een onvoorspelbare toestand komt.
\end{itemize}
We zijn in principe niet verplicht om ons aan deze voorwaarden te houden, het ontwerpen van schakelingen die deze restricties niet volgen is echter zeer moeilijk.
\paragraph{}
Ook bij het ontwerpen van een asynchrone schakeling zullen we een stappenplan volgen. In grote lijnen lijkt deze procedure op het ontwikkelen van synchrone schakelingen. De stappen zelf wijken echter sterk af van hun synchrone variant. In de meeste stappen zullen we ook met de complexe problemen die eigen zijn aan een asynchrone schakeling moeten rekening houden. Het plan bestaat uit 4 stappen:
\begin{enumerate}
 \item Opstellen van een toestandstabel.
 \item Minimaliseren van het aantal toestanden.
 \item Coderen van de toestanden.
 \item Realisatie van de schakeling met digitale logica.
\end{enumerate}
\paragraph{Terminologie}
Bij synchrone schakelingen werkten we met een klokflank. Vaak gebruikt men bij een kloksignaal de termen ``stijgende klokflank'' en ``dalende klokflank''. Dit zijn immers de tijdstippen waarop gebeurtenissen plaatsvinden in de schakeling die relevant zijn. Een asynchrone schakeling heeft geen kloksignaal. Toch worden de termen ``\termen{stijgende flank}'' en ``\termen{dalende flank}'' ook in deze context gebruikt. De gebeurtenissen in een asynchrone schakeling zijn immers de omschakeling van \'e\'en van de ingangssignalen.
\subsection{Leidend voorbeeld}
Een oplettende lezer heeft misschien al opgemerkt dat de geheugencomponenten die we gebruiken in synchrone schakelingen -- latches en flipflops -- op hun beurt weer asynchrone schakelingen zijn. Daarom zullen we een nieuw type latch ontwerpen, die uiteraard louter fictief is. De latch wordt gebruikt als een interrupt-register. In de meeste computers voert een programma niet constant controles uit of een gebruiker bijvoorbeeld een toets aanslaat. De CPU bevat een component die bijhoudt of er een bepaalde gebeurtenis plaatsvindt. Hardwarematig controleert de CPU frequent of zo'n gebeurtenis is opgetreden. In dat geval voert de processor een procedure uit gedefinieerd door het besturingssysteem. Op dat moment dient de component uiteraard gereset te worden. Verder kan een besturingssysteem ook een masker plaatsen op een interrupt. In dat geval zal de CPU niet reageren op deze gebeurtenis. Deze component heeft twee ingangen $I$ en $E$. $I$ zal een stijgende flank vertonen wanneer de gebruiker een toets aanslaat, maar zal na enige tijd weer dalen. $E$ bepaalt of de CPU luistert naar deze interrupt. Indien $E=0$ heeft het besturingssysteem een masker gezet. Indien we een laag signaal op $E$ aanleggen wordt een reset op de module uitgevoerd. Dit is het geval wanneer de processor de procedure van het besturingssysteem zal uitvoeren. Op het moment dat de procedure uitgevoerd is, zetten we $E$ dan terug hoog om te luisteren naar een mogelijke gebeurtenis. De uitgang $Q$ vertelt ons of er een gebeurtenis is opgetreden op het moment dat er geen masker op de interrupt.
\paragraph{Formele beschrijving}
De vorige paragraaf is relatief informeel beschreven. We zetten deze beschrijving om in een formeler equivalent. Dit doen we aan de hand van tabel \ref{tbl:formalDescriptionAsyncExample}. In de eerste kolom plaatsen we de gebeurtenis. Dit is dus een ingang die van waarde verandert. De tweede kolom bevat de voorwaarden die relevant zijn voor deze regel. Dit zijn testen in verband met ingangen en uitgangen. De derde kolom ten slotte bevat het effect. Een effect houdt in dat de uitgang verandert.
\begin{table}[hbt]
\centering
\begin{tabular}{c|c|c}
Gebeurtenis&Voorwaarde(n)&Effect(en)\\\hline
$I:0\rightarrow 1$&$E=1$&$Q\rightarrow 1$\\
$E:1\rightarrow 0$&&$Q\rightarrow 0$
\end{tabular}
\caption{Formele beschrijving van het leidend voorbeeld.}
\tbllab{formalDescriptionAsyncExample}
\end{table}
Merk verder op dat het onmogelijk is dat $Q$ een 1 bevat en $E$ een 0.
\subsection{Stap 1: Opstellen van een toestandstabel}
Net zoals bij de synchrone schakelingen stellen we eerst een toestandstabel op. Een eerste vraag die zich stelt is wat een toestand is. Een toestand is elke combinatie van in- en uitgangen waarin de schakeling kan terechtkomen. De schakeling die we zullen implementeren telt 3 verschillende signalen. Dit betekent dus dat we hooguit 8 toestanden kunnen bekomen. Aangezien configuraties met $\left(Q,E\right)=\left(1,0\right)$ onmogelijk kunnen voorkomen houden we 6 toestanden over. We stellen een configuratietabel op die voor elke mogelijke configuratie een toestand voorziet in tabel \ref{tbl:configTableAsync}.
\begin{table}[hbt]
\centering
\subtable[Configuratietabel]{\begin{tabular}{c|ccc}
Toestand&$Q$&$I$&$E$\\\hline
$a$&$0$&$0$&$0$\\
$b$&$0$&$0$&$1$\\
$c$&$0$&$1$&$0$\\
$d$&$0$&$1$&$1$\\
(onmogelijk)&$0$&$0$&$0$\\
$e$&$0$&$0$&$1$\\
(onmogelijk)&$0$&$1$&$0$\\
$f$&$0$&$1$&$1$
\end{tabular}
\tbllab{configTableAsync}}
\subtable[Toestandstabel]{\begin{tabular}{c|cccc|c}
\multirow{2}{*}{Toestand}&\multicolumn{4}{c|}{$I\ E$}&\multirow{2}{*}{$Q$}\\&00&01&11&10&\\\hline
$a$&$a$&$b$&$-$&$c$&$0$\\
$b$&$a$&$b$&$f$&$-$&$0$\\
$c$&$a$&$-$&$d$&$c$&$0$\\
$d$&$-$&$b$&$d$&$c$&$0$\\
$e$&$a$&$e$&$f$&$-$&$1$\\
$f$&$-$&$e$&$f$&$c$&$1$
\end{tabular}
\tbllab{stateTableAsync}
}
\caption{Configuratie- en toestandstabel van het leidend voorbeeld.}
\end{table}
Op basis van deze configuratietabel kunnen we nu een toestandstabel maken. Deze toestandtabel loopt erg gelijk met een toestandstabel van een Moore-machine. De tabel bestaat uit drie delen: toestand, invoer en uitvoer. Een verschil met de toestandtabel van de Moore-machine is dat we niet alle kolommen invullen bij elke toestand. Een toestand is immers gekoppeld aan een bepaalde invoerconfiguratie\footnote{Beschreven in de configuratietabel.}, de fundamentele modus zorgt ervoor dat er slechts \'e\'en bit tegelijk kan omslaan, bijgevolg kunnen we kolommen die op meerdere bits afwijken niet invullen. Concreet betekent dit dat we voor een invoer van $n$-bits, $n+1$ van de $2^n$ kolommen kunnen invullen. In de overige kolommen schrijven we een liggende streepje (``-''). Veelal zal men ook de kolommen van de toestandtabel anders schikken. Men probeert configuraties die slechts \'e\'en bit verschillen naast elkaar te plaatsen. Dit is echter niet vereist.
\paragraph{}
We stellen de toestandstabel op met behulp van de configuratietabel en de tabel met mogelijke gebeurtenissen. Toestand $a$ betekent in dit geval bijvoorbeeld dat we $\left(I,E\right)=\left(0,0\right)$ aan de ingangen aanleggen. Het spreekt dus voor zich dat we bij de kolom met dezelfde configuratie ook een $a$ schrijven. In dat geval is de toestand \termen{stabiel}. Bij de tweede kolom geldt $\left(I,E\right)=\left(0,1\right)$, dit betekent dus dat we een stijgende flank van $E$ beschouwen. Uit de gebeurtenissentabel leiden we af dat in dat geval $Q$ op 0 komt te staan. Dit is reeds het geval. We migreren dus naar de toestand met $\left(Q,I,E\right)=\left(0,0,1\right)$. Op de configuratietabel zien we dat dit toestand $b$ is. Bij de laatste kolom beschouwen we een stijgende flank van $I$, opnieuw gebeurt er niets vermits er niet aan de voorwaarden wordt voldaan in de gebeurtenissentabel. Op deze manier kunnen we de volledige toestandstabel opstellen zoals in tabel \ref{tbl:stateTableAsync}. We kunnen ook met behulp van een toestandsdiagram een grafische voorstelling van deze tabel geven. Het toestandsdiagram staat op figuur \ref{fig:stateDiagramAsync}.
\begin{figure}[hbt]
\centering
\begin{tikzpicture}[->,shorten >=1pt,auto,node distance=2cm,on grid,semithick,state/.style=state with output,every state/.style={draw=black!50,very thick,fill=black!20,scale=0.75}]
\clip (-1.5,-5.5) rectangle (4,1);
\node[state] (E) {$e$\nodepart{lower} $1$};
\node[state] (F) [right=of E] {$f$\nodepart{lower} $1$};
\node[state] (A) [below=of E] {$a$\nodepart{lower} $0$};
\node[state] (B) [below=of F] {$b$\nodepart{lower} $0$};
\node[state] (C) [below=of A] {$c$\nodepart{lower} $0$};
\node[state] (D) [below=of B] {$d$\nodepart{lower} $0$};
\path (A) edge[loop left] node {00} (A)
	  edge[bend left] node {01} (B)
	  edge[bend left] node {10} (C)
      (B) edge[bend left] node {00} (A)
	  edge[loop right] node {01} (B)
	  edge node {11} (F)
      (C) edge[bend left] node {00} (A)
	  edge[loop left] node {10} (C)
	  edge[bend left] node {11} (D)
      (D) edge node {00} (B)
	  edge[bend left] node {10} (C)
	  edge[loop right] node {11} (D)
      (E) edge node {00} (A)
	  edge[loop left] node {01} (E)
	  edge[bend left] node {11} (F)
      (F) edge[bend left] node {01} (E)
	  edge[loop right] node {11} (F);
	  %edge node {10} (C);
\draw (F) .. controls ([shift={(4cm,-4cm)}] F) and ([shift={(2cm,-3cm)}] C).. (C) node[midway] {10};
\end{tikzpicture}
\caption{Toestandsdiagram van het leidend voorbeeld.}
\figlab{stateDiagramAsync}
\end{figure}
\paragraph{}Een toestandtabel bij asynchrone schakelingen wordt ook een ``\termen{flow table}'' genoemd. Indien de tabel nog niet geminimaliseerd is, spreekt met van een ``\termen{primitive flow table}''.
\subsection{Stap 2: Minimaliseren van de toestanden}
Net als bij synchrone schakelingen loont het meestal de moeite om de toestandsruimte te minimaliseren. Dit leidt immers in de meeste gevallen tot een goedkopere schakeling. Vermits het probleem anders is, vertoont de minimalisatie van asynchrone schakelingen ook verschillen tegenover synchrone schakelingen. Een belangrijk verschil is dat we in de flow toestandstabel ook rekening moeten houden met don't cares. Elke toestand heeft er immers $2^n-n-1$ bij $n$-bit invoer. Daarnaast kunnen er uiteraard nog bijkomende don't cares optreden die probleemafhankelijk zijn. Meestal laten deze don't cares toe om op een flexibele manier de toestandsruimte te minimaliseren.
\paragraph{}
Er bestaan verschillende methodes om de toestandsruimte te minimaliseren. We opteren voor een methode die werkt met twee stappen. Allereerst is er de partitionering. Deze partitionering is volledig analoog aan de sequenti\"ele schakelingen (zie \ref{ss:minimizeFSMSeq} op pagina \pageref{ss:minimizeFSMSeq}).
%Het verschil tussen deze regels en de regels bij een sequenti\"ele machine, is dat we hier afdwingen dat beide toestanden naar dezelfde toestand $S_k$ gaan. Bij het minimaliseren van een sequenti\"ele machine mochten beide toestanden een andere volgende toestand hebben, zolang deze toestanden zich in dezelfde partitie bevinden. Een tweede verschil is dat we hier rekening houden met de don't cares. Deze regel zorgt er dan ook voor dat we meer toestanden kunnen tegenkomen die equivalent zijn met elkaar.
We dienen echter de voorwaarden van deze partitionering minimaal aan te passen. De nieuwe voorwaarde wordt dan:
\begin{enumerate}
 \item De toestanden hebben dezelfde uitgangscombinatie.
 \item De don't cares bevinden zich in dezelfde kolommen.
\end{enumerate}
Vervolgens passen we opnieuw de iteratiestap toe zoals bij synchrone schakelingen. De voorwaarden veranderen hierbij niet.
\paragraph{}
Nadat we de partitionering berekend hebben, kunnen we een eerste toestandsreductie toepassen. Bij de synchrone schakelingen betekent de toestandreductie dan ook meteen het einde van de minimalisatie. Bij synchrone schakelingen kunnen we verder reduceren. Hiertoe berekenen we \termen{compatibele toestanden}\footnote{Bemerkt het verschil met ``equivalente toestanden'' bij de partitionering.}. Twee toestanden $S_i$ en $S_j$ zijn equivalent indien:
\begin{enumerate}
 \item Dezelfde uitvoerconfiguratie hebben
 \item Voor elke ingangscombinatie geldt ofwel:
 \begin{enumerate}
  \item $S_i$ en $S_j$ zijn beide stabiel
  \item De volgende toestand van minstens \'e\'en van de twee toestanden $S_i$ of $S_j$ is niet gespecificeerd, er staat dus een don't care in de kolom van minstens \'e\'en van de toestanden.
  \item Beide toestanden gaan naar dezelfde toestand $S_k$.
 \end{enumerate}
\end{enumerate}
Merk op dat de toestanden $S_i$, $S_j$ en $S_k$ niet de toestanden zijn uit de initi\"ele toestandstabel, maar uit de door partitionering reeds gereduceerde toestandstabel. Een andere mogelijke valkuil is dat beide toestanden naar dezelfde toestand moeten gaan, niet naar twee compatibele of equivalente toestanden. De reden hiervoor is dat de compatibiliteitsrelatie niet transitief is: Als toestand $A$ compatibel is met toestand $B$ en $B$ is compatibel met toestand $C$, is $A$ niet noodzakelijk compatibel met toestand $C$. De relatie is echter wel symmetrisch\footnote{Dit kunnen we eenvoudig aantonen door $S_i$ en $S_j$ om te wisselen. In dat geval zien we dat de voorwaarden niet veranderen.} en uiteraard reflexief.
\paragraph{}
Het feit dat er de compatibiliteitsrelatie niet transitief is introduceert ook een nieuw probleem: we willen verschillende compatibele toestanden samennemen in \'e\'en nieuwe toestand, een zogenaamde \termen{kliek $K_n$}. Vermits hiervoor elke twee toestanden compatibel met elkaar moeten zijn, is dit proces niet deterministisch. Men kan dit probleem op verschillende methodes oplossen. In de praktijk lost men dit vaak op met behulp van programma's die zoeken naar een zo goed mogelijke groepering. Nadat we een groepering hebben bepaald, minimaliseren we opnieuw de toestandentabel. Dit betekent echter nog niet dat het minimalisatiealgoritme ten einde is: door het groeperen van toestanden in een nieuwe toestand, kunnen deze nieuwe toestanden weer compatibiliteit vertonen. We dienen dus opnieuw een compatibiliteitsrelatie op te bouwen en eventueel te minimaliseren. De minimalisatie stopt op het moment dat geen enkele nieuwe toestand meer compatibel is met een andere. Schematisch geven we het hele proces weer in een flowchart op figuur \ref{fig:flowchartMinimizeAsynchrone}.
\begin{figure}[hbt]
\centering
\begin{tikzpicture}[node distance = 2cm, auto]
\node [block] (A) {Partitioneren van de toestanden};
\node [block, below of=A] (B) {Bepaal compatibele toestanden};
\node [block, below of=B] (C) {Groepeer compatibele toestanden};
\node [block, below of=C] (D) {Samenvoegen van de toestanden in een groep};
\node [decision,below of=D] (E) {Toestands\-tabel veranderd?};
\node [block] (F) at (E -| 3,0) {Stop};
\node [block] (O) at (A -| -3,0) {Start};
\path [line] (O) -- (A);
\path [line] (A) -- (B);
\path [line] (B) -- (C);
\path [line] (C) -- (D);
\path [line] (D) -- (E);
\path [line] (E) -- node [near start,scale=0.75] {yes} ++(-3,0) |- (B);
\path [line] (E) -- node [near start,scale=0.75] {no} (F);
\end{tikzpicture}
\caption{Flowchart van het minimalisatieproces van asynchrone schakelingen.}
\figlab{flowchartMinimizeAsynchrone}
\end{figure}
\paragraph{Samenvoegen van toestanden in een nieuwe toestand}
Hoe bouwen we vanuit een groep toestanden een nieuwe toestand op? Dit kunnen we in principe al afleiden uit de definitie van compatibele toestanden. Uit de toestanden $\left\{s_1,s_2,\ldots,s_n\right\}$ construeren we een toestand $t$ indien alle toestanden met elkaar compatibel zijn. We dienen bij $t$ ook de overgangen te defini\"eren: $s$ gaat onder een invoer $I$ over naar $\delta\left(s,I\right)$. Deze functie noemen we ook de \termen{transitie-functie}. Verder voeren we ook een functie $\mathcal{G}\left(s\right)$ in, deze functie geeft de nieuwe toestand weer van de groep waar $s$ toe behoort. We kunnen dan volgende regels op $t$ defini\"eren\footnote{We voeren de regels in de volgorde van verschijnen uit. Het kan zijn dat een situatie aan twee criteria voldoet, in dat geval voeren we de eerste regel uit.}:
\begin{enumerate}
 \item $\delta\left(t,I\right)=-$ indien $\forall i:\delta\left(s_i,I\right)=-$. We plaatsen een don't care bij een transitie van $t$ met invoer $I$ als elke toestanden $s_i$ onder deze invoer ook een don't care bevat.
 \item $\delta\left(t,I\right)=t$ indien $\forall i:\delta\left(s_i,I\right)=-\vee\delta\left(s_i,I\right)\in\left\{s_1,s_2,\ldots,s_n\right\}$. Indien alle toestanden onder invoer $I$ binnen de groep blijven of een don't care bevatten, vormt $t$ onder invoer $I$ een stabiele toestand.
 \item $\delta\left(t,I\right)=u$ indien $\forall i:\delta\left(s_i,I\right)=-\vee\mathcal{G}\left(\delta\left(s_i,I\right)\right)=u$. Indien alle toestanden onder invoer $I$ naar dezelfde groep transformeren, transformeert $t$ onder deze invoer naar de toestand die deze groep voorstelt.
\end{enumerate}
% Door dit nieuwe criterium is de transitiviteit van de compatibiliteitsrelatie niet meer gewaarborgd. Bij de sequenti\"ele schakelingen geldt immers de eigenschap als $A$ compatibel is met $B$ en $B$ met $C$ is $A$ ook compatibel met $C$. De transitiviteit vervalt omdat we toelaten dat don't cares.
\subsubsection{Voorbeeld}
Om alle speciale gevallen in te sluiten gebruiken we een andere toestandstabel beschreven in tabel \ref{tbl:stateTableAsyncMiniOrig}.
\begin{table}[hbt]
\centering
\subtable[Initi\"ele tabel]{\begin{tabular}{c|cccc|c}
\multirow{2}{*}{Toestand}&\multicolumn{4}{c|}{$I\ E$}&\multirow{2}{*}{$Q$}\\&00&01&11&10&\\\hline
$a$&$a$&$f$&$-$&$c$&$0$\\
$b$&$-$&$b$&$h$&$-$&$1$\\
$c$&$-$&$c$&$g$&$j$&$0$\\
$d$&$-$&$f$&$d$&$-$&$1$\\
$e$&$g$&$-$&$d$&$e$&$1$\\
$f$&$-$&$f$&$k$&$-$&$0$\\
$g$&$l$&$g$&$j$&$-$&$0$\\
$h$&$-$&$l$&$h$&$e$&$1$\\
$i$&$i$&$e$&$-$&$f$&$1$\\
$j$&$b$&$-$&$-$&$j$&$0$\\
$k$&$-$&$b$&$k$&$e$&$1$\\
$l$&$-$&$l$&$k$&$-$&$1$\\
$m$&$m$&$l$&$-$&$e$&$1$
\end{tabular}
\tbllab{stateTableAsyncMiniOrig}}
\subtable[Tabel na partitionering]{\begin{tabular}{c|cccc|c}
\multirow{2}{*}{Toestand}&\multicolumn{4}{c|}{$I\ E$}&\multirow{2}{*}{$Q$}\\&00&01&11&10&\\\hline
$a$&$a$&$f$&$-$&$c$&$0$\\
$b$&$-$&$b$&$h$&$-$&$1$\\
$c$&$-$&$c$&$g$&$j$&$0$\\
$d$&$-$&$f$&$d$&$-$&$1$\\
$e$&$g$&$-$&$d$&$e$&$1$\\
$f$&$-$&$f$&$h$&$-$&$0$\\
$g$&$b$&$g$&$j$&$-$&$0$\\
$h$&$-$&$b$&$h$&$e$&$1$\\
$i$&$i$&$e$&$-$&$f$&$1$\\
$j$&$b$&$-$&$-$&$j$&$0$\\
$m$&$m$&$b$&$-$&$e$&$1$
\end{tabular}
\tbllab{stateTableAsyncMiniParty}}
\caption{Evolutie van de toestandstabel bij het minimaliseren voor en na partitioneren.}
\end{table}
Allereerst passen we de initialisatiestap van het partitiealgoritme toe. We dienen hierbij buiten een verschillende uitvoerconfiguratie ook rekening te houden met verschillende don't care configuraties. Dit leidt tot de volgende partitie:
\begin{equation}
\begin{array}{lr}
\mathcal{P}_0=\left\{\left\{a\right\},\left\{b,d,l\right\},\left\{c\right\},\left\{e\right\},\left\{f\right\},\left\{g\right\},\left\{h,k\right\},\left\{i,m\right\},\left\{j\right\}\right\}&\mbox{(voorbeeld)}
\end{array}
\end{equation}
Na het iteratieproces bij de partitionering die wel volledig behouden blijft bekomen we volgende partitie:
\begin{equation}
\begin{array}{lr}
\mathcal{P}_1=\left\{\left\{a\right\},\left\{b,l\right\},\left\{c\right\},\left\{d\right\},\left\{e\right\},\left\{f\right\},\left\{g\right\},\left\{h,k\right\},\left\{i\right\},\left\{j\right\},\left\{m\right\}\right\}&\mbox{(voorbeeld)}
\end{array}
\end{equation}
Op basis van deze partitionering kunnen we een nieuwe toestandstabel opstellen: tabel \ref{tbl:stateTableAsyncMiniParty}. We gebruiken hier telkens de alfabetisch eerste letter van de toestanden die in een bepaalde partitie zitten. Dit is echter een arbitraire keuze. Vervolgens minimaliseren we verder door een compatibiliteitsrelatie op te stellen. Zo is $a$ duidelijk niet compatibel met $b$ net als $a$ en $d$, ze hebben immers een verschillende uitvoer. Ook $\left(a,c\right)$ en $\left(a,e\right)$ zijn duidelijk geen elementen van de relatie: $a$ en $c$ gaan onder invoer $\left(0,1\right)$ naar een verschillende toestand, hetzelfde geldt voor $a$ en $e$ onder invoer $\left(1,0\right)$. $\left(a,f\right)$ behoort dan weer wel tot de relatie. Het zijn beide toestanden met dezelfde uitvoer, en in alle mogelijke invoercombinaties heeft ofwel minstens \'e\'en van de toestanden een don't care, of wijzen ze naar dezelfde toestand. Alle andere toestanden zijn niet compatibel met $a$ omwille van hiervoor opgesomde redenen. We kunnen zo verder alle toestanden tegen elkaar uitspelen, en defini\"eren zo de equivalentierelatie $\equiv_0$ als volgt\footnote{Bij de definitie hebben we de symmetrische redundantie weggelaten. Het spreekt echter voor zich dat als $x\equiv_0y$, dat ook geldt $y\equiv_0x$}:
\begin{equation}
\left\{\begin{array}{cc}
a\equiv_0f&b\equiv_0h\\
b\equiv_0m&c\equiv_0j\\
d\equiv_0e&f\equiv_0j\\
g\equiv_0j&h\equiv_0m
\end{array}\right.
\end{equation}
We kunnen deze relatie ook grafisch voorstellen, dit doen we met behulp van een \termen{Merger diagram}. Hierbij stellen we de toestanden voor als knopen, en indien twee toestanden compatibel zijn, tekenen we een ongerichte boog tussen de knopen die deze toestanden voorstellen. Een dergelijk Merger diagram staat op figuur \ref{fig:mergerDiagram0}.
\begin{figure}
\centering
\hfill{}
\subfigure[Iteratie 1]{\begin{tikzpicture}[shorten >=1pt,auto,node distance=1.3cm,on grid,semithick,every state/.style={draw=black!50,very thick,fill=black!20,scale=0.75}]
\node[state] (A) {$a$};
\node[state] (F) [below=of A] {$f$};
\node[state] (J) [right=of F] {$j$};
\node[state] (C) [below=of J] {$c$};
\node[state] (G) [above=of J] {$g$};
\node[state] (B) [right=of G] {$b$};
\node[state] (H) [right=of B] {$h$};
\node[state] (M) [below=of B] {$m$};
\node[state] (D) [right=of C] {$d$};
\node[state] (E) [right=of D] {$e$};
\node[state] (I) [left=of C] {$i$};
\path (A) edge (F)
      (F) edge (J)
      (J) edge (G)
      (J) edge (C);
\path (B) edge (H)
      (H) edge (M)
      (M) edge (B);
\path (D) edge (E);
\node[draw,dashed,fit=(A) (F)] {};
\node[draw,dashed,fit=(B) (H) (M)] {};
\node[draw,dashed,fit=(C) (J)] {};
\node[draw,dashed,fit=(D) (E)] {};
%\draw[rounded corners=3mm,dashed] (A.north west) -- (A.north east) -- (F.south east) -- (F.south west) -- cycle;
\end{tikzpicture}
\figlab{mergerDiagram0}}
\hfill{}
\subfigure[Iteratie 2]{\begin{tikzpicture}[shorten >=1pt,auto,node distance=1.3cm,on grid,semithick,every state/.style={draw=black!50,very thick,fill=black!20,scale=0.75}]
\node[state] (A) {$a$};
\node[state] (B) [right=of A] {$b$};
\node[state] (C) [right=of B] {$c$};
\node[state] (D) [below=of A] {$d$};
\node[state] (I) [right=of D] {$i$};
\node[state] (G) [right=of I] {$g$};
\path (C) edge (G);
\node[draw,dashed,fit=(C) (G)] {};
\end{tikzpicture}
\figlab{mergerDiagram1}}
\hfill{}
\subfigure[Iteratie 3]{\begin{tikzpicture}[shorten >=1pt,auto,node distance=1.3cm,on grid,semithick,every state/.style={draw=black!50,very thick,fill=black!20,scale=0.75}]
\node[state] (A) {$a$};
\node[state] (B) [right=of A] {$b$};
\node[state] (C) [right=of B] {$c$};
\node[state] (D) [below=of A] {$d$};
\node[state] (I) [right=of D] {$i$};
\end{tikzpicture}
\figlab{mergerDiagram2}}
\hfill{}
\caption{Merger-diagrammen van het leidend voorbeeld.}
\end{figure}
We kunnen vervolgens een keuze maken welke compatibele toestanden we samennemen. Merk op dat we enkel toestanden kunnen samennemen als elke twee toestanden compatibel met elkaar zijn. Zo kunnen we $\left\{a,f\right\}$ samennemen, maar $\left\{a,f,j\right\}$ is niet mogelijk omdat $a$ en $j$ niet compatibel met elkaar zijn. Op figuur \ref{fig:mergerDiagram0} duiden we met behulp van de stippelijnen aan welke toestanden we hebben samengenomen. Of formeler:\begin{equation}
\begin{array}{lr}
\mathcal{P}_2=\left\{\left\{a,f\right\},\left\{b,h,m\right\},\left\{c,j\right\},\left\{d,e\right\},\left\{g\right\},\left\{i\right\}\right\}&\mbox{(voorbeeld)}
\end{array}
\end{equation}
We hadden echter in plaats van $\left\{c,j\right\}$ ook voor $\left\{g,j\right\}$ kunnen opteren. Of $\left\{a,f\right\}$ en $\left\{c,j\right\}$ kunnen inwisselen voor $\left\{f,j\right\}$. Men kan argumenteren dat we met de laatste configuratie minder toestanden wegwerken. Merk echter op dat we eventueel na deze iteratie nog iteraties kunnen uitvoeren en toestanden wegwerken. Een strategie waarin we in elke iteratie het aantal toestanden maximaal reduceren zal niet altijd tot het beste resultaat leiden.
\paragraph{}
We genereren vervolgens voor elke voorgestelde groep een nieuwe toestand. Deze toestand stellen we met de alfabetisch laagste letter voor van de toestanden die de groep omvat. De toestandstabel die hieruit voortkomt staat in tabel \ref{tbl:stateTableIteration1}.
\begin{table}[hbt]
\centering
\subtable[Iteratie 1]{\small{\begin{tabular}{c|cccc|c}
\multirow{2}{*}{Toestand}&\multicolumn{4}{c|}{$I\ E$}&\multirow{2}{*}{$Q$}\\&00&01&11&10&\\\hline
$a$&$a$&$a$&$b$&$c$&$0$\\
$b$&$b$&$b$&$b$&$d$&$1$\\
$c$&$b$&$c$&$g$&$c$&$0$\\
$d$&$g$&$a$&$d$&$d$&$1$\\
$g$&$b$&$g$&$c$&$-$&$0$\\
$i$&$i$&$d$&$-$&$a$&$1$
\end{tabular}}
\tbllab{stateTableIteration1}}
\subtable[Iteratie 2 en 3]{\small{\begin{tabular}{c|cccc|c}
\multirow{2}{*}{Toestand}&\multicolumn{4}{c|}{$I\ E$}&\multirow{2}{*}{$Q$}\\&00&01&11&10&\\\hline
$a$&$a$&$a$&$b$&$c$&$0$\\
$b$&$b$&$b$&$b$&$d$&$1$\\
$c$&$b$&$c$&$c$&$c$&$0$\\
$d$&$c$&$a$&$d$&$d$&$1$\\
$i$&$i$&$d$&$-$&$a$&$1$
\end{tabular}}
\tbllab{stateTableIteration2}}
\caption{Evolutie van de toestandstabel bij het minimaliseren voor en na twee iteraties.}
\end{table}
Hierbij voegen we dus de toestanden samen. We zullen in deze tekst de groep $\left\{b,h,m\right\}$ volledig uitwerken. Deze groep hebben we voorgesteld door $b$. Bij de invoer $\left(0,0\right)$ zien we dat zowel $b$ als $h$ een don't care bevatten. Voor $m$ is dit echter een stabiele toestand, bijgevolg is deze ingang ook een stabiele toestand. Bij de configuratie $\left(0,1\right)$ wijzen alle toestanden naar $b$, vermits $b$ in de groep zit die later $b$ zal worden, plaatsen we $b$ in de tabel. Indien $\left(1,1\right)$ op de ingang wordt aangelegd gaan $b$ en $h$ naar $h$, $m$ bevat een don't care. We kunnen dus zonder problemen ook $h$ invullen bij die don't care. $h$ is een onderdeel van de groep die later toestand $b$ zal worden. Bijgevolg kunnen we ook voor deze kolom $b$ invullen. In de laatste configuratie ten slotte -- $\left(1,0\right)$ -- gaan $h$ en $m$ naar $e$. We zien dat $e$ een onderdeel is van de groep $\left\{d,e\right\}$. Deze groep zal dus later voorgesteld worden door de toestand $d$. Bijgevolg vullen we $d$ in. Indien we deze procedure ook toepassen op de andere groepen resulteert dit in tabel \ref{tbl:stateTableIteration1}. Hiermee zijn we aan het einde gekomen van de eerste iteratie. Vermist we echter een andere toestandstabel hebben tegenover het begin van de iteratie, dienen we een nieuwe iteratie aan te vatten. Hierbij berekenen we opnieuw een compatibiliteitsrelatie: $\equiv_1$. We zullen de berekening van deze relatie achterwege laten, vermits we reeds de vorige relatie uitgebreid hebben beschreven. We defini\"eren $\equiv_1$ op symmetrie na als volgt:
\begin{equation}
c\equiv_1g
\end{equation}
Het Merger-diagram van deze relatie staat op figuur \ref{fig:mergerDiagram1}. Hier is de keuze van de groep wel deterministisch. De nieuwe partitie is dus:
\begin{equation}
\begin{array}{lr}
\mathcal{P}_3=\left\{\left\{a\right\},\left\{b\right\},\left\{c,g\right\},\left\{d\right\},\left\{i\right\}\right\}&\mbox{(voorbeeld)}
\end{array}
\end{equation}
Op basis van deze partitie ruilen we dus $\left\{c,g\right\}$ in voor een nieuwe toestand $c$. De vernieuwde toestandstabel is dan tabel \ref{tbl:stateTableIteration2}. Merk op dat we niet enkel de rij van toestand $c$ moeten aanpassen. Ook toestanden die naar $g$ kunnen springen, springen nu naar $c$, bijvoorbeeld toestand $d$. Opnieuw hebben we dus de toestandstabel aangepast. Dit impliceert dat we nogmaals een iteratie uitvoeren. De compatibiliteitsrelatie $\equiv_2$ blijkt echter leeg te zijn, geen enkele toestand is dus compatibel met een andere. Dit leidt tot een Merger-diagram zoals op figuur \ref{fig:mergerDiagram2}. We kunnen bijgevolg geen toestanden groeperen waardoor de tabel onveranderd blijft. We hebben dus het aantal toestanden geminimaliseerd tot 5 zoals in tabel \ref{tbl:stateTableIteration2}.
\paragraph{Alternatieve groepering}Bij de eerste iteratie bij het samenvoegen van de toestanden, konden we kiezen tussen twee vormen van groeperingen. Ofwel groeperen we $c$ en $j$ ofwel $g$ en $j$. Anderzijds hadden we ook $f$ en $j$ kunnen samenvoegen, maar deze configuratie is minder voordelig. We zullen bij wijze van extra voorbeeld ook het alternatieve scenario bespreken. Wanneer we dit alternatieve scenario uitwerken bekomen we na \'e\'en iteratie de waarden op \tblref{stateTableAltIteration1}.
\begin{table}[hbt]
\centering
\subtable[Iteratie 1]{\small{\begin{tabular}{c|cccc|c}
\multirow{2}{*}{Toestand}&\multicolumn{4}{c|}{$I\ E$}&\multirow{2}{*}{$Q$}\\&00&01&11&10&\\\hline
$a$&$a$&$a$&$b$&$c$&$0$\\
$b$&$b$&$b$&$b$&$d$&$1$\\
$c$&$-$&$c$&$g$&$g$&$0$\\
$d$&$g$&$a$&$d$&$d$&$1$\\
$g$&$b$&$g$&$g$&$g$&$0$\\
$i$&$i$&$d$&$-$&$a$&$1$
\end{tabular}}
\tbllab{stateTableAltIteration1}}
\subtable[Iteratie 2 en 3]{\small{\begin{tabular}{c|cccc|c}
\multirow{2}{*}{Toestand}&\multicolumn{4}{c|}{$I\ E$}&\multirow{2}{*}{$Q$}\\&00&01&11&10&\\\hline
$a$&$a$&$a$&$b$&$c$&$0$\\
$b$&$b$&$b$&$b$&$d$&$1$\\
$c$&$b$&$c$&$c$&$c$&$0$\\
$d$&$c$&$a$&$d$&$d$&$1$\\
$i$&$i$&$d$&$-$&$a$&$1$
\end{tabular}}
\tbllab{stateTableAltIteration2}}
\caption{Evolutie van de toestandstabel bij een alternatieve minimalisering voor en na twee iteraties.}
\end{table}
Op basis van deze tabel kunnen we verder bepalen welke toestanden compatibel zijn. Door de definitie van compatibele toestanden toe te passen zien we dat enkel $c$ en $g$ compatibel met elkaar zijn. Vermits er geen alternatieven zijn, voegen we beide toestanden samen en bekomen we \tblref{stateTableAltIteration2}. We kunnen opmerken dat deze tabel volledig identiek is aan de \tblref{stateTableIteration2}. Bijgevolg kunnen we deze tabel ook niet verder minimaliseren. We kunnen ook besluiten dat zelfs wanneer we een keuze kunnen maken tussen verschillende alternatieven, dit niet noodzakelijk betekent dat we een andere ofwel minder minimale configuratie zullen uitkomen. Soms kan men via verschillende keuzes toch hetzelfde of een even goedkope configuratie bekomen. Anderzijds is dit geen algemeen geldend principe: de keuze welke toestanden zullen worden samengevoegd kan wel degelijk verschillende eindconfiguraties opleveren.
\paragraph{Leidend voorbeeld} Bij wijzen van oefening kan de lezer de toestandstabel van het leidend voorbeeld minimaliseren. De oplossing staat in \sscref{asynchronousSequentialMinimalisation}.
\subsection{Stap 3: Codering van de toestanden}
Net als bij synchrone schakelingen moeten we elk van deze toestanden op een manier coderen in het geheugen. Het probleem is dat we bij het coderen van toestanden bij asynchrone schakelingen op nieuwe problemen stuiten.
\subsubsection{Terminologie}
\label{term:race}
Deze problemen brengen nieuwe terminologie met zich mee die we eerst zullen introduceren:
\begin{itemize}
 \item \termen{Race}: Een fenomeen dat optreedt wanneer door \'e\'en ingangsbit te veranderen er minstens twee toestandsvariabelen moeten veranderen.
 \item \termen{Critical race}: Indien een race tot een tijdelijk verkeerde toestand leidt (de codering van een andere toestand dus) spreken we van een critical race.
 \item \termen{Cycle}: Een sequentieel circuit werkt met terugkoppeling. Door deze terugkoppeling kunnen er oscillerende effecten optreden. Een race die in een oscillerend effect uitmondt heet een cycle.
\end{itemize}
Het optreden van deze een critical race of cycle hangt af van de vertragingskarakteristieken van de poorten. Het spreekt voor zich dat we trachten om deze fenomenen te voorkomen. Daarom zullen we ook technieken ontwikkelen die ons toelaten een goede toestandscodering te ontwikkelen waarbij we deze problemen reeds kunnen voorkomen.
\begin{figure}[hbt]
\centering
\importtikzsubfigure{asynchrone-trans}{Toestandstabel.}
\importtikzsubfigure{asynchrone-impl1}{Implementatie 1.}
\importtikzsubfigure{asynchrone-impl2}{Implementatie 2.}
\importtikzsubfigure{asynchrone-expected}{Verwacht gedrag.}
\importtikzsubfigure{asynchrone-cycle}{Cycle.}
\importtikzsubfigure{asynchrone-criticalrace}{Critical race.}
\caption{Voorbeelden van een cycle en critical race.}
\figlab{asynchrone-problems}
\end{figure}
\figref{asynchrone-problems} illustreert het principe van een critical race en cycle. Op \figref{asynchrone-impl1} en \figref{asynchrone-impl2} stellen we twee equivalente asynchrone schakelingen voor. Het enige wat we hebben aangepast is de twee NAND-poorten omzetten naar AND-poorten alsook een NAND-poort naar een OR-poort. In een combinatorische schakeling is dit een perfect geldige transformatie die tot equivalente resultaten leidt. In de implementatie bepalen de signalen $s_0$ en $s_1$ samen de toestand, $q$ bepaald de uitvoer en $i$ en $e$ de invoer. Bij wijze van voorbeeld beschouwen we het systeem in een toestand $\tupl{s_0,s_1}=\tupl{1,0}$ met aan de ingang een signaal $\tupl{i,e}=\tupl{0,1}$. We veranderen vervolgens het signaal van $i$ waardoor we volgens de toestandstabel op \figref{asynchrone-trans} in toestand $\tupl{s_0,s_1}=\tupl{1,0}$ zullen terechtkomen. Wanneer we dit doen volgens de eerste schakeling bekomen we het tijdsgedrag zoals op \figref{asynchrone-expected}. We zien dat beide toestandsignalen tegelijk veranderen. Zelfs wanneer er een klein tijdsverschil op de verandering zit zal dit echter niet tot grote problemen leiden.
\paragraph{}
Wanneer we echter werken met de tweede oplossing is het tijdsverschil veel groter. Hierdoor ontstaat er een terugkoppeling tussen het signaal $b'$ en $s_1$. Beide reageren telkens op de verandering van de ander waardoor we in een cycle terechtkomen. De toestandsverandering wordt dus $\tupl{s_0,s_1}=\tupl{0,1}\rightarrow\tupl{0,0}\rightarrow\tupl{0,1}\rightarrow\ldots$.
\paragraph{}

\paragraph{}Het aantal bits dat verandert tussen twee toestanden is een belangrijke eigenschap, deze wordt vaak ook de ``\termen{Hamming distance}'' ofwel ``\termen{Hammingafstand}'' genoemd, en is enkel gedefinieerd tussen twee bitreeksen van dezelfde lengte.
\subsubsection{Elimineren van critical races}
Men kan critical races elimineren door ervoor te zorgen dat er nooit twee of meer toestandssignalen tegelijk moeten veranderen. Overgangen zoals van $00$ naar $11$ zijn dus uit den boze. Er zijn grofweg drie methodes waarmee we dit kunnen realiseren. We zullen deze methodes ordenen volgens het vermogen om problemen op te lossen. Zo is de laatste methode in staat om alle problemen op te lossen, maar zal deze meestal een hoge kostprijs met zich meebrengen. We proberen dus de problemen op te lossen in de volgorde waarin de methodes worden voorgesteld. Verder zullen de methodes meestal in staat zijn een gedeelte van het probleem op te lossen, waarna de andere methodes de overige problemen kunnen oplossen. De methodes zijn:
\begin{enumerate}
 \item Het kiezen van een toestandscodering die elke overgang realiseert door hooguit \'e\'en bit te veranderen.
 \item Werken met zogezegde ``\termen{tussentoestanden}'': via een reeds bestaande toestand toch de finale toestand bereiken, bij de verschillende overgangen verandert dan telkens slechts \'e\'en bit.
 \item Het introduceren van ``\termen{overgangstoestanden}'': nieuw toestanden toevoegen die geen functionaliteit hebben buiten deze van het doorverwijzen naar een andere toestand.
\end{enumerate}
We zullen elk van deze methodes in de volgende subsubsecties bespreken en toepassen op twee voorbeelden. De toestandstabellen van beide voorbeelden zijn gegeven in \tblref{asynchrone-code-exa} en \tblref{asynchrone-code-exb}.
\begin{table}[hbt]
\centering
\importtabularsubtable{asynchrone-code-exa}{Voorbeeld 1.}
\importtabularsubtable{asynchrone-code-exb}{Voorbeeld 2.}
\caption{Toestandstabellen van de leidende voorbeelden bij de asynchrone toestandscodering.}
\end{table}
In de tabellen werden sommige overgangen geannoteerd met een subscript. Deze overgangen zijn stabiele configuraties: een set van toestand- en ingangbits die tot dezelfde toestand zullen leiden. In het eerste geval zijn er zeven van deze stabiele configuraties, in het tweede voorbeeld zes. Deze configuraties zijn:
\begin{equation}
T_1=\acclarray{
\tupl{a,\tupl{0,0}}_1\\
\tupl{a,\tupl{1,1}}_2\\
\tupl{b,\tupl{0,0}}_3\\
\tupl{b,\tupl{0,1}}_4\\
\tupl{c,\tupl{0,1}}_5\\
\tupl{c,\tupl{1,1}}_6\\
\tupl{d,\tupl{1,0}}_7
}\ \ \ 
T_2=\acclarray{
\tupl{a,\tupl{0,0}}_1\\
\tupl{a,\tupl{1,1}}_2\\
\tupl{a,\tupl{1,0}}_3\\
\tupl{b,\tupl{0,1}}_4\\
\tupl{c,\tupl{0,1}}_5\\
\tupl{c,\tupl{1,0}}_6
}
\end{equation}
Deze stabiele configuraties spelen een belangrijke rol in de verschillende methodes en het transitiediagram.
\subsubsection{Methode 1: Zoeken naar een goede codering}
Hierbij kunnen we terugdenken aan de ``minimal-bit-change'' en de ``Gray-code teller'' uit \ref{term:minimalBitChange}. Deze methode lost in de meeste gevallen reeds heel wat problemen op. Een methode die het beste resultaat oplevert is alle mogelijk toestandscoderingen afgaan en vervolgens het aantal overgangen met 1 veranderende bit tellen. Deze methode is niet effici\"ent vermits we \bigoh{n!} verschillende configuraties moeten analyseren. We kunnen uiteraard met behulp van heuristische methodes reeds tot een acceptabele configuratie komen. Bovendien is de kans groot dat niet elke overgang tot 1 veranderende bit is te herleiden, in dat geval moeten we de andere methodes gebruiken. Een optimale configuratie in methode 1 leidt niet noodzakelijk tot het beste eindresultaat.
\paragraph{Voorbeeld}
Bij wijze van voorbeeld zullen we een goede codering zoeken voor beide voorbeelden. Het eerste geval (\tblref{asynchrone-code-exa}) kunnen we oplossen met behulp van een heuristiek. Zo kennen we de toestand $d$ de codering $10$ toe. Omdat er slechts \'e\'en overgang is tussen $c$ en $d$, zullen we $c$ de encodering $01$ geven. $a$ krijgt de encodering $00$ omdat er zowel overgangen tussen $a$ en $c$, als tussen $a$ en $d$ zijn. $b$ krijgt ten slotte de overblijvende codering $11$. Het resultaat van deze codering staat in de coderingstabel in \tblref{asynchrone-code-exa-m1a}. We voeren ook wat nieuwe syntax in die het ons in de volgende stappen makkelijker zal maken: stabiele configuraties zullen we noteren tussen twee vertical bars (``$|$'') in de tabel. Cellen waarbij de toestand naar een andere toestand gaat waarbij de encodering minstens twee bits verandert worden onderlijnd. Deze configuraties zullen we nog trachten aan te passen met de volgende methodes.
\begin{table}[hbt]
\centering
\importtabularsubtable{asynchrone-code-exa-m1a}{Voorbeeld 1, alternatief 1.}
\importtabularsubtable{asynchrone-code-exa-m1b}{Voorbeeld 1, alternatief 2.}
\importtabularsubtable{asynchrone-code-exb-m1}{Voorbeeld 2.}
\caption{Coderingstabellen van het voorbeeld na het toepassen van de eerste methode.}
\end{table}
\paragraph{}
In het geval van een klein aantal toestanden kunnen we exhaustief zoeken. Dit wordt vergemakkelijkt omdat we symmetrie\"en kunnen uitbuiten. De concrete codering maakt immers niet zoveel uit, zolang de Hammingafstand maar dezelfde blijft. We introduceren hiervoor een \termen{transitiediagram}. Een transitiediagram is een grafe waarbij de knopen toestanden voorstellen. We plaatsen bogen tussen twee toestanden wanneer er transities tussen twee toestanden kunnen plaatsvinden. Deze bogen bevatten de annotaties naar welke stabiele configuratie deze transitie uiteindelijk zal migreren. Deze annotaties worden opgedeeld in twee types. Wanneer we na de overgang meteen in een stabiele configuratie terecht komen zetten we de bijbehorende annotatie op de boog zonder deze te onderlijnen. Wanneer we echter na deze toestand niet in een stabiele configuratie terechtkomen, noteren we de annotatie van de stabiele toestand waar we uiteindelijk in zullen terecht komen en wordt deze onderlijnd.
\paragraph{}
Het eerste leidende voorbeeld telt vier toestanden. We kunnen bijgevolg een grafe beschouwen met vier toestanden. Rotaties en spiegelingen bij deze grafes dienen we niet te beschouwen. De Hammingafstand blijft immers onder deze transformaties gelijk. Bijgevolg zijn er slechts twee mogelijke configuraties voorgesteld op \figref{asynchrone-code-exa-m1a} en \figref{asynchrone-code-exa-m1b}.
\begin{figure}[hbt]
\centering
\importtikzsubfigure{asynchrone-code-exa-m1a}{Voorbeeld 1, alternatief 1.}
\importtikzsubfigure{asynchrone-code-exa-m1b}{Voorbeeld 1, alternatief 2.}
\importtikzsubfigure{asynchrone-code-exb-m1}{Voorbeeld 2.}
\caption{Transitiediagramma van het voorbeeld na het toepassen van de eerste methode.}
\end{figure}
Zoals we kunnen vaststellen komt \figref{asynchrone-code-exa-m1a}. We kunnen door twee horizontale of verticale buren om te wisselen de andere versie bekomen. Indien we dit doen -- bijvoorbeeld met $B$ en $C$ -- bekomen we het alternatief op \figref{asynchrone-code-exa-m1b}. We hebben dit alternatief ook in tabelvorm geformaliseerd in \tblref{asynchrone-code-exa-m1b}. Een belangrijk aspect bij deze diagrammen is dat we het aantal overgangen waarbij twee of meer toestand-bits willen minimaliseren. Dit komt dus neer op de diagonale transities. We kunnen opmerken dat het eerste alternatief bijgevolg beter is dan het tweede.
\paragraph{}
Ook het tweede voorbeeld minimaliseren we met behulp van een transitiediagram zoals op \figref{asynchrone-code-exb-m1}. Men kan opnieuw stellen dat dit systeem equivalent is onder spiegeling. Onder rotatie is dit echter niet het geval. Hoe we immers de toestanden ook alloceren, er zal steeds een boog zijn met twee te veranderen bits. We kunnen echter kiezen welke twee toestanden er verbonden zijn met deze boog. In het geval van $b$ en $c$ is er slechts sprake van \'e\'en transitie. Bijgevolg alloceren we de toestanden zoals weergegeven op \figref{asynchrone-code-exb-m1} en op \tblref{asynchrone-code-exb-m1}.
\subsubsection{Methode 2: Gebruik maken van een tussentoestand}
Vermits er geen kloksignaal is en we dus te maken hebben met een schakeling die blijft zoeken naar een stabiele toestand, hoeven we niet noodzakelijk meteen de uiteindelijke toestand in een cel in te vullen. We kunnen ook een tijdelijke transitie naar een andere toestand ondernemen - waarbij slechts \'e\'en toestandsbit verandert - waarna we met een reeks tussentoestanden in de uiteindelijke toestand belanden. Dit principe is dus niet beperkt tot \'e\'en tussentoestand. Een extra hulpmiddel dat we hierbij kunnen hanteren zijn de don't cares die nog in de tabel staan. Vermits deze configuraties toch niet kunnen voorkomen, kunnen we een transitie invullen bij de bijbehorende don't cares. In het andere geval moeten we op zoek gaan naar een toestand die slechts \'e\'en bit verschilt van de huidige toestand en die dezelfde stabiele eindtoestand voor de invoer-bits stelt.
\paragraph{}
In het eerste voorbeeld is er sprake van twee transities die we moeten aanpassen: wanneer we ons bevinden in toestand $\tupl{0,0}_a$ met invoer $\tupl{0,1}$ en in toestand $\tupl{0,1}_c$ met invoer $\tupl{1,0}$. Om deze problemen op te lossen dien we de relevante kolommen van de invoer te inspecteren. In het eerste geval zoeken we naar een toestand met een Hammingafstand van 1 bit die ons onder invoer $\tupl{0,1}$ naar de toestand $\tupl{1,1}_b$ zal brengen. In het eerste geval lijkt zo'n toestand niet te bestaan. We kunnen echter toestand $\tupl{1,0}_d$ opmerken. Deze toestand verschilt slechts \'e\'en bit en bevat in de relevante kolom een don't care. Vermits de ingang toch niet kan voorkomen in deze toestand kunnen we deze gebruiken om de overgang te bewerkstelligen. We vullen dus $\tupl{1,1}_b$ in in de rij van toestand $\tupl{1,0}_d$ met invoer $\tupl{0,1}$. Verder passen we de rij van de originele toestand aan: de cel van toestand $\tupl{0,0}_a$ met invoer $\tupl{0,1}$ overschrijven we met de waarde $\tupl{1,0}_d$. Concreet betekent dit dus wanneer we ons in toestand $b$ bevinden en we leggen de relevante invoer aan, we eerst naar toestand $d$ zullen springen. Toestand $d$ is echter niet stabiel onder deze invoer waardoor we meteen naar toestand $b$ migreren: de oorspronkelijk bedoelde toestand.
\paragraph{}
We proberen ook de tweede problematische transitie van het eerste voorbeeld op te lossen: invoer $\tupl{1,0}$ in toestand $\tupl{0,1}_c$. We gaan opnieuw op zoek naar een toestand die in dezelfde kolom ook tot toestand $\tupl{1,0}_d$ leidt. Er bestaat hiervoor \'e\'en toestand: $\tupl{1,1}_b$ heeft een Hamming-afstand van $1$ en we zien dat in de kolom voor invoer $\tupl{1,0}$ deze ook een transitie naar $\tupl{1,0}_d$ leidt. Bijgevolg modificeren we de eerste cel zodat deze naar $\tupl{1,1}_b$ leidt. Na deze stappen bekomen we de coderingstabel in \tblref{asynchrone-code-exa-m2} en het transitiediagram in \figref{asynchrone-code-exa-m2}. De schuingedrukte cellen zijn cellen die we hebben aangepast. Zoals we kunnen zien zijn er geen transities meer die meer dan \'e\'en bit aanpassen. De toestandscodering is dus volledig aangepast.
\begin{table}[hbt]
\centering
\importtabularsubtable{asynchrone-code-exa-m2}{Voorbeeld 1.}
\importtabularsubtable{asynchrone-code-exa-m2c}{Voorbeeld 1 met compressie.}
\importtabularsubtable{asynchrone-code-exb-m2}{Voorbeeld 2.}
\caption{Coderingstabellen van het voorbeeld na het toepassen van de tweede methode.}
\end{table}
\begin{figure}[hbt]
\centering
\importtikzsubfigure{asynchrone-code-exa-m2}{Voorbeeld 1.}
\importtikzsubfigure{asynchrone-code-exb-m2}{Voorbeeld 2.}
\caption{Transitiediagramma van het voorbeeld na het toepassen van de tweede methode.}
\end{figure}
\paragraph{}
We kunnen in deze methode ook een ander aspect bewerkstelligen: compressie. Compressie probeert de sequentie van toestanden in te korten die de schakeling zal overlopen wanneer de ingang verandert. Stel dat de schakeling zich in toestand $\tupl{0,0}_a$ bevinden en de ingang vanuit een stabiele configuratie naar de invoer-bits $\tupl{1,0}$ verandert. In dat geval doorloopt de schakeling de volgende toestanden: $\tupl{0,0}_a\rightarrow\tupl{0,1}_c\rightarrow\tupl{1,1}_b\rightarrow\tupl{1,0}_d$. We doorlopen dus vier toestanden alvorens we in de uiteindelijke stabiele configuratie terecht komen. Dit terwijl de Hammingafstand tussen $a$ en $d$ slechts \'e\'en bit bedraagt. Dit betekent dus dat we rechtstreeks naar $d$ kunnen migreren. Hiervoor dienen we dus enkel de transitie van toestand $\tupl{0,0}_a$ met invoer $\tupl{1,0}$ aan te passen naar $\tupl{1,0}_d$. We bekomen dus de coderingstabel in \tblref{asynchrone-code-exa-m2c}.
\paragraph{}
We zullen ook proberen deze methode toe te passen op voorbeeld $2$. Meer bepaald wanneer we ons in toestand $\tupl{0,1}_b$ bevinden en we veranderen de ingang-bits naar $\tupl{1,1}$ beschouwen we momenteel een transitie waarbij twee bits veranderen. De enige toestand met een Hammingafstand van $1$ is echter $\tupl{0,0}_a$. Vermits we in deze toestand met ingang-bits $\tupl{1,1}$ in een stabiele configuratie zitten, kunnen we geen transitie bewerkstelligen naar $\tupl{1,0}_c$. Het probleem kan dus niet opgelost worden met deze methode. We zullen dus methode $3$ hiervoor moeten aanwenden.
\subsubsection{Methode 3: Invoeren van extra overgangstoestand}
In de laatste methode introduceren we om de problematische transities op te lossen een nieuwe toestand. Deze toestand mag -- net als in de vorige methode -- slechts een Hammingafstand van $1$ optekenen met de toestand waartussen de problematisch codering zich bevindt. Dit is niet altijd mogelijk. Daarom zal men soms zelfs een pad van verschillende extra overgangstoestanden moeten ontwerpen die telkens \'e\'en bit van elkaar verschillen.
\paragraph{}
Omdat alle coderingen soms al in gebruik zijn of om aan de voorwaarde van de Hamming-afstand te voldoen is het daarom niet altijd eenvoudig of zelfs mogelijk om een codering met hetzelfde aantal bits te voorzien. In dat geval moeten we soms de toestandcodering uitbreiden naar meerdere bits. Wanneer we extra bits toevoegen betekent dit traditioneel dat we de volledige codering herbekijken wat veel werk met zich meebrengt. Door echter bits vooraan toe te voegen kunnen we de oude codering behouden (met leidende $0$-bits bijvoorbeeld) en maken we tegelijk ruimte om meer toestanden voor te stellen.
\paragraph{}
Eenmaal we een nieuwe codering hebben beschouwd is de realisatie eenvoudig: in de kolom van de relevante invoer-configuratie plaatsen we het pad door elke extra overgangstoestand te laten wijzen naar de volgende overgangstoestand in het pad. Ook passen we de rij van de oorspronkelijke toestand aan zodat deze wijst naar de eerste overgangstoestand in het pad. De overige kolommen vullen we op met don't cares alsook de uitvoer in de overgangstoestanden.
\paragraph{}
We hoeven niet voor elke problematische transitie meteen extra overgangstoestanden te voorzien. Soms kan men bijvoorbeeld eerst \'e\'en overgang oplossen en vervolgens met methode $2$ bijvoorbeeld proberen de nieuwe toestand als tussentoestand te beschouwen bij het oplossen van een andere problematische transitie.
\paragraph{}
Met deze methode zullen we ten slotte de laatste problematische transitie oplossen van het tweede voorbeeld. De twee toestanden waartussen deze transitie zich afspeelt zijn $\tupl{0,1}_b$ en $\tupl{1,0}_c$. Zonder de toestandscodering uit te breiden met extra bits zijn er twee coderingen die een Hammingafstand van $1$ hebben met beide toestanden: $\tupl{0,0}$ en $\tupl{1,1}$. We kunnen $\tupl{0,0}$ niet gebruiken omdat deze codering reeds gebruikt wordt door toestand $a$. De andere codering is echter nog vrij. We introduceren dus een toestand $\tupl{1,1}_d$. Bij deze toestand vullen we de transitie-kolommen en de uitvoer-kolom met don't cares behalve de relevante ingang: $\tupl{1,1}$. Deze kolom laten we verwijzen naar de doeltoestand $\tupl{1,0}_c$. Tot slot passen we de rij van toestand $\tupl{0,1}_b$ aan: we laten de relevante invoer-kolom verwijzen naar de ingevoerde overgangstoestand. Wanneer we deze wijzigingen doorvoeren bekomen we de coderingstabel op \tblref{asynchrone-code-exb-m3} en het transitiediagram op \figref{asynchrone-code-exb-m3}. We zien in de coderingstabel dat er geen problematische transities meer zijn. Alle problemen zijn bijgevolg opgelost.
\importtabulartable{asynchrone-code-exb-m3}{Coderingstabel van het voorbeeld na het toepassen van de derde methode.}
\importtikzfigure{asynchrone-code-exb-m3}{Transitiediagram van het voorbeeld na het toepassen van de derde methode.}
\paragraph{Initi\"ele toestand}
Een laatste aspect die we moeten behandelen wanneer we methode $3$ toepassen is de initi\"ele toestand: de schakeling van waaruit de schakeling vertrekt wanneer de spanning opkomt. Net als bij flipflops en latches is dit niet te voorspellen. De originele toestand hangt dan ook af van verschillende factoren: inductieve spanning ten gevolge van eerder gebruik, kleine verschillen in de vertragingen op poorten, de temperatuur van de poorten op dat moment, thermische ruis op de verbindingen. Het kan echter gebeuren dat we dus in een codering terechtkomen van een overgangstoestand. Het probleem is dat in het voorbeeld de codering $\tupl{1,1}$ niet overeenkomt met een werkelijke toestand. Wanneer de schakeling dus met dergelijke toestand wordt ge\"initialiseerd moet de schakeling dus meteen een transfer maken naar een andere -- wel geldige -- toestand. Een probleem is echter dat in \tblref{asynchrone-code-exb-m3} er don't cares in de kolommen staan. Het is dus mogelijk dat dit later wordt ge\"implementeerd zodat er stabiele configuraties ontstaan voor een codering die niet overeenkomt met een toestand. Het is duidelijk dat dit niet de bedoeling is van een overgangstoestand. Om dit te vermijden kunnen  we in de kolommen opgevuld met don't cares concrete waarden invullen.
\paragraph{}
Welke waarden we precies invullen maakt niet zoveel uit. Zolang het stabiele configuraties met de relevante invoer-bit en voor een geldige toestand: dit betekent geen overgangstoestand. We zoeken dus in dezelfde kolom naar een stabiele configuratie en verwijzen in de tabel naar de overeenkomstige toestand.
\paragraph{}
In het leidend voorbeeld hebben we een overgangstoestand ge\"introduceerd. Wanneer we $\tupl{0,0}$ aanleggen aan de invoer zien we \'e\'en stabiele configuratie bij toestand $\tupl{0,0}_a$. In de tweede kolom zijn er twee stabiele configuraties we kunnen dus kiezen om ofwel $\tupl{0,1}_b$ ofwel $\tupl{1,0}_c$ in te vullen. We kiezen hier voor het eerste. De volgende kolom is er slechts \'e\'en stabiele configuratie: een configuratie met toestand $\tupl{0,0}_a$. Men kan argumenteren dat we deze toestand niet kunnen gebruiken: er veranderen immers twee bits in vergelijking met de eerste bit. Anderzijds staat er nog een don't care in de kolom. We kunnen deze don't care ook laten verwijzen naar $\tupl{0,0}_a$. Hierdoor is de volledig kolom nu gevuld met verwijzingen naar $\tupl{0,0}_a$. Vermits de volledige kolom uniform met dezelfde waardes gevuld is, is de Hamming-afstand tussen de toestanden niet meer van belang: Stel dat we ons in toestand $\tupl{1,1}$ bevinden en we een transitie naar $\tupl{0,0}_a$ maken, dan kan het gebeuren dat door een verschil in vertraging we in de toestand $\tupl{0,1}_b$ ofwel $\tupl{0,0}_c$ terechtkomen. Beide toestanden vormen echter geen probleem omdat ze zelf ook instabiele configuraties zijn die uiteindelijk toch naar dezelfde toestand zullen transformeren. Er is dus sprake van een race, maar deze is niet critical. De laatste kolom ten slotte is reeds ingevuld en bijgevolg bekomen we de uiteindelijke coderingstabel in \tblref{asynchrone-code-exb-m3i}.
\importtabulartable{asynchrone-code-exb-m3i}{Coderingstabel van het voorbeeld met initi\"ele toestand voor de overgangstoestand.}
\subsubsection{Initi\"ele toestand (bis)}
Naast eventuele overgangstoestanden kan het gebeuren dat niet alle coderingen zijn toegewezen aan werkelijke of overgangstoestanden. Stel bijvoorbeeld dat we een coderingstabel beschouwen met $6$ rijen. Hoewel er slechts zes toestanden zijn, zullen er in werkelijkheid meer toestandscoderingen mogelijk zijn. Zoals we al hebben aangehaald kan men niet voorspellen in welke toestandscodering de schakeling zich initieel zal bevinden wanneer de stroom opkomt. Daarom dienen we de overige coderingen dus ook te beschouwen en transities te voorzien naar eerder ge\"introduceerde stabiele configuraties. De werkwijze is dan ook volledig identiek aan deze bij het aanpassen van de don't cares van de overgangstoestanden. We gaan er dan ook niet verder op in.
\subsection{Stap 4: Realisatie met digitale logica}
Wie denkt dat met een geldige toestandscodering alle problemen van de baan zijn moeten we teleurstellen. We hebben dan wel kritische races ge\"elimineerd, maar bij het realiseren van de schakeling dienen we het volgende problemen op te lossen: \termen{hazards}. Een hazard vormt een ander probleem dan een race. Een race betekent dat door het tijdsverschil waartussen twee of meer signalen van waarde veranderen we in een foute toestand kunnen terechtkomen. Combinatorische logica kan echter ook andere problemen met zich meebrengen: namelijk dat de signalen alvorens de definitieve waarde aan te namen eerst enkele malen veranderen.
\subsubsection{Hazards}
De vorige definitie is nogal abstract. We zullen deze dan ook verder toelichten aan de hand van een voorbeeld. Alvorens dit te doen introduceren we extra terminologie. Zo worden hazards opgedeeld in twee soorten: een \termen{statische hazard} en een \termen{dynamische hazard}.
\paragraph{}
In het geval van een statische hazard zou het signaal eigenlijk niet moeten veranderen. Stel bijvoorbeeld dat we volgende expressie beschouwen:
\begin{equation}
\fun{f}{x,y,z}=x\cdot y+y'\cdot z
\end{equation}
Wanneer we $\tupl{x,y,z}=\tupl{1,1,1}$ aanleggen is het duidelijk dat dit resulteert in een uitgang $\fun{f}{1,1,1}=1$. Stel nu dat we het signaal van $y$ aanpassen naar laag, dan geldt nog steeds dat $\fun{f}{1,0,1}=1$. We kunnen dus stellen dat het aanpassen van $y$ in deze context geen verschil zal maken. Een gevaar voor een race is er dus bijgevolg zeker niet. Wanneer we deze formule implementeren met behulp van logica kunnen we een schakeling ontwerpen zoals op \figref{asynchrone-hazard-stat-ex}. Wanneer we de verandering van $y$ echter simuleren zien we op de bijbehorende tijdsgrafiek (\figref{asynchrone-hazard-stat-extim}) dat het signaal even naar $0$ gaat. We hebben dit fenomeen al eerder omschreven als glitch.
\begin{figure}[hbt]
\centering
\importtikzsubfigure{asynchrone-hazard-stat-ex}{Voorbeeldimplementatie.}
\importtikzsubfigure{asynchrone-hazard-stat-extim}{Tijdsgedrag van het voorbeeld.}
\caption{Statische hazards.}
\end{figure}
Men deelt statische hazards verder onder in een \termen{statische 0-hazard} en een \termen{statische 1-hazard}. De waarde verwijst naar de waarde die de uitgang oorspronkelijk had (en de waarde die dus ook moet worden aangehouden). Wanneer men schakelingen implementeert volgens het sum-of-products principe kan men enkel statische 1-hazards bekomen. Wanneer men werkt volgens de product-of-sums methodologie treden enkel statische 0-hazards op. We zullen dit principe verderop uitleggen.
\paragraph{}
We beschrijven dynamische hazards aan de hand van een andere formule:
\begin{equation}
\fun{f}{x,y,z,t}=\brak{\brak{x\wedge\brak{x\wedge y}'}'\wedge\brak{\brak{\brak{x \wedge y}'\wedge z}'\wedge t}'}'
\end{equation}
Deze formule kunnen we implementeren met een schakeling zoals op \figref{asynchrone-hazard-dyna-ex}. Wanneer we deze formule uitrekenen met de invoer $\tupl{x,y,z,t}=\tupl{0,1,1,1}$ zal het uitgang-signaal laag zijn: $\fun{f}{0,1,1,1}=0$. Wanneer we echter $x$ aanpassen naar een hoog signaal bekomen we $\fun{f}{1,1,1,1}=1$. Het signaal zal dus sowieso omkeren. Wanneer we dit echter simuleren in de tijd bekomen we de grafiek op \figref{asynchrone-hazard-dyna-extim}. We zien dat uiteindelijk het signaal naar hoog gaat, maar dat er eerst storingen op de uitgang verschijnen. Een dynamische hazard is dan ook een storing waarbij de uitgang niet eenmaal verandert van signaal, maar een oneven aantal keer (groter dan $1$). Een voordeel van een sum-of-products of product-of-sums implementatie te gebruiken is echter dat dynamische hazards niet kunnen voorkomen.
\begin{figure}[hbt]
\centering
\importtikzsubfigure{asynchrone-hazard-dyna-ex}{Voorbeeldimplementatie.}
\importtikzsubfigure{asynchrone-hazard-dyna-extim}{Tijdsgedrag van het voorbeeld.}
\caption{Dynamische hazards.}
\end{figure}
\paragraph{Oorzaak}
Bij het defini\"eren van de terminologie rond hazards hebben we niet stilgestaan hoe deze fenomenen tot stand komen. Hazards worden veroorzaakt door een tijdsverschil waarin een verandering doorheen de verschillende poorten wordt gepropageerd. Als voorbeeld nemen we opnieuw de schakeling bij statische hazards op \figref{asynchrone-hazard-stat-ex}. We zien dat het resultaat berekend wordt door een OR-poort die het resultaat van twee AND-poorten binair optelt. Wanneer we $y$ aanpassen zal de bovenste AND-poort deze verandering meteen waarnemen. De onderste AND-poort zal dit echter nog niet waarnemen: het signaal moet eerst nog door de NOT-poort propageren. Daardoor zal de bovenste AND-poort een $0$ op de OR-poort kunnen aanleggen alvorens de onderste AND-poort dit kan goedmaken door terug een $1$ op een ingang van de OR-poort aan te leggen. Wanneer we dus $y$ van $1$ naar $0$ zouden aanpassen treden er geen problemen op: de bovenste AND-poort zal een $1$ aanleggen op \'e\'en van de ingangen van de OR-poort alvorens de onderste dit doet waardoor dit niet aan de uitgang merkbaar zal zijn. Samenvattend kunnen we dus stellen dat de oorzaak van een hazard een verschil in vertraging is van eenzelfde ingang naar eenzelfde uitgang langs verschillende paden.
\paragraph{Gestroomlijnd tijdsgedrag}
Omdat hazards veroorzaakt wordt door tijdsverschillen doorheen de schakeling zouden we ervoor kunnen opteren om een schakeling te ontwerpen waar alle signalen gestroomlijnd door de schakeling propageren. Dit is echter onmogelijk te realiseren. Allereerst zou men allerhande poorten moeten tussenvoegen om bepaalde signalen voldoende te vertragen waardoor de schakeling duurder wordt. Daarnaast is het theoretische vertragingsmodel slechts een benadering. De werkelijke vertraging van een poort is ook afhankelijk van bijvoorbeeld de lengte van de verbindingen, de temperatuur van de poort, enzovoort. Sommige parameters zoals de temperatuur zijn bovendien op voorhand niet gekend.
\paragraph{Toestandssignalen}
In combinatorische en synchrone sequenti\"ele schakelingen komen hazards natuurlijk ook voor. Nochtans vormen hazards in deze schakelingen geen probleem. Dit komt omdat een combinatorische schakeling maar na een bepaalde tijd het correcte resultaat op de uitgangen moet kunnen aanleggen. Bij een synchrone schakeling bepaalt het kloksignaal dan weer wanneer de signalen aan de uitgang zullen worden ingelezen. Zolang het signaal dus correct berekend is voor de rising-edge die het resultaat in de flipflops zal opslaan, is er geen enkel probleem. In het geval van asynchrone sequenti\"ele schakelingen is dit niet geval. Dit komt door de toestandssignalen, de signalen die het terugkoppelingsmechanisme vormen. De signalen worden berekend door de logica van de schakeling, maar vormen ook een deel van de invoer van de schakeling. Wanneer de signalen dus tijdelijk een foutieve waarde aannemen kan dit tot problemen leiden: er kunnen oscillaties ontstaan of de schakeling kan in een foute toestand terechtkomen.
\paragraph{}
Het detecteren van hazards is een niet triviaal probleem en vereist doorgaans het simuleren van overgangen. Vooral in het geval van dynamische hazards is dit problematisch.
\subsubsection{Hazards en Karnaugh-kaarten}
Combinatorische schakelingen komen meestal tot stand met behulp van een Karnaugh-kaart volgens het sum-of-products concept. We hebben reeds aangehaald dat wanneer de logica het sum-of-products principe volgt, uitsluitend statische 1-hazards kunnen optreden. In deze subsubsectie gaan we hier dieper op in. Bovendien kunnen we op basis van Karnaugh-kaarten een methode voorstellen om combinatorische schakelingen te ontwikkelen waar geen hazards kunnen optreden.
\paragraph{}
Op \figref{asynchrone-hazard-karnaugh-ex} tonen we de bijbehorende Karnaugh-kaart voor de schakeling op \figref{asynchrone-hazard-stat-ex}. Op de figuur tonen we ook voor elke AND-poort welke gevallen worden bedekt. In het voorbeeld veranderden we de invoer van $\tupl{x,y,z}=\tupl{1,1,1}$ naar $\tupl{x,y,z}=\tupl{1,0,1}$. Zoals we op de figuur zien behoort dit tot een ander gebied. Men zou kunnen stellen dat de AND-poorten controleren of de invoer zich in hun overeenkomstige rechthoek bevindt en zo ja, komt er een \'e\'en op de uitgang. Een statische 1-hazard wordt veroorzaakt wanneer de rechthoek die we verlaten dit eerder ``opmerkt'' en dus de uitvoer op $0$ brengt alvorens de rechthoek waarin we toekomen dit merkt en de uitvoer terug hoog maakt. Door deze analogie kunnen we ook verklaren waarom we bij een schakeling volgens het sum-of-product-principe nooit een statisch 0-hazard zullen realiseren: we verplaatsen ons van cel naar cel. Wanneer beide cellen $0$ zijn, zal geen enkel rechthoek ooit actief worden. Bijgevolg zal op geen enkel moment er een $1$ aan de ingang van de OR-poort verschijnen.
\importtikzfigure{asynchrone-hazard-karnaugh-ex}{Karnaugh-kaart bij het leidende voorbeeld.}
\paragraph{Hazards voorkomen}
We kunnen een hazard voorkomen door ervoor te zorgen dat als we tussen twee cellen een transitie uitvoeren, er een rechthoek bestaat die beide cellen omvat. Een concrete oplossing wordt hiervoor voorgesteld in \figref{asynchrone-hazard-karnaugh-exred}. Wanneer we in dit voorbeeld een transitie uitvoeren, zullen de AND-poorten van de oorspronkelijke schakeling nog steeds een tijdsverschil optekenen. We introduceren echter een AND-poort die onder de transitie telkens een $1$ zal blijven aanleggen op de OR-poort. Bijgevolg zal de uitvoer altijd $1$ blijven.
\begin{figure}[hbt]
\centering
\importtikzsubfigure{asynchrone-hazard-karnaugh-exred}{Kaart.}
\importtikzsubfigure{asynchrone-hazard-impl-exred}{Implementatie.}
\caption{Het invoeren van redundante termen elimineert statische 1-hazards.}
\end{figure}
Om dus hazards te vermijden introduceren we \termen{redundante termen}: AND-poorten die niet strikt gezien nodig zijn om de correcte combinatorische logica voor te stellen, maar poorten die bij een wijziging aan de invoer ervoor zorgen dat er nooit een statische 1-hazard kan optreden.
\paragraph{}
Is het altijd mogelijk om zo'n AND-poort te realiseren? Wat indien we bijvoorbeeld een dambord patroon beschouwen zoals op \figref{asynchrone-hazard-karnaugh-exreddam}?
\importtikzfigure{asynchrone-hazard-karnaugh-exreddam}{Dambord patroon.}
In dat geval zouden we geen redundante termen kunnen introduceren: we kunnen dus bijvoorbeeld geen overgang van $\tupl{x,y,z}=\tupl{1,0,1}$ naar $\tupl{x,y,z}=\tupl{1,1,0}$ nemen en voorkomen dat we in een $0$ terecht komen. Merk echter op dat we hier twee ingangen hebben aangepast, iets wat volgens de specificaties niet mag. Indien we dus een dergelijke overgang nemen, zullen we altijd eerst \'e\'en van de ingangen eerst aanpassen en wachten tot de effecten zijn uitgewerkt. Dan pas passen we de andere ingang aan. Bijgevolg vormt dit geen probleem.
\paragraph{Product of sums}
In het geval van sum-of-product voegen we mintermen toe voor disjuncte $1$-gebieden. Soms is het echter goedkoper om een product-of-sum te implementeren. In dat geval voorzien we maxtermen voor disjuncte $0$-gebieden. We gaan hier niet verder op in.
\subsubsection{Realisatie leidend voorbeeld}
We zullen voor het leidend voorbeeld doorheen deze sectie (coderingstabel in \tblrefpag{asynchrone-code-exb-m3}) een schakeling realiseren. Hiervoor zullen we eerst de Karnaugh-kaarten opstellen zoals op \figref{asynchrone-real-karnaugh}. We doen dit voor zowel $s_0$, $s_1$ en $Q$. De volgende toestand hangt af van de originele toestand en van de ingangen. De variabelen zijn dus $s_0$, $s_1$, $I$ en $E$. Bij de uitgang $Q$ is enkel de toestand van belang. Bijgevolg vermelden we enkel $s_0$ en $s_1$ op de Karnaugh-kaart.
\begin{figure}[hbt]
\centering
\importtikzsubfigure{asynchrone-real-karnaugh}{Karnaugh-kaarten.}
\importtikzsubfigure{asynchrone-real-impl}{Implementatie.}
\caption{Realisatie van het leidend voorbeeld.}
\figlab{asynchrone-real}
\end{figure}
Vervolgens dienen we de logica zelf te implementeren. Hiervoor realiseren we de combinatorische schakelingen die we hebben gespecificeerd met behulp van de Karnaugh-kaarten. In dit geval doen we dit met een $3$-OR poort voor $s_0$ en een $2$-OR poort voor $s_1$. In het geval van $Q$ hebben we bovendien geen poorten nodig: de uitgang komt immers overeen met \'e\'en van de toestand-bits $s_0$. Daarna verbinden we de uitgang van de combinatorische schakelingen met de ingangen die de respectievelijke toestand-bits voorstellen. We bekomen dus een implementatie zoals op \figref{asynchrone-real-impl}.
\subsubsection{Problemen ten gevolge van skew op ingangen}
Met het voorkomen van hazards zijn de theoretische problemen opgelost bij het realiseren van een asynchrone schakeling: wanneer we een digitale schakeling simuleren zal de schakeling correct werken. Maar zoals reeds verschillende malen werd aangehaald is het theoretische model niet helemaal correct.
\paragraph{}\termen{skew} is een fenomeen waarbij een extra vertraging wordt ge\"introduceerd op een lokale plaats in de schakeling. Of anders gesteld, een verandering op een lijn wordt niet overal op hetzelfde moment opgemerkt. Verder in deze cursus zullen we een speciale vorm van skew beschouwen: clock skew\footnote{Zie \sscref{clockSkew}.}. Een mogelijk gevolg is een \termen{essenti\"ele hazard}: de verandering van slechts \'e\'en ingangssignaal brengt de schakeling in een foute toestand.
\paragraph{}
Algemeen verloopt een essenti\"ele hazard altijd als volgt:
\begin{enumerate}
 \item Het ingangssignaal wordt aangepast (door \'e\'en bit te veranderen, dus conform de regels).
 \item Enkele poorten merken de verandering op en een bit van de toestand wordt aangepast.
 \item Een poort merkt de verandering van de toestand op, maar heeft de aanpassing van de invoer nog niet opgemerkt.
 \item Deze poort verandert op zijn beurt een toestand-bit.
 \item Pas later wordt de verandering van de invoer opgemerkt.
\end{enumerate}
Empirisch zijn essenti\"ele hazards moeilijk op te sporen: het gebeurt niet zelden dat tijdelijk in de verkeerde toestand terechtkomen geen probleem vormt, de foute toestand kan immers dienst doen als een via-toestand waardoor de schakeling alsnog in de correcte toestand terecht komt. Er bestaat wel een algoritmische manier om essenti\"ele hazards te ontdekken.
\paragraph{Voorbeeld}
Bij wijze van voorbeeld beschouwen we de schakeling op \figref{asynchrone-real-impl}. We stellen op basis van de Karnaugh-kaarten een coderingstabel op in \tblref{asynchrone-real-skew-ex}.
\importtabulartable{asynchrone-real-skew-ex}{Coderingstabel van de schakeling uit \figref{asynchrone-real}.}
We zullen eerst het scenario in abstracto beschouwen:
\begin{enumerate}
 \item We bevinden ons in toestand $\tupl{s_0,s_1}=\tupl{1,0}$ met invoer $\tupl{I,E}=\tupl{0,1}$.
 \item De invoer van $I$ verandert naar $\tupl{I,E}=\tupl{1,1}$.
 \item De toestand wordt omgeschakeld naar $\tupl{s_0,s_1}=\tupl{0,0}$ maar de eerste AND poort heeft de omschakeling van de ingang nog niet opgemerkt.
 \item De AND-poort wordt hierdoor actief, bijgevolg schakelt de schakeling nu om naar toestand $\tupl{s_0,s_1}=\tupl{0,1}$.
 \item Deze toestand is stabiel met zowel de oude en de nieuwe invoer. Bijgevolg blijft de schakeling in toestand $\tupl{s_0,s_1}=\tupl{0,1}$ stabiel.
\end{enumerate}
Dit scenario treedt op wanneer op de verbinding tussen $I$ en de eerste AND-poort een significante vertraging plaatsgrijpt. Dit kan bijvoorbeeld het gevolg zijn van een lange lijn. \figref{asynchrone-real-skew-ex} toont de schakeling met de skew en \figref{asynchrone-real-skew-extime} het tijdsgedrag van de schakeling.
\begin{figure}[hbt]
\centering
\importtikzsubfigure{asynchrone-real-skew-ex}{Implementatie.}
\importtikzsubfigure{asynchrone-real-skew-extime}{Tijdsgedrag.}
\caption{Essenti\"ele hazard van het leidend voorbeeld.}
\figlab{asynchrone-real}
\end{figure}
\paragraph{Detectie}
Een eenvoudige manier om essenti\"ele hazards op te sporen is de ingang niet \'e\'en maar drie keer aanpassen. Toegepast op het voorbeeld betekent dit dat we opnieuw beginnen vanuit $\tupl{s_0,s_1,I,E}=\tupl{1,0,0,1}$. Wanneer we $I$ eenmaal aanpassen bekomen we volgens de coderingstabel: $\tupl{s_0,s_1,I,E}=\tupl{1,0,0,1}\rightarrow\tupl{1,0,1,1}\rightarrow\tupl{0,0,1,1}$. We passen vervolgens nog tweemaal de invoer aan en bekomen: $\tupl{0,0,1,1}\rightarrow\tupl{0,0,0,1}\rightarrow\tupl{0,1,0,1}\rightarrow\tupl{0,1,1,1}\rightarrow\tupl{0,1,1,1}$. Zoals we zien is de eindtoestand niet dezelfde. We hoeven het signaal nooit meer dan drie keer aan te passen.
\paragraph{Oplossing}
Een elegante oplossing voor het probleem bestaat er niet. Men moet proberen te voorkomen dat toestandsvariabelen veranderen alvorens het ingangssignaal op alle poorten is aangekomen. Door de coderingstabel aan te passen kan men het probleem meestal reduceren, bijvoorbeeld door het aantal wijzigingen van toestandsvariabelen beperkt te houden. Maar bij sommige specificaties kan men dit effect niet wegnemen. In dat geval moet men op elektronisch niveau het plan aanpassen: men kan bijvoorbeeld vertragingen introduceren bij de andere poorten om het verschil weg te nemen. Meestal vereist dit zorgvuldige en complexe bewerkingen, deze liggen buiten het bereik van deze cursus.
\subsection{Besluit}
Als algemene conclusie kunnen we stellen:
\begin{quote}
Vermijd asynchrone sequenti\"ele schakelingen.
\end{quote}
Asynchrone schakelingen zijn immers complexer te ontwerpen: twee ingangen mogen niet tegelijk aangepast worden dus dient men in sommige gevallen dit op te lossen aan de hand van extra (synchrone) logica voor de ingangen. Verder dient men verschillende problemen indachtig te zijn: de coderingstabel moet worden aangepast aan races en hazards. Tot slot wordt de meeste elektronica gebouwd aan de hand van CAD software. In deze software zijn asynchrone sequenti\"ele schakelingen meestal beperkt ondersteund.
\paragraph{}
Asynchrone schakelingen worden dan ook enkel gebruikt in twee gevallen: wanneer snelheid van cruciaal belang is en een synchrone sequenti\"ele schakeling op geen enkele manier de gewenste doorvoer kan bereiken. Verder is het niet altijd mogelijk om een sequentieel systeem te implementeren. We kunnen bijvoorbeeld denken aan een computernetwerk: elke machine heeft een eigen klok. Het synchroniseren van klokken is een onmogelijk opdracht. In dat geval zal men met behulp van een kleine en eenvoudige asynchrone schakeling data tussen de twee synchrone ``eilanden'' uit wisselen.
\part{Processoren}
\chapter{Niet-Programmeerbare Processoren}
\chplab{nonprogramming}
\chapterquote{We accepteren nu het feit dat leren een levenslang proces is om op de hoogte te blijven van veranderingen. En de meest urgente taak is mensen te leren hoe te leren.}{Peter F. Drucker, Amerikaans management consultant en auteur (1909-)}
\begin{chapterintro}
In de twee vorige hoofdstukken hebben we componenten gebouwd met een beperkte functionaliteit. De combinatorische schakelingen laten ons toe om schakelingen te ontwerpen die een rekenkundige operatie uitvoeren, maar we hebben geen geheugen beschikbaar om tussenresultaten in op te slaan. Het hoofdstuk over sequenti\"ele schakelingen maakt het mogelijk om schakelingen te ontwerpen met een geheugen. De meeste problemen hebben echter zeer grote toestandsruimtes (een 32-bit getal heeft meer dan vier miljard toestanden). Daarom volstaan de methodes uit dit hoofdstuk niet om een component te ontwikkelen die iets functioneel doet. Daarvoor zullen we methodes op een hoger niveau introduceren, dat van een niet-programmeerbare processor. Een niet programmeerbare processor voert een algoritme uit die op voorhand gekend is. Hierdoor kunnen we optimaal gebruik maken van de hardware en zoveel mogelijk instructies tegelijk uitvoeren. Het nadeel is dat eenmaal de processor geproduceerd is, we geen andere problemen met het
component kunnen uitvoeren.
\end{chapterintro}
\minitoc[n]
\section{De Niet-Programmeerbare Processor}
Alvorens we de bouw van zo'n processor verder uitwerken, dienen we eerst enkele concepten te formaliseren. Allereerst ontleden we in deze sectie uit welke delen zo'n processor is opgebouwd. Vervolgens zullen we in sectie \ref{s:descriptionFSMD} een methode ontwikkelen om een algoritme formeel weer te geven. Deze beschrijving zal toelaten het algoritme later om te zetten naar een processor. In sectie \ref{s:memoryFSMD} ten slotte zullen we extra geheugencomponenten introduceren die we nodig zullen hebben bij de bouw van een processor.
\subsection{Algemene Structuur}
\label{ss:specialProcessorGeneralStructure}
Een \termen{Niet-programmeerbare processor}, ofwel \termen{Finite State Machine with Data path (FSMD)} bestaat grofweg uit twee delen:
\begin{itemize}
 \item Een \termen{datapad}: een component die bewerkingen (rekenkundig, aritmetisch,...) uitvoert en de resultaten opslaat in tijdelijk geheugen.
 \item Een \termen{controller}: een component die het datapad aanstuurt. Het zegt welke actie op welk moment moet ondernomen worden.
\end{itemize}
In dit hoofdstuk is de controller niet programmeerbaar. Dat wil zeggen dat de controller telkens hetzelfde programma uitvoert. Dit betekent echter niet dat er een vaste cyclus in de controller zit. De controller kan afhankelijk van de waarden die in de geheugens van het datapad zitten, of van ingangen van de processor beslissen om andere acties te ondernemen. Een controller is dus een sequenti\"ele schakeling ofwel finite state machine. De synthese van een finite state machine werd in het hoofdstuk \ref{ch:SeqComp} reeds besproken. Uiteraard zullen we de karakteristieken die eigen zijn aan controllers in dit hoofdstuk bespreken.
\paragraph{}
Het spreekt voor zich dat de controller en het datapad continu data met elkaar uitwisselen. Enerzijds geeft de controller instructies aan het datapad. De groep signalen waarmee een controller een datapad aanstuurt noemen we het ``\termen{instructiewoord}'' ofwel ``\termen{controle-signalen}''. Anderzijds zullen de instructies vaak afhangen van de toestand van variabelen opgeslagen in het datapad. De verzameling van signalen die het datapad over zijn variabelen doorstuurt naar de controller noemen we ``\termen{statussignalen}''.
\paragraph{}
Een processor voert operaties uit op data. Deze data moet op de een of andere manier ingelezen worden in de processor. De verzameling ingangen waarmee we data vanuit de omgeving in het datapad injecteren noemen we de ``\termen{data-ingangen}''. Verder zullen we vaak ook informatie aan de controller moeten meedelen: we denken bijvoorbeeld aan een signaal dat actief wordt wanneer alle data ingelezen is, en het algoritme kan uitgevoerd worden. Deze signalen noemen we ``\termen{controle-ingangen}''. Daarnaast willen we ook de resultaten kunnen uitlezen. Hiervoor voorzien we een reeks signalen vanuit het datapad, deze signalen noemen we ``\termen{data-uitgangen}''. Tot slot zijn we soms ook ge\"interesseerd in de toestand van het algoritme. We zullen bijvoorbeeld enkel data uitlezen indien het algoritme afgelopen is. De controller kan informatie over het algoritme naar buiten brengen via ``\termen{controle-uitgangen}''. De verschillende informatiestromen tussen het datapad en de controller en de processor en zijn
omgeving beschrijven we op figuur \ref{fig:processorInformationStreams}.
\begin{figure}[hbt]
\centering
\subfigure[Processor]{
\begin{tikzpicture}[scale=0.8]
\draw[gray,dashed,thick] (-2.5,-3) rectangle (2.5,3);
\draw (-2.5,0) node[rotate=-90,gray,anchor=south]{Processor};
\draw (-2.5,0) node[rotate=-90,gray,anchor=north]{Omgeving};
\node[rectangle,thick,draw=black,minimum width=2 cm,minimum height=0.75 cm] (D) at (0,2) {Datapad};
\node[rectangle,thick,draw=black,minimum width=2 cm,minimum height=0.75 cm] (C) at (0,-2) {Controller};
\draw[->,thick] (D.south -| 0.3333,0) to node[midway,sloped,above,scale=0.8]{status-signalen} (C.north -| 0.3333,0);
\draw[->,thick] (C.north -| -0.3333,0) to node[midway,sloped,above,scale=0.8]{instructiewoord} (D.south -| -0.3333,0);
\draw[<-,thick] (C.west) to node[below,midway,scale=0.8]{controle-ingangen} (-4.25,-2);
\draw[->,thick] (C.east) to node[below,midway,scale=0.8]{controle-uitgangen} (4.25,-2);
\draw[<-,thick] (D.west) to node[above,midway,scale=0.8]{data-ingangen} (-4.25,2);
\draw[->,thick] (D.east) to node[above,midway,scale=0.8]{data-uitgangen} (4.25,2);
\end{tikzpicture}
\figlab{processorInformationStreams}}
\subfigure[Datapad]{
\begin{tikzpicture}[scale=0.8]
\draw[gray,dashed,thick] (-3.5,-3) rectangle (3.5,3);
\draw (-3.5,1.3) node[rotate=-90,gray,anchor=south]{Datapad};
\draw (-3.5,1.3) node[rotate=-90,gray,anchor=north]{Omgeving};
\node[rectangle,thick,draw=black,minimum width=2 cm] (T) at (0,1.25) {Tijdelijk geheugen};
\node[trapezium,thick,draw=black] (O) at (0,0) {Operatorverbindingen};
\node[rectangle,thick,draw=black,minimum width=2 cm] (F) at (0,-1.25) {Functionele eenheden};
\node[minimum width=2 cm] (RA) at (0,-2.5) {Resultaatverbindingen};
\node[minimum width=2 cm,white] (RB) at (0,2.5) {Resultaatverbindingen};
\draw[thick] (RA.north west) -- (O.bottom left corner |- RA.south) -- (RA.south east) -- ++(1,0) |- (RB.north west) -- (RB.south -| O.bottom left corner) -- (RB.south east) -- ++(0.5,0) |- (RA.north west);
\draw[thick,->] (RB) -- (T);
\draw[thick,->] (T) -- (O);
\draw[thick,->] (O) -- (F);
\draw[thick,->] (F) -- (RA);
\draw[<-,thick] (RA.west -| O.west) -- ++(-3,0);
\draw[<-,thick] (RB.west -| O.west) -- ++(-3,0);
\draw (-4,0) node[scale=0.8,anchor=south]{instructiewoord};
\draw (-4,-2.5) node[scale=0.8,anchor=south]{instructiewoord};
\draw (-4,2.5) node[scale=0.8,anchor=south]{externe-ingangen};
\draw (1.75,-3.5) node[scale=0.8,anchor=south]{externe-uitgangen};
\draw[<-,thick] (O.west) -- ++(-3,0);
\draw[->,thick] (RA) |- ++(3.5,-1);
\end{tikzpicture}
\figlab{datapadInformationStreams}}
\caption{Opbouw van een processor en datapad.}
\end{figure}
\paragraph{}
Door de controle-ingangen wordt de definitie van ``niet-programmeerbaar'' natuurlijk vaag. We zouden immers het toestandswoord van de controller in grote mate laten afhangen van de invoer die de controle-ingangen. Hierdoor kunnen we de processor toch programmeren. Het onderscheid is dan ook eerder een common-sense.
\subsection{Het Datapad}
Zoals we reeds hebben vermeld, kunnen we een controller modelleren als een eindige toestandsautomaat ofwel finite state machine. Een datapad daarentegen bestaat uit verschillende componenten:
\begin{itemize}
 \item \termen{Functionele Eenheden} ofwel \termen{Functional Units (FU)}: dit zijn schakelingen die berekeningen en aritmetisch operaties uitvoeren. Dit zijn dus de componenten die we in hoofdstuk \ref{ch:combinatoric} hebben besproken: optellers, ALU, schuifoperator,... Uiteraard kunnen we ook zelf functionele eenheden bouwen op de manier die we gezien hebben.
 \item \termen{Tijdelijke geheugens}: dit zijn componenten die de waarden waarop we bewerkingen uitvoeren voor enkele klokcycli kunnen vasthouden. Dit zijn bijvoorbeeld de registerbanken en RAM die we in sectie \ref{s:memoryFSMD} zullen invoeren. Het zijn groepen van flipflops die ons toelaten om op een hoger niveau te redeneren.
 \item \termen{Verbindingen}: de tijdelijke geheugens en de functionele eenheden wisselen informatie uit. Daarom hebben we twee types verbindingen nodig:
 \begin{itemize}
  \item \termen{Operandverbindingen}: dit zijn verbindingen die de waardes van de tijdelijke geheugens overbrengen als operanden van de functionele eenheden. De waarde van een register kan op die manier bijvoorbeeld gebruikt worden bij een optelling.
  \item \termen{Resultaatverbindingen}: het is de bedoeling dat de resultaten vervolgens in een tijdelijk geheugen opgeslagen worden. Resultaatverbindingen transporteren de resultaten van de functionele eenheden terug naar de tijdelijke geheugens. Sommige uitvoer kan ook weggeschreven worden naar de data-uitgangen. Ook de invoer van de data-ingangen wordt door deze verbindingen verwerkt.
 \end{itemize}
 Het spreekt voor zich dat de verbindingen beslissen welke geheugens als operanden en resultaatgeheugens dienen. Daarom zullen we ze implementeren als bussen met multiplexers en 3-state buffers. Deze bussen zullen dan worden aangestuurd door de controller.
\end{itemize}
Dit concept beschrijven we op figuur \ref{fig:datapadInformationStreams}.
\paragraph{}
In het datapad doen we dan ook niets anders dan waardes uit het tijdelijke geheugen inlezen, er een operatie van een functionele eenheid op uitvoeren en vervolgens in een tijdelijk geheugen plaatsen. Dit proces noemen we ook wel de ``\termen{registertransfer}'' en formaliseren we als:
\begin{equation}
\mbox{register}_a\leftarrow\mbox{FU}_a\left(\mbox{register}_{a_1},\mbox{register}_{a_2},\ldots,\mbox{register}_{a_n}\right)
\end{equation}
In het eerste hoofdstuk hebben we reeds vermeld dat we schakelingen bij het bouwen van een processor beschrijven op registertransfer-niveau. Dit betekent dat we bijvoorbeeld abstractie zullen maken van flipflops en zullen werken met registers. Ook zullen we details als het aantal bits die een opteller nodig heeft verwaarlozen. Deze nieuwe notatiestijl zullen we geleidelijk invoeren.
\section{Formeel Beschrijven van een Algoritme}
\label{s:descriptionFSMD}
Alvorens we een processor kunnen bouwen die een algoritme uitvoert, moeten we eerst een formeel algoritme kunnen opstellen. Dit algoritme vertrekt altijd vanuit een probleemstelling. Hoe we een probleemstelling omzetten naar een algoritme behoort niet tot de inhoud van deze cursus\footnote{Het omzetten van een probleem in een algoritme is geen exacte wetenschap. Het is een vaardigheid die wel geoefend kan worden. Hiervoor bestaan er andere cursussen.}. We zullen altijd stellen dat het algoritme vooraf gekend moet zijn.
\subsection{Leidend Voorbeeld: Deler}
Als leidend voorbeeld doorheen dit hoofdstuk zullen we een processor bouwen die natuurlijke getallen kan delen. Uiteraard zouden we hiervoor een combinatorische schakeling kunnen bouwen. We zullen echter een algoritme beschouwen om de berekening te maken. De processor heeft 2 4-bit ingangen die het deeltal en de deler inlezen. Verder bevat het ook een controle-ingang. Zolang we een laag signaal op de controle-ingang aanleggen betekent dit dat er geen correcte invoer op de data-ingangen staat. Pas wanneer we een hoog signaal aanleggen zal het algoritme dus een deling uitvoeren. Verder bevat de processor ook 2 4-bit uitgang om het quoti\"ent en de rest naar buiten te brengen, en een controle uitgang die hoog wordt op het moment dat het algoritme het quoti\"ent en de rest heeft berekend. Zolang de controle-uitgang dus laag is, is het algoritme nog bezig met de berekening. Verder zal de processor ook wachten totdat de controle-ingang eerst laag is geweest alvorens opnieuw te beginnen. We maken de assumptie dat
de deler nooit gelijk is aan 0. Het algoritme dat dan vervolgens het deeltal en de deler omzet in het quoti\"ent en rest staat op \algoref{alg:devisionFSMD}.
\begin{algorithm}[hbt]
\caption{Delen van twee $n$-bit getallen.}\alglab{devisionFSMD}
\begin{algorithmic}[1]
\Function{Division}{$N,D$}
\State $Q\gets 0$
\State $R\gets 0$
\For{$I=n-1\mbox{ \textbf{to} }0$}
\State $R\gets R\shlcmd{}1$\Comment{Logische shift $R$ naar links over 1 positie}
\State $R\left[0\right]\gets N\left[n-1\right]$
\State $N\gets N\shlcmd{}1$\Comment{Logische shift $N$ naar links over 1 positie}
\State $Q\gets Q\shlcmd{}1$\Comment{Logische shift $Q$ naar links over 1 positie}
\If{$R\geq D$}
\State $R\gets R-D$
\State $Q\left[0\right]\gets 1$
\EndIf
\EndFor
\State \Return $\left(Q,R\right)$
\EndFunction
\end{algorithmic}
\end{algorithm}
We gaan niet in op de precieze werking van het algoritme. Indien we een index opvragen of zetten bij een variabele zoals $V\left[i\right]$ betekent dit dat we een operatie op de $i$-de bit uitvoeren. We tellen we van rechts naar links, $V[0]$ is dus de minst beduidende bit van $V$ die rechts staat in de encodering. Verder zullen we ook de subscript-notatie gebruiken wanneer we bits samen nemen. Zo betekent $v_2v_1w_2w_2v_2w_1$ dat we een getal samenstellen uit de eerste twee bits van $V$, gevolgd door de eerste en de derde bit van $W$, daarna volgen nog de derde bit van $V$ en de tweede bit van $W$. Dit algoritme is echter niet geschikt voor een processor. Een processor voert immers continu het programma uit. Bovendien wordt hier niet gewacht tot er invoer op de data-ingangen staat. Een laatste opmerking is dat we geen berekeningen op invoer kunnen uitvoeren. Anders zouden we immers de uitgangen van functionele eenheden met de ingangen van de processor verbinden. Daarom dienen we variabelen te introduceren
die we $X$, $Y$ en $Z$. Daarom zullen we het algoritme herschrijven\footnote{Strikt genomen is dit geen algoritme meer, vermits het nooit eindigt en er geen echt resultaat is. Een betere bewoording is waarschijnlijk procedure.}. We beschouwen hierbij de controle-ingang $ci$ en de controle-uitgang $co$. De herschreven procedure staat in \algoref{alg:devisionFSMDRev}.
\begin{algorithm}[hbt]
\caption{Procedure voor het delen van twee $4$-bit getallen.}\label{alg:devisionFSMDRev}
\begin{algorithmic}[1]
\Procedure{Division}{}
\While{true}
\Repeat\label{algl:s1start}
\State $co\gets 0$
\State $X\gets N$
\State $Y\gets 0$
\State $Z\gets 0$
\Until{$ci$}\label{algl:s1stop}
\For{$I=3\mbox{ \textbf{to} }0$}\label{algl:s2start}
\State $Z\gets Z\shlcmd{}1$\Comment{Logische shift $Z$ naar links over 1 positie}
\State $Z\left[0\right]\gets X\left[3\right]$
\State $X\gets X\shlcmd{}1$\Comment{Logische shift $X$ naar links over 1 positie}
\State $Y\gets Y\shlcmd{}1$\Comment{Logische shift $Y$ naar links over 1 positie}
\If{$Z\geq D$}\label{algl:devisionFSMDComp}
\State $Z\gets Z-D$\label{algl:devisionFSMDSub}
\State $Y\left[0\right]\gets 1$
\EndIf
\EndFor\label{algl:s2stop}
\Repeat\label{algl:s3start}
\State $co\gets 1$
\State $Q\gets Y$
\State $R\gets Z$
\Until{$\neg ci$}\label{algl:s3stop}
\EndWhile
\EndProcedure
\end{algorithmic}
\end{algorithm}
\subsection{Toestandsbeschrijving}
Een eerste probleem dient zich aan hoe we de procedure omzetten in een reeks toestanden. De vraag is immers wat we in \'e\'en zo'n toestand zullen realiseren. Zo kunnen we \algoref{alg:devisionFSMDRev} uitvoeren en per instructie een nieuwe toestand bouwen. Als we echter het programma onder de loep nemen, zien we dat dit algoritme zich uitstekend leent om verschillende instructies samen uit te voeren. Allereerst voeren we in de \textbf{for}-lus hoofdzakelijke shift operaties uit. Deze shiftoperaties vinden plaats over een vast aantal posities. We dienen dus helemaal geen schuifoperator te implementeren, en kunnen eenvoudigweg schuiven met verbindingen. Vervolgens doen we ook aan bitmanipulaties. Omdat deze bitmanipulaties opnieuw op vaste plaatsen plaatsvinden kunnen we dit realiseren met behulp van verbindingen. De enige twee aspecten die enige logica vereisen is de test of $Z\geq D$ is (lijn \ref{algl:devisionFSMDComp}), en het eventueel aftrekken van $D$ uit $Z$ (lijn \ref{algl:devisionFSMDSub}).
Daarnaast moeten we tijdens de uitvoer van de \textbf{for} lus ook controleren of $I\geq 0$. In dat geval dienen we immers nogmaals de for-lus uit te voeren. We kunnen dit echter controleren met een simpele OR-poort die alle bits van $I$ samenneemt. Indien minstens \'e\'en van de bits een 1 is, zal de OR-lus een 1 teruggeven, en dienen we dus nog een cyclus uit te voeren. Dit leidt ertoe dat we ons algoritme in drie toestanden opdelen:
\begin{enumerate}
 \item $S_0$: inlezen van invoer, initialiseren van variabelen en wachten totdat $ci$ hoog wordt (lijnen \ref{algl:s1start}-\ref{algl:s1stop}).
 \item $S_1$: uitvoeren van een cyclus van de \textbf{for} lus en $I$ met 1 verlagen (lijnen \ref{algl:s2start}-\ref{algl:s2stop}).
 \item $S_2$: resultaten op de uitgang plaatsen en wachten tot $ci$ laag wordt (lijnen \ref{algl:s3start}-\ref{algl:s3stop}).
\end{enumerate}
\subsection{Toestand-Actie Tabel}
Nu we de toestanden hebben vastgesteld kunnen we het algoritme verder formaliseren. Dit zouden we kunnen doen met een grafische voorstelling zoals we gedaan hebben met een eindige toestandsautomaat. Het probleem is dat een rij in het toestandsdiagram niet enkel de voorwaarden en eventuele uitgangen bevat, daarnaast dient het ook nog de acties die door het datapad moeten worden uitgevoerd weer te geven. Dit zou leiden tot een complex en chaotisch diagram. Daarom verkiezen we een tabel: de ``\termen{Toestand-Actie Tabel}''. De tabel bestaat grofweg uit drie gedeeltes:
\begin{itemize}
 \item de huidige toestand: de huidige toestand waarin de controller zich bevindt.
 \item Een toestandsgedeelte: die we kunnen vergelijken met de toestandstabel van een eindige toestandsautomaat. De tabel bevat volgende kolommen:
 \begin{itemize}
  \item condities (afhankelijk van controle- en status-signalen)
  \item de volgende toestand
  \item de uitvoer (van eventuele controle-uitgangen).
 \end{itemize}
 \item een controle-actie gedeelte. Dit gedeelte bevat twee kolommen:
 \begin{itemize}
  \item conditie: een set voorwaarden wanneer een bepaalde set acties (gespecificeerd in de volgende kolom) moet worden uitgevoerd.
  \item actie: afhankelijk van de conditie welke opdrachten uitgevoerd worden op de variabelen in een klokcyclus.
 \end{itemize}
\end{itemize}
Tabel \ref{tbl:stateActionTableRunningExample} toont de toestand-actie tabel van \algoref{alg:devisionFSMDRev}. Een belangrijke opmerking is dat het onderverdelen van toestanden in condities bij het toestandsgedeelte niet verder loopt bij het onderverdelen van diezelfde toestand in het controle-actie gedeelte. Dit betekent dus dat de conditie bij de volgende toestand niet de conditie bij de controle- en datapad-acties impliceert en omgekeerd. Concreet betekent dit dus dat indien $I>0$, dit niet betekent dat $z_2z_1z_0x_3<D$ of dat we de bijbehorende datapad-acties moeten uitvoeren. Het toestandsgedeelte en het controle-actie gedeelte zijn dus onafhankelijk en zijn enkel afhankelijk van de huidige toestand.
\begin{table}[hbt]
\centering
\begin{tabular}{c!{\vrule width 1pt}c|c|c!{\vrule width 1pt}c|c}
Huidige&\multicolumn{2}{c|}{Volgende toestand}&Uit-&\multicolumn{2}{c}{Controle- \& datapad-acties}\\
Toestand&Conditie&Toestand&gang&Conditie&Acties\\\noalign{\hrule height 1pt}
\multirow{4}{*}{$S_0$}&\multirow{2}{*}{$ci=0$}&\multirow{2}{*}{$S_0$}&\multirow{2}{*}{$co=0$}&&$X\gets N$\\
&&&&&$Y\gets 0$\\\cline{2-4}
&\multirow{2}{*}{$ci=1$}&\multirow{2}{*}{$S_1$}&\multirow{2}{*}{$co=0$}&&$Z\gets 0$\\
&&&&&$I\gets 3$\\\noalign{\hrule height 1pt}
\multirow{8}{*}{$S_1$}&\multirow{4}{*}{$I>0$}&\multirow{4}{*}{$S_1$}&\multirow{4}{*}{$co=0$}&\multirow{4}{*}{$z_2z_1z_0x_3<D$}&$Z\gets z_2z_1z_0x_3$\\
&&&&&$X\gets X\shlcmd 1$\\
&&&&&$Y\gets Y\shlcmd 1$\\
&&&&&$I\gets I-1$\\\cline{2-6}
&\multirow{4}{*}{$I=0$}&\multirow{4}{*}{$S_2$}&\multirow{4}{*}{$co=0$}&\multirow{4}{*}{$z_2z_1z_0x_3\geq D$}&$Z\gets z_2z_1z_0x_3-D$\\
&&&&&$X\gets X\shlcmd 1$\\
&&&&&$Y\gets y_2y_1y_01$\\
&&&&&$I\gets I-1$\\\noalign{\hrule height 1pt}
\multirow{6}{*}{$S_2$}&\multirow{3}{*}{$ci=0$}&\multirow{3}{*}{$S_0$}&$co=1$&&\\
&&&$Q=Y$&&\\
&&&$R=Z$&&\\\cline{2-4}
&\multirow{3}{*}{$ci=1$}&\multirow{3}{*}{$S_2$}&$co=1$&&\\
&&&$Q=Y$&&\\
&&&$R=Z$&&\\
\end{tabular}
\caption{Toestand-actie tabel van het leidend voorbeeld.}
\label{tbl:stateActionTableRunningExample}
\end{table}
\paragraph{Simulatie}Om ons meer vertrouwd te maken met het concept van een toestand-actie tabel zullen we een deling simuleren met behulp van de tabel. We zullen $N=12$ delen door $D=7$ op de processor en stap per stap kijken wat er verandert. Dit doen we met behulp van tabel \ref{tbl:stateActionTableRunningExampleSim}. Omdat we met bitoperaties werken zullen we alle variabelen in de tabel in binaire notatie zetten.
\begin{table}[hbt]
\centering
\begin{tabular}{c|c|c|c}
Toestand&Voldane Condities&Acties&Uitvoer\\\hline
\multirow{4}{*}{$S_0$}&\multirow{4}{*}{$ci=1$}&$X\gets N=1100$&\multirow{4}{*}{$co=0$}\\
&&$Y\gets0000$&\\
&&$Z\gets0000$&\\
&&$I\gets11$&\\\hline

\multirow{4}{*}{$S_1$}&\multirow{2}{*}{$I=11>00$}&$Z\gets z_2z_1z_0x_3=0001$&\multirow{4}{*}{$co=0$}\\
&&$X\gets X\shlcmd{} 1=1000$&\\
&\multirow{2}{*}{$z_2z_1z_0x_3=0001<D=0101$}&$Y\gets Y\shlcmd{} 1=0000$&\\
&&$I\gets I-1=10$&\\\hline

\multirow{4}{*}{$S_1$}&\multirow{2}{*}{$I=10>00$}&$Z\gets z_2z_1z_0x_3=0011$&\multirow{4}{*}{$co=0$}\\
&&$X\gets X\shlcmd{} 1=0000$&\\
&\multirow{2}{*}{$z_2z_1z_0x_3=0011<D=0101$}&$Y\gets Y\shlcmd{} 1=0000$&\\
&&$I\gets I-1=01$&\\\hline

\multirow{4}{*}{$S_1$}&\multirow{2}{*}{$I=01>00$}&$Z\gets z_2z_1z_0x_3-D=0001$&\multirow{4}{*}{$co=0$}\\
&&$X\gets X\shlcmd{} 1=0000$&\\
&\multirow{2}{*}{$z_2z_1z_0x_3=0110\geq D=0101$}&$Y\gets y_2y_1y_01=0001$&\\
&&$I\gets I-1=00$&\\\hline

\multirow{4}{*}{$S_1$}&\multirow{2}{*}{$I=00$}&$Z\gets z_2z_1z_0x_3=0010$&\multirow{4}{*}{$co=0$}\\
&&$X\gets X\shlcmd{} 1=0000$&\\
&\multirow{2}{*}{$z_2z_1z_0x_3=0010<D=0101$}&$Y\gets Y\shlcmd{} 1=0010$&\\
&&$I\gets I-1=11$&\\\hline

\multirow{3}{*}{$S_2$}&\multirow{3}{*}{$ci=0$}&&$co=1$\\
&&&$Q=Y=0010$\\
&&&$R=Z=0010$\\
\end{tabular}
\caption{Simulatie van het algoritme met behulp van de toestand-actie tabel (tabel \ref{tbl:stateActionTableRunningExample}).}
\label{tbl:stateActionTableRunningExampleSim}
\end{table}
We stellen dat vanaf dat we de simulatie beginnen, reeds de data op de ingangen van de processor aangelegd staat. Bijgevolg is $ci=1$, $N=1100$ en $D=0101$. We maken verder ook een assumptie dat $X$, $Y$ en $Z$ 4-bit geheugens zijn, dit is redelijk vermits we uitsluitend bits in dit bereik gebruiken, en we geen schuifoperaties naar rechts uitvoeren waardoor hogere bits in het bereik zouden komen te liggen. $I$ is een 2-bit geheugen vermits het uitsluitend waardes tussen 0 en 3 moet aannemen. Initieel vertrekt de processor vanuit toestand $S_0$ we zien op die toestand-actie tabel dat in toestand $S_0$ de ingangen in de geheugens worden ingelezen. Vermits er data op de ingangen staat, krijgt $X$ de waarde van de teller $N=1100$. De overige variabelen worden ge\"initialiseerd zoals beschreven staat in de toestand-actie tabel. Omdat $ci=1$ kunnen we afleiden dat de volgende toestand $S_1$ is. Verder specificeert de tabel ook dat we een laag signaal op de controle-uitgang moeten aanleggen (momenteel staat
er immers geen uitkomst op de uitgangen). In de volgende stap bevinden we ons in toestand $S_1$. We evalueren eerst de verschillende condities ??.
\subsection{ASM-Schema}
Zoals reeds gezegd is de visualisatie van een eindige toestandsautomaat niet toereikend om een algoritme weer te geven. Een grafisch voorstelling die we wel kunnen gebruiken is een ``\termen{Algorithmic-State-Machine Chart}'' ofwel ``\termen{ASM-schema}''.
\subsubsection{ASM-Elementen}
Een ASM-schema lijkt op een flow-chart en bestaat drie verschillende soorten ``\termen{ASM-elementen}'':
\begin{itemize}
 \item \termen{Toestandskader} ofwel \termen{state box}: dit is een set niet-conditionele toekenning. We stellen een toestandskader voor door middel van een rechthoek waarin we de toekenningen schrijven. De toekenningen in \'e\'en toestandskader worden parallel uitgevoerd. Daarnaast bevat een toestandskader ook de status- en data-uitgangen van de processor. Men maakt een onderscheid doordat toekenningen met een pijl ($\gets$) weergegeven worden en uitgangen met een gelijkheidsteken ($=$). Figuur \ref{fig:asmElementState} toont een voorbeeld van een toestandskader.
 \item \termen{Beslissingskader} ofwel \termen{decision box}: dit is de voorstelling van een bepaalde conditie. Een conditie wordt voorgesteld met behulp van een ruit, waarin de conditie wordt geschreven. Vanuit een beslissingskader vertrekken er twee pijlen: voor het geval waarin de voorwaarde waar of vals is. We noteren de pijlen dan ook respectievelijk met ``[True]'' en ``[False]'', soms wordt ook 1 en 0 gebruikt. Figuur \ref{fig:asmElementDecision} toont een voorbeeld van een beslissingskader.
 \item \termen{Conditioneel kader} ofwel \termen{conditional box}: Dit is een toestandskader die enkel onder voorwaarden gespecificeerd door een beslissingskader worden uitgevoerd. Ook deze toekenningen worden in parallel uitgevoerd. Verder bevat een conditioneel kader ook de conditionele uitvoer op de processoruitgangen. We noteren toekenningen en uitgangen op dezelfde manier als bij toestandskaders. Men stelt een conditioneel kader voor als een rechthoek met afgeronde hoeken. Figuur \ref{fig:asmElementConditional} toont een voorbeeld van een conditioneel kader.
\end{itemize}
\begin{figure}[hbt]
\centering
\subfigure[Toestandskader]{
\begin{tikzpicture}
%\node[asmS] (S) at (0,0) {$\begin{array}{c}\mbox{niet conditioneel commando 1}\\\mbox{niet conditioneel commando 2}\\\cdots\\\mbox{niet conditioneel commando $n$}\end{array}$};
\node[asmS] (S) at (0,0) {$\begin{array}{c}\mbox{variabele$_1\gets$ expressie$_1$}\\\cdots\\\mbox{variabele$_m\gets$ expressie$_m$}\\\mbox{uitgang$_1=$ expressie$_{m+1}$}\\\cdots\\\mbox{uitgang$_n=$ expressie$_{m+n}$}\end{array}$};
\draw[<-] (S) -- ++(0,1.5) node[anchor=south]{in};
\draw[->] (S) -- ++(0,-1.5) node[anchor=north]{uit};
\end{tikzpicture}
\figlab{asmElementState}
}
\subfigure[Beslissingskader]{
\begin{tikzpicture}
\node[asmD] (D) at (0,0) {test};
\draw[<-] (D) -- ++(0,1.5) node[anchor=south]{in};
\draw[->] (D) -| ++(-1.75,-1.5) node[anchor=north]{uit 1};
\draw[->] (D) -| ++(1.75,-1.5) node[anchor=north]{uit 2};
\draw (-1.75,0) node[anchor=south west,scale=0.75]{[True]};
\draw (1.75,0) node[anchor=south east,scale=0.75]{[False]};
\end{tikzpicture}
\figlab{asmElementDecision}
}
\subfigure[Conditioneel kader]{
\begin{tikzpicture}
%\node[asmC] (C) at (0,0) {$\begin{array}{c}\mbox{conditioneel commando 1}\\\mbox{conditioneel commando 2}\\\cdots\\\mbox{conditioneel commando $n$}\end{array}$};
\node[asmC] (C) at (0,0) {$\begin{array}{c}\mbox{variabele$_1\gets$ expressie$_1$}\\\cdots\\\mbox{variabele$_m\gets$ expressie$_m$}\\\mbox{uitgang$_1=$ expressie$_{m+1}$}\\\cdots\\\mbox{uitgang$_n=$ expressie$_{m+n}$}\end{array}$};
\draw[<-] (C) -- ++(0,1.5) node[anchor=south]{in};
\draw[->] (C) -- ++(0,-1.5) node[anchor=north]{uit};
\end{tikzpicture}
\figlab{asmElementConditional}
}
\caption{Voorstelling van de verschillende ASM-elementen}
\figlab{asmElementsEnum}
\end{figure}
\subsubsection{Het ASM-Blok}
Deze ASM-elementen worden gegroepeerd in een ``\termen{ASM-blok}''. Alle ASM-elementen die in eenzelfde ASM-blok zitten, worden dan in \'e\'en klokcyclus uitgevoerd. Hierdoor voorzien we per toestand in de Toestand-Actie tabel een ASM-blok. Het blok zelf moet dan specificeren wat er in de toestand gebeurt. We zullen deze component voorstellen met behulp van een vierkant met streepjeslijnen. Vermits alle acties in \'e\'en klokflank uitgevoerd worden, bevat elk ASM-blok exact \'e\'en toestandskader. Dit toestandskader bevat dan alle toekenningen en uitgangen die onafhankelijk van condities in die toestand worden uitgevoerd. Indien er geen onafhankelijke operaties zijn, is het vierkant leeg. Indien er naast onafhankelijke acties ook onafhankelijk acties gebeuren (zowel in het toestand- als het actie-gedeelte), zullen we vervolgens enkele beslissingskaders plaatsen. We voeren testen uit op variabelen in het datapad door middel van status-signalen en eventuele signalen aan de controle-ingangen. Vermits we al deze
testen reeds in de toestand-actie-tabel hebben gedefinieerd kunnen we eenvoudig de toestand-actie-tabel omvormen tot een ASM-schema. Zo staat op figuur \ref{fig:aSMSchemaRunningExample} het ASM-schema voor het leidend voorbeeld.
\begin{figure}[hbt]
\centering
\begin{tikzpicture}
\def\ds{6};
\def\di{1.75};
\def\dy{-2};
\def\offset{1};
\node[asmS] (S1S) at (0,0) {$\begin{array}{c}X\gets N\\Y\gets 0\\Z\gets 0\\I\gets 3\\co=0\end{array}$};
\node[asmD] (S1D) at (0,\dy) {$ci=0$};
\draw[->] (S1S) -- (S1D);
\draw[->] (S1D) -| (-\di,0) -- (S1S);
\setTrueFalseLabels{S1D};

\node[asmS] (S2S) at (\ds,0) {$\begin{array}{c}X\gets X\shlcmd 1\\I\gets I-1\\co=0\end{array}$};
\node[asmD] (S2D1) at (\ds,\dy) {$z_2z_1z_0x_3<D$};
\draw[->] (S2S) -- (S2D1);
\node[asmC] (S2C1) at (\ds-\di,2*\dy) {$\begin{array}{c}Z\gets z_2z_1z_0x_3\\Y\gets Y\shlcmd{} 1\end{array}$};
\draw[->] (S2D1) -| (S2C1);
\node[asmC] (S2C2) at (\ds+\di,2*\dy) {$\begin{array}{c}Z\gets z_2z_1z_0x_3-D\\Y\gets y_2y_1y_01\end{array}$};
\draw[->] (S2D1) -| (S2C2);
\draw (S2C1) |- (\ds,2.5*\dy) -| (S2C2);
\node[asmD] (S2D2) at (\ds,3*\dy) {$I>0$};
\draw[->] (\ds,2.5*\dy) -- (S2D2);
\draw[->] (S1D) -| (\di,\offset) -| (S2S);
\draw[->] (S2D2) -| (\ds-2*\di,0) -- (S2S);
\setTrueFalseLabels{S2D1};
\setTrueFalseLabels{S2D2};

\node[asmS] (S3S) at (2*\ds,0) {$\begin{array}{c}co=1\end{array}$};
\node[asmD] (S3D) at (2*\ds,\dy) {$ci=1$};
\node[asmN] (N1) at (2*\ds,2*\dy) {$\begin{array}{c}Q=Y\\R=Z\end{array}$};
\draw[->] (S2D2) -| (\ds+2*\di,\offset) -| (S3S);
\draw[->] (S3S) -- (S3D);
\draw[->] (S3D) -| (2*\ds-\di,0) -- (S3S);
\draw[->] (S3D) -| (\di+2*\ds,1.5) -| (S1S);
\setTrueFalseLabels{S3D};
\begin{pgfonlayer}{background}
\node[asmB, fit=(S1S) (S1D)] (S1) {};
\node[asmB, fit=(S2S) (S2D1) (S2D2) (S2C1) (S2C2)] (S2) {};
\node[asmB, fit=(S3S) (S3D)] (S3) {};
\end{pgfonlayer}
\draw (S1.south west) node[anchor=south west]{$S_1$};
\draw (S2.south west) node[anchor=south west]{$S_2$};
\draw (S3.south west) node[anchor=south west]{$S_3$};
\end{tikzpicture}
\caption{ASM-schema van het leidend voorbeeld.}
\figlab{aSMSchemaRunningExample}
\end{figure}
\subsubsection{Traditionele Valkuilen}
Traditioneel maakt men een aantal fouten tegen ASM-schemas. In deze subsubsectie zullen we een overzicht geven van de meest gemaakte fouten.
\begin{figure}[hbt]
\centering
\subfigure[Meerdere volgende toestanden]{
\begin{tikzpicture}
\node[asmS] (A0) at (0,0) {$A_0$};
\node[asmD] (T0) at (-1.75,-1.25) {$t_1$};
\node[asmD] (T1) at (1.75,-1.25) {$t_2$};
\node[asmS] (A1) at (-3,-2.375) {$A_1$};
\node[asmS] (A2) at (0,-2.375) {$A_2$};
\node[asmS] (A3) at (3,-2.375) {$A_3$};
\setTrueFalseLabels{T0};
\setTrueFalseLabels{T1};
\draw[->] (A0) -- (0,-0.5) -| (T0);
\draw[->] (0,-0.5) -| (T1);
\draw[->] (T0) -| (A1);
\draw[->] (T1) -| (A3);
\draw[->] (T0) -| (A2);
\draw[<-] (A0) -- ++(0,0.75);
\draw (T1) -| (0,-1.25);
\pdot{0,-0.5};
\pdot{0,-1.25};
\begin{pgfonlayer}{background}
\node[asmB, fit=(A0) (T0) (T1)] (S1) {};
\node[asmB, fit=(A1)] (S2) {};
\node[asmB, fit=(A2)] (S3) {};
\node[asmB, fit=(A3)] (S4) {};
\end{pgfonlayer}
\draw (S1.south west) node[anchor=south west]{$S_1$};
\draw (S2.south west) node[anchor=south west]{$S_2$};
\draw (S3.south west) node[anchor=south west]{$S_3$};
\draw (S4.south west) node[anchor=south west]{$S_4$};
\end{tikzpicture}
\figlab{badASMMultipleFlows}
}
\subfigure[Geen volgende toestanden]{
\begin{tikzpicture}
\node[asmS] (A0) at (0,0) {$A_0$};
\node[asmD] (T0) at (0,-1.25) {$t_1$};
\node[asmS] (A1) at (1.25,-2.375) {$A_1$};
\setTrueFalseLabels{T0};
\draw[->] (A0) -- (T0);
\draw[->] (T0) -| (A1);
\draw[<-] (A0) -- ++(0,0.75);
\draw (T0) -| (-1.25,-0.5) -- (0,-0.5);
\pdot{0,-0.5};
\begin{pgfonlayer}{background}
\node[asmB, fit=(A0) (T0)] (S1) {};
\node[asmB, fit=(A1)] (S2) {};
\end{pgfonlayer}
\draw (S1.south west) node[anchor=south west]{$S_1$};
\draw (S2.south west) node[anchor=south west]{$S_2$};
\end{tikzpicture}
\figlab{badASMNoFlows}
}
\subfigure[Meerdere toekenningen]{
\begin{tikzpicture}
\node[asmS] (A0) at (0,0) {$X\gets 0$};
\node[asmD] (T0) at (0,-0.75) {$t_1$};
\node[asmC] (A1) at (1.25,-1.5) {$X\gets 1$};
\node[asmS] (A2) at (0,-2.5) {$A_0$};
\setTrueFalseLabels{T0};
\draw[->] (A0) -- (T0);
\draw[->] (T0) -| (A1);
\draw[<-] (A0) -- ++(0,0.75);
\draw[->] (T0) -| (-1.25,-2) -| (A2);
\draw (A1) |- (0,-2);
\pdot{0,-2};
\begin{pgfonlayer}{background}
\node[asmB, fit=(A0) (T0) (A1)] (S1) {};
\node[asmB, fit=(A2)] (S2) {};
\end{pgfonlayer}
\draw (S1.south west) node[anchor=south west]{$S_1$};
\draw (S2.south west) node[anchor=south west]{$S_2$};
\end{tikzpicture}
\figlab{badASMMultipleAssignments}
}
\subfigure[Testen op nieuwe waarden]{
\begin{tikzpicture}
\node[asmS] (A0) at (0,0.25) {$a\gets a-1$};
\node[asmD] (T0) at (0,-0.75) {$a>3$};
\node[asmC] (A1) at (-1.25,-1.5) {$b\gets 2$};
\node[asmS] (A2) at (0,-2.5) {$A_0$};
\setTrueFalseLabels{T0};
\draw[->] (A0) -- (T0);
\draw[->] (T0) -| (A1);
\draw[<-] (A0) -- ++(0,0.75);
\draw[->] (T0) -| (1.25,-2) -| (A2);
\draw (A1) |- (0,-2);
\pdot{0,-2};
\begin{pgfonlayer}{background}
\node[asmB, fit=(A0) (T0) (A1)] (S1) {};
\node[asmB, fit=(A2)] (S2) {};
\end{pgfonlayer}
\draw (S1.south west) node[anchor=south west]{$S_1$};
\draw (S2.south west) node[anchor=south west]{$S_2$};
\end{tikzpicture}
\figlab{badASMTests}
}
\subfigure[Gebruik nieuwe waarde]{
\begin{tikzpicture}
\node[asmS] (A00) at (0,0) {$y\gets x+1$};
\node[asmS] (A01) at (0,-1) {$z\gets y+2$};
\draw[->] (0,0.75) -- (A00);
\draw[->] (A00) -- (A01);
\draw[->] (A01) -- (0,-1.75);
\node (A) at (1.5,-0.5) {$\neq$};
\node[asmS] (A1) at (3,-0.5) {$\begin{array}{c}y\gets x+1\\z\gets y+2\end{array}$};
\draw[->] (3,0.75) -- (A1);
\draw[->] (A1) -- (3,-1.75);
\end{tikzpicture}
\figlab{badASMUsage}
}
\subfigure[Conditioneel na toestandskader]{
\begin{tikzpicture}
\node[asmS,minimum width=3cm] (A00) at (0,0) {$A_0$};
\node[asmC,minimum width=3cm] (A01) at (0,-1) {$A_1$};
\draw[->] (0,0.75) -- (A00);
\draw[->] (A00) -- (A01);
\draw[->] (A01) -- (0,-1.75);
\end{tikzpicture}
\figlab{badCAfterS}
}
\subfigure[Toewijzen uitgang]{
\begin{tikzpicture}
\node[asmS] (A1) at (3,-0.5) {$\mbox{uitgang}\gets\mbox{expressie}$};
\draw[->] (3,0.75) -- (A1);
\draw[->] (A1) -- (3,-1.75);
\end{tikzpicture}
}
\caption{Traditionele valkuilen bij het maken van ASM-schema's.}
\figlab{badASM}
\end{figure}
\paragraph{Meerdere volgende toestanden}Men kan in een ASM-schema een flow chart tekenen waarbij onder bepaalde condities, men twee verschillende pijlen kan volgen. Een voorbeeld van zo'n flow chart staat op figuur \ref{fig:badASMMultipleFlows}. Indien bijvoorbeeld test $t_1$ slaagt en $t2$ faalt, dienen we de stromen naar $A_1$ \'en $A_3$ te volgen. Dit is niet zo problematisch wanneer dit in hetzelfde ASM-blok gebeurd (we kunnen argumenteren dat we dan alle toewijzingen uit $A_1$ en $A_3$ uitvoeren). Indien we echter later naar verschillende toestanden gaan krijgen we problemen. We kunnen dit probleem makkelijk verhelpen door geen vertakkingen met pijlen toe te staan. Enkel uit het beslissingskader vertrekken twee pijlen. Uit een toestands- en conditioneel kader vertrekt altijd slechts \'e\'en pijl. Het samenbrengen van pijlen is wel toegelaten.
\paragraph{Geen volgende toestand}Ook het omgekeerde kan voorkomen: een ASM-blok waarbij we geen volgende toestand bekomen bij een bepaalde situatie. Figuur \ref{fig:badASMNoFlows} toont een minimaal voorbeeld: indien aan $t_1$ wordt voldaan zullen we nooit naar een volgende toestand overgaan. Dit komt omdat de pijl nooit een toestandskader bereikt (en we dus in een volgende toestand komen. Ook dit probleem kunnen we eenvoudig voorkomen: in elk ASM-blok gaan alle mogelijke lussen doorheen het toestandskader.
\paragraph{Verschillende toekenningen aan dezelfde variabele} Tijdens \'e\'en klokflank kan een variabele slechts \'e\'enmaal van waarde veranderen. We kunnen in een ASM-blok echter meerdere kaders plaatsen die elk een waarde aan dezelfde variabele toekennen. Figuur \ref{fig:badASMMultipleAssignments} toont zo'n situatie: indien $t_1$ niet waar is, kennen we zowel $0$ als $1$ toe aan $X$. Men kan argumenteren dat $X$ dan de waarde $1$ krijgt, omdat dit de laatste toekenning is aan $X$ in het diagram. De componenten die in een ASM-blok staan kunnen dus vrij verandert worden in volgorde. Bovendien zullen we in sectie \ref{s:syntheseFSMD} een mechanisme ontwikkelen om deze ASM-schema's om te zetten in hardware. Incorrecte ASM-schema's zullen leiden tot implementatiefouten. We kunnen dit voorkomen door bij elk ASM-blok alle mogelijke paden te analyseren en te controleren dat geen variabele twee toewijzingen krijgt.
\paragraph{Testen van nieuwe waarde} Omdat alles in een ASM-blok tegelijk gebeurt zijn de waardes van variabelen ook nog niet aangepast wanneer we een toestandsblok verlaten. Zolang we ons echter nog in hetzelfde ASM-blok bevinden, zijn die aanpassingen nog niet doorgevoerd. Stel dat we bijvoorbeeld volgende \texttt{C} programma beschouwen:
\begin{verbatim}
a--;
if(a > 3) {
  b = 2;
}
\end{verbatim}
Dan kunnen we dit vertalen naar \'e\'en toestand in het ASM-schema. Figuur \ref{fig:badASMTests} is echter niet de juiste vertaling. Stel immers dat $a=4$ dan zal in het \texttt{C}-programma de \texttt{if}-lus niet uitgevoerd worden, $a$ heeft immers voor het \texttt{if}-statement de waarde $3$. In het ASM-schema krijgt $a$ ook de waarde $3$, maar alleen nadat we het ASM-blok verlaten hebben. Bijgevolg zal bij de voorwaarde $a$ nog steeds de waarde $4$ hebben. En zal de \texttt{if}-lus uitgevoerd worden. Een oplossing is om in dit geval gewoon te testen op $a-1>3$ of dus $a>4$.
\paragraph{Gebruiken van een nieuwe waarde} Een verwante traditionele fout is het gebruiken van de nieuwe waarde in de volgende berekening. Een voorbeeld van dit concept staat op figuur \ref{fig:badASMUsage}. Hier zien we twee ASM-schema's die niet equivalent zijn. Indien bijvoorbeeld $x=2$ en $y=1$ zal in de eerste flow $z=5$. In het tweede geval is $z=3$. Indien beide kaders in een verschillend ASM-blok of -element staan, is dit uiteraard toegelaten. Indien de toekenningen in hetzelfde ASM-element of ASM-blok staan, worden de opdrachten parallel uitgevoerd, en zullen we dus de oude waarde gebruiken. Men kan dit fenomeen testen door de volgorde van toekenningen in een ASM-element te wijzigen of de beslissingskaders en hun bijbehorende conditionele kaders anders te schikken. Nadat deze volgorde dan wijzigt, zou het programma nog steeds op dezelfde manier moeten werken.
\paragraph{Aanduiden van controller-uitgangen}We zijn reeds kort ingegaan op de notatie van uitgangen in de toestand- en conditionele kaders. We noteren de waarde van een uitgang met behulp van een gelijkheidsteken ($=$). Indien we de uitgang niet vermelden, staat er een 0 op die uitgang (in het geval van meerdere bits, zijn alle bits dus 0). Soms komt het ook voor dat een uitgang in elke toestand een combinatorische schakeling van enkele variabelen is. In dat geval moeten we deze uitgang niet in elk toestandskader vermelden, maar volstaat het om een nota te maken, zoals we ook op figuur \ref{fig:aSMSchemaRunningExample}. Deze nota is geen onderdeel van het ASM-schema, en wordt makkelijk vergeten.
\paragraph{Conditioneel kader na toestandskader}Een laatste fout die regelmatig terugkomt is het plaatsen van een conditioneel kader na een toestandskader zoals op figuur \ref{fig:badCAfterS}. Vermits er geen beslissingskader aan vooraf gaat, is dit conditioneel kader helemaal niet gebonden aan een voorwaarde. Dit probleem lossen we op door het conditioneel kader om te vormen tot een toestandskader. Indien beide kaders bovendien in eenzelfde ASM-blok staan, kunnen we de inhoud van beide kaders samennemen in \'e\'en toestandskader.
\subsubsection{Inputgebaseerde en Toestandsgebaseerde ASM-schema's}
We hebben het reeds kort gehad over het toekennen van toestanden aan delen van een programma besproken. Een belangrijk aspect daarbij is dat we ASM-schema's kunnen onderverdelen in twee categorie\"en:
\begin{itemize}
 \item \termen{Inputgebaseerd ASM-schema}: In dit schema kunnen we de waarde van testen (status-signale) en controle-ingangen onmiddellijk gebruiken. Een inputgebaseerd ASM-schema van het leidend voorbeeld stond op figuur \ref{fig:aSMSchemaRunningExample}.
 \item \termen{Toestandsgebaseerd ASM-schema}: Hierbij kunnen we de waardes van testen (status-signalen) en controle-ingangen pas in de volgende klokflank gebruiken. Het betekent dus dat elke voorwaardelijke uitvoering van een opdracht gepaard gaat met de overgang naar een nieuwe toestand. Bijgevolg bevat dit diagram ook geen conditionele kaders.
\end{itemize}
Inputgebaseerde ASM-schema's kunnen meer opdrachten uitvoeren in een klokflank, omdat we niet moeten wachten op het kloksignaal om conditionele operaties uit te voeren. Anderzijds zal dit ASM-schema tot een langere klokcyclus leiden. Dit komt omdat de testen eerst moeten berekend worden alvorens we sommige opdrachten kunnen uitvoeren. Een nadeel van toestandsgebaseerde ASM-schema's is dat we meer toestanden nodig hebben, wat zal leiden tot een groter geheugen en mogelijk ook logica.
\paragraph{}
De termen inputgebaseerd en toestandsgebaseerd komen van de controller. We hebben reeds besproken dat een controller een eindige toestandsautomaat is. Ook bij deze eindige toestandsautomaten hebben we deze indeling gemaakt. Een toestandsgebaseerd ASM-schema zal aanleiding geven tot een toestandsgebaseerde controller en vice versa. Bij wijze van voorbeeld zullen we het leidend voorbeeld ook met een toestandsgebaseerd ASM-schema visualiseren op figuur \ref{fig:aSMSchemaRunningExampleState}.
\begin{figure}[hbt]
\centering
\begin{tikzpicture}
\def\ds{6};
\def\di{1.75};
\def\dy{-2};
\def\offset{1};
\node[asmS] (S1S) at (0,0) {$\begin{array}{c}X\gets N\\Y\gets 0\\Z\gets 0\\I\gets 3\\co=0\end{array}$};
\node[asmD] (S1D) at (0,\dy) {$ci=0$};
\draw[->] (S1S) -- (S1D);
\draw[->] (S1D) -| (-\di,0) -- (S1S);
\setTrueFalseLabels{S1D};

\node[asmS] (S2S) at (\ds,0) {$\begin{array}{c}X\gets X\shlcmd 1\\co=0\end{array}$};
\node[asmD] (S2D1) at (\ds,\dy) {$z_2z_1z_0x_3<D$};
\draw[->] (S2S) -- (S2D1);
\node[asmS] (S4S) at (\ds-\di,2*\dy) {$\begin{array}{c}Z\gets z_2z_1z_0x_3\\I\gets I-1\\Y\gets Y\shlcmd{} 1\end{array}$};
\draw[->] (S2D1) -| (S4S);
\node[asmS] (S5S) at (\ds+\di,2*\dy) {$\begin{array}{c}Z\gets z_2z_1z_0x_3-D\\I\gets I-1\\Y\gets y_2y_1y_01\end{array}$};
\draw[->] (S2D1) -| (S5S);
\node[asmD] (S4D) at (\ds-\di,3*\dy) {$I>0$};
\draw[->] (S4S) -| (S4D);
\node[asmD] (S5D) at (\ds+\di,3*\dy) {$I>0$};
\draw[->] (S5S) -| (S5D);
\draw[->] (S1D) -| (\di,\offset) -| (S2S);
\draw[->] (S4D) -| (\ds-2*\di,0) -- (S2S);
\setTrueFalseLabels{S2D1};
\setTrueFalseLabels{S4D};
\setTrueFalseLabels{S5D};

\node[asmS] (S3S) at (2*\ds,0) {$\begin{array}{c}co=1\end{array}$};
\node[asmD] (S3D) at (2*\ds,\dy) {$ci=1$};
\node[asmN] (N1) at (2*\ds,2*\dy) {$\begin{array}{c}Q=Y\\R=Z\end{array}$};
\draw[->] (S4D) -| (\ds-0.375*\di,3.75*\dy) -| (\ds+2*\di,\offset) -| (S3S);
\draw (S5D) -| (\ds+0.375*\di,3.5*\dy) -| (\ds-2*\di,0 |- S4D);
\draw (S5D) -- (S5D -| \ds+2*\di,0);
\pdot{S5D -| \ds+2*\di,0}
\pdot{\ds-2*\di,0 |- S4D}
\draw[->] (S3S) -- (S3D);
\draw[->] (S3D) -| (2*\ds-\di,0) -- (S3S);
\draw[->] (S3D) -| (\di+2*\ds,1.5) -| (S1S);
\setTrueFalseLabels{S3D};
\begin{pgfonlayer}{background}
\node[asmB, fit=(S1S) (S1D)] (S1) {};
\node[asmB, fit=(S2S) (S2D1)] (S2) {};
\node[asmB, fit=(S3S) (S3D)] (S3) {};
\node[asmB, fit=(S4S) (S4D)] (S4) {};
\node[asmB, fit=(S5S) (S5D)] (S5) {};
\end{pgfonlayer}
\draw (S1.south west) node[anchor=south west]{$S_1$};
\draw (S2.south west) node[anchor=south west]{$S_2$};
\draw (S3.south west) node[anchor=south west]{$S_3$};
\draw (S5.south west) node[anchor=south west]{$S_4$};
\draw (S4.south west) node[anchor=south west]{$S_5$};
\end{tikzpicture}
\caption{Toestandsgebaseerd ASM-schema van het leidend voorbeeld.}
\figlab{aSMSchemaRunningExampleState}
\end{figure}
\section{Geheugencomponenten}
\label{s:memoryFSMD}
Alvorens we processoren kunnen implementeren zullen we eerst nieuwe componenten moeten introduceren. We zullen we deze componenten introduceren in een logische volgorde waarbij componenten gebruik maken van eerder ge\"introduceerde componenten.
\subsection{Register File Cell (RFC)}
Een \termen{register file cell} is een uitbreiding op een geklokte D-flipflop. Het component bevat een klok-ingang $\mbox{Clk}$, data-ingangen $D_{\mbox{\small{in}}1},\ldots,D_{\mbox{\small{in}}m}$, data-uitgangen $D_{\mbox{\small{out}}1},\ldots,D_{\mbox{\small{out}}n}$, \termen{leespoorten $\mbox{RE}_1,\mbox{RE}_{n}$} (ook wel ``\termen{Read-Enabled}'' genoemd) en \termen{schrijfpoorten $\mbox{WE}_1,\ldots,\mbox{WE}_{\left\lceil\log_2m+1\right\rceil}$} (ook wel ``\termen{Write-Enabled}'' genoemd). Deze ingangen laten ons toe om te kiezen uit welke data-ingang we data willen inlezen en deze bij de klokflank willen opslaan. We zullen data geklokt wegschrijven, net zoals bij een D-flipflop. Indien alle schrijfpoorten $\mbox{WE}_i=0$ lezen we geen nieuwe waarde in, en blijft de oude waarde behouden. In de andere gevallen dienen de schrijfpoorten een binaire getal $a$ te bepalen vanuit welke data-ingang $d_{\mbox{\small{in}}a}$ we data inlezen. Indien er voor het aantal data-ingangen $m$ geen natuurlijk getal
$l$ bestaat zodat $l=\log_2 m+1$, zal er bij alle overige write-enable configuraties, data ingelezen worden uit de laatste data-ingang. Daarnaast kunnen we op eender welke uitgang de inhoud van het geheugen plaatsen. Vandaar dat er per uitgang ook een read-enable ingang is voorzien. Indien we een laag signaal aanleggen op een read-enable ingang $\mbox{RE}_i$, zal de overeenkomstige uitgang $D_{\mbox{\small{out}}i}$ hoog impedant zijn. De toestand van de uitgangen is niet geklokt: indien we tijdens twee klokflanken in een read-enable ingang aanpassen zullen we mits enige vertraging het resultaat op de data-uitgang zien, we hoeven dus niet op een klokflank te wachten. Op figuur \ref{fig:registerFileCell} tonen we een implementatie van een Register File Cell met $m=n=2$.
\begin{figure}[hbt]
\centering
\begin{tikzpicture}
\node[dff] (D) at (0,0) {};
\node[mux4to1,rotate=90] (M) at (-1.75,0 |- D.D) {};
\draw (M.output) -- (D.D);
\node[tris] (T1) at (1.75,0 |- D.Q) {};
\node[tris] (T2) at (2.5,-0.25) {};
\draw (D.Q) -- (T1.x);
\draw (D.Q -| 1.25,0) |- (T2.x);
\draw (D.Q -| 1.25,0) |- (-2.25,1.5) |- (M.data0);
\draw (M.data3) -| (M.data2 -| -2.25,0);
\draw (M.data1) -- ++(-1,0) node[anchor=east,scale=0.85]{$D_{\mbox{\small{in}}1}$};
\draw (M.data2) -- ++(-1,0) node[anchor=east,scale=0.85]{$D_{\mbox{\small{in}}2}$};
\draw (D.Clk) -| (-1,-2) node[anchor=north,scale=0.85]{Clk};
\draw (T1.z) -- (T1.z -| 3.5,0) node[anchor=west,scale=0.85]{$D_{\mbox{\small{out}}1}$};
\draw (T2.z) -- (T2.z -| 3.5,0) node[anchor=west,scale=0.85]{$D_{\mbox{\small{out}}2}$};
\draw (T1.c) -- (T1.c |- 0,-1.75) node[anchor=north,scale=0.85]{$\mbox{RE}_1$};
\draw (T2.c) -- (T2.c |- 0,-1.75) node[anchor=north,scale=0.85]{$\mbox{RE}_2$};
\draw (M.selout1) -- (M.selout1 |- 0,2) node[anchor=south,scale=0.85]{$\mbox{WE}_2$};
\draw (M.selout0) -| (-1.125,2) node[anchor=south,scale=0.85]{$\mbox{WE}_1$};
\pdot{D.Q -| 1.25,0};
\pdot{M.data2 -| -2.25,0};
\setIndexLabelsMuxD[west]{M};
\node[rectangle,draw=black,dashed,inner sep=0.3cm, fit=(M) (D) (T1) (T2)] (S1) {};
\end{tikzpicture}
\caption{Implementatie van een Register File Cell (RFC) met $2$ lees- en $2$ schrijfpoorten}
\figlab{registerFileCell}
\end{figure}
\subsection{Registerbank}
Een belangrijke toepassing van een Register File Cell is een \termen{registerbank}. Een registerbank bevat verschillende register file cellen, die geordend worden in matrixstructuur. We spreken dan ook over een \termen{$m\times n$ registerbank} met $k$ \termen{schrijfpoorten} en $l$ \termen{leespoorten}. Dit betekent dat de component $m$ sequenties van $n$ bits opslaat. We kunnen hierbij data op $k$ verschillende sequenties tegelijk schrijven, en de inhoud van $l$ verschillende sequenties uitlezen. Hiervoor dienen we volgende in- en uitgangen te voorzien:
\begin{itemize}
 \item \termen{invoer-ingangen $I_{ij}$}: een set van $k\times n$ ingangen om $k$ sequenties van $n$ bits te kunnen inlezen in de registerbank.
 \item \termen{write-enable-ingangen $\mbox{WE}_i$}: $k$ verschillende signalen waarmee we aangeven of de invoer op $I_{ij}$ ingangen moet worden ingelezen.
 \item \termen{write-address-ingangen $\mbox{WA}_{ia}$}: $k$ groepen van $\lceil\log_2m\rceil$ bits waarmee we aangeven op welk adres we de $n$ bits die op $I_{ij}$ staan zullen wegschrijven.
 \item \termen{uitvoer-uitgangen $O_{ij}$}: een reeks van $l\times n$ signalen die we gebruiken om data in de registerbank uit te lezen.
 \item \termen{read-enable-ingangen $\mbox{RE}_i$}: $l$ signalen die bepalen of we op de uitgangen $O_{ij}$ iets zullen uitlezen. Analoog aan de write-enable-ingangen.
 \item \termen{read-address-ingangen $\mbox{RA}_{ia}$}: $l$ groepen van $\left\lceil\log_2m\right\rceil$ bits bepalen welke sequentie -- binair voorgesteld op de adres-ingangen -- wordt uitgelezen. Dit is analoog aan de write-address-ingangen.
\end{itemize}
Een registerbank omvat twee scenarios: het inlezen van data en het uitlezen van data. Indien we een hoog signaal aanleggen op $\mbox{WE}_i$, zullen we de data die op de ingangen $I_{ij}$ staan wegschrijven naar het adres dat binair ge\"encodeerd is met de write-access ingangen $\mbox{WA}_{ia}$ voor $j=0\ldots n-1$ en $a=0\ldots\left\lceil\log_2m\right\rceil-1$. Dit doen we op de klokflank. Indien we een laag signaal aan de write-enable ingang plaatsen, wordt de inhoud die op de bijbehorende invoer-ingangen staat genegeerd. Bij het uitlezen van data is het signaal van de read-enable-ingang van belang. Indien we een hoog signaal aanleggen op $\mbox{RE}_i$ zullen de uitvoer-uitgangen $O_{ij}$ de waardes van de data opgeslagen in een adres, binair gevormd door de read-address-ingangen $\mbox{WA}_{ia}$, aannemen voor $j=0\ldots n-1$ en $a=0\ldots\left\lceil\log_2m\right\rceil-1$. Indien we een laag signaal aanleggen, zijn deze uitvoer-uitgangen hoog impedant. Uitlezen van data gebeurt ongeklokt: indien we
bijvoorbeeld de read-adress signalen aanpassen, zullen de uitvoer-uitgangen zich aanpassen ongeacht de toestand van de kok op dat moment. Tot slot beschouwen we vaak een speciaal geval van een registerbank: de \termen{dual port registerbank} is een registerbank met \'e\'en lees- en \'e\'en schrijfpoort. Bijgevolg is in dat geval $k=l=1$.
\begin{figure}[hbt]
\centering
\begin{tikzpicture}
\def\dy{-2.2 cm};
\def\dx{2.2 cm};
\def\ra{0.0725};
\def\rb{0.03625};
\foreach \a in {0,1} {
  \node[decoder2to4,rotate=90] (DW\a) at (-1.25*\dx,0.5*\dy+2*\a*\dy) {decoder};
%  \node[sigo] (SgW\a) at (-1.5*\dx,0 |- DW\a) {};
  \draw (DW\a.a0) -- ++(-0.5*\dx,0) node[anchor=east]{$\mbox{WA}_{\a0}$};
  \draw (DW\a.a1) -- ++(-0.5*\dx,0) node[anchor=east]{$\mbox{WA}_{\a1}$};
  \draw (DW\a.enable) |- ++(-0.5*\dx,\ra*\dy) node[anchor=east]{$\mbox{WE}_{\a}$};
}
\foreach \a in {0,1,2} {
  \node[decoder2to4,rotate=-90] (DR\a) at (3.5*\dx,0.5*\dy+\dy*\a) {decoder};
  \draw (DR\a.a0) -- ++(0.5*\dx,0) node[anchor=west]{$\mbox{RA}_{\a0}$};
  \draw (DR\a.a1) -- ++(0.5*\dx,0) node[anchor=west]{$\mbox{RA}_{\a1}$};
  \draw (DR\a.enable) |- ++(0.5*\dx,-\ra*\dy) node[anchor=west]{$\mbox{RE}_{\a}$};
}
\foreach \y in {0,...,3} {
  \foreach \x in {0,1,2} {
    \node[rfcbc] (RFC\y\x) at (\dx*\x,\dy*\y) {RFC};
  }
}
\foreach \x in {0,...,3} {
  \coordinate (MI\x0) at (DW0.s\x -| -0.5*\dx-4*\ra*\dx-\ra*\dx*\x,0);
  \coordinate (MI\x1) at (DW1.s\x -| -0.5*\dx-3*\ra*\dx+\ra*\dx*\x,0);
  \foreach \y in {0,1} {
    \draw (DW\y.s\x) -- (MI\x\y);
  }
  \foreach \y in {0,1,2} {
    \coordinate (MO\x\y) at (DR\y.s\x -| 2.5*\dx+\rb*\dx+4*\ra*\dx*\y+\ra*\dx*\x,0);
    \draw (DR\y.s\x) -- (MO\x\y);
  }
}
\foreach \y in {0,...,3} {
  \foreach \x/\p in {0/a,1/b,2/c} {
    \coordinate (RLH\y\x) at (0,\dy*\y+0.5*\dy-\rb*\dy-\ra*\dy*\x -| RFC\y2.re\p);
    \draw (RLH\y\x) -- (RLH\y\x -| RFC\y0.re\p);
    \foreach \z in {0,1,2} {
      \draw (RFC\y\z.re\p) -- (RLH\y\x -| RFC\y\z.re\p);
    }
    \foreach \z in {1,2} {
      \pdot{RLH\y\x -| RFC\y\z.re\p};
    }
  }
  \foreach \x/\p in {0/a,1/b} {
      \coordinate (WLH\y\x) at (0,\dy*\y-0.5*\dy+\rb*\dy+\ra*\dy+\ra*\dy*\x -| RFC\y0.we\p);
      \draw (WLH\y\x) -- (WLH\y\x -| RFC\y2.we\p);
      \foreach \z in {0,1,2} {
	\draw (RFC\y\z.we\p) -- (WLH\y\x -| RFC\y\z.we\p);
      }
      \foreach \z in {0,1} {
	\pdot{WLH\y\x -| RFC\y\z.we\p};
      }
    }
}
\foreach \x in {0,1,2} {
  \foreach \y/\p/\an in {0/a/west,1/b/north,2/c/east} {
    \coordinate (RLV\x\y) at (\dx*\x+0.5*\dx-\rb*\dx-\ra*\dx*\y,0 |- RFC0\x.dout\p);
    \draw (RLV\x\y) -- (RLV\x\y |- 0,3.75*\dy) node[anchor=\an,scale=0.95]{$O_{\y\x}$};

    \foreach \z in {0,1,2,3} {
      \draw (RLV\x\y |- RFC\z\x.dout\p) -- (RFC\z\x.dout\p);
    }
    \foreach \z in {1,2,3} {
      \pdot{RLV\x\y |- RFC\z\x.dout\p};
    }
  }
  \foreach \y/\p/\an in {0/a/east,1/b/west} {
    \coordinate (WLV\x\y) at (\dx*\x-0.5*\dx+\rb*\dx+\ra*\dx+\ra*\dx*\y,0 |- RFC3\x.din\p);
    \draw (WLV\x\y) -- (WLV\x\y |- 0,-0.75*\dy) node[anchor=\an,scale=0.95]{$I_{\y\x}$};
    \foreach \z in {0,1,2,3} {
      \draw (WLV\x\y |- RFC\z\x.din\p) -- (RFC\z\x.din\p);
    }
    \foreach \z in {0,1,2} {
      \pdot{WLV\x\y |- RFC\z\x.din\p};
    }
  }
}
\foreach \y in {0,...,3} {
  \foreach \x in {0,1} {
    \draw (MI\y\x) |- (WLH\y\x);
  }
}
\foreach \y in {0,...,3} {
  \foreach \x in {0,1,2} {
    \draw (MO\y\x) |- (RLH\y\x);
  }
}
\node[fit=(DR0) (DR1) (DR2) (DW0) (DW1) (RFC00) (RFC32),inner sep=0.625cm,draw=black,rectangle,dashed] {};
\end{tikzpicture}
\caption{Implementatie van een $4\times 3$ registerbank met $2$ schrijf- en $3$ leespoorten.??}
\figlab{registerbank}
\end{figure}
\paragraph{}
Op figuur \ref{fig:registerbank} beschouwen we een $4\times 3$ registerbank met $2$ schrijf- en $3$ leespoorten. Bij de constructie van een registerbank met $k$ schrijf- en $l$ leespoorten, gebruiken we logischerwijs file register cellen met $k$ schrijf- en $l$ leespoorten. We zullen hier het kloksignaal negeren vermits de klokingang van de registerbank het kloksignaal enkel verderpropageert naar de klokingangen van alle register file cellen. We construeren vervolgens $k+l$ $m$-bit decoders te introduceren. Bij elk van de decoders verbinden we een read-enable $\mbox{RE}_i$ of write-enable $\mbox{WE}_i$ met de enable-ingang van de decoder. De read-address $\mbox{RA}_{ij}$ en write-address $\mbox{WA}_{ij}$ ingangen leggen vervolgens signalen aan op de adres-ingangen van de bijbehorende decoders.??
\paragraph{}
Verder zullen we ook een nieuwe notatie invoeren die we vanaf hier frequent zullen gebruiken: vaak zullen een groot aantal signalen parallell verschillende bits van het ene component naar het andere overbrengen. Vermits door de ori\"entatie van de de component meestal duidelijk is om welke signalen het gaat, zullen we niet elk signaal individueel tekenen. In dat geval stellen we de groep signalen voor met een brede lijn, en schrijven naast een dwarse streep het aantal signalen op die deze lijn voorstelt.
\subsection{Random Access Memory (RAM)}
Een variant van een registerbank is ``\termen{Random Access Memory (RAM)}''. De term ``\termen{Random Access}'' slaat op het feit dat we het geheugen niet sequentieel moeten uitlezen. We kunnen dus een adres meegeven die bepaald welke cellen we uitlezen. Verder wijkt een RAM ook af van ``Read-Only Memory (ROM)'' omdat we data naar het geheugen kunnen schrijven. Bij ROM branden we de data via een ingewikkelde procedure op de chip, waarna we enkel data kunnen uitlezen. Beide eigenschappen zijn ook eigen aan een registerbank. Indien we echter naar de implementatie van een registerbank kijken, is deze niet goedkoop. RAM geheugens bieden een gelijkaardige functionaliteit met minder hardware. Hiervoor bestaan er twee soorten implementaties:
\begin{itemize}
 \item \termen{Statisch RAM}: hier realiseren we een bit geheugen met een flipflop (wat dus neerkomt op 4 tot 6 transistoren per bit).
 \item \termen{Dynamisch RAM}: een implementatie met behulp van een condensator. Indien er stroom op de condensator staat bevat de cel een 1, in het ander geval een 0. Dynamisch RAM heeft de eigenschap dat door de data van een bit op te vragen, we de stroom uit de condensator halen, en deze dus opnieuw moeten opladen. Daarnaast kent een condensator ook lekstroom, waardoor we aan een zekere frequentie de cellen die een 1 voorstellen terug moeten opladen.
\end{itemize}
Vermits we minder hardware per bit nodig hebben, zal RAM bepaalde functionaliteit van een registerbank niet aanbieden. Allereerst werkt RAM geheugen trager. Dit kan alleen al verklaard worden door een grotere adres-decoder - RAM geheugens zijn immers groter - en is dus niet geschikt voor het opslaan van tussenresultaten in processoren. Daarnaast heeft RAM-geheugen een \termen{gecombineerde lees-schrijfpoort $R/W^*$}. Om aan te geven dat we iets willen schrijven of uitlezen bevat RAM-geheugen daarnaast ook een \termen{Chip Select-ingang $CS$}. Deze ingang functioneert ook als een vorm van klok-ingang. RAM geheugen is bijgevolg niet geklokt\footnote{Of tenminste niet op de globale klok van de schakeling.}.
\paragraph{}
RAM-geheugens hebben complex tijdsgedrag. Daarom zullen we de twee scenario's: het uitlezen en wegschrijven van data bespreken samen met de verschillende vormen van vertraging. Verder zullen we ook enkele typische grenzen van vertragingstijden voor vergelijkbare RAM-geheugen geven in tabel \ref{tbl:rAMDelaySamples}.
\paragraph{Uitlezen van data}Hiervoor dienen we het adres op de adres-ingang aan te leggen en een hoog signaal op de Chip Select-ingang $\mbox{CS}$ en de lees-schrijfpoort $\mbox{R/W}^*$. Op het moment dat we dit doen, zal de data-uitgang\footnote{Meestal heeft RAM-geheugen geen enkele data-uitgang maar een reeks uitgangen waardoor we bijvoorbeeld 8 bit tegelijk kunnen uitlezen.} $\mbox{D}_{\mbox{\small{out}}}$ hoog impedant zijn. Na enige tijd zal er op deze uitgang een ongeldig signaal komen te staan dit wordt meestal veroorzaakt door de interne logica van de component. Vervolgens komt de werkelijke data op de data-uitgangen te staan. Kenmerkend zijn hier de \termen{toegangstijd $t_{\mbox{\small{AA}}}$} en de \termen{$\mbox{CS}$ toegangstijd $t_{\mbox{\small{ACS}}}$}. Het zijn de maximale tijdsverschillen tussen het aanleggen van het signaal en het verschijnen van de geldige data op de data-uitgangen. Stel dat de lees-schrijfpoort en de chip-select ingang al hoog aangestuurd worden, en we zetten een adres
op de adres-ingang zal het hoogstens de tijd gespecificeerd door de toegangstijd duren alvorens er geldige data op data-uitgang komt te staan. Omgekeerd kan er ook een geldig adres op de adres-ingang staan en dienen we nog een hoog signaal op de $\mbox{CS}$- en $\mbox{R/W}^*$ aan te leggen. In dat geval duurt het hoogstens de $\mbox{CS}$ toegangstijd alvorens geldige data op de data-uitgang verschijnt. Eenmaal er geldige data op de data-uitgang verschijnt kunnen we deze uitlezen en ergens in de schakeling gebruiken. Om nieuwe data uit te lezen of weg te schrijven zullen we ofwel het adres moeten veranderen, ofwel de een laag signaal om de $\mbox{CS}$- of $\mbox{R/W}^*$-ingang aanleggen. Er treed enige vertraging op alvorens het signaal van de data-uitgang dan terugvalt op de hoog impedante toestand. Deze vertraging noemen we de \termen{\mbox{CE} Off to Output High Impedance State $t_{\mbox{\small{HZ}}}$}. Een laatste tijdseenheid is de \termen{leescyclustijd $t_{\mbox{\small{RC}}}$}. Deze vertraging bepaald
de minimale duur van het volledige leesproces. Het specificeert de minimale tijd die tussen twee leesopdrachten ligt. Vaak is deze bij DRAM langer dan de som van alle vertragingen die we hierboven beschouwd hebben. Dit komt omdat we na het uitlezen van de data, de condensatoren terug moeten opladen. Een schematische voorstelling van deze tijden staat op figuur \ref{fig:timeRAMRead}.
\begin{figure}[hbt]
\centering
\subfigure[Uitlezen]{
\begin{tikzpicture}
\begin{wave}[0.5]{3}{5.71428}
 \nextwave{Adres} \knownS{}{0.6857} \known{Adres}{4} \knownE{}{1.02857}
 \nextwave[0.5]{$\mbox{CS}\cdot \mbox{R/W}^*$} \bitS{0}{0.6857} \bit{1}{4} \bitE{0}{1.02857}
 \nextwave[1.0]{$\mbox{D}_{\mbox{\small{out}}}$} \unknownbit{2.3571428} \unknown{0.5} \known{geldig}{2.3428572} \unknownbit{0.51428}
 \waveTime{1}{1}{1}{2}{0.5}{$t_{\mbox{\small{RC}}}$}
 \waveTime{1}{1}{3}{2}{-1}{$t_{\mbox{\small{AA}}}$}
 \waveTime{2}{1}{3}{2}{-2.5}{$t_{\mbox{\small{ACS}}}$}
 \waveTime{2}{2}{3}{3}{-2.5}{$t_{\mbox{\small{HZ}}}$}
\end{wave}
\end{tikzpicture}
\figlab{timeRAMRead}}
\centering
\subfigure[Wegschrijven]{
\begin{tikzpicture}
\begin{wave}[0.5]{3}{5.71428}
 \nextwave{Adres} \knownS{}{0.6857} \known{Adres}{4} \knownE{}{1.02857}
 \nextwave[0.5]{$\mbox{CS}\cdot\left(\mbox{R/W}^*\right)'$} \bitS{0}{0.6857} \bit{1}{4} \bitE{0}{1.02857}
 \nextwave[1.0]{$\mbox{D}_{\mbox{\small{in}}}$} \knownS{-}{2.8571428} \known{geldig}{2.3428572} \knownE{-}{0.51428}
 \waveTime{1}{1}{1}{2}{0.5}{$t_{\mbox{\small{WC}}}$}
 \waveTime{2}{2}{3}{1}{-2.5}{$t_{\mbox{\small{DW}}}$}
 \waveTime{2}{2}{3}{2}{-2.5}{$t_{\mbox{\small{DH}}}$}
\end{wave}
\end{tikzpicture}
\figlab{timeRAMWrite}}
\caption{Tijdsgedrag van RAM-geheugens.}
\end{figure}
\paragraph{Wegschrijven van data}
Een gelijkaardig scenario treed op bij het wegschrijven van data. Alleen beschouwen we hier de data-ingang\footnote{Vermits er bij het inlezen op data-uitgang een hoog-impedant signaal wordt aangelegd, combineren sommige RAM-geheugens de data-ingang en de data-uitgang. In dat geval kunnen de signalen dus in beide richtingen stromen. We gaan hier niet verder op in.} $\mbox{D}_{\mbox{\small{in}}}$. Opnieuw dienen we eerst het adres aan te leggen om de adres ingang. Verder zetten we een 0 op de lees-schrijfpoort om aan te geven dat we een schrijf-operatie uitvoeren, en zetten we een 1 op de Chip Select-ingang om aan te geven dat we een operatie zullen uitvoeren. Vervolgens kunnen we data op de data-ingang aanleggen. Wanneer we de correcte data wegschrijven is in principe irrelevant. Vanaf het moment dat we de operatie starten, zal het geheugencomponenten beginnen met data weg te schrijven. Als we data aan de data-ingang veranderen zal met enige vertraging de nieuwe data weggeschreven worden. We dienen alleen de
tijdkarakteristieken van geheugencomponenten in het algemeen in de gaten te houden: de \termen{set-up-tijd $t_{\mbox{\small{DW}}}$} en de \termen{data houdtijd $t_{\mbox{\small{DH}}}$}. Kort voor het afronden van de operatie dient immers de data niet meer te veranderen aan de ingang van de data-ingang. Dit is de set-up tijd. Anderzijds dient de data bovendien nog een zekere periode na het afronden van de operatie te blijven staan: de houdtijd. Tot slot spreken we ook over een \termen{schrijf cyclustijd $t_{\mbox{\small{WC}}}$}. Dit is de minimale tijd dat het adres op de adres-ingang moet blijven staan en de chip select-ingang en lees-schrijfpoort dezelfde configuratie blijven behouden. Een schematische voorstelling van een schrijfoperatie staat op figuur \ref{fig:timeRAMWrite}. Typische vertragingen voor een $4\mbox{k}\times1$-RAM\footnote{Dit betekent dat het geheugencomponent 4096 adressen bevat en elk adres 1 bit bijhoudt.} geheugen voor SRAM en DRAM staan in tabel \ref{tbl:rAMDelaySamples}.
\begin{table}[hbt]
\centering
\begin{tabular}{cl|rr|rr}
\multirow{2}{*}{$\Delta t$}&\multirow{2}{*}{Vertraging}&\multicolumn{2}{c|}{2147H SRAM}&\multicolumn{2}{c}{MM5280 DRAM}\\
&&min&max&min&max\\
\hline
$t_{RC}$&cyclustijd lezen&35 ns&&400 ns&\\
$t_{AA}$&toegangstijd&&35 ns&&200 ns\\
$t_{ACS}$&$CS$ toegangstijd&&35 ns&&180 ns\\
$t_{HZ}$&$CS'\rightarrow Z$&0 ns&30 ns&0 ns&\\\hline
$t_{WC}$&cyclustijd schrijven&35 ns&&0 ns&\\
$t_{DW}$&data set-up-tijd&20 ns&&150 ns&\\
$t_{DH}$&data houdtijd&0 ns&&0 ns&\\
\end{tabular}
\caption{Typische vertragingstijden voor RAM-geheugens.}
\label{tbl:rAMDelaySamples}
\end{table}
\subsection{Geheugens met Impliciete Adressering}
RAM geheugens vereisen dat we een absoluut adres meegeven. In heel wat programmeertalen hebben we echter datastructuren beschikbaar die zonder expliciete adressen werken. We spreken dan over een \termen{stapelgeheugen} ofwel \termen{stack} en een \termen{buffergeheugen} ofwel \termen{queue}. Omdat deze datastructuren bijzonder nuttig zijn in sommige algoritmen, werden hiervoor ook hardware-equivalenten ontwikkeld. We zullen in de volgende subsubsecties eerst deze datastructuren bespreken en vervolgens een hardwarecomponent bouwen die deze structuren implementeert. Studenten met een uitgebreide kennis over deze datastructuren kunnen de definitie overslaan.
\subsubsection{Stack (LIFO: Last-In-First-Out)}\ssclab{stack}
\paragraph{Definitie}Een stack ofwel stapelgeheugen is een datastructuur die een lineare lijst voorstelt. Alle methodes van de stapel kunnen enkel bewerkingen uitvoeren op de \'e\'en uiteinde van deze lijst: de \termen{top}. De lijst van aangeboden functionaliteiten verschilt nogal. Toch dient een stapel minstens volgende functionaliteiten aan te bieden:
\begin{itemize}
 \item \termen{push}: het toevoegen van een element aan de stapel. Dit element wordt dan de nieuwe top.
 \item \termen{pop}: het weghalen van het element die zich bij de top bevindt. Het element net onder de top wordt dan de nieuwe top.
 \item \termen{reset}: ook wel clear genoemd. Een operatie die de volledige stapel verwijdert. We kunnen dit opvatten als het herhalen van een pop-operatie totdat de stapel leeg is, in de meeste gevallen kunnen we deze operatie echter sneller uitvoeren.
 \item \termen{leeg/empty}: controleren of een stapel leeg is en dus geen enkel element bevat. Indien dit het geval is, kan men geen pop-operaties meer uitvoeren.
 \item \termen{vol/full}: controleren of het volledige geheugen van de stapel gebruikt wordt. Indien dit het geval is, kan men geen push-operaties meer uitvoeren. Deze functionaliteit wordt meestal niet in software-implementaties van stapels aangeboden vermits men het geheugen dynamisch kan uitbreiden tot het volledige RAM geheugen in gebruik genomen is. Een fenomeen die men doorgaans niet beschouwd in software.
\end{itemize}
In hardware is een element een hoeveelheid data met een vast aantal bits. Men spreekt over het ``\termen{Last-In-First-Out (LIFO)}''-principe omdat de elementen die het laatst aan een stapel toegevoegd worden (door middel van een push-operatie), de eerste elementen zijn die terug uit de stapel gehaald worden (door middel van een pop-operatie).
\paragraph{Conceptueel voorbeeld}
\begin{figure}[hbt]
\centering
\subfigure[Push 6]{
\begin{stackConcept}{}\scpush{6}\end{stackConcept}
}
\subfigure[Push 2]{
\begin{stackConcept}{6}\scpush{2}\end{stackConcept}
}
\subfigure[Pop]{
\begin{stackConcept}{6,2}\scpop{}\end{stackConcept}
}
\subfigure[Push 1]{
\begin{stackConcept}{6}\scpush{1}\end{stackConcept}
}
\subfigure[Pop]{
\begin{stackConcept}{6,1}\scpop{}\end{stackConcept}
}
\subfigure[Pop]{
\begin{stackConcept}{6}\scpop{}\end{stackConcept}
}
\caption{Conceptueel voorbeeld van een stapelgeheugen.}
\figlab{stackConceptExample}
\end{figure}
We introduceren ook een conceptueel voorbeeld van een stapelgeheugen dat staat op figuur \ref{fig:stackConceptExample}. Initieel is de stapel leeg en bevat deze dus geen elementen. Vervolgens voeren we een push-operatie uit met als argument 6. Het gevolg is een stapel met als enig element 6. Daarna voeren we een push operatie uit met 2. We voegen dus 2 toe aan de top van de stapel. Hierna voeren we een pop-operatie uit. Vermits het laatste toegevoegde element -- die nog niet van de stapel is gehaald -- 2 is, zullen we 2 terugkrijgen. Bij de push-operatie van 1 voegen we 1 toe aan de stapel. De stapel heeft dan als top 1 en subtop 6. Tot slot voeren we twee pop-operaties uit. Omdat 1 het laatste toegevoegde element is, is dit het resultaat van de eerste pop-operatie. Tot slot zullen we ook 6 van de stapel halen.
\paragraph{Implementatie met een schuifregister}
We kunnen een stapel implementeren met behulp van een schuifregister. Voorwaarde is wel dat het geheugen dan niet bijzonder groot kan worden. Bij een push operatie zullen we in dat geval de bits van het element inschuiven en dit aantal bits naar links opschuiven. Bij een pop-operatie schuiven we dit aantal bits terug naar rechts, en stellen de meest rechtse bits dit element voor. Deze implementatie is vrij kostelijk: in \ref{sss:shiftregisters} hebben we reeds register ge\"implementeerd. Per bit geheugen hebben we een flipflop en enkele multiplexers nodig. Bij grote stapelgeheugens is deze kost onacceptabel. Daarom zullen we bij grote stapels meestal gebruik maken van RAM-geheugens.
\begin{figure}[hbt]
\centering
\subfigure[Push 6]{
\stackImpl{3}{3}{0}{}
}
\subfigure[Push 2]{
\stackImpl{3}{0}{1}{6}
}
\subfigure[Pop]{
\stackImpl{3}{1}{2}{6,2}
}
\subfigure[Push 1]{
\stackImpl{3}{0}{1}{6}
}
\subfigure[Pop]{
\stackImpl{3}{1}{2}{6,1}
}
\subfigure[Pop]{
\stackImpl{3}{0}{1}{6}
}
\caption{Demonstratie van een stapelgeheugen met tellers.}
\figlab{stackConceptExample}
\end{figure}
\begin{figure}[hbt]
\centering
\begin{tikzpicture}[circuit logic US]
\def\ec{-5};
\def\epp{-5.25};
\def\er{-5.625};
\def\efe{-1};
\node[counterdbitrev] (CW) at (-3,1.5) {$\begin{array}{c}n+1\mbox{-bit}\\\mbox{schrijfteller}\end{array}$};
\node[mux2to1,rotate=90,scale=1.5,anchor=data0,thick] (M) at ($(CW.Q2)+(2.75,-0.25)$) {};
\draw[thick] (CW.Q2) -- (CW.Q2 |- M.data0);
\node[counterdbitrev] (CR) at (-3,-1.75) {$\begin{array}{c}n\mbox{-bit}\\\mbox{leesteller}\end{array}$};
\node (CRV) at (CR.D2 |- 0,-0.25) {$2^n-1$};\draw[thick] (CRV) -- (CR.D2);
\draw[thick] (CR.Q2) |- ++(2.5,-0.25);
\draw[decoration={sigo,lines={$n$}},decorate,thick] ($(CR.Q2)+(2.5,-0.25)$)  |- (M.data1);
\node[ramm,anchor=A] (R) at ($(M.output)+(0.75,0)$) {$\begin{array}{c}2^n\times w\mbox{-bit}\\\mbox{RAM}\end{array}$};
\draw[decoration={sigo,lines={$n$}},decorate,thick] (M.output) -- (R.A);
\node[not gate,scale=0.75,rotate=90] (N1) at ($(R.RW)+(-0.25,-1)$) {};%($(CR.Q2)+(0,-0.875)$)
\node[not gate,scale=0.75,rotate=-90] (N2) at (\er,0) {};
\node[nor gate,inputs={normal,normal,normal}] (NO) at ($(R.north)+(0,1)$) {};
\coordinate (FEO) at (\efe,0 |- NO.input 2);
\coordinate (FEOM) at (FEO |- M.data0);
\coordinate (CRD) at ($(CR.Q2)+(0,-0.75)$);
\coordinate (MRW) at (M.selin0 |- CRD);
\pdot{MRW};
\draw (M.selin0) -- (MRW);
\draw[decoration={sigo,lines={$n+1$}},decorate,thick] (CW.Q2 |- M.data0) -- (FEOM);
\draw[decoration={sigo,lines={$n$}},decorate,thick] (FEOM) -- (M.data0);
\pdot{FEO};\pdot{FEOM};
\draw[decoration={sigo,lines={$n+1$}},decorate,thick] (FEOM) -- (FEO);
\draw[decoration={sigo,lines={$n$}},decorate,thick] (FEO) -- (NO.input 2);
\node[anchor=east,scale=0.9] (IR) at (-7,0 |- CW.CLR) {Reset$^*$};\draw (IR) -- (CW.CLR);\draw (IR -| \er,0) -- (N2.input);\draw(N2.output) |- (CR.LD);\pdot{IR -| \er,0};
\node[anchor=east,scale=0.9] (IE) at (-7,0 |- CR.CEIN) {Enable};\draw (IE) -- (CR.CEIN);\draw (IE -| \ec,0) |- (CW.CEIN);\draw (IE -| \ec,0) |- ($(CR.Q2)+(2.5,-0.5)$) -| ($(R.CS)+(-0.625,0)$) -- (R.CS);\pdot{IE -| \ec,0};
\node[anchor=east,scale=0.9] (IPP) at (-7,0 |- CR.DU) {Push/Pop$^*$};\draw (IPP) -- (CR.DU);\draw (IPP -| \epp,0) |- (CW.DU);\draw (IPP -| \epp,0) |- ($(CR.Q2)+(2.5,-0.75)$) -| (N1.input); \draw (N1.output) -| ($(R.RW)+(-0.25,0)$) -- (R.RW);\pdot{IPP -| \epp,0};
\node[anchor=west,scale=0.9] (OF) at (5,0 |- CW.north east) {Full};\draw[decoration={sigo,lines={$1$ MSB}},decorate] (FEO) |- (OF);
\node[anchor=west,scale=0.9] (OE) at (5,0 |- NO.output) {Empty};\draw (NO.output) -- (OE);
\node[anchor=west,scale=0.9] (OD) at (R.D -| 5,0) {Data In/Out};\draw[decoration={sigo,lines={$w$}},decorate,thick] (R.D) -- (OD);
\node[fit=(R) (M) (CW) (CR) (N1) (N2) (NO) (MRW),inner sep=0.625cm,draw=black,rectangle,dashed] {};
\end{tikzpicture}
\caption{Implementatie van een stapelgeheugen met behulp van RAM-geheugen.}
\figlab{stackImplRAM}
\end{figure}
\paragraph{Implementatie met RAM-geheugens}
Een goedkopere manier is gebruik maken van RAM-geheugens. Hierbij is de kost beperkt tot een 4 \`a 6 transistoren per bit. We realiseren dan een stapelgeheugen met behulp van een RAM-geheugen -- die tevens ook de grote van het stapelgeheugen bepaald -- en twee tellers: het \termen{leesadres} en het \termen{schrijfadres}. Indien het RAM-geheugen $2^n$ adressen kent, beschouwen we een $n$-bit leesteller en $n+1$-bit schrijftellers\footnote{Op die manier kunnen ze elk adres voorstellen. We voegen aan de schrijfteller een extra bit toe voor overflow.}. De schrijfteller staat initieel op $0$, de leesteller staat initeel op $2^n-1$. Wanneer we een push operatie uitvoeren schrijven we de data weg naar het adres voorgesteld door de schrijfteller, en verhogen we beide tellers. Bij een pop-operatie lezen we de data uit het RAM-geheugen op het adres voorgesteld door de leesteller. Vervolgens verlagen we beide tellers. Merk op dat we bij een push- en pop-operatie dus niet alle data in het RAM-geheugen moeten opschuiven.
We verhogen en verlagen enkel de tellers. Indien we het volledige geheugen volgeschreven hebben, zal de schrijfteller op $2^n$ komen te staan\footnote{Vermits de schrijfteller een $n+1$-bit teller is.}. Dit betekent dus dat de $n+1$-de bit van die teller een hoog signaal bevat. We kunnen dit hoog signaal naar buiten brengen als een full-signaal. Op figuur \ref{fig:stackImplRAM} implementeren we dit stapelgeheugen. Indien het reset-signaal actief wordt, wordt de schrijfteller gereset en komt deze op 0 te staan, de leesteller voert een load-operatie uit en laad de waarde $2^n-1$ in. Bij een push-operatie wordt het push/pop$^*$ signaal hoog, hierdoor wordt het increment-signaal bij beide tellers geactiveerd en worden deze opgehoogd in de volgende klokflank. De multiplexer zal intussen de huidige waarde van de schrijfteller doorsturen naar het RAM-geheugen, die de waarde die op de data-ingangen staat zal inlezen en wegschrijven. Bij een pop-operatie ontvangen beide tellers een decrement-signaal, de multiplexer
stuurt het leesadres door naar het RAM-geheugen die de waarde die op dit adres-staat op de data-uitgang zal zetten. Ons component bevat daarnaast nog een enable-ingang: indien deze op 0 staat, worden de tellers en het RAM-geheugen niet aangepast en gebeurt er dus niets. De full-uitgang brengt zoals eerder vermeld de hoogste bit van de schrijfteller naar buiten, en is hoog indien alle geheugenplaatsen opgebruikt zijn. Daarnaast bevat ons component ook een empty-uitgang. Deze voert een NOR-operatie uit op alle bits van de schrijfteller. Indien alle bits 0 zijn (en de teller dus op 0 staat) is de stapel leeg, en geeft empty dus een laag signaal. Tot slot kunnen we opmerken dat we voor de implementatie strikt genomen geen twee tellers nodig hebben. Vermits de leesteller telkens gelijk is aan de gedecrementeerde waarde schrijfteller (en modulo $2^n$), zouden we ook een combinatorische moduler kunnen voorzien die de waarde van de leesteller uit de schrijfteller afleid. In dit geval zal de vertraging van de pop-
operaties wel groter worden.
\subsubsection{Queue (FIFO: First-In-First-Out)}
\paragraph{Definitie}Een queue of buffergeheugen is een datastructuur die een lineare lijst voorstelt. Deze lijst heeft twee uiteindes. Aan het ene uiteinde voegen we elementen toe bij een schrijfoperatie, aan het andere uiteinde zullen we bij een leesoperatie elementen weghalen. Zoals de naam al doet vermoeden heeft een queue hoofdzakelijk de taak om data te bufferen in de tijd\footnote{Men dient een buffergeheugen niet te verwarren met een buffer-component. Een buffercomponent staat in voor het gelijkmatig verdelen van de spanning over verschillende uitgangen (en buffert dus het elektrisch potentiaal).}. Dit kan bijvoorbeeld nuttig zijn indien we een component soms aan hoge snelheid invoer data krijgt (zogenaamde bursts) en men deze data niet aan dezelfde snelheid kan verwerken. Een queue dient minstens volgende functionaliteiten aan te bieden:
\begin{itemize}
 \item \termen{Write} (ofwel enqueue): het inlezen van data en dit aan \'e\'en uiteinde van het buffergeheugen plaatsen.
 \item \termen{Read} (ofwel dequeue): het uitlezen van data aan het andere uiteinde van het buffergeheugen.
 \item \termen{Reset}: het buffergeheugen in de initi\"ele toestand plaatsen (waarbij het dus geen elementen bevat). Dit kan opgevat worden als het herhalen van read operaties tot alle elementen uit de queue verdwenen zijn.
 \item \termen{Full}: een indicator of alle beschikbare geheugenlocaties bezet zijn.
 \item \termen{Empty}: een indicator dat het buffergeheugen leeg is.
\end{itemize}
In hardware is een element een hoeveelheid data met een vast aantal bits. Men spreekt over het ``\termen{First-In-First-Out (LIFO)}''-principe omdat de elementen die het eerst aan een queue toegevoegd worden (door middel van een write-operatie), de eerste elementen zijn die terug uit de queue gehaald worden (door middel van een read-operatie).
\begin{figure}[hbt]
\centering
\subfigure[Write 6]{
\begin{queueConcept}{4}{}\qupush{6}\end{queueConcept}
}
\subfigure[Write 2]{
\begin{queueConcept}{4}{6}\qupush{2}\end{queueConcept}
}
\subfigure[Read]{
\begin{queueConcept}{4}{6,2}\qupop{}\end{queueConcept}
}
\subfigure[Write 1]{
\begin{queueConcept}{4}{2}\qupush{1}\end{queueConcept}
}
\subfigure[Read]{
\begin{queueConcept}{4}{2,1}\qupop{}\end{queueConcept}
}
\subfigure[Read]{
\begin{queueConcept}{4}{1}\qupop{}\end{queueConcept}
}
\caption{Conceptueel voorbeeld van een buffergeheugen.}
\figlab{queueConceptExample}
\end{figure}
\paragraph{Conceptueel voorbeeld}
Analoog aan het stapelgeheugen introduceren we een conceptueel voorbeeld voor een buffergeheugen op figuur \ref{fig:queueConceptExample}. We voeren dezelfde operaties uit als bij het stapelgeheugen (push is write en pop is read). Toch zien we dat de elementen op een andere manier uit de datastructuur verdwijnen. Aanvankelijk is de queue leeg. Vervolgens plaatsen we met write operaties een 6 en een 2 in de queue. Bij de read operatie zullen we in tegenstelling tot bij een stapel, de 6 uit het geheugen halen.
\begin{figure}[hbt]
\centering
\subfigure[Push 6]{
\queueImpl{2}{0}{0}{}
}
\subfigure[Push 2]{
\queueImpl{2}{0}{1}{6}
}
\subfigure[Pop]{
\queueImpl{2}{0}{2}{6,2}
}
\subfigure[Push 1]{
\queueImpl{2}{1}{2}{6}
}
\subfigure[Pop]{
\queueImpl{2}{1}{3}{6,1}
}
\subfigure[Pop]{
\queueImpl{2}{2}{3}{6}
}
\caption{Demonstratie van een buffergeheugen met tellers.}
\figlab{queueConceptExample}
\end{figure}
\paragraph{Implementatie met RAM-geheugens}
Analoog kunnen we ook een buffergeheugen implementeren met behulp van een RAM-geheugen en een lees- en schrijfteller. De schrijfteller houdt de positie bij van het uiteinde van de rij, waar we elementen zullen bijschrijven. De leesteller houdt de positie van het andere uiteinde bij, waar we elementen zullen uitlezen. We gebruiken het RAM-geheugen op een manier waarbij we de elementen niet dienen te verplaatsen. Dit kunnen we doen door zowel de lees als de schrijfteller aanvankelijk op 0 te zetten. Indien we een schrijf-operatie dienen uit te voeren, schrijven we het elementen op het adres gespecificeerd door de schrijfteller, en incrementeren we deze teller. Bij een lees-operatie lezen we de locatie van het geheugen uit en incrementeren we de leesteller. Indien \'e\'en van de tellers aan het einde van het geheugen komt, voeren we eenvoudigweg een wrap-around uit, en schrijven we dus opnieuw op het begin van het geheugen. Een probleem stelt zich op het moment dat de lees- en schrijftellers beide dezelfde
waarde hebben. In dat geval zijn er immers twee situaties mogelijk: ofwel zijn op dat moment alle geheugencellen bezet, ofwel is op dat moment het buffergeheugen leeg (zoals bijvoorbeeld bij de initi\"ele toestand van de tellers). We verhelpen dit probleem opnieuw door bij een $2^n\times w$-bit RAM geheugen, de tellers uit te breiden naar $n+1$-bit tellers. De hoogste bit wordt niet gebruikt om het adres voor het RAM-geheugen aan te wijzen, we beschouwen voor adressering dus enkel de laagste $n$ bits. Indien deze $n$ laagste bits van de lees- en schrijfteller aan elkaar gelijk zijn, speelt zich opnieuw het scenario van vol of leeg af. Indien daarenboven de hoogste bits aan elkaar gelijk zijn, is de queue leeg, indien ze niet gelijk zijn is het buffergeheugen vol. Het conceptueel principe van deze tellers staat op figuur \ref{fig:queueConceptExample}. Op figuur \ref{fig:queueImplRAM} implementeren we een schakeling die een buffergeheugen realiseert. Vermits de tellers in \'e\'en richting tellen, makken we
gebruik van uptellers. Indien de enable-inang hoog is, zal afhankelijk van het signaal aan de \mbox{\termen{read/write$^*$-ingang}} \'e\'en van de tellers een increment uitvoeren. Verder bepaald deze ingang ook, welke teller de multiplexer naar het RAM-geheugen doorstuurd. De multiplexer stuurt wel enkel de laagste $n$ bits door. De enable ingang zal verder ook het RAM-geheugen activeren. Indien de reset-ingang geactiveerd is, zullen beide tellers terug de initi\"ele waarde 0 krijgen. Tot slot implementeren we de indicators zoals eerder besproken: een buffergeheugen is leeg indien alle $n+1$ bits van de twee tellers gelijk zijn. Een buffergeheugen is vol indien op de hoogste bits na de tellers gelijk zijn.
\begin{figure}[hbt]
\centering
\begin{tikzpicture}[circuit logic US]
\def\el{-6.75};
\def\ela{5.25};
\def\ec{-5.125};
\def\epp{-6.125};
\def\epq{-6.375};
\def\er{-5};
\def\efe{-1.25};
\def\eff{2.75};
\def\af{3.5};
\def\afa{3.25};
\def\afb{3};
\node[upcounterdbit,minimum width=3cm] (CW) at (-3,1.5) {$\begin{array}{c}n+1\mbox{-bit}\\\mbox{schrijfteller}\end{array}$};
\node[mux2to1,rotate=90,scale=1.5,anchor=data0,thick] (M) at ($(CW.Q2)+(3,-0.5)$) {};\setIndexLabelsMuxB[west]{M};
\draw[thick] (CW.Q2) -- (CW.Q2 |- M.data0);
\node[upcounterdbit,minimum width=3cm] (CR) at (-3,-1.75) {$\begin{array}{c}n+1\mbox{-bit}\\\mbox{leesteller}\end{array}$};
\draw[thick] (CR.Q2) |- ++(\eff,-0.25);
\draw[decoration={sigo,lines={$n$}},decorate,thick] ($(CR.Q2)+(\eff,-0.25)$)  |- (M.data1);
\node[ramm,anchor=A] (R) at ($(M.output)+(0.75,0)$) {$\begin{array}{c}2^n\times w\mbox{-bit}\\\mbox{RAM}\end{array}$};
\draw[decoration={sigo,lines={$n$}},decorate,thick] (M.output) -- (R.A);
\node[and gate,anchor=output] (AR) at (CR.CEIN -| \ec,0) {};\draw (AR.output) -- (CR.CEIN);
\node[and gate,inputs={inverted,normal},anchor=output] (AW) at (CW.CEIN -| \ec,0) {};\draw (AW.output) -- (CW.CEIN);
\node[comp,rotate=90,thick,anchor=north,xshift=-0.125cm] (CO) at (R.west |- CW.CEIN) {Comp};
\node[xor gate,anchor=south,yshift=0.25cm] (X) at (CO.east) {};
\node[and gate,anchor=input 1] (A0) at (X -| \af,0) {};\draw (A0.output) -- (A0.output -| \ela,0) node[anchor=west]{Full};
\node[and gate,inputs={inverted,normal},anchor=input 2] (A1) at (CO -| \af,0) {};\draw (A1.output) -- (A1.output -| \ela,0) node[anchor=west]{Empty};
\draw (X.output -| \afa,0) |- (A1.input 1);\pdot{X.output -| \afa,0};\draw (CO.eq -| \afb,0) |- (A0.input 2);\pdot{CO.eq -| \afb,0};
\coordinate (FEO) at (\efe,0 |- CR.CEIN);
\coordinate (FEOM) at (FEO |- M.data0);\coordinate (FEOC) at (FEO |- CO.y1);\draw[decoration={sigo,lines={$n+1$}},decorate,thick] (FEOM) -- (FEOC);\pdot{FEOC};\draw[decoration={sigo,lines={$n+1$}},decorate,thick] (FEOC) -- (CO.y1);\draw (FEOC) |- (X.input 1);
\coordinate (FFO) at ($(CR.Q2)+(\eff,-0.25)$);
\coordinate (FFOM) at (FFO |- M.data1);\coordinate (FFOC) at (FFO |- CO.x1);\draw[decoration={sigo,lines={$n+1$}},decorate,thick] (FFOM) -- (FFOC);\pdot{FFOC};\draw[decoration={sigo,lines={$n+1$}},decorate,thick] (FFOC) -- (CO.x1);\draw (FFOC) |- (X.input 2);
\coordinate (CRD) at ($(CR.Q2)+(0,-0.75)$);
\coordinate (MRW) at (M.selin0 |- CRD);
\pdot{MRW};\pdot{FFOM};
\draw (M.selin0) -- (MRW);
\draw[decoration={sigo,lines={$n+1$}},decorate,thick] (CW.Q2 |- M.data0) -- (FEOM);
\draw[decoration={sigo,lines={$n$}},decorate,thick] (FEOM) -- (M.data0);
\pdot{FEOM};
\node[anchor=east,scale=0.9] (IR) at (\el,0 |- CW.CLR) {Reset$^*$};\draw (IR) -- (CW.CLR);\draw (IR -| \er,0) |- (CR.CLR);\pdot{IR -| \er,0};
\node[anchor=east,scale=0.9] (IE) at (\el,0 |- AR.input 2) {Enable};\draw (IE) -- (AR.input 2);\draw (IE -| \epq,0) |- (AW.input 2);\draw (IE -| \epq,0) |- ($(CR.Q2)+(2.5,-0.5)$) -| ($(R.CS)+(-0.625,0)$) -- (R.CS);\pdot{IE -| \epq,0};
\draw (X.output) -- (A0.input 1);\draw (CO.eq) -- (A1.input 2);
\node[anchor=east,scale=0.9] (IPP) at (\el,0 |- AW.input 1) {Read/Write$^*$};\draw (IPP) -- (AW.input 1);\draw (IPP -| \epp,0) |- (AR.input 1);\pdot{IPP -| \epp,0};\draw (AR.input 1 -| \epp,0) |- (MRW) -| ($(R.RW)+(-0.25,0)$) -- (R.RW);\pdot{AR.input 1 -| \epp,0};
\node[anchor=west,scale=0.9] (OD) at (R.D -| \ela,0) {Data In/Out};\draw[decoration={sigo,lines={$w$}},decorate,thick] (R.D) -- (OD);
\node[fit=(R) (M) (CW) (CR) (CW) (AR) (AW) (MRW),inner sep=0.625cm,draw=black,rectangle,dashed] {};
\end{tikzpicture}
\caption{Implementatie van een buffergeheugen met behulp van RAM-geheugen.}
\figlab{queueImplRAM}
\end{figure}
\section{Synthese van een Niet-Programmeerbare Processor}
\label{s:syntheseFSMD}
Nu we een ASM-schema gedefinieerd hebben en nieuwe geheugencomponenten ter beschikkingen hebben kunnen we eindelijk een processor implementeren. We vertrekken vanuit het ASM-schema, en zullen in subsectie \ref{ss:syntheseFSMDBasis} eerst een methode bespreken om op een mechanische manier een ASM-schema om te zetten in een processor. Deze methode leidt tot een straightforward resultaat, maar dit is verre van optimaal. In subsecties \ref{ss:syntheseFSMDController}, \ref{ss:syntheseFSMDDatapad} en \ref{ss:syntheseFSMDOptimal} zullen we dan ook optimalisaties bespreken om de processor goedkoper en effici\"enter te maken.
\subsection{Basisprincipes}
\label{ss:syntheseFSMDBasis}
\subsubsection{Principes}
We kunnen door volgende principes toe te passen op het ASM-schema een processor bouwen:
\begin{enumerate}
 \item Elke variabele is een register.
 \item Voor elk register maakt de set-ingang deel uit van het instructiewoord.
 \item Elke operatie komt overeen met een functionele eenheid (FU) die logischerwijs dezelfde opdracht uitvoert (indien er dus een optelling in het ASM-schema voorkomt, zullen we een opteller in het datapad plaatsen).
 \item Indien een operatie de waarde van een variabele nodig heeft, maken we een verbinding van de uitgang van het bijbehorende register naar de bijbehorende ingang van functionele eenheid.
 \item Indien het resultaat van een operatie wordt weggeschreven naar een variabele, bouwen we een verbinding van de uitgang van de bijbehorende functionele eenheid naar de data-ingang van het register.
 \item Indien data-ingangen of constanten ingelezen kunnen worden bij een variabele, bouwen we een verbinding tussen de data-ingang/constante en het bijbehorende register, tenzij de constante 0 is.
 \item Indien een variabele op 0 gezet kan worden, bevat het instructiewoord de reset-ingang van het bijbehorende register.
 \item Indien verschillende functionele eenheden waardes wegschrijven naar eenzelfde register, voorzien we een multiplexer om de waarde te kiezen. De selectie-ingangen van deze multiplexers behoren tot het instructiewoord.
 \item Elk ASM-blok komt overeen met een toestand van de controller.
 \item Voor elke test waarbij een variabele betrokken is, dienen we een combinatorische module te bouwen die op basis van de waarde van de registers de test kan uitvoeren. De uitgang van deze test module is \'e\'en van het status-signalen, en bijgevolg een deel van de controller-ingangen.
 \item De reset- en load-ingangen van de registers vormen samen met de selectie-ingangen de het instructiewoord: een deel van de uitgangen van de controller.
 \item Data-uitgangen worden berekend aan de hand van een combinatorische schakeling met als invoer de waarde van de relevante registers, en eventuele instructiewoord-bits.
\end{enumerate}
\subsubsection{Leidend voorbeeld}
\paragraph{Datapad}
Op basis van het ASM-schema op figuur \ref{fig:aSMSchemaRunningExample} op pagina \pageref{fig:aSMSchemaRunningExample} kunnen we een processor bouwen. Het resultaat van van deze omzetting staat op figuur \ref{fig:sprocessorGenericLeadingExample}. In het ASM-schema is er sprake van vier variabelen: $X$, $Y$, $Z$ en $I$. Daarom introduceren we vier registers. Uit het ASM-schema kunnen we afleiden dan zowel $Y$ als $Z$ op 0 gezet kunnen worden. Daarom voorzien we op basis van de register reeds 6 instructie-bits: de set-ingangen van de $X$ en $I$-registers en de set- en reset-ingangen van de $Y$ en $Z$ registers. Voor elk van de registers voorzien we vervolgens een lijn (deze tekenen we onder de registers). We kunnen nu de functionele eenheden introduceren. Alle operaties uit het ASM-schema gebeuren in toestand $S_2$. In totaal zijn er 6 functionele eenheden: een shift voor zowel $X$ als $Y$ samen met de speciale recombinaties en decrement. We maken geen verschil of de instructies al dan niet voorwaardelijk
zijn. Elk van deze functionele eenheden plaatsen we onder de lijnen van de registers. Vervolgens tekenen we verbindingen vanuit de registers en eventuele data-ingangen naar de functionele eenheden. Zo tekenen we een signaal vanuit $I$ naar de decrementor, $X$ en $Y$ naar een shift-operator, en andere registers naar hun specifieke recombinaties. Verder tekenen we ook de testen: $I>0$ en $z_2z_1z_0x_3<D$. Ook deze testen verbinden we met de respectievelijk registers en data-ingangen. Elk van deze testen vormen een deel van het status-signaal. We duiden deze bits aan met onderlijnde grote letters. Vervolgens dienen we de uitgang van de functionele eenheden samen met data-ingangen te verbinden met de ingang van de registers. Zo zien we in het ASM-schema dat de variable $X$ vanuit twee bronnen waardes kan krijgen: vanuit de data-ingang $N$ en vanuit een schuifregister $X\shlcmd1$. Bijgevolg voorzien we een multiplexer die zowel de data-ingang $N$ als de uitgang van een schuifoperator combineert. De selectie-
ingang van deze multiplexer maakt deel uit van het instructiewoord. Bits van het instructiewoord duiden we aan met onderlijnde kleine letters\footnote{De voorstelling van bits van het toestand- en instructiewoord zijn louter arbitrair, andere notaties zijn eveneens toegelaten.}. $Y$ kan worden gezet op drie soorten waardes: de waarde 0, het resultaat van een andere schuifoperator en de waarde na de recombinatiestap. Vermits we 0 kunnen activeren door een hoog signaal op de reset-ingang van het register aan te leggen, zullen we 0 niet toevoegen aan de multiplexer, en combineert de multiplexer alleen de uitvoer van de schuifoperator met deze van recombinatie-functionele eenheid. Op een gelijkaardige manier realiseren we de rest van de multiplexers. Tot slot dienen we nog combinatorische logica voor de data-uitgangen te bouwen. In dit geval is het simpel: de notitie bij het ASM-schema stelt dat de data-uitgangen $Q$ en $R$ louter de waardes van $Y$ en $Z$ naar buiten brengen. We specificeren dus deze uitgangen
en verbinden de bussen van de registers met deze data-uitgangen. We hebben nu het volledige datapad gerealiseerd zoals op figuur \ref{fig:sprocessorDatapathGenericLeadingExample}.
\begin{figure}[hbt]
\centering
\begin{sprocessor}{x/$X$/1/2,y/$Y$/3/2,z/$Z$/3/2,i/$I$/1/2}{sa/{$\shlcmd 1$},sb/{$\shlcmd 1$},rcc/$y_2y_1y_01$,rca/$z_2z_1z_0x_3$,rcb/{$z_2z_1z_0x_3$\\$-D$},de/$-1$}{ta/{$z_2z_1z_0x_3$\\$<D$},tb/{$I>0$}}
\node[anchor=south] (VN) at ($(RRx-1)+(0,0.5)$) {$N$};
\node[anchor=south] (V3) at (RRi-1 |- VN.south) {$3$};
\spcorf{VDC}{rcb}{-0.3};
\node[anchor=south] (VD) at (VDC |- VN.south) {$D$};
\draw[thick] (RRx-1) -- (VN.south);
\draw[thick] (RRi-1) -- (V3.south);
\draw[thick] (VDC) -- (VD.south);
\draw[thick] (FURde) -| (RRi-0);
\draw[thick] (FURsa) -| (RRx-0);
\draw[thick] (FURrcb) -| (RRz-0);
\draw[thick] (FURrca) -| (RRz-1);
\draw[thick] (FURrcc) -| (RRy-0);
\draw[thick] (FURsb) -| (RRy-1);
\spcort{TAD}{ta}{0.3};
\spcort{OQB}{tb}{0.75};\coordinate (OQ) at (OQB |- 0,-4);
\spcort{ORB}{ta}{0.75};\coordinate (OR) at (ORB |- 0,-4);
\draw (OQ) node[anchor=north]{$Q$};\sprbtc{y}{OQ};
\draw (OR) node[anchor=north]{$R$};\sprbtc{z}{OR};
\draw[thick] (TAD) |- ($(VD.south)+(0,-0.3)$);
\pdot{$(VD.south)+(0,-0.3)$};
\sprbtf{i}{de}{0};
\sprbtf{z}{rcb}{0.3};
\sprbtf{x}{rcb}{0};
\sprbtf{z}{rca}{0.2};
\sprbtf{x}{rca}{-0.2};
\sprbtf{y}{rcc}{0};
\sprbtf{y}{sb}{0};
\sprbtf{x}{sa}{0};
\sprbtt{x}{ta}{-0.3};
\sprbtt{z}{ta}{0};
\sprbtt{i}{tb}{0};
\end{sprocessor}
\caption{Implementatie van het datapad van het leidend voorbeeld via de basisprincipes.}
\figlab{sprocessorDatapathGenericLeadingExample}
\end{figure}
\paragraph{Controller}
Naast het datapad moeten we ook de controller realiseren. De controller bouwen we op basis van het ASM-schema. Elke toestand van het ASM-schema komt overeen met een toestand van de controller. De invoer van de controller bestaat enerzijds uit controle-ingangen en anderzijds status-signalen. Voor het voorbeeld van de deler betekent dit dus:
\begin{enumerate}
 \item [\underline{ci}] het signaal van de controle-ingang dat aangeeft dat de data-ingangen gereed zijn voor verwerking.
 \item [\underline{A}] het resultaat van de test $z_2z_1z_0x_3<D$
 \item [\underline{B}] het resultaat van de test $I>0$
\end{enumerate}
Daarnaast dient de controller zowel een instructiewoord naar
\begin{enumerate}
 \item[\underline{co}] het signaal van de controle-uitgang dat aangeeft dat het algoritme was uitgevoerd en de resultaten op de data-uitgangen staat.
 \item[\underline{a}] indien hoog wordt de waarde van de variabele $X$ gezet op de waarde van de multiplexer.
 \item[\underline{b}] indien hoog wordt de waarde $N$ doorgegeven aan het register $X$, anders wordt $X\shlcmd1$ doorgegeven.
 \item[\underline{c}] indien hoog wordt de waarde van de variabele $Y$ gezet op de waarde van de multiplexer.
 \item[\underline{d}] indien hoog wordt de waarde van de variabele $Y$ gezet op 0.
 \item[\underline{e}] indien hoog wordt de waarde $Y\shlcmd1$ doorgegeven aan het register $Y$, anders wordt $y_2y_1y_01$ doorgegeven.
 \item[\underline{f}] indien hoog wordt de waarde van de variabele $Z$ gezet op de waarde van de multiplexer.
 \item[\underline{g}] indien hoog wordt de waarde van de variabele $Z$ gezet op 0.
 \item[\underline{h}] indien hoog wordt de waarde $z_2z_1z_0x_3$ doorgegeven aan het register $Z$, anders wordt $z_2z_1z_0x_3-D$.
 \item[\underline{i}] indien hoog wordt de waarde van de variabele $I$ gezet op de waarde van de multiplexer.
 \item[\underline{j}] indien hoog wordt de waarde $3$ doorgegeven aan het register $I$, anders wordt $I-1$ doorgegeven.
\end{enumerate}
Op basis van deze beschrijving kunnen we een toestandstabel bouwen die de volledige controller beschrijft zoals in tabel \ref{tbl:sprocessorControllerGenericLeadingExample} of figuur \ref{fig:sprocessorControllerGenericLeadingExample}.
\begin{table}[hbt]
\centering
\begin{tabular}{c|ccc|ccccccccccc|c}
\multirow{2}{*}{Toestand}&\multicolumn{3}{c|}{Ingang}&\multicolumn{11}{c|}{Uitgang}&\multirow{2}{*}{Volgende toestand}\\
&ci&A&B&co&a&b&c&d&e&f&g&h&i&j&\\\hline
\multirow{2}{*}{$S_1$}
&0&-&-&		0	&1&1	&0&1&-	&0&1&-	&1&1	&$S_1$\\
&1&-&-&		0	&1&1	&0&1&-	&0&1&-	&1&1	&$S_2$\\\hline
\multirow{4}{*}{$S_2$}
&-&0&0&		0	&1&0	&1&0&0	&1&0&0	&1&0	&$S_3$\\
&-&0&1&		0	&1&0	&1&0&0	&1&0&0	&1&0	&$S_2$\\
&-&1&0&		0	&1&0	&1&0&1	&1&0&1	&1&0	&$S_3$\\
&-&1&1&		0	&1&0	&1&0&1	&1&0&1	&1&0	&$S_2$\\\hline
\multirow{2}{*}{$S_3$}
&0&-&-&		1	&0&-	&0&0&-	&0&0&-	&0&-	&$S_1$\\
&1&-&-&		1	&0&-	&0&0&-	&0&0&-	&0&-	&$S_3$\\
\end{tabular}
\caption{Toestandstabel van de controller van het leidend voorbeeld via de basisprincipes.}
\label{tbl:sprocessorControllerGenericLeadingExample}
\end{table}
\begin{figure}[hbt]
\centering
\begin{tikzpicture}[->,shorten >=1pt,auto,node distance=4cm,on grid,semithick,every state/.style={draw=black!50,very thick,fill=black!20,scale=0.75}]
\node[state] (S1) {$S_1$};
\node[state] (S2) [right=of S1] {$S_2$};
\node[state] (S3) [right=of S2] {$S_3$};
\path	(S1) edge node[swap] {\texttt{1--/01101-01-11}} (S2)
	   edge[loop below] node {\texttt{0--/01101-01-11}} (S1)
	(S2) edge node[swap] {\begin{varwidth}{5cm}\begin{center}\texttt{-00/01010010010}\\\texttt{-10/01010110110}\end{center}\end{varwidth}} (S3)
	   edge[loop below] node {\begin{varwidth}{5cm}\begin{center}\texttt{-01/01010010010}\\\texttt{-11/01010110110}\end{center}\end{varwidth}} (S2)
	(S3) edge[bend right] node {\texttt{0--/10-00-00-0-}} (S1)
	   edge[loop below] node {\texttt{1--/10-00-00-0-}} (S3);
%bend left
%0--/01101-01-11
\end{tikzpicture}
\caption{Toestandsdiagram van de controller van het leidend voorbeeld via de basisprincipes.}
\figlab{sprocessorControllerGenericLeadingExample}
\end{figure}
Deze tabel genereren we opnieuw op basis van het ASM-schema. Het instructiewoord -- een sequentie aan bits die bepaald welke instructies op het datapad worden uitgevoerd -- is afhankelijk van zowel de toestand als de controller-ingang en statussignalen. Bij wijze van voorbeeld zullen we de controller-tabel van het leidend voorbeeld opbouwen. Bij toestand $S_1$ zijn alle opdrachten onafhankelijk van testen. Bijgevolg zal het instructiewoord ook onafhankelijk zijn van de invoer. Dit instructiewoord kunnen we bouwen aan de hand van de instructies bij toestand 1. Zo is de controller-uitgang $co=0$, logischerwijs is dit ook het geval de uitvoer. Daarnaast bevat de toestand de instructie $X\gets N$. Bijgevolg zetten we het set-signaal van het $X$-register hoog, samen met het selectie-signaal van de multiplexer, waardoor deze $N$ doorgeeft. Dit resulteert in $\left(a,b\right)=\left(1,1\right)$. Daarnaast worden $Y$ en $Z$ op $0$ gezet. Dit betekent dat we het set-signaal op laag zetten van beide registers en het
reset-signaal op hoog. Vermits de waarde van de multiplexers niet wordt ingelezen, maakt het niet uit welke waarde de multiplexer doorgeeft. Daarom is het volgende deel van het instructiewoord $\left(c,d,e,f,g,h\right)=\left(0,1,-,0,1,-\right)$. De instructie $I\gets N$ bepaald tot slot de rest van het instructiewoord. Het complete instructiewoord wordt dan:
\begin{equation}
S_1:\left(co,a,b,c,d,e,f,g,h,i,j\right)=\left(0,1,1,0,1,-,0,1,-,1,1\right)
\end{equation}
Ondanks het feit dat alle instructies onafhankelijk van testen worden uitgevoerd, is de volgende toestand wel afhankelijk van de $ci$-controller-ingang. Bijgevolg plaatsen we twee regels in de toestandstabel: indien $ci=0$ is de volgende toestand opnieuw $S_1$, in het andere geval is de volgende toestand $S_2$. In het geval van toestand $S_2$ zijn sommige instructies wel afhankelijk van de resultaten van enkele testen. Er zijn echter wel enkele instructies die onafhankelijk van testen worden uitgevoerd. Op analoge wijze bekomen we voor deze instructiebits: $\left(co,a,b,i,j\right)=\left(0,1,0,1,0\right)$. De andere instructies zijn afhankelijk van de test die we gespecificeerd hebben als de status-bit \underline{A} ofwel $z_2z_1z_0x_3<D$. In het geval deze test slaagt zijn de rest van de instructie-bits $\left(c,d,e,f,g,h\right)=\left(1,0,0,1,0,0\right)$. In het andere geval is $\left(c,d,e,f,g,h\right)=\left(1,0,1,1,0,1\right)$. Dit kunnen we formaliseren als:
\begin{equation}
S_2:\left(co,a,b,c,d,e,f,g,h,i,j\right)=\left\{\begin{array}{ll}
\left(0,1,0,1,0,0,1,0,0,1,0\right)&\ifun\mbox{\underline{A}}\\
\left(0,1,0,1,0,1,1,0,1,1,0\right)&\otherwiseun\\
\end{array}\right.
\end{equation}
Opnieuw wordt de volgende toestand gedifferentieerd op basis van een test: \underline{B}. Indien \underline{B} waar is, is $I>0$ bijgevolg is de volgende toestand $S_2$, in het geval de test faalt, is de volgende toestand $S_3$. Tot slot beschouwen we de laatste toestand $S_3$: hierbij worden er geen instructies uitgevoerd. Bijgevolg staan alle set- en reset-signalen van de registers op 0. Opnieuw is ook de selectie-ingang van de multiplexers irrelevant: de inhoud wordt immers niet opgeslagen in de registers. Tot slot is de controller-uitgang $co=1$. Bijgevolg is de volledige uitvoer:
\begin{equation}
S_3:\left(co,a,b,c,d,e,f,g,h,i,j\right)=\left(1,0,-,0,0,-,0,0,-,0,-\right)
\end{equation}
Ook hier bepaald de controller-ingang $ci$ de volgende toestand. Bijgevolg differenti\"eren we de volgende toestand opnieuw op basis van deze test. Op basis van deze paragraaf kunnen we dus de controller toestandstabel op tabel \ref{tbl:sprocessorControllerGenericLeadingExample} opstellen. Voor de concrete implementatie van de controller met behulp van flipflops verwijzen we naar het vorige hoofdstuk. Eenmaal de controller en het datapad gebouwd zijn, dienen we beide componenten alleen nog met elkaar te verbinden: de instructie-uitgangen van de controller worden gelinkt met de set- en reset-signalen van de registers en de selectie-ingangen van de multiplexers. De testen die op het datapad worden uitgevoerd vormen dan weer de invoer van de controller. Merk verder ook op dat we hier een input-gebaseerd ASM-schema hebben verwezenlijkt. Een toestandsgebaseerd ASM-schema wordt op compleet analoge wijze gesynthetiseerd, alleen zal het instructiewoord niet afhangen van eventuele testen. Opdelingen zoals bij
toestand $S_2$ komen dus niet voor.
\subsubsection{Betere implementatie}
Zoals reeds eerder vermeld leidt de procedure die we in deze subsectie hebben ontwikkeld hebben niet tot een optimale oplossing. Er zijn verschillende problemen die afhankelijk van het algoritme tot een ineffici\"ente implementatie leiden:
\begin{itemize}
 \item In deze implementatie zullen we voor elke operatie een aparte functionele eenheid voorzien. Merk echter op dat indien we in eender welke toestand slechts \'e\'en optelling uitvoeren we dezelfde opteller zouden kunnen gebruiken.
 \item Nogal wat functionele eenheden hebben gelijkaardige kenmerken. Zo valt $Y\shlcmd1$ te combineren met $y_2y_1y_01$. Functionele eenheden te combineren leidt tot een complexere schakeling maar minder redundante delen.
 \item In sommige algoritmen wordt slechts een subset van de register tegelijk gebruikt. In dat geval kunnen we een registerbank gebruiken en enkel de adressen uitlezen die bij een bepaalde toestand relevant zijn.
 \item We dienen niet steeds per register een bus te voorzien. Indien bij elke toestand een deel van de register belangrijk zijn, kunnen we op een beperkt aantal bussen de waardes van de relevante registers plaatsen.
 \item Soms kunnen we het ASM-schema verder optimaliseren. Dit sluit enigzinds aan bij de discussie over het aantal toestanden die we gebruikten voor de vertaling van het algoritme naar een ASM-schema.
 \item Logischerwijs kunnen we de toestandstabel van de controller eerst minimaliseren alvorens deze te implementeren.
\end{itemize}
In de volgende subsecties zullen we een groot deel van deze problemen bespreken en toepassen op het leidend voorbeeld.
\subsection{Ontwerp Controller}
\label{ss:syntheseFSMDController}
\subsubsection{Patronen in een algoritme}
Het ontwerp van een controller is in wezen equivalent met de implementatie van een Finite State Machine (FSM). De controller in het leidende voorbeeld is dan ook vrij eenvoudig te implementeren met behulp van de methodes uit hoofdstuk \ref{ch:SeqComp}. Bij complexe programma's is een implementatie van een controller echter niet meer zo triviaal: deze kennen een groot aantal toestanden en tests, en doorgaans een groot instructiewoord. Daarom werd al snel naar een methode gezocht om snel complexe controllers te bouwen. Dit doen we door volgende patronen in een ontwerp (ASM-schema) te zoeken:
\begin{itemize}
 \item \termen{Natuurlijke volgorde} van toestanden.
 \item Terugkerende sets van toestanden veroorzaakt door een \termen{subroutine}.
 \item Eenvoud van het implementeren van logica met behulp van one-hot coderingen.
\end{itemize}
Elk van deze elementen wordt in de volgende subsubsecties verder uitgewerkt. Op basis van deze drie elementen zullen we een component bouwen die een algemene controller vertegenwoordigt: de \termen{microprogrammeerbare controller}. Een component die we in een schakeling inpluggen en die we gegeven een ASM-schema vervolgens kunnen programmeren en daardoor omvormen tot een concrete controller.
\subsubsection{Algemene vorm van een controller}
\begin{figure}[hbt]
\centering
\begin{tikzpicture}[subcomp/.style={minimum width=1.5cm,minimum height=2 cm,draw=black,fill=white,thick,rectangle}]
\node[subcomp] (SR) at (0,0) {SReg};
\node[subcomp] (NSL) at (3,1.25) {\begin{varwidth}{2 cm}\begin{center}Next\\State\\Logic\end{center}\end{varwidth}};
\node[subcomp] (OL) at (3,-1.25) {\begin{varwidth}{2 cm}\begin{center}Output\\Logic\end{center}\end{varwidth}};
%\coordinate (CSP) at ($0.5*(SR)+0.25*(NSL)+0.25*(OL)$);
\coordinate (CSP) at ($(SR.east)+(1,0)$);
\coordinate (TL) at (SR.west |- NSL.north);\coordinate (TL) at ($(TL)+(-0.25,0.25)$);
\pdot{CSP};
\coordinate (NSLA) at ($0.75*(NSL.south west)+0.25*(NSL.north west)$);
\coordinate (NSLB) at ($0.5*(NSL.south west)+0.5*(NSL.north west)$);
\coordinate (NSLC) at ($0.25*(NSL.south west)+0.75*(NSL.north west)$);
\coordinate (OLA) at ($0.75*(OL.south west)+0.25*(OL.north west)$);
\coordinate (OLB) at ($0.5*(OL.south west)+0.5*(OL.north west)$);
\coordinate (OLC) at ($0.25*(OL.south west)+0.75*(OL.north west)$);
\draw[thick] (SR) -- (CSP);\draw[thick,->] (CSP) |- (NSLA);\draw[thick,->] (CSP) |- (OLC);
\draw[thick,->] (NSL.east) -- ++(0.25,0) |- (TL) |- (SR.west);
\begin{pgfonlayer}{background}
\node[subcomp,fill=gray!20, fit=(SR) (NSL) (OL),inner sep=1cm] (CT) {};
\end{pgfonlayer}
\coordinate (CS) at ($0.333*(CT.north west)+0.667*(CT.north east)$);\draw[->,thick] (CS) -- ++(0,0.5) node[anchor=south,text width=1.5 cm]{Controle-Signalen (CS)};
\coordinate (SS) at ($0.667*(CT.north west)+0.333*(CT.north east)$);\draw[<-,thick] (SS) -- ++(0,0.5)
 node[anchor=south,text width=1.5 cm]{Status-Signalen (SS)};
\coordinate (CI) at (CT.west);\draw[<-,thick] (CI) -- ++(-0.5,0) node[anchor=east,text width=1.5 cm]{Controle-Ingangen (CI)};
\coordinate (CO) at (CT.east);\draw[->,thick] (CO) -- ++(0.5,0) node[anchor=west,text width=1.5 cm]{Controle-Uitgangen (CO)};
\coordinate (COD) at ($0.667*(OL.south east)+0.333*(OL.north east)$);\draw[thick] (COD) -| ($(CO)+(-0.25,0)$) -- (CO);
\coordinate (CSD) at ($0.333*(OL.south east)+0.667*(OL.north east)$);\coordinate (CSE) at ($0.75*(NSL.south east)+0.25*(NSL.north east)+(0.5,0)$);\coordinate (CSF) at ($(CS)+(0,-0.25)$);\draw[thick] (CSD) -| (CSE) |- (CSF) -- (CS);
\coordinate (NSLSS) at ($(NSLC)+(-0.75,0)$);
\draw[thick,->] (SS) |- (NSLC);\draw[thick,->] (CI) -- ++(0.25,0) |- (NSLB);\draw[thick,->] ($(CI)+(0.25,0)$) |- (OLA);\pdot{$(CI)+(0.25,0)$};\pdot{NSLSS};\draw[thick,dashed,->] (NSLSS) |- (OLB);
\end{tikzpicture}
\caption{Algemene vorm van een controller.}
\figlab{generalStructureController}
\end{figure}
Allereerst grijpen we terug naar de definitie van een controller. En combineren we dit met de defintie van een sequenti\"ele schakeling. In subsectie \ref{ss:specialProcessorGeneralStructure} hebben we reeds de invoer en uitvoer van een controller besproken. We zien dezelfde structuur op figuur \ref{fig:generalStructureController}. Daarnaast weten we uit hoofdstuk \ref{ch:SeqComp} hoe de algemene sequenti\"ele structuur eruit ziet:
\begin{itemize}
 \item \termen{State Register} ofwel \termen{toestandsregister}: een register die de huidige toestand bijhoudt.
 \item \termen{Next State Logic}: combinatorische logica die de volgende toestand berekend.
 \item \termen{Output Logic} ofwel \termen{uitvoer logica}: een component die het instructiewoord ofwel de controle-signalen berekend, samen met de controle-uitgangen.
\end{itemize}
Beide combinatorische componenten zijn afhankelijk van zowel de controle-ingangen en de huidige toestand bijgehouden door het toestandregister. Daarnaast zullen de statussignalen (de resultaten van de testen in het datapad) de volgende toestand mee helpen bepalen. In het geval van een inputgebaseerd ASM-schema is dit zelfs het geval voor de uitvoer logica. We kunnen dus stellen dat figuur \ref{fig:generalStructureController} de algemene structuur van de controller weergeeft. We zullen dit ontwerp in de volgende subsubsecties aanpassen tot een structuur die gericht is op specifieke eigenschappen van een controller.
\subsubsection{Natuurlijke volgorde van toestanden}
\begin{figure}[hbt]
\centering
\begin{tikzpicture}[subcomp/.style={minimum width=1.5cm,minimum height=2 cm,draw=black,fill=white,thick,rectangle}]
\node[subcomp] (SR) at (0,0) {SReg};
\node[mux2to1,fill=white,rotate=90,thick] (M) at (-1.5,0) {};
\node[subcomp,minimum width=0.5 cm,minimum height=0.25] (INC) at ($(M.data1)+(-0.75,0)$) {$+1$};
\draw[thick,->] (M.output) -- (SR.west);
\draw[thick,->] (INC.east) -- (M.data1);
\node[subcomp] (NSL) at (3,1.25) {\begin{varwidth}{2 cm}\begin{center}Next\\State\\Logic\end{center}\end{varwidth}};
\node[subcomp] (OL) at (3,-1.25) {\begin{varwidth}{2 cm}\begin{center}Output\\Logic\end{center}\end{varwidth}};
\coordinate (CSP) at ($(SR.east)+(1,0)$);
\coordinate (TL) at (M.north |- NSL.north);\coordinate (TL) at ($(TL)+(-0.5,0.5)$);\coordinate (TLA) at ($(TL)+(0.25,-0.25)$);
\pdot{CSP};
\coordinate (NSLA) at ($0.75*(NSL.south west)+0.25*(NSL.north west)$);
\coordinate (NSLB) at ($0.5*(NSL.south west)+0.5*(NSL.north west)$);
\coordinate (NSLC) at ($0.25*(NSL.south west)+0.75*(NSL.north west)$);
\coordinate (OLA) at ($0.75*(OL.south west)+0.25*(OL.north west)$);
\coordinate (OLB) at ($0.5*(OL.south west)+0.5*(OL.north west)$);
\coordinate (OLC) at ($0.25*(OL.south west)+0.75*(OL.north west)$);
\draw[thick] (SR) -- (CSP);\draw[thick,->] (CSP) |- (NSLA);\draw[thick,->] (CSP) |- (OLC);
\draw[thick,->] (NSL.east) -- ++(0.5,0) |- (TL) |- (M.data0);
\draw[thick,->] ($0.5*(NSL.east)+0.5*(NSL.north east)$) -- ++(0.25,0) |- (TLA -| M.selout0) -- (M.selout0);
\begin{pgfonlayer}{background}
\node[subcomp,fill=gray!20, fit=(SR) (NSL) (OL) (M) (INC),inner sep=1cm] (CT) {};
\end{pgfonlayer}
\coordinate (CS) at ($0.333*(CT.north west)+0.667*(CT.north east)$);\draw[->,thick] (CS) -- ++(0,0.5) node[anchor=south,text width=1.5 cm]{Controle-Signalen (CS)};
\coordinate (SS) at ($0.667*(CT.north west)+0.333*(CT.north east)$);\draw[<-,thick] (SS) -- ++(0,0.5)
 node[anchor=south,text width=1.5 cm]{Status-Signalen (SS)};
\coordinate (CI) at (CT.west);\draw[<-,thick] (CI) -- ++(-0.5,0) node[anchor=east,text width=1.5 cm]{Controle-Ingangen (CI)};
\coordinate (CO) at (CT.east);\draw[->,thick] (CO) -- ++(0.5,0) node[anchor=west,text width=1.5 cm]{Controle-Uitgangen (CO)};
\coordinate (COD) at ($0.667*(OL.south east)+0.333*(OL.north east)$);\draw[thick] (COD) -| ($(CO)+(-0.25,0)$) -- (CO);
\coordinate (CSD) at ($0.333*(OL.south east)+0.667*(OL.north east)$);\coordinate (CSE) at ($0.75*(NSL.south east)+0.25*(NSL.north east)+(0.75,0)$);\coordinate (CSF) at ($(CS)+(0,-0.25)$);\draw[thick] (CSD) -- ++(0.25,0) |- (CSE) |- (CSF) -- (CS);
\coordinate (NSLSS) at ($(NSLC)+(-0.75,0)$);\coordinate (SRINC) at ($(SR.east)+(0.25,0)$);\coordinate (SRINCA) at ($(INC.west)+(-0.25,0)$);\coordinate (SRINCA) at (SRINCA |- OLB);
\pdot{SRINC};
\draw[thick,->] (SRINC) |- (SRINCA) |- (INC.west);
\draw[thick,->] (SS) |- (NSLC);\draw[thick,->] (CI) -- ++(0.25,0) |- (NSLB);\draw[thick,->] ($(CI)+(0.25,0)$) |- (OLA);\pdot{$(CI)+(0.25,0)$};\pdot{NSLSS};\draw[thick,dashed,->] (NSLSS) |- (OLB);
\begin{pgfonlayer}{background}
\node[fill=gray!10,draw=black,dotted,thin, fit=(SR) (M) (INC),inner sep=0.5cm] (LR) {};
\end{pgfonlayer}
\end{tikzpicture}
\caption{Controller met natuurlijke volgorde van toestanden.}
\figlab{incrementStructureController}
\end{figure}
Men kan de toestand van een controller enigzinds vergelijken met een programmateller. In de meeste gevallen zal een programma dan ook bestaan uit een reeks toestanden waarbij de volgende toestand vaststaat. Omdat het zoeken van een minimale encodering voor grote toestandstabellen tijdrovend is, zullen we deze optie niet beschouwen. Een logisch gevolg hiervan is dat we de opeenvolgende toestanden opeenvolgende toestandscoderingen geven. In plaats van deze volgende toestandscoderingen voor elke toestand uit te rekenen. Kunnen we dit probleem eenvoudig oplossen door een incrementor en een multiplexer te voorzien. Een deel van de volgende toestandlogica bestaat er dan uit om te bepalen of we de toestandcodering incrementeren, in het ander geval dient de volgende toestand logica zelf een adres te berekenen. In veel gevallen leidt dit tot een vereenvoudiging van de uitvoer logica: er zijn immers heel wat toestanden die deze increment functie kunnen gebruiken, indien dit het geval is kunnen we bovendien don't cares
invoeren bij de logica die de volgende encodering berekend. De volgende toestandlogica dient dan ook enkel conditionele volgende toestanden genereren. Figuur \ref{fig:incrementStructureController}. Een oplettende lezer zal misschien reeds opgemerkt hebben dat een register gecombineerd met increment neerkomt op een teller. Vermits we met behulp van de multiplexer kunnen kiezen om het register met 1 op te hogen, ofwel een nieuwe waarde in te laden, hebben we hier een laadbare teller gerealiseerd. Op figuur \ref{fig:incrementStructureController} hebben we de componten die leiden tot deze laadbare teller gemarkeerd met behulp van een lichtgrijze rechthoek.
\subsubsection{Ondersteunen van subroutines}
\begin{figure}[hbt]
\centering
\begin{tikzpicture}[subcomp/.style={minimum width=1.5cm,minimum height=2 cm,draw=black,fill=white,thick,rectangle}]
\node[subcomp] (SR) at (0,0) {SReg};
\node[mux4to1,fill=white,rotate=90,thick] (M) at (-1.35,0) {};
\node[subcomp,minimum width=1.1 cm,minimum height=0.25] (INC) at ($(M.data3)+(-1.25,-0.25)$) {$+1$};
\coordinate (LFD) at (M.data2 -| INC.north);
\node[subcomp,minimum width=1.1 cm,minimum height=1.55 cm,anchor=south] (LIFO) at ($(LFD)+(0,-0.25)$) {LIFO};
\draw[thick,->] (M.output) -- (SR.west);
\draw[thick,->] (LIFO.east |- M.data1) -- (M.data1);
\node[subcomp] (NSL) at (3,1.25) {\begin{varwidth}{2 cm}\begin{center}Next\\State\\Logic\end{center}\end{varwidth}};
\coordinate (TL) at (M.north |- NSL.north);\coordinate (TL) at ($(TL)+(-0.4,0.5)$);\coordinate (TLA) at ($(TL)+(0.25,-0.25)$);
\draw[thick,->] (INC.east) -- (INC.east -| TL) |- (M.data3);\pdot{M.data3 -| TL};\draw[thick,->] (M.data3 -| TL) |- (M.data2 -| LIFO.east);
\node[subcomp] (OL) at (3,-1.25) {\begin{varwidth}{2 cm}\begin{center}Output\\Logic\end{center}\end{varwidth}};
\coordinate (CSP) at ($(SR.east)+(1,0)$);
\pdot{CSP};
\coordinate (NSLA) at ($0.8*(NSL.south west)+0.2*(NSL.north west)$);
\coordinate (NSLB) at ($0.6*(NSL.south west)+0.4*(NSL.north west)$);
\coordinate (NSLC) at ($0.4*(NSL.south west)+0.6*(NSL.north west)$);
\coordinate (NSLD) at ($0.2*(NSL.south west)+0.8*(NSL.north west)$);
\coordinate (OLA) at ($0.75*(OL.south west)+0.25*(OL.north west)$);
\coordinate (OLB) at ($0.5*(OL.south west)+0.5*(OL.north west)$);
\coordinate (OLC) at ($0.25*(OL.south west)+0.75*(OL.north west)$);
\draw[thick] (SR) -- (CSP);\draw[thick,->] (CSP) |- (NSLA);\draw[thick,->] (CSP) |- (OLC);
\draw[thick,->] (NSL.east) -- ++(0.5,0) |- (TL) |- (M.data0);
\draw[thick,->] ($0.5*(NSL.east)+0.5*(NSL.north east)$) -- ++(0.25,0) |- (TLA -| M.selout0) -- (M.selout0);
\begin{pgfonlayer}{background}
\node[subcomp,fill=gray!20, fit=(SR) (NSL) (OL) (M) (INC),inner sep=1cm] (CT) {};
\end{pgfonlayer}
\coordinate (CS) at ($0.333*(CT.north west)+0.667*(CT.north east)$);\draw[->,thick] (CS) -- ++(0,0.5) node[anchor=south,text width=1.5 cm]{Controle-Signalen (CS)};
\coordinate (SS) at ($0.667*(CT.north west)+0.333*(CT.north east)$);\draw[<-,thick] (SS) -- ++(0,0.5)
 node[anchor=south,text width=1.5 cm]{Status-Signalen (SS)};
\coordinate (CI) at (CT.west);\draw[<-,thick] (CI) -- ++(-0.5,0) node[anchor=east,text width=1.5 cm]{Controle-Ingangen (CI)};
\coordinate (CO) at (CT.east);\draw[->,thick] (CO) -- ++(0.5,0) node[anchor=west,text width=1.5 cm]{Controle-Uitgangen (CO)};
\coordinate (COD) at ($0.667*(OL.south east)+0.333*(OL.north east)$);\draw[thick] (COD) -| ($(CO)+(-0.25,0)$) -- (CO);
\coordinate (PuPoD) at ($(NSL.north west)+(-0.25,0)$);\coordinate (PuPo) at (LIFO.north);
\draw[thick,->] (NSLB) -| (PuPoD) -| (PuPo);\draw ($(PuPo)+(0,0.2)$) node[anchor=south west,rotate=90,scale=0.85] {Push/Pop$^*$};
\coordinate (CSD) at ($0.333*(OL.south east)+0.667*(OL.north east)$);\coordinate (CSE) at ($0.75*(NSL.south east)+0.25*(NSL.north east)+(0.75,0)$);\coordinate (CSF) at ($(CS)+(0,-0.25)$);\draw[thick] (CSD) -- ++(0.25,0) |- (CSE) |- (CSF) -- (CS);
\coordinate (NSLSS) at ($(NSLD)+(-0.75,0)$);\coordinate (SRINC) at ($(SR.east)+(0.25,0)$);\coordinate (SRINCA) at ($(INC.west)+(-0.25,0)$);\coordinate (SRINCA) at (SRINCA |- OLB);
\pdot{SRINC};
\draw[thick,->] (SRINC) |- (SRINCA) |- (INC.west);
\draw[thick,->] (SS) |- (NSLD);\draw[thick,->] (CI) -- ++(0.25,0) |- (NSLC);\draw[thick,->] ($(CI)+(0.25,0)$) |- (OLA);\pdot{$(CI)+(0.25,0)$};\pdot{NSLSS};\draw[thick,dashed,->] (NSLSS) |- (OLB);
\begin{pgfonlayer}{background}
\node[fit=(LIFO) (INC) (M),inner sep=0.125cm] (NSTPA) {};
\node[fit=(NSL),inner sep=0.125cm] (NSTPB) {};
\coordinate (NSTPC) at ($(SR.north east)+(0,0.25)$);
\filldraw[dotted,fill=gray!10] (NSTPA.south west) -- (NSTPA.south east) |- (NSTPC) -- (NSTPB.south west) -- (NSTPB.south east) -- (NSTPB.north east) -- (NSTPB.north west) -| (NSTPA.north west) -- cycle;
\end{pgfonlayer}
\end{tikzpicture}
\caption{Controller met natuurlijke volgorde van toestanden en subroutines.}
\figlab{subroutineStructureController}
\end{figure}
Bij het bouwen van een programma wordt vaak gebruik gemaakt van een subroutine: een klein programma die we oproepen om een bepaalde opdracht uit te voeren en nadien terugkeert naar de toestand die volgt op de toestand waar we de subroutine hebben opgeroepen. Een probleem bestaat erin dat we niet eenvoudigweg de subroutine kunnen omzetten naar een reeks toestanden en vervolgens naar de eerste toestand van deze subroutine kunnen springen. In dat geval weet de subroutine immers niet meer naar welke toestand men dient terug te springen (deze toestand wordt de \termen{terugkeertoestand} genoemd) nadat de subroutine is uitgevoerd. Een oplossing zou erin kunnen bestaan om in dat geval per oproep van een subroutine de subroutine te kopi\"eren in de structuur. Deze oplossing heeft echter enkele nadelen:
\begin{itemize}
 \item Iedere keer wanneer we een subroutine oproepen dienen we een kopie te maken van alle toestanden. Indien de subroutine $T_s$ verschillende toestanden heeft en we roepen de routine $n$ keer op, resulteert dit in een controller waarvan $n\cdot T_s$ toestanden door de subroutine zijn gegenereerd. Dit leidt dus tot een groot aantal toestanden en complexe logica voor het berekenen van de volgende toestand en de uitvoer.
 \item In sommige programma's roept een subroutine zichzelf op. Dit fenomeen wordt \termen{recursie} genoemd. We kunnen dit fenomeen niet rechtstreeks omzetten in de aangereikte oplossing, omdat we niet altijd vooraf weten hoe diep deze recursie gaat. In dat geval zullen we de subroutine eerst moeten herschrijven in een subroutine zonder recursie.
\end{itemize}
De oplossing bestaat dan ook uit een datastructuur genaamd de \termen{call stack} die -- zoals de naam reeds doet vermoeden -- en vorm van stapelgeheugen is. De oplossing bestaat eruit bij elke oproep naar het stapelgeheugen de terugkeertoestand op de call stack te zetten. Op het moment dat de subroutine is afgelopen, wordt de terugkeertoestand terug uit het stapelgeheugen uitgelezen en in het toestandsregister geplaatst. De callstack ondersteund ook recursie omdat indien een subroutine zichzelf oproept, ook de terugkeertoestand van deze subroutine op de stapel wordt gezet. Indien we deze wijziging combineren met bekomen we een structuur met een incrementor, een stapelgeheugen en een multiplexer zoals op figuur \ref{fig:subroutineStructureController}. De volgende toestand logica heeft dan vier verschillende types instructies:
\begin{itemize}
 \item Natuurlijke volgende toestand: het toestandregister wordt met 1 opgehoogd. Dit gebeurt bij toestanden waarbij de volgende toestand vast staat (onafhankelijk van testen).
 \item Specifieke volgende toestand: indien de volgende toestand wel afhankelijk is van testen, of we springen naar een bepaalde toestand in een programma kan de volgende toestand logica zelf een toestand berekenen.
 \item Uitvoeren van een subroutine: in dat geval wordt de natuurlijke volgende toestand van de huidige toestand op de call stack geplaatst, en lezen we de begintoestand van de subroutine in het toestandsregister in.
 \item Terugkeren uit een subroutine: dit is de laatste instructie bij een subroutine. We voeren een pop-operatie uit op de call stack en de terugkeertoestand wordt in het toestandsregister ingelezen. We maken dus de veronderstelling dat we terugspringen naar de volgende natuurlijke toestand van de procedure die we hebben opgeroepen\footnote{We kunnen echter de component aanpassen dat we ook zelf een terugkeertoestand kunnen invoeren. Anderzijds kunnen we, indien dit nodig is, van de terugkeertoestand ook een lege toestand maken en de volgende toestandlogica dan naar het juiste adres laten springen.}.
\end{itemize}
We realiseren deze schakeling door naast de incrementor, ook een een stapelgeheugen te plaatsen. De volgende toestandslogica heeft controle over de multiplexer (die kan kiezen uit het stapelgeheugen, de incrementor of een toestandencodering die hij zelf heeft gegenereerd) samen met de push/pop$^*$-ingang van het stapelgeheugen. Verder kan de volgende toestand logica ook zelf een toestandcodering genereren door deze aan \'e\'en van de data-ingangen van de multiplexer aan te leggen.
\subsubsection{Werken met ``One-Hot'' coderingen}
In het vorige hoofdstuk hebben we het reeds uitgebreid gehad over de one-hot codering. Een one-hotcodering kan de volgende toestand logica drastisch vereenvoudigen. Een probleem is echter dat we bij een one-hot codering het aantal flipflops opdrijven. Een tussenoplossing zou er echter uit kunnen bestaan om een decoder tussen het toestandsregister en de volgende toestand logica te zetten. Merk op dat we verder ook de uitvoer-logica op een gelijkaardige manier kunnen vereenvoudigen.
\begin{figure}[hbt]
\centering
\begin{tikzpicture}[subcomp/.style={minimum width=1.5cm,minimum height=2 cm,draw=black,fill=white,thick,rectangle}]
\node[subcomp] (SR) at (0,0) {SReg};
\node[mux4to1,fill=white,rotate=90,thick] (M) at (-1.35,0) {};
\node[subcomp,minimum width=1.1 cm,minimum height=0.25] (INC) at ($(M.data3)+(-1.25,-0.25)$) {$+1$};
\coordinate (LFD) at (M.data2 -| INC.north);
\node[subcomp,minimum width=1.1 cm,minimum height=1.55 cm,anchor=south] (LIFO) at ($(LFD)+(0,-0.25)$) {LIFO};
\draw[thick,->] (M.output) -- (SR.west);
\draw[thick,->] (LIFO.east |- M.data1) -- (M.data1);
\node[subcomp] (DEC) at (2,0) {Decoder};
\node[subcomp] (NSL) at (4,1.25) {\begin{varwidth}{2 cm}\begin{center}Next\\State\\Logic\end{center}\end{varwidth}};
\coordinate (TL) at (M.north |- NSL.north);\coordinate (TL) at ($(TL)+(-0.4,0.5)$);\coordinate (TLA) at ($(TL)+(0.25,-0.25)$);
\draw[thick,->] (INC.east) -- (INC.east -| TL) |- (M.data3);\pdot{M.data3 -| TL};\draw[thick,->] (M.data3 -| TL) |- (M.data2 -| LIFO.east);
\node[subcomp] (OL) at (4,-1.25) {\begin{varwidth}{2 cm}\begin{center}Output\\Logic\end{center}\end{varwidth}};
\coordinate (CSP) at ($(DEC.east)+(0.25,0)$);
\pdot{CSP};
\coordinate (NSLA) at ($0.667*(NSL.south west)+0.333*(NSL.north west)$);
\coordinate (NSLB) at ($0.333*(NSL.south west)+0.667*(NSL.north west)$);
\coordinate (OLA) at (OL.west);
\draw[thick,->] (SR.east) -- (DEC.west);
\draw[thick] (DEC.east) -- (CSP);\draw[thick,->] (CSP) |- (NSLA);\draw[thick,->] (CSP) |- (OLA);
\draw[thick,->] (NSL.east) -- ++(0.5,0) |- (TL) |- (M.data0);
\draw[thick,->] ($0.5*(NSL.east)+0.5*(NSL.north east)$) -- ++(0.25,0) |- (TLA -| M.selout0) -- (M.selout0);
\begin{pgfonlayer}{background}
\node[subcomp,fill=gray!20, fit=(SR) (NSL) (OL) (M) (INC),inner sep=1cm] (CT) {};
\end{pgfonlayer}
\coordinate (CS) at ($0.333*(CT.north west)+0.667*(CT.north east)$);\draw[->,thick] (CS) -- ++(0,0.5) node[anchor=south,text width=1.5 cm]{Controle-Signalen (CS)};
\coordinate (SS) at ($0.667*(CT.north west)+0.333*(CT.north east)$);\draw[<-,thick] (SS) -- ++(0,0.5)
 node[anchor=south,text width=1.5 cm]{Status-Signalen (SS)};
\coordinate (CI) at (CT.west);\draw[<-,thick] (CI) -- ++(-0.5,0) node[anchor=east,text width=1.5 cm]{Controle-Ingangen (CI)};
\coordinate (CO) at (CT.east);\draw[->,thick] (CO) -- ++(0.5,0) node[anchor=west,text width=1.5 cm]{Controle-Uitgangen (CO)};
\coordinate (COD) at ($0.667*(OL.south east)+0.333*(OL.north east)$);\draw[thick] (COD) -| ($(CO)+(-0.25,0)$) -- (CO);
\coordinate (PuPo) at (LIFO.north);\draw[thick,->] (NSLB) -| (PuPo);\draw ($(PuPo)+(0,0.2)$) node[anchor=south west,rotate=90,scale=0.85] {Push/Pop$^*$};
\coordinate (CSD) at ($0.333*(OL.south east)+0.667*(OL.north east)$);\coordinate (CSE) at ($0.75*(NSL.south east)+0.25*(NSL.north east)+(0.75,0)$);\coordinate (CSF) at ($(CS)+(0,-0.25)$);\draw[thick] (CSD) -- ++(0.25,0) |- (CSE) |- (CSF) -- (CS);
\coordinate (NSLSS) at ($(NSLD)+(-0.75,0)$);\coordinate (SRINC) at ($(SR.east)+(0.25,0)$);\coordinate (SRINCA) at ($(INC.west)+(-0.25,0)$);\coordinate (SRINCA) at (SRINCA |- OLB);
\pdot{SRINC};
\draw[thick,->] (SRINC) |- (SRINCA) |- (INC.west);
\coordinate (SRCI) at ($(SR.south)+(0,-0.5)$);\draw[thick,->] (CI) -- ++(0.25,0) |- (SRCI -| DEC) -- (DEC);\draw[thick,->] (SS) |- (NSL.east -| DEC.north) -- (DEC.north);
\begin{pgfonlayer}{background}
\node[subcomp,dotted,fill=gray!10,fit=(NSL) (DEC) (OL),inner sep=0.125cm] (ROM) {};
%\node[fit=(NSL),inner sep=0.125cm] (NSTPB) {};
%\coordinate (NSTPC) at ($(SR.north east)+(0,0.25)$);
%\filldraw[dotted,fill=gray!10] (NSTPA.south west) -- (NSTPA.south east) |- (NSTPC) -- (NSTPB.south west) -- (NSTPB.south east) -- (NSTPB.north east) -- (NSTPB.north west) -| (NSTPA.north west) -- cycle;
\end{pgfonlayer}
\end{tikzpicture}
\caption{Controller met one-hotcodering.}
\figlab{oneHotStructureController}
\end{figure}
Figuur \ref{fig:oneHotStructureController} toont een algemene controller waarbij we een decoder tussen het toestandsregister en de volgende toestand- en uitvoer-logica's plaatsen.
\subsubsection{Programmeerbare controller}
In plaats van specifieke logica te voorzien voor de volgende toestand logica en de uitvoer logica, kunnen we deze componenten ook vervangen door een meer algemene structuur: een vorm van look-up tables. Indien we deze look-up table vervolgens combineren met de decoder bekomen we een geheugen (deze componenten staan gemarkeerd op figuur \ref{fig:oneHotStructureController}). Welk geheugen we gebruiken is in principe arbitrair. Indien we een ROM-geheugen gebruiken kunnen we de controller eenmaal programmeren naar een specifieke controller. Een andere mogelijkheid is om een RAM-geheugen te gebruiken waarbij we de specifieke controller telkens inladen bij het opstarten van het elektronisch circuit. Een voordeel van het gebruik van een RAM-geheugen is dat we bijgevolg het algoritme kunnen wijzigen, op voorwaarde dat alle benodigde hardware op het datapad aanwezig is. Daarentegen betalen we een prijs dat bij het opstarten van de component we eerste de controller moeten inladen wat tijd vergt.
\begin{figure}[hbt]
\centering
\begin{tikzpicture}[subcomp/.style={minimum width=1.5cm,minimum height=2 cm,draw=black,fill=white,thick,rectangle}]
\node[subcomp] (SR) at (0,0) {SReg};
\node[mux4to1,fill=white,rotate=90,thick] (M) at (-1.35,0) {};
\node[subcomp,minimum width=1.1 cm,minimum height=0.25] (INC) at ($(M.data3)+(-1.25,-0.25)$) {$+1$};
\coordinate (LFD) at (M.data2 -| INC.north);
\node[subcomp,minimum width=1.1 cm,minimum height=1.55 cm,anchor=south] (LIFO) at ($(LFD)+(0,-0.25)$) {LIFO};
\draw[thick,->] (M.output) -- (SR.west);
\draw[thick,->] (LIFO.east |- M.data1) -- (M.data1);
\node[subcomp] (ROM) at (2,0) {\begin{varwidth}{1.5cm}ROM-geheugen\end{varwidth}};
\coordinate (TL) at (M.north |- NSL.north);\coordinate (TL) at ($(TL)+(-0.4,0.5)$);\coordinate (TLA) at ($(TL)+(0.25,-0.25)$);
\coordinate (ROMSA) at ($0.667*(ROM.south west)+0.333*(ROM.south east)$);
\coordinate (ROMSB) at ($0.333*(ROM.south west)+0.667*(ROM.south east)$);
\coordinate (ROMNA) at ($0.8*(ROM.north west)+0.2*(ROM.north east)$);
\coordinate (ROMNB) at ($0.6*(ROM.north west)+0.4*(ROM.north east)$);
\coordinate (ROMNC) at ($0.4*(ROM.north west)+0.6*(ROM.north east)$);
\coordinate (ROMND) at ($0.2*(ROM.north west)+0.8*(ROM.north east)$);
\draw[thick,->] (INC.east) -- (INC.east -| TL) |- (M.data3);\pdot{M.data3 -| TL};\draw[thick,->] (M.data3 -| TL) |- (M.data2 -| LIFO.east);
\draw[thick,->] (SR.east) -- (ROM.west);
\draw[thick,->] (ROMSA) -- ++(0,-0.5) -| (M.selin0);
\draw[thick,->] (ROMNA) -- ++(0,0.25) -| ($(M.data0)+(-0.25,0)$) -- (M.data0);
\begin{pgfonlayer}{background}
\node[subcomp,fill=gray!20, fit=(SR) (ROM) (M) (INC),inner sep=1.25cm] (CT) {};
\end{pgfonlayer}
\coordinate (CS) at ($0.333*(CT.north west)+0.667*(CT.north east)$);\draw[->,thick] (CS) -- ++(0,0.5) node[anchor=south,text width=1.5 cm]{Controle-Signalen (CS)};
\coordinate (SS) at ($0.667*(CT.north west)+0.333*(CT.north east)$);\draw[<-,thick] (SS) -- ++(0,0.5)
 node[anchor=south,text width=1.5 cm]{Status-Signalen (SS)};
\coordinate (CI) at (CT.west);\draw[<-,thick] (CI) -- ++(-0.5,0) node[anchor=east,text width=1.5 cm]{Controle-Ingangen (CI)};
\coordinate (CO) at (CT.east);\draw[->,thick] (CO) -- ++(0.5,0) node[anchor=west,text width=1.5 cm]{Controle-Uitgangen (CO)};
\draw[thick] (ROM.east) -- (CO);
\coordinate (PuPo) at (LIFO.north);\draw[thick,->] (ROMNB) -- ++(0,0.5) -| (PuPo);

\coordinate (NSLSS) at ($(NSLD)+(-0.75,0)$);\coordinate (SRINC) at ($(SR.east)+(0.25,0)$);\coordinate (SRINCA) at ($(SR.south)+(0,-0.25)$);\coordinate (SRINCB) at ($(INC.west)+(-0.25,0)$);
\pdot{SRINC};
\draw[thick,->] (SRINC) |- (SRINCA) -| (SRINCB) |- (INC.west);
\coordinate (SRCI) at ($(SR.south)+(0,-0.75)$);\draw[thick,->] (CI) -- ++(0.25,0) |- (SRCI -| ROMSB) -- (ROMSB);\draw[thick,->] (SS) -- ++(0,-0.5) -| (ROMNC);
\draw[thick] (CS) -- ++(0,-0.25) -| (ROMND);
\end{tikzpicture}
\caption{Microprogrammeerbare controller.}
\figlab{microprogrammableController}
\end{figure}
Op figuur \ref{fig:microprogrammableController} tonen we een realisatie de microprogrammeerbare controller. Merk op dat we geen aparte geheugens voor de volgende toestand en de uitvoer voorzien. We kunnen beide componenten immers combineren tot \'e\'en geheugen component. Het ROM-geheugen dient dan ook verschillende rijen van bits op te slaan:
\begin{itemize}
 \item Push/Pop bits: wordt gebruikt bij het oproepen of verlaten van een subroutine, ofwel blijft het stapelgeheugen ongewijzigd.
 \item Multiplexer bits: bepalen welke van de drie bronnen van toestandscoderingen we gebruiken: natuurlijke opvolging, eigen adressering of stapelgeheugen.
 \item Toestandsencodering: in het geval men naam een speciale toestand wil gaan (geen natuurlijke opvolging).
 \item Controle-uitgang: bits die de toestand van een algoritme buiten de processor brengen.
 \item Controle-signalen ofwel het instructiewoord: bepaald welke opdrachten het datapad zullen uitvoeren.
\end{itemize}
\subsection{Minimaliseren Datapad}
\label{ss:syntheseFSMDDatapad}
Naast het introduceren van een algemene controller zullen we enkele verbeteringen bij een datapad bespreken. Hiervoor zullen we vier technieken bespreken:
\begin{itemize}
 \item Het samenvoegen van variabelen.
 \item Het samenvoegen van bewerkingen.
 \item Het samenvoegen van verbindingen ofwel ``\termen{bus sharing}''.
 \item Het samenvoegen van registers in een registerbank.
\end{itemize}
Alvorens we al deze optimalisaties bespreken zullen we eerst een ander leidend voorbeeld introduceren. We zullen telkens \'e\'en van de leidende voorbeelden gebruiken op de plaatsen waar dit relevant is.
\subsubsection{Leidend voorbeeld: vierkantswortel benadering}
Een tweede leidende voorbeeld is het benaderen van een vierkantswortel. Veel programma's berekenen vaak de vierkantswortel van de som van twee kwadraten (bijvoorbeeld bij het berekenen van een afstand tussen twee punten). Voor de meeste programma's is het echter niet belangrijk dat deze vierkantswortel met een hoge nauwkeurigheid berekend wordt. Daarom maakt men meestal gebruik van een volgende benadering:
\begin{equation}
\sqrt{x^2+y^2}\approx\max\left(\max\left(\left|x\right|,\left|y\right|\right),0.875\cdot\max\left(\left|x\right|,\left|y\right|\right)+0.5\cdot\min\left(\left|x\right|,\left|y\right|\right)\right)
\end{equation}
Figuur \ref{fig:asmSqrt} toont een ASM-schema die deze vierkantswortel kan uitrekenen. Bemerk dat we ook een controle-ingang $\mbox{start}$ voorzien, die op hoog dient gezet te worden alvorens het algoritme wordt uitgevoerd, en een controle-uitgang $\mbox{ready}$ die aangeeft dat het resultaat berekend is. Daarnaast zijn er twee data-ingangen $A$ en $B$. Daarnaast voorzien we ook een data-uitgang $R$ die het berekende resultaat naar buiten brengt.
\begin{figure}[hbt]
\centering
\begin{tikzpicture}[yscale=-1]
\node[asmS] (S01) at (0,0) {\begin{varwidth}{3 cm}$t_1\gets A$\\$t_2\gets B$\end{varwidth}};
\node[asmD] (D01) at (0,1.5) {$\mbox{start}$};\setTrueFalseLabels{D01}
\node[asmS] (S02) at (0,3) {\begin{varwidth}{3 cm}$t_3\gets \left|t_1\right|$\\$t_4\gets \left|t_2\right|$\end{varwidth}};
\node[asmS] (S03) at (0,4.5) {\begin{varwidth}{3 cm}$t_5\gets \max\left(t_3,t_4\right)$\\$t_6\gets \min\left(t_3,t_4\right)$\end{varwidth}};
\node[asmS] (S04) at (3.375,4.5) {\begin{varwidth}{3 cm}$t_7\gets t_5\shrcmd3$\\$t_8\gets t_6\shrcmd1$\end{varwidth}};
\node[asmS] (S05) at (3.375,3) {$t_9\gets t_5-t_7$};
\node[asmS] (S06) at (3.375,1.5) {$t_{10}\gets t_8+t_9$};
\node[asmS] (S07) at (3.375,0) {\begin{varwidth}{3 cm}$t_{11}\gets\max\left(t_5,t_{10}\right)$\end{varwidth}};
\node[asmS] (S08) at (3.375,-1.5) {$\mbox{ready=1}$};
\node[asmN] (N01) at (0,-1.5) {$R=t_{11}$};
\path[->] (S01) edge (D01) (S02) edge (S03) (S03) edge (S04) (S04) edge (S05) (S05) edge (S06) (S06) edge (S07) (S07) edge (S08);
\draw[->] (S08.west) -- (S01.north);
\draw[->] (D01.west) -| ++(-0.75,0.75) -| (S02.north);
\draw[->] (D01.east) -- ++(0.75,0) |- (S01.east);
\begin{pgfonlayer}{background}
\node[asmB,fit=(S01) (D01)] (B00) {};\draw (B00.north west) node[anchor=north east]{$S_0$};
\node[asmB,fit=(S02)] (B01) {};\draw (B01.north west) node[anchor=north east]{$S_1$};
\node[asmB,fit=(S03)] (B02) {};\draw (B02.north west) node[anchor=north east]{$S_2$};
\node[asmB,fit=(S04)] (B03) {};\draw (B03.north east) node[anchor=north west]{$S_3$};
\node[asmB,fit=(S05)] (B04) {};\draw (B04.north east) node[anchor=north west]{$S_4$};
\node[asmB,fit=(S06)] (B05) {};\draw (B05.north east) node[anchor=north west]{$S_5$};
\node[asmB,fit=(S07)] (B06) {};\draw (B06.north east) node[anchor=north west]{$S_6$};
\node[asmB,fit=(S08)] (B07) {};\draw (B07.north east) node[anchor=north west]{$S_7$};
\end{pgfonlayer}
\end{tikzpicture}
\caption{ASM-schema van het vierkantswortel-benaderingsalgoritme.}
\figlab{asmSqrt}
\end{figure}
Indien we dit ASM-schema op de eerder genoemde wijze zouden implementeren zouden we volgende componenten nodig hebben:
\begin{itemize}
 \item 11 registers (van een arbitrair aantal bits).
 \item 2 absolute waarde functionele eenheden.
 \item 2 maximum functionele eenheden.
 \item 1 minimum functionele eenheid.
 \item 1 opteller functionele eenheden.
 \item 1 aftrekker functionele eenheid.
 \item 1 shift 3 posities naar rechts functionele eenheid.
 \item 1 shift 1 positie naar rechts functionele eenheid.
\end{itemize}
Het is duidelijk dat dit ASM-schema een grote kost met zich teweeg brengt. We zullen in de volgende subsubsecties dan ook de implementatie significant verbeteren.
\begin{figure}[hbt]
\centering
\begin{sprocessor}[0.75/1.25/1.3/1.4/0.25/0.8]{t1/$t_1$/1/1,t2/$t_2$/1/1,t3/$t_3$/1/1,t4/$t_4$/1/1,t5/$t_5$/1/1,t6/$t_6$/1/1,t7/$t_7$/1/1,t8/$t_8$/1/1, t9/$t_9$/1/1,t10/$t_{10}$/1/1,t11/$t_{11}$/1/1}{abs1/$\abs$,abs2/$\abs$,max1/$\max$,min/$\min$,shr3/$\shrcmd3$,shr1/$\shrcmd1$,sub/$-$,add/$+$,max2/$\max$}{}
\sprbtf{t1}{abs1}{0};
\sprbtf{t2}{abs2}{-0.2};
\sprbtf{t3}{max1}{-0.3};
\sprbtf{t4}{max1}{0.3};
\sprbtf{t3}{min}{0};
\sprbtf{t4}{min}{0.3};
\sprbtf{t5}{shr3}{0};
\sprbtf{t6}{shr1}{0};
\sprbtf{t5}{sub}{-0.2};
\sprbtf{t7}{sub}{0.2};
\sprbtf{t8}{add}{-0.2};
\sprbtf{t9}{add}{0.2};
\sprbtf{t5}{max2}{-0.3};
\sprbtf{t10}{max2}{0};
\spfutr{max2}{t11}{0};
\spfutr{add}{t10}{0};
\spfutr{sub}{t9}{0};
\spfutr{shr1}{t8}{0};
\spfutr{shr3}{t7}{0};
\spfutr{min}{t6}{0};
\spfutr{max1}{t5}{0};
\spfutr{abs2}{t4}{0};
\spfutr{abs1}{t3}{0};
\end{sprocessor}
\caption{Implementatie van het datapad van de benaderende vierkantswortel volgens de basisprincipes.}
\figlab{datapad-generic}
\end{figure}
\subsubsection{Kostprijsberekening}
Alvorens we een goedkopere realisatie van de processor in kwestie kunnen bekomen, dienen we eerst te weten hoeveel elk component precies kost. Hiervoor kunnen we de implementatie van bijvoorbeeld een multiplexer in \'e\'en van de vorige hoofdstukken herbekijken, en de prijs uitrekenen. We zullen in deze subsubsectie echter een bondig overzicht geven die de kostprijs van de belangrijkste componenten die we zullen tegenkomen samenvat. Per component zullen we de kostprijs zowel in transistoren als in het aantal logische cellen in een FPGA weergeven. Verder zullen we ook telkens de componenten in functie van 1 bit uitdrukken, bijvoorbeeld een 1 bit register. Het vermenigvuldigen met het aantal bits geeft een ruwe schatting maar ervaring uit vorige hoofdstukken zou moeten leren dat we soms ook heel wat transistoren kunnen uitsparen omdat we sommige deelcircuits kunnen hergebruiken.
\begin{table}[hbt]
\centering
\begin{tabular}{lrrrrrr}
\toprule
&3-state buffer&\multicolumn{4}{c}{multiplexer}&register\\
\midrule
\# Transistoren&10&12&18&24&30&44\\
\# Logische cellen&0&1&2&2&3&1\\
\bottomrule
\toprule
&Half Adder&Full Adder&Aftrekker&Opteller/Aftrekker\\
\midrule
\# Transistoren&18&36&38&48\\
\# Logische cellen&2&2&2&2\\
\end{tabular}
\caption{Samenvatting van de kostprijs van de belangrijkste componenten}
\label{tbl:costComponentsProcessor}
\end{table}
\subsubsection{Variabelen samenvoegen}
Een belangrijke verbetering is het reduceren van het aantal variabelen en dus het aantal registers. We merken immers op in figuur \ref{fig:asmSqrt} dat de meeste variabelen slechts \'e\'enmaal op een waarde gezet worden, en daarna -- meestal in de volgende toestand -- uitgelezen worden. Vanaf dat moment worden deze variabelen niet meer gebruikt, bijgevolg levert het niets op om hun waarde effectief bij te houden. We kunnen de registers die eerst de waarde van de ene variabele bijhielden echter hergebruiken om de waarde van een andere variabele bij te houden. Hiervoor maken we gebruik van een \termen{levensduurtabel}. Een levensduurtabel is een tabel waarin de variabelen horizontaal worden voorgesteld, en de toestanden verticaal. We markeren een gegeven cel indien we in de gegeven of een latere toestand de waarde van de variabele uitlezen nadat de waarde is toegewezen. Zo zien we bijvoorbeeld op het ASM-schema (figuur \ref{fig:asmSqrt}), dat de variabele $t_5$ toegewezen wordt in toestand $S_2$ en we de
waarde uitlezen in toestanden $S_3$, $S_4$ en $S_6$. Dit betekent dus dat de \termen{levensduur} van $t_5$ zich uitstrekt van $S_3$ tot en met $S_6$. Men kan zich misschien afvragen waarom de toestand waarin we de waarde toewijzen niet ook deel uitmaakt van de levensduur. De rede is omdat op dat moment we nog geen register hebben waarin we de variabele opslaan. Let wel: er dient een register te bestaan waarin we plannen de waarde te zullen opslaan. In toestand $S_2$ wordt echter enkel de waarde van $t_5$ berekent een aangelegd op \'e\'en van de register. Het register heeft echter de waarde nog niet opgeslagen. Dit doet het bij de klokflank tussen $S_2$ en $S_3$. We stellen de levensduur tabel op door voor elke variabele uit te rekenen wanneer welke variabelen actief zijn. De levensduurtabel van het leidend voorbeeld staat in tabel \ref{tbl:variableLifespanExample}. Onder de rij van variabelen vermelden we telkens het totaal aantal variabelen die in de gegeven toestand actief zijn.
\begin{table}[hbt]
\centering
\begin{tabular}{c|cccccccc}
&$S_0$&$S_1$&$S_2$&$S_3$&$S_4$&$S_5$&$S_6$&$S_7$\\\hline
$t_1$&	&$\bullet$&	&	&	&	&	&\\
$t_2$&	&$\bullet$&	&	&	&	&	&\\
$t_3$&	&	&$\bullet$&	&	&	&	&\\
$t_4$&	&	&$\bullet$&	&	&	&	&\\
$t_5$&	&	&	&$\bullet$&$\bullet$&$\bullet$&$\bullet$&\\
$t_6$&	&	&	&$\bullet$&	&	&	&\\
$t_7$&	&	&	&	&$\bullet$&	&	&\\
$t_8$&	&	&	&	&$\bullet$&$\bullet$&	&\\
$t_9$&	&	&	&	&	&$\bullet$&	&\\
$t_{10}$&	&	&	&	&	&	&$\bullet$&\\
$t_{11}$&$\cdot$&	&	&	&	&	&	&$\bullet$\\\hline
$\#$&1&2&2&2&3&3&2&0\\
\end{tabular}
\caption{Levensduurtabel van het vierkantswortel voorbeeld.}
\label{tbl:variableLifespanExample}
\end{table}
\paragraph{Formeel algoritme}We hebben tabel \ref{tbl:variableLifespanExample} opgesteld door te redeneren over de levensduurte van een bepaalde variabele. We kunnen de levensduur echter ook berekenen met een formeel algoritme. Dit algoritme is geen leerstof die behoort tot het opleidingsonderdeel ``Digitale Elektronica en Processoren'', maar is desalniettemin ook nuttig voor andere opleidingsonderdelen zoals bijvoorbeeld ``compilerconstructies'', In compilerconstructies wordt het genereren van een dergelijke tabel ook wel de ``liveness analyse genoemd''. In \cite{books/cu/Appel2002} wordt zo'n algoritme aangereikt. In het algoritme zal men eerst code opdelen in toestanden. Dit hoeven we hier niet meer te doen: dit zijn immers de toestanden van het ASM-schema. Verder dient elke toestand ook drie verzamelingen bij te houden:
\begin{itemize}
 \item $\succst{n}$: dit is een verzameling toestanden die op toestand $n$ kan volgen. Voor een ASM-blok zonder beslissingskader bestaat de toestand uit juist \'e\'en toestand. Voor ASM blokken met beslissingskaders uit \'e\'en of meerdere toestanden.
 \item $\usestt{n}$: dit is een verzameling van variabelen die in toestand $n$ gebruikt worden om berekeningen uit te voeren. Voor toestand $S_5$ is dit dus: $\usestt{S_5}=\left\{t_8,t_9\right\}$
 \item $\defstt{n}$: dit is de verzameling van alle variabelen die in een toestand $n$ gedefinieerd worden. Indien we niet weten of een variabele in een bepaalde toestand zal gedefinieerd worden
\end{itemize}
We dienen vervolgens de volgende twee sets te berekenen per toestand:
\begin{itemize}
 \item $\instt{n}$: een variabele zit in $\instt{n}$ voor een toestand $n$ wanneer er een uitvoer-pad bestaat waarbij de variabele levend is bij het binnenkomen van die toestand.
 \item $\outstt{n}$: een variabele zit in $\outstt{n}$ voor een toestand $n$ wanneer er een uitvoer-pad bestaat waarbij de variabele levend is bij het verlaten van die toestand.
\end{itemize}
Het algoritme die deze sets kan berekenen staat in \algoref{alg:calculatingLiveness}.
\begin{algorithm}[hbt]
\caption{Berekenen van liveness.}\label{alg:calculatingLiveness}
\begin{algorithmic}[1]
\Function{Liveness}{$\succst{n}$,$\usestt{n}$,$\defstt{n}$}
\ForAll{$n$}
\State $\instt{n}\gets\emptyset$
\State $\outstt{n}\gets\emptyset$
\EndFor
\Repeat
\ForAll{$n$}
\State $\inastt{n}\gets\instt{n}$
\State $\outastt{n}\gets\outstt{n}$
\State $\instt{n}\gets\usestt{n}\cup\left(\outstt{n}\setminus\defstt{n}\right)$
\State $\outstt{n}\gets\displaystyle\cup_{s\in\succst{n}}\instt{s}$
\EndFor
\Until{$\forall n: \inastt{n}=\instt{n}\wedge\outastt{n}=\outstt{n}$}
\State \Return $\left(\instt{n},\outstt{n}\right)$
\EndFunction
\end{algorithmic}
\end{algorithm}
Het algoritme werkt op basis van least-fixed-point theorie. Hierbij berekenen we de sets door ze te initialiseren als lege sets. En vervolgens deze telkens te updaten. Wanneer we een iteratie bereiken waarbij geen enkele set aangepast wordt, weten we dat dit in de volgende iteratie ook niet zal gebeuren. Bijgevolg kunnen we stoppen. In tabel \ref{tbl:lifenessEvolution} staan de $\succst{n}$, $\usestt{n}$ en $\defstt{n}$ tabellen samen met de nodige iteraties om de $\instt{n}$ en $\outstt{n}$ set uit te rekenen. We kunnen opmerken dat de $\instt{n}$ set dus overeenkomt met de levensduurtabel (zie tabel \ref{tbl:variableLifespanExample}).
\begin{table}[hbt]
\centering
\small{\begin{tabular}{c|ccc|cc|cc|cc}
$n$&$\succst{n}$&$\usestt{n}$&$\defstt{n}$&$\instti{n}{0}$&$\outstti{n}{0}$&$\instti{n}{1}$&$\outstti{n}{1}$&$\instti{n}{2}$&$\outstti{n}{2}$\\\hline
$S_0$&\accol{S_0,S_1}&$\emptyset$&\accol{t_1,t_2}&	$\emptyset$&		\accol{t_1,t_2}&	$\emptyset$&		\accol{t_1,t_2}&	$\emptyset$&		\accol{t_1,t_2}\\
$S_1$&\accol{S_2}&\accol{t_1,t_2}&\accol{t_3,t_4}&	\accol{t_1,t_2}&	\accol{t_3,t_4}&	\accol{t_1,t_2}&	\accol{t_3,t_4}&	\accol{t_1,t_2}&	\accol{t_3,t_4}\\
$S_2$&\accol{S_3}&\accol{t_3,t_4}&\accol{t_5,t_6}&	\accol{t_3,t_4}&	\accol{t_5,t_6}&	\accol{t_3,t_4}&	\accol{t_5,t_6}&	\accol{t_3,t_4}&	\accol{t_5,t_6}\\
$S_3$&\accol{S_4}&\accol{t_5,t_6}&\accol{t_7,t_8}&	\accol{t_5,t_6}&	\accol{t_5,t_7}&	\accol{t_5,t_6}&	\accol{t_5,t_7,t_8}&	\accol{t_5,t_6}&	\accol{t_5,t_7,t_8}\\
$S_4$&\accol{S_5}&\accol{t_5,t_7}&\accol{t_9}&	\accol{t_5,t_7}&	\accol{t_8,t_9}&	\accol{t_5,t_7,t_8}&	\accol{t_5,t_8,t_9}&	\accol{t_5,t_7,t_8}&	\accol{t_5,t_8,t_9}\\
$S_5$&\accol{S_6}&\accol{t_8,t_9}&\accol{t_{10}}&	\accol{t_8,t_9}&	\accol{t_5,t_{10}}&	\accol{t_5,t_8,t_9}&	\accol{t_5,t_{10}}&	\accol{t_5,t_8,t_9}&	\accol{t_5,t_{10}}\\
$S_6$&\accol{S_7}&\accol{t_5,t_{10}}&\accol{t_{11}}&	\accol{t_5,t_{10}}&	\accol{t_{11}}&		\accol{t_5,t_{10}}&	\accol{t_{11}}&		\accol{t_5,t_{10}}&	\accol{t_{11}}\\
$S_7$&\accol{S_0}&\accol{t_{11}}&$\emptyset$&	\accol{t_{11}}&		$\emptyset$&		\accol{t_{11}}&		$\emptyset$&		\accol{t_{11}}&		$\emptyset$
\end{tabular}}
\caption{De evolutie van de $\instt{n}$- en $\outstt{n}$-set op basis van het leidend voorbeeld (zie figuur \ref{fig:asmSqrt})}
\label{tbl:lifenessEvolution}
\end{table}
\paragraph{Nut}Men kan zich terecht afvragen wat het nut is van het opstellen van zo'n tabel. Wanneer welke variabele actief is, is niet zo nuttig. Het interessante aspect zit hem eerder in de onderste rij: het aantal actieve variabelen per toestand. Dit getal specificeert immers hoeveel registers we in die toestand eigenlijk nodig hebben om de variabelen te bewaren. De overige variabelen dienen we immers niet meer op te slaan, ze worden verder nergens meer gebruikt. Vermits we in elke toestand slechts een aantal registers nodig hebben, is het totaal aantal benodigde registers niets anders dan het maximum van de onderste rij over alle toestanden. In het geval van het leidend voorbeeld is dit dus:
\begin{equation}
\begin{array}{l|l}
N_{\small{\mbox{reg.}}}=\max{\left(1,2,2,2,3,3,2,0\right)}=3&\mbox{(leidend voorbeeld)}
\end{array}
\end{equation}
We kunnen dus onze schakeling implementeren met drie registers op voorwaarde dat een register de waarde van verschillende variabelen op verschillende tijdstippen bijhoudt. Een eenvoudige methode is aan elk van deze drie registers een lijst met variabelen associ\"eren die dit register zal bijhouden. De enige voorwaarde is dat de variabelen in geen enkele toestand allebei actief zijn. Deze methode zal sowieso tot het minimaal aantal registers leiden, maar introduceert een nieuw probleem: we zullen multiplexers voor de ingang van deze registers moeten plaatsen en bovendien lopen we een gelegenheid mis om multiplexers te elimineren die de waarde van een register doorgeven aan een functionele eenheid. Daarom zullen we extra tabellen opstellen.
\paragraph{Functionele-eenhedentabel}
Een tweede tabel die we zullen opstellen is de \termen{functionele-eenhedentabel}. Deze tabel bevat horizontaal de toestanden en verticaal de types functionele eenheden. In elke cel schrijven we in, hoeveel functionele eenheden van het specifieke type we in de specifieke toestand nodig hebben. Zo zien we bijvoorbeeld op figuur \ref{fig:asmSqrt} dat we in toestand $S_2$ zowel een component nodig hebben die het minimum berekend en component die het maximum berekend. In toestand $S_1$ hebben we twee functionele eenheden nodig die elk de absolute waarde berekenen. Indien we dit voor alle toestanden doen, bekomen we tabel \ref{tbl:functionalunittableexample}.
\begin{table}[hbt]
\centering
\begin{tabular}{c|cccccccc|c}
	&$S_0$	&$S_1$	&$S_2$	&$S_3$	&$S_4$	&$S_5$	&$S_6$	&$S_7$	&$\#$\\\hline
abs	&	&2	&	&	&	&	&	&	&2\\
max	&	&	&1	&	&	&	&1	&	&1\\
min	&	&	&1	&	&	&	&	&	&2\\
shr	&	&	&	&2	&	&	&	&	&2\\
$-$	&	&	&	&	&1	&	&	&	&1\\
$+$	&	&	&	&	&	&1	&	&	&1\\\hline
$\#$	&0	&2	&2	&2	&1	&1	&1	&0	&9\\
\end{tabular}
\caption{Functionele-eenhedentabel van het leidend voorbeeld (zie figuur \ref{fig:asmSqrt})}
\label{tbl:functionalunittableexample}
\end{table}
De enige potenti\"ele moeilijkheid aan het opstellen van deze tabel is de definitie van ``type functionele eenheid''. Men kan zich afvragen of bijvoorbeeld een shift over 3 bits bijvoorbeeld tot hetzelfde type behoort als een shift over 1 bit. In geval van twijfel kan het geen kwaad om een onderscheid te maken (en dus een rij in de tabel te voorzien voor $\mbox{shr1}$ en $\mbox{shr3}$). Als algemene definitie kunnen we stellen dat twee functionele eenheden tot hetzelfde type behoren wanneer we hardware kunnen voorzien die beide operaties op hetzelfde component uitvoert. Indien we dus beslissen om de shift te implementeren met behulp van de verbindingen, is een shift over drie bits een ander operatie dan een shift over 1 bit. Wanneer we echter met een schuifregister werken, kunnen beide operaties op hetzelfde schuifregister worden uitgevoerd.
\paragraph{Verbindingentabel, Invoer- en uitvoertabellen}
Een andere tabel is de zogenaamde \termen{verbindingentabel}. Deze tabel bevat horizontaal de variabelen, en verticaal de functionele eenheden (niet verwarren met ``type functionele eenheid''). In een cel plaatsen we een ``I'' wanneer de variabele als invoer dient voor de functionele eenheid, en ``O'' wanneer het resultaat van deze functionele eenheid wordt weggeschreven in de variabele. We gaan er hier altijd van uit dat we in de verschillende toestanden verschillende functionele eenheden gebruiken. Hierdoor ligt het aantal functionele eenheden in deze tabel meestal hoger dan het uiteindelijke aantal functionele eenheden in de implementatie. In deze cursus zullen we de functionele eenheden dan ook nummeren. Zo maken we in toestand $S_1$ gebruik van twee functionele eenheden die de absolute waarde berekenen: abs1 en abs2. Hierbij zorgen $t_1$ en $t_2$ respectievelijk voor de invoer bij abs1 en abs2. De resultaten worden weggeschreven naar respectievelijk $t_3$ en $t_4$. Daarom noteren we in de tabel bij
cellen \brak{t_1,\mbox{abs1}} en \brak{t_2,\mbox{abs2}} een ``I'' en in cellen \brak{t_3,\mbox{abs1}} en \brak{t_4,\mbox{abs2}} een ``O''. Indien we dit voor alle toestanden doen, bekomen we tabel \ref{tbl:connectiontableexample}.
\begin{table}[hbt]
\centering
\begin{tabular}{c|ccccccccccc}
	&$t_1$	&$t_2$	&$t_3$	&$t_4$	&$t_5$	&$t_6$	&$t_7$	&$t_8$	&$t_9$	&$t_{10}$	&$t_{11}$\\\hline
abs1	&I	&	&O	&	&	&	&	&	&	&		&\\
abs2	&	&I	&	&O	&	&	&	&	&	&		&\\
max1	&	&	&I	&I	&O	&	&	&	&	&		&\\
min	&	&	&I	&I	&	&O	&	&	&	&		&\\
shr 3	&	&	&	&	&I	&	&O	&	&	&		&\\
shr 1	&	&	&	&	&	&I	&	&O	&	&		&\\
$-$	&	&	&	&	&I	&	&I	&	&O	&		&\\
$+$	&	&	&	&	&	&	&	&I	&I	&O		&\\
max2	&	&	&	&	&I	&	&	&	&	&I		&O
\end{tabular}
\caption{Verbindingentabel van het leidend voorbeeld (zie figuur \ref{fig:asmSqrt})}
\label{tbl:connectiontableexample}
\end{table}
Als laatste stellen we ook een set \termen{invoer- en uitvoertabellen} op. Deze tabellen worden gedeeltelijk afgeleid uit de verbindingentabel (in dit voorbeeld tabel \ref{tbl:connectiontableexample}). In deze tabel hebben we enkel een onderscheid gemaakt tussen invoer en uitvoer. Sommige functionele elementen hebben echter verschillende ingangen. Door variabelen toe te wijzen aan een specifieke ingang maken we de tabel dus concreter. In sommige gevallen staat deze ingang vast. In toestand $S_4$ trekken we bijvoorbeeld $t_7$ van $t_5$ af. Het spreekt voor zich dat we dus $t_5$ verbinden met de eerste ingang en $t_7$ met de tweede ingang. Bij sommige functionele eenheden kunnen we echter kiezen. In dat geval is de operatie die door de functionele eenheid wordt uitgevoerd ``commutatief''. Formeler kunnen we stellen dat een functie $f$ commutatief is indien:
\begin{equation}
f\left(\vec{x}\right)=f\left(\sigma\vec{x}\right)\mbox{ voor elke permutatie $\sigma$}
\end{equation}
Concrete voorbeelden van zulke operaties zijn optellen, minimum en maximum. Sommige functionele eenheden hebben meer dan twee ingangen. Ook hier kan er commutativiteit spelen. Soms is de operatie volledig commutatief en is dus elke permutatie geldig. In andere gevallen is er sprake van gedeeltelijke commutativiteit en wordt slechts een subset van permutaties ondersteund. Het toewijzen van ingangen kunnen we natuurlijk willekeurig doen. Toch loont het de moeite om dit op een intelligentere manier te doen. In het leidend voorbeeld (zie figuur \ref{fig:asmSqrt}) zien we bijvoorbeeld dat we zowel in toestand $S_2$ en $S_6$ een maximum berekenen. De kans is relatief groot, dat we in de toekomst deze twee bewerkingen op dezelfde functionele eenheid zullen uitvoeren. In toestand $S_2$ berekenen we $\max\left(t_3,t_4\right)$, in toestand $S_6$ $\max\left(t_5,t_{10}\right)$. Indien we echter in toestand $S_6$, $\max\left(t_5,t_3\right)$ hadden willen uitrekenen, was het zinvol geweest om deze operatie aan te passen
naar $\max\left(t_3,t_5\right)$. Indien we dan later de twee functionele eenheden zouden samenvoegen, dienen we geen multiplexer voor de eerste ingang te plaatsen. In het algemeen geldt meestal volgende regel:
\begin{quote}
Wijs dezelfde variabele zoveel mogelijk aan dezelfde ingangen en uitgangen toe.
\end{quote}
Bij wijze van voorbeeld zullen we deze regel toepassen op het leidend voorbeeld. Hierbij zijn variabelen $t_1$,$t_2$, $t_6$ triviaal: ze worden uitsluitend in functionele eenheden met maar \'e\'en ingang gebruikt. Daarom kennen we dus deze ingangen toe aan:
\begin{equation}
\begin{array}{lr}
\left(\left\langle t_1,\mbox{abs1}\right\rangle,\left\langle t_2,\mbox{abs2}\right\rangle,\left\langle t_6,\mbox{shr 1}\right\rangle\right)\mapsto\left(I_1,I_1,I_1\right)&\mbox{(leidend voorbeeld)}
\end{array}
\end{equation}
Vervolgens kunnen we $t_5$ en $t_7$ toewijzen: beide variabelen zijn immers betrokken in een aftrekking. In deze bewerking is $t_5$ verplicht gebonden aan ingang $I_1$, $t_7$ is gebonden aan ingang $t_7$:
\begin{equation}
\begin{array}{lr}
\left(\left\langle t_5,-\right\rangle,\left\langle t_7,-\right\rangle\right)\mapsto\left(I_1,I_2\right)&\mbox{(leidend voorbeeld)}
\end{array}
\end{equation}
We kijken eerst of we de toewijzingen van $t_5$ en $t_7$ kunnen uitbreiden. $t_5$ is verder nog betrokken in $\mbox{shr 3}$ en $\mbox{max2}$, dit zijn beide commutatieve functionele eenheden. We zullen dus ook voor deze operaties $t_5$ toekennen aan ingang $I_1$. $t_7$ is in geen enkele andere operatie betrokken. We vatten samen:
\begin{equation}
\begin{array}{lr}
\left(\left\langle t_5,\mbox{shr 3}\right\rangle,\left\langle t_5,\mbox{max2}\right\rangle\right)\mapsto\left(I_1,I_1\right)&\mbox{(leidend voorbeeld)}
\end{array}
\end{equation}
Nu we $t_7$ aan deze functionele eenheden hebben toegewezen, kunnen we de andere variabelen die in deze operaties betrokken zijn toewijzen: $\mbox{max2}$ heeft een twee ingangen en aan de andere ingang dient $t_{10}$ te staan:
\begin{equation}
\begin{array}{lr}
\left(\left\langle t_{10},\mbox{max2}\right\rangle\right)\mapsto\left(I_2\right)&\mbox{(leidend voorbeeld)}
\end{array}
\end{equation}
Voor de overige variabelen hebben we niet meteen een aanwijzing. Daarom zullen we deze eenvoudigweg aan een ingang toewijzen. Zo is $t_3$ betrokken in $\mbox{max1}$ en $\mbox{min}$. Bij beide operaties kennen we $t_3$ toe aan $I_1$. Een logisch gevolg is dat $t_4$ toegewezen wordt aan ingang $I_2$. $t_8$ wijzen we ook arbitrair toe aan $I_1$ voor de $+$ operatie, bijgevolg wijzen we $t_9$ toe aan $I_2$.
\begin{equation}
\begin{array}{lr}
\left(\left\langle t_3,\mbox{max1}\right\rangle,\left\langle t_3,\mbox{min}\right\rangle,\left\langle t_4,\mbox{max1}\right\rangle,\left\langle t_4,\mbox{min}\right\rangle,\left\langle t_8,+\right\rangle,\left\langle t_9,+\right\rangle\right)\mapsto\left(I_1,I_1,I_2,I_2,I_1,I_2\right)&\mbox{(leidend voorbeeld)}
\end{array}
\end{equation}
Vermits elke eenheid maar \'e\'en uitgang heeft kunnen we ook deze aspecten triviaal oplossen:
\begin{equation}
\begin{array}{lr}
\left(\begin{array}{c}
\left\langle t_3,\mbox{abs1}\right\rangle\\
\left\langle t_4,\mbox{abs2}\right\rangle\\
\left\langle t_5,\mbox{max1}\right\rangle\\
\left\langle t_6,\mbox{min}\right\rangle\\
\left\langle t_7,\mbox{shr 3}\right\rangle\\
\left\langle t_8,\mbox{shr 1}\right\rangle\\
\left\langle t_9,-\right\rangle\\
\left\langle t_{10},+\right\rangle\\
\left\langle t_{11},\mbox{max2}\right\rangle
\end{array}\right)\mapsto\left(\begin{array}{c}
O_1\\O_1\\O_1\\O_1\\O_1\\O_1\\O_1\\O_1\\O_1
\end{array}\right)&\mbox{(leidend voorbeeld)}
\end{array}
\end{equation}
Indien we nu alle voorgaande informatie samenvatten in een tabel, bekomen we de invoer- en uitvoertabellen zoals in tabel \ref{tbl:inputtableexample}.
\begin{table}[hbt]
\centering
\small{\begin{tabular}{c|cccccc|cccc|ccccccccc}
	&\multicolumn{6}{c|}{\bf In $I_1$}		&\multicolumn{4}{c|}{\bf In $I_2$}&\multicolumn{9}{c}{\bf Out $O_1$}\\
	&$t_1$	&$t_2$	&$t_3$	&$t_5$	&$t_6$	&$t_8$	&$t_4$	&$t_7$	&$t_8$	&$t_9$	&$t_3$	&$t_4$	&$t_5$	&$t_6$	&$t_7$	&$t_8$	&$t_9$	&$t_{10}$&$t_{11}$\\\hline
abs1	&$\bullet$&	&	&	&	&	&	&	&	&	&$\bullet$&	&	&	&	&	&	&	&\\
abs2	&	&$\bullet$&	&	&	&	&	&	&	&	&	&$\bullet$&	&	&	&	&	&	&\\
max1	&	&	&$\bullet$&	&	&	&$\bullet$&	&	&	&	&	&$\bullet$&	&	&	&	&	&\\
min	&	&	&$\bullet$&	&	&	&$\bullet$&	&	&	&	&	&	&$\bullet$&	&	&	&	&\\
shr 3	&	&	&	&$\bullet$&	&	&	&	&	&	&	&	&	&	&$\bullet$&	&	&	&\\
shr 1	&	&	&	&	&$\bullet$&	&	&	&	&	&	&	&	&	&	&$\bullet$&	&	&\\
$-$	&	&	&	&$\bullet$&	&	&	&$\bullet$&	&	&	&	&	&	&	&	&$\bullet$&	&\\
$+$	&	&	&	&	&	&$\bullet$&	&	&$\bullet$&	&	&	&	&	&	&	&	&$\bullet$&\\
max2	&	&	&	&$\bullet$&	&	&	&	&	&$\bullet$&	&	&	&	&	&	&	&	&$\bullet$
\end{tabular}}
\caption{Invoer- en uitvoertabellen van het leidend voorbeeld (zie figuur \ref{fig:asmSqrt})}
\label{tbl:inputtableexample}
\end{table}
We zetten hier de variabelen op de horizontale as, en de functionele eenheden op de verticale as. Variabelen die aan geen enkele functionele eenheid bijdragen hebben we weggelaten uit de kolommen: bijvoorbeeld variabele $t_1$ is nergens verbonden met een ingang $I_2$, bijgevolg staat deze ook niet in dat deel van de tabel. Het weglaten van deze kolommen is vanzelfsprekend optioneel.
\paragraph{Onsamenvoegbaarheid en Prioriteit}
Nu we deze tabellen hebben opgesteld, kunnen we met het echte werk beginnen: het samenvoegen van registers. Een eerste probleem die we moeten oplossen is bepalen wanneer twee variabelen eenzelfde register kunnen delen. Allereerst kunnen we opmerken dat een variabele eigenlijk niet gebonden is aan een register: we zouden bijvoorbeeld dezelfde variabele in de ene toestand in register 1 kunnen bewaren en de andere toestand in register 2. Dit kan er toe leiden dat we inderdaad tot zeer compacte configuraties komen, maar in de meeste gevallen, maakt het het optimaliseren hopeloos ingewikkeld. Daarom gebruiken we doorgaans de vuistregel:
\begin{quote}
We kennen een variabele toe aan strikt \'e\'en register.
\end{quote}
De vraag blijft in welke omstandigheden we twee variabelen niet in hetzelfde register kunnen opslaan. Een algemene regel is:
\begin{quote}
Twee variabelen kunnen niet in hetzelfde register worden opgeslagen indien de twee variabelen leven in \'e\'en of meer toestanden.
\end{quote}
We kunnen dit formaliseren met behulp van de sets die we berekend hebben in het deel over de lifeness analyse:
\begin{equation}
\forall t_i,t_j\in\mbox{Variabelen}, t_i\neq t_j:\mbox{NietSamenvoegen}\left(t_i,t_j\right)\Leftrightarrow\left(\exists s\in S: t_i\in\instt{s}\wedge t_j\in\instt{s}\right)
\end{equation}
Op basis van deze definitie kunnen we de volledige relatie berekenen van variabelen die we kunnen niet kunnen samenvoegen:
\begin{equation}
\begin{array}{ccc}
\brak{t_1,t_2}&\brak{t_3,t_4}&\brak{t_5,t_6}\\
\brak{t_5,t_7}&\brak{t_5,t_8}&\brak{t_5,t_9}\\
\brak{t_5,t_{10}}&\brak{t_7,t_8}&\brak{t_8,t_9}
\end{array}
\label{eqn:asmSqrtIncompatible}
\end{equation}
Toegegeven dat er negen paren van variabelen botsen, maar dit geeft ruimte voor een groot aantal manieren om variabelen samen te voegen. De schaduwzijde van het feit dat veel configuraties mogelijk zijn, is dat er een serieuze kans bestaat dat we een foute configuratie uitkiezen. Om meer gericht op zoek te gaan naar welke variabelen we kunnen samennemen, zullen we een heuristiek ontwikkelen: een functie die aangeeft in welke mate het de moeite loont om twee variabelen samen te nemen. Deze functie heeft de signatuur $h:V\times V\rightarrow\mathbb{N}$. Waarbij $V$ de set van variabelen voorstelt, en $\mathbb{N}$ de set van natuurlijke getallen (met 0). Een interessante metriek kunnen we opstellen op basis van de invoer- en uitvoertabellen (zie tabel \ref{tbl:inputtableexample}). De heuristiek defini\"eren we dan als:
\begin{quote}
Het aantal in- en uitgangen die de twee variabelen gemeenschappelijk hebben met elkaar bij functionele eenheden die het potentieel hebben om samengenomen te worden.
\end{quote}
De argumentatie is de volgende: indien we effectief functionele eenheden samennemen. Zal dit er meestal toe leiden dat verschillende registers naar eenzelfde ingang van deze samengestelde functionele eenheid worden geleid. Indien alle variabelen die deze ingang gemeenschappelijk hebben, echter in hetzelfde register opgeslagen zitten, zijn deze multiplexers overbodig. Formeel kunnen we deze functie dus defini\"eren als:
\begin{equation}
\forall t_i,t_j,t_i\neq t_j:\mbox{prioriteit}\left(t_i,t_j\right)=\#\left\{\begin{array}{l|l}
&\forall\brak{t_i,f_a,c},\brak{t_j,f_b,c}\in\mbox{Verbindingen}, a<b,\\
\brak{f_a,f_b,c}&\exists F\in\mbox{SamenvoegbareFunctioneleEenheden},\\
&f_a,f_b\in F
\end{array}\right\}
\label{eqn:priorityMergingVariables}
\end{equation}
Ook deze definitie zullen we toepassen op leidend voorbeeld. We verwachten dat we $\mbox{max1}$ en $\mbox{max 2}$ kunnen samennemen alsook $+$ en $-$. Daarom wordt de set \mbox{SamenvoegbareFunctioneleEenheden} gelijk aan:
\begin{equation}
\begin{array}{ll}
\mbox{SamenvoegbareFunctioneleEenheden}=\accol{\accol{\mbox{max1},\mbox{max2}},\accol{+,-}}&\mbox{(leidend voorbeeld)}
\end{array}
\end{equation}
De \mbox{Verbindingen}-set hebben we eigenlijk al opgesteld en kunnen we bekijken in de in- en uitvoertabellen (zie tabel \ref{tbl:inputtableexample}). De set bestaat uit tuples met als volgorde: \brak{\mbox{variable},\mbox{functionele eenheid},\mbox{verbinding}}. De \mbox{Verbindingen} bevat dus onder andere de volgende elementen:
\begin{equation}
\begin{array}{ll}
\mbox{Verbindingen}=\accol{\brak{t_1,\mbox{abs1},I_1},\brak{t_3,\mbox{abs1},O_1},\brak{t_2,\mbox{abs2},I_1},\brak{t_4,\mbox{abs1},O_1},\ldots}&\mbox{(leidend voorbeeld)}
\end{array}
\end{equation}
??
\begin{equation}
\begin{array}{lll}
\mbox{prioriteit}\brak{t_3,t_5}=1&\mbox{prioriteit}\brak{t_4,t_{10}}=1&\mbox{prioriteit}\brak{t_5,t_8}=1\\
\mbox{prioriteit}\brak{t_5,t_{11}}=1&\mbox{prioriteit}\brak{t_7,t_9}=1&\mbox{prioriteit}\brak{t_9,t_{10}}=1
\end{array}
\label{eqn:asmSqrtPriority}
\end{equation}
\paragraph{Opstellen van een compatibiliteitsgraaf}
Met de informatie uit de vorige paragraaf kunnen we een \termen{compatibiliteitsgraaf} opstellen. Een compatibiliteitsgraaf is een graaf waarbij de \termen{knopen} (ofwel \termen{nodes}) entiteiten voorstellen die al dan niet compatibel met elkaar zijn (in dit geval dus de variabelen). Verder bevat de graaf twee soorten bogen:
\begin{enumerate}
 \item \termen{Incompatibiliteitsranden}: dit zijn bogen tussen twee knopen die onverenigbaar met elkaar zijn. In het geval van de compatibiliteitsgraaf die we hier aan het opstellen zijn, zijn dit dus de paren uit vergelijking (\ref{eqn:asmSqrtIncompatible}). We stellen de bogen voor met stippellijnen.
 \item \termen{Prioriteitsranden}: dit zijn bogen tussen twee knopen die net goed verenigbaar zijn, en die dus waarschijnlijk een belangrijk voordeel opleveren wanneer ze samengevoegd worden. We hebben deze bogen berekend met de prioriteitsfunctie uit vergelijking (\ref{eqn:priorityMergingVariables}). In dit geval komt dit dus neer op de bogen in vergelijking (\ref{eqn:asmSqrtPriority}). We stellen de bogen voor met volle lijnen. Bovendien hebben de bogen natuurlijk een label: de waarde van de prioriteit.
\end{enumerate}
Een uitgewerkte compatibiliteitsgraaf voor het leidend voorbeeld staat op figuur \ref{fig:asmSqrtCompatibilityGraph} (de gevulde gebieden laten we voorlopig nog buiten beschouwing). De bedoeling van een compatibiliteitsgraaf is om een grafisch hulpmiddel te bieden om de variabelen optimaal samen te voegen. Op basis van de graaf moeten we op zoek gaan naar een \termen{max-cut graph partitioning}: het opdelen van de graaf in een minimaal aantal groepen zodat het totale gewicht van alle groepen samen maximaal is. Hierbij is het \termen{totale gewicht} van een groep de som van alle prioriteitsranden waarbij beide knopen in de groep zitten. De partitionering van de graaf dient verder volledig te zijn: elke knoop dient tot juist \'e\'en groep te behoren. Het bekomen van een max-cut graph partitioning is een niet-triviaal probleem en valt niet onder deze cursus. We zullen in de volgende paragraaf een methode ontwikkelen waarmee we grafisch tot een redelijke partitionering zullen komen.
\begin{figure}[hbt]
\centering
\begin{tikzpicture}[auto,node distance=2cm]
\node[compatibilitynode] (t1) {$t_1$};
\node[compatibilitynode,below of=t1] (t2) {$t_2$};
\node[compatibilitynode,right of=t1] (t3) {$t_3$};
\node[compatibilitynode,below of=t3] (t4) {$t_4$};
\node[compatibilitynode,right of=t3] (t11) {$t_{11}$};
\node[compatibilitynode,below of=t11] (t10) {$t_{10}$};
\node[compatibilitynode,right of=t11] (t5) {$t_5$};
\node[compatibilitynode,below of=t5] (t9) {$t_9$};
\node[compatibilitynode,right of=t9] (t7) {$t_7$};
\node[compatibilitynode,right of=t7] (t8) {$t_8$};
\node[compatibilitynode,above of=t8] (t6) {$t_6$};
\foreach \a/\b/\o in {t1/t2/,t3/t4/,t5/t10/,t5/t9/,t5/t7/,t5/t8/,t5/t6/,t7/t8/,t8/t9/bend left} {
  \path [thick,dashed] (\a) edge [\o] (\b);
}
\foreach \a/\b/\t/\o in {t3/t5/1/bend left,t11/t5/1/,t4/t10/1/,t10/t9/1/,t9/t7/1/} {
  \path [thick] (\a) edge [\o] node {\t} (\b);
}
\begin{pgfonlayer}{bg}
\foreach \tim/\off/\vars in {1/t5/,1/t7/,1/t8/,2/t5/{t3,t11},2/t8/{t6,t8},2/t7/{t4,t9,t10},3/t5/{t1,t3,t11},3/t7/{t2,t4,t9,t10}} {
  \node[fit=(\off),inner sep=3*\tim pt] (fitnode) {};
  \foreach \var in \vars {
    \node[fit=(\var),inner sep=3*\tim pt] (thick\var) {};
    \node[fit=(fitnode) (thick\var),inner sep=0pt] (fitnode) {};
  }
  \node [compatibilitygroup,fit=(fitnode)] {};
}
\end{pgfonlayer}
\end{tikzpicture}
\caption{De evolutie van de compatibiliteitsgraaf voor de variabelen.}
\figlab{asmSqrtCompatibilityGraph}
\end{figure}
\paragraph{Partitioneren van de compatibiliteitsgraaf}
Het bekomen van een optimale partitionering zullen we in drie stappen bewerkstelligen:
\begin{enumerate}
 \item Zoek een \termen{kliek}: een kliek is een groep knopen waarbij er tussen elke twee knopen een incompatibiliteitsrand zit. In het leidend voorbeeld kunnen we dus een kliek beschouwen met de knopen $\accol{\accol{t_5},\accol{t_7},\accol{t_8}}$. Deze variabelen vormen de oorsprong van de partities.
 \item \label{item:partitioningCompatibilityStepB} Voeg knopen toe op basis van de prioriteitsranden: we gaan de prioriteitsranden van groot naar klein af. Telkens wanneer er juist \'e\'en van de knopen al toegevoegd is aan een partitie zullen we proberen de andere knoop ook aan de partitie toe te voegen. Dit kan enkel wanneer er geen incompatibiliteitsrand tussen de knoop van de partitie en de knoop die we willen toevoegen zit. Telkens wanneer we erin slagen om zo'n knoop toe te voegen, verlopen we alle bogen opnieuw. We stoppen wanneer we bij geen enkele boog meer een knoop meer kunnen toevoegen. Indien twee bogen hetzelfde gewicht hebben, is de volgorde arbitrair. Toegepast op het voorbeeld kunnen we dus de volgorde volledig zelf kiezen. We kiezen de volgorde op basis van de knoop met de laagste index. In het geval beide bogen dezelfde kleinste knoop gemeenschappelijk hebben, dient de andere knoop als ``tiebreak''. We behandelen de knopen dus in volgende volgorde:
\begin{equation}
\begin{array}{ll}
\brak{t_3,t_5,1}<\brak{t_4,t_{10},1}<\brak{t_5,t_{11},1}<\brak{t_7,t_9,1}<\brak{t_9,t_{10},1}&\mbox{(leidend voorbeeld)}
\end{array}
\end{equation}
Vervolgens beginnen we met toevoegen. De eerste boog die aan de voorwaarden voldoet is $\brak{t_3,t_5,1}$, we voegen dus $t_3$ toe aan de partitie van $t_5$. Daarna volgt $t_{11}$ die eveneens aan $t_5$ wordt toegevoegd. De volgende bogen is $\brak{t_7,t_9,1}$, $\brak{t_9,t_{10},1}$ en $\brak{t_4,t_{10},1}$. We voegen dus $t_9$, $t_10$ en $t_4$ toe aan de partitie van $t_7$. Hiermee hebben we alle prioriteitsranden met succes kunnen oplossen. De partitie achter deze stap is dan ook: $\accol{\accol{t_3,t_5,t_{11}},\accol{t_4,t_7,t_9,t_{10}},\accol{t_8}}$.
 \item \label{item:partitioningCompatibilityStepC} Wijs de overige knopen toe: de laatste stap bestaat er uit om de knopen toe te wijzen die nog vrij zijn. Dit kunnen we arbitrair doen: we voegen een knoop toe aan een partitie naar keuze. Dit kan zolang er geen incompatibiliteitsrand tussen een knoop in de partitie en de knoop die we willen toevoegen zit. Indien dit wel zo is, zullen we een andere partitie moeten kiezen. In het geval voor iedere partitie er een incompatibiliteitsrand bestaat, richt de knoop zijn eigen partitie op. Vervolgens kunnen we weer naar stap \ref{item:partitioningCompatibilityStepB} terugkeren om eventuele prioriteitranden op te lossen met de nieuwe partitie. Om vervolgens weer stap \ref{item:partitioningCompatibilityStepC} uit te voeren. Zo zullen we in het leidend voorbeeld $t_1$ aan de partitie van $t_5$ toevoegen. $t_2$ kunnen we niet aan diezelfde partitie toevoegen, want tussen $t_1$ en $t_2$ ligt immers een incompatibiliteitsrand. Daarom voegen we $t_2$ toe aan de partitie van $t_7$. Tot slot dienen we nog $t_6$ toe te voegen. Hiervoor hebben we de keuze tussen de partities van $t_7$ en $t_8$. We kiezen hier voor de partitie van $t_8$, maar het alternatief is ook correct. Na deze stap hebben we alle knopen toegewezen en is de partitie de volgende:
\begin{equation}
\begin{array}{ll}
\accol{\accol{t_1,t_3,t_5,t_{11}},\accol{t_2,t_4,t_7,t_9,t_{10}},\accol{t_6,t_8}}&\mbox{(leidend voorbeeld)}
\end{array}
\end{equation}
\end{enumerate}
\paragraph{Minimalisering}
\begin{figure}[hbt]
\centering
\begin{sprocessor}[1.25/4.6/1.3/1.4/0.25/0.8]{t1/{$t_1$}/1/4,t2/{$t_2$}/1/5,t6/{$t_6$}/1/2}{abs1/$\abs$,abs2/$\abs$,max1/$\max$,min/$\min$,shr3/$\shrcmd3$,shr1/$\shrcmd1$,sub/$-$,add/$+$,max2/$\max$}{}
\sprbtf{t1}{abs1}{0};
\sprbtf{t2}{abs2}{-0.2};
\sprbtf{t1}{max1}{-0.3};
\sprbtf{t2}{max1}{0.3};
\sprbtf{t1}{min}{0};
\sprbtf{t2}{min}{0.3};
\sprbtf{t1}{shr3}{0};
\sprbtf{t6}{shr1}{0};
\sprbtf{t1}{sub}{-0.2};
\sprbtf{t2}{sub}{0.2};
\sprbtf{t6}{add}{-0.2};
\sprbtf{t2}{add}{0.2};
\sprbtf{t1}{max2}{-0.3};
\sprbtf{t2}{max2}{0};
\spfutr{max2}{t1}{0};
\spfutr{add}{t2}{0};
\spfutr{sub}{t2}{1};
\spfutr{shr1}{t6}{0};
\spfutr{shr3}{t2}{2};
\spfutr{min}{t6}{1};
\spfutr{max1}{t1}{1};
\spfutr{abs2}{t2}{3};
\spfutr{abs1}{t1}{2};
\end{sprocessor}
\caption{Implementatie van het datapad van het leidend voorbeeld na het minimaliseren van de variabelen.}
\figlab{datapad-minimal-variables}
\end{figure}
\figref{datapad-minimal-variables} toont de implementatie van het datapad nadat we de variabelen hebben gereduceerd. De figuur toont drie registers. Elke registers krijgt de naam van een van de element van de partitie\footnote{In werkelijkheid vertegenwoordigt het register alle variabelen in de partitie. We hadden het register $t_1$ dus ook bijvoorbeeld $t_3$ kunnen noemen.}. Vermits het aantal registers op drie staat, zal ook het aantal verbindingen van de registers naar de functionele eenheden gereduceerd worden. Een registers vertegenwoordigt een groep variabelen, bijgevolg trekken we een lijn van een register naar alle functionele eenheden die invoer uitlezen uit minstens \'e\'en van de vertegenwoordigde variabelen. De verbinding gebeurt uiteraard met de relevante invoerlijnen. Bijvoorbeeld het min component berekent het minimum tussen $t_3$ en $t_4$ respectievelijk op ingang 1 en 2. Omdat $t_3$ vertegenwoordigt wordt door het register $t_1$, verbinden we dit register met de eerste ingang. $t_4$ wordt dan weer vertegenwoordigt door register $t_2$. Bijgevolg verbinden we dit register met de tweede ingang.
\paragraph{}
Het samenvoegen van registers levert ook nog een ander probleem op: wat doet men met de uitvoer van de functionele eenheden. In \figref{datapad-generic} komt de uitvoer van een functionele eenheid altijd in \'e\'en register terecht. Omdat we hier de registers samenvoegen in drie algemene registers lukt dit niet. We kunnen de uitvoer van de functionele eenheden verbinden met het register die de relevante variabele vertegenwoordigt. We kunnen echter geen uitgangen combineren. Dit kan tot kortsluiting leiden en bovendien is het onduidelijk welke uitvoer er in dat geval in het register zou worden ingelezen. We kunnen dit probleem oplossen met behulp van multiplexers. Bij iedere register waar meerdere uitgangen van functionele eenheden toekomen plaatsen we een multiplexer met voldoende data-ingangen. De uitgang van de multiplexer verbinden we vervolgens met het register. Ten slotte zijn de selectie-ingangen van de multiplexer de verantwoordelijkheid van de controller. De controller dient bijgevolg te bepalen welke uit welke functionele eenheid de data moet worden ingelezen die in het register zal worden geplaatst.
\paragraph{}
Wanneer we een multiplexer introduceren, zal de controller extra logica moeten voorzien om de selectie-ingangen te besturen. Anderzijds verdwijnt de logica voor de set- en reset-ingangen van de registers die wegvallen. Ook moet de we logica voorzien voor de set- en reset-ingangen van de gegroepeerde registers. In deze paragraaf beschrijven we kort hoe we de logica bepalen voor de selectie-ingangen van de multiplexers. Als voorbeeld nemen we register $t_1$. Een eerste aspect is het voorzien van de set- en reset-ingangen. Indien \'e\'en van de oorspronkelijke registers zo'n ingang voorzag. Dus voor het uiteindelijke register $t_1$ moeten we enkel een set-ingang voorzien (verbonden met de controller). Dit komt omdat de originele registers voor $t_1$, $t_3$, $t_5$ en $t_{11}$ uitsluitend een set-ingang voorzien. Een tweede aspect is wanneer set- en reset-ingangen actief moeten worden. Omdat een samengevoegd register alle oorspronkelijke registers vertegenwoordigt, moeten de set- en reset-ingangen actief zijn, wanneer dit bij minstens \'e\'en van de originele registers het geval was. Concreet laadt het originele register $t_1$ een waarde in, in toestand $0$, $t_3$ doet dit in toestand $1$, $t_5$ in toestand $2$ en $t_{11}$ ten slotte in toestand $6$. Bijgevolg moet de set-ingang van het samengevoegde register $t_1$ dus actief zijn in toestanden $0$, $1$, $2$ en $6$. In alle andere toestand moet het signaal laag zijn. Hetzelfde geldt voor (hier niet aanwezige) reset-ingangen. Indien we de controller dus minimaal willen aanpassen kunnen we een OR-poort plaatsen die als invoer de uitvoer naar de register neemt en zelf het signaal op de set- en reset-ingangen van de samengevoegde registers bepaald. In de praktijk zal het opnieuw synthetiseren van de logica goedkoper zijn. Een tweede type logica die we moeten synthetiseren is het signaal op de selectie-ingangen. We weten dat het originele register $t_1$ in toestand $0$ data moest inladen die op dat moment op data-ingang $A$ van de processor staat. Deze data staat op $11$ bij de multiplexer. Bijgevolg moet op dat moment ook deze waarde aan de selectie-ingangen van multiplexer staan. Analoog staat in toestand $1$, $10$ op de selectie-ingangen. In toestand $2$ is het signaal $01$, en in toestand $6$ is dit $00$. In de andere toestanden moeten we natuurlijk ook een signaal op de selectie-ingangen zetten. Omdat de waarde die de multiplexer doorgeeft toch niks uitmaakt, kunnen we hiervoor gebruik maken van don't cares. Meestal maakt men multiplexers waarbij het aantal data-ingangen machten van twee is. Wanneer we er echter minder data-ingangen gebruikt worden, kunnen we op verschillende ingangen dezelfde data aanleggen. In zo'n situatie kunnen we dan kiezen tussen \'e\'en van de data-ingangen die het signaal aanlegt. Dit leidt dus tot meer don't cares en bijgevolg een mogelijk nog compactere implementatie. Zo zien we dat bij de multiplexer van $t_2$ we meermaals data-ingang $B$ aanleggen. Dit zorgt ervoor dat behalve de hoogste bit, we vrije keuze hebben welke waarde wordt aangelegd. \tblref{datapad-minimal-variables-controller} geeft de stuursignalen van alle componenten weer.
\begin{table}[hbt]
\centering
\begin{tabular}{c|ccc|cccc|cc}
Toestand&$a$&$b$&$c$&$d$&$e$&$f$&$g$&$h$&$i$\\\hline
0&1&1&1&1&-&-&1&0&-\\
1&1&0&1&1&1&1&0&0&-\\
2&1&1&0&0&-&-&-&1&1\\
3&0&-&-&1&0&1&0&1&0\\
4&0&-&-&1&1&0&0&0&-\\
5&0&-&-&1&0&0&0&0&-\\
6&1&0&0&0&-&-&-&0&-\\
7&0&-&-&0&-&-&-&0&-
\end{tabular}
\caption{Stuursignalen voor de verschillende toestanden na samenvoegen van registers.}
\tbllab{datapad-minimal-variables-controller}
\end{table}
\paragraph{}
Het samenvoegen van registers gaat dus gepaard met de introductie van multiplexers en extra combinatorische logica in de controller. Dit is een van de redenen waarom het moeilijk is om te voorspellen of een optimalisatie effectief tot goedkopere implementaties zal leiden.
\subsubsection{Bewerkingen samenvoegen (``functional-unit sharing'')}
In het leidend voorbeeld zien we verschillende functionele eenheden. Sommige van deze functionele eenheden voeren echter dezelfde bewerking in een andere toestand uit. Het is mogelijk goedkoper deze bewerkingen met dezelfde functionele eenheid uit te voeren. Anderzijds zijn sommige functionele eenheden zeer gelijkaardig. In \sscref{??} hebben we de implementatie van een schakeling besproken die getallen kan optellen en aftrekken. Functionele eenheden die gelijkaardige bewerkingen uitvoeren kunnen meestal worden samengevoegd in schakelingen die beide bewerkingen kunnen uitvoeren, en waarbij de resulterende schakeling goedkoper is dan de som van de twee aparte functionele eenheden. Het samenvoegen van functionele eenheden wordt \termen{functional-unit sharing} genoemd omdat een functionele eenheid tussen verschillende toestanden wordt gedeeld.
\paragraph{Gelijkaardige bewerkingen samenvoegen}
Een probleem in dit proces is dat het samenvoegen van twee functionele eenheden in een nieuwe functionele eenheid geen sinecure is, en meestal enige ervaring vereist. We zullen in doorheen deze subsectie verschillende functionele eenheden combineren om de lezer een notie te geven over de principes hierachter. Meestal zoekt men naar logica die beide functionele eenheden gemeenschappelijk hebben. Daarna moeten extra ingangen bepalen welke bewerking wordt uitgevoerd door de overige logica aan te sturen.
\paragraph{Compatibiliteitsgrafe}
Eenmaal we de kostprijs berekend hebben van het samenvoegen van verschillende functionele eenheden, dienen we een keuze te maken welke bewerkingen we effectief zullen samenvoegen. Hiervoor zullen we dezelfde werkwijze met de compatibiliteitsgrafe hanteren. In de compatibiliteitsgrafe hebben de componenten de volgende betekenis:
\begin{enumerate}
 \item Knopen: de knopen stellen bewerkingen voor, dus de originele functionele eenheden.
 \item Incompatibiliteitsranden: twee bewerkingen zijn niet verenigbaar, wanneer ze een toestand gemeenschappelijk hebben waarin ze beiden actief zijn.
 \item Prioriteitsranden: twee of meer\footnote{Soms kan men drie of meer bewerkingen samenvoegen in \'e\'en nieuwe functionele eenheid.} bewerkingen die men kan samenvoegen delen een prioriteitsrand. Het gewicht is de winst in kostprijs (transistors of logische blokken) bij het samenvoegen.
\end{enumerate}
\figref{compatibilitygraph-functionals-original} toont de initi\"ele compatibiliteitsgraaf met de incompatibiliteitsranden.
\begin{figure}[hbt]
\centering
\begin{tikzpicture}[auto,node distance=2cm]
\node[compatibilitynode] (f1) {abs1};
\node[compatibilitynode,below of=f1] (f2) {abs2};
\node[compatibilitynode,right of=f1] (f3) {min};
\node[compatibilitynode,below of=f3] (f4) {max1};
\node[compatibilitynode,right of=f4] (f5) {max2};
\node[compatibilitynode,right of=f5] (f9) {+};
\node[compatibilitynode,above of=f9] (f8) {-};
\node[compatibilitynode,right of=f8] (f6) {shr 3};
\node[compatibilitynode,right of=f9] (f7) {shr 1};
\foreach \a/\b/\o in {f1/f2/,f3/f4/,f6/f7/} {
  \path [thick,dashed] (\a) edge [\o] (\b);
}
\end{tikzpicture}
\caption{Oorspronkelijke compatibiliteitsgrafe bij het samenvoegen van bewerkingen.}
\figlab{compatibilitygraph-functionals-original}
\end{figure}
\paragraph{Multiplexers}
Wanneer twee functionele eenheden worden samengevoegd, delen ze een deel van de ingangen. Het is mogelijk dat de ingangen van beide functionele eenheden verbonden zijn met andere registers. In dat geval moeten extra multiplexers bepalen welk register de invoer levert voor de functionele eenheid. Multiplexers brengen echter een extra kost met zich mee. Zelfs wanneer de samengestelde functionele eenheid goedkoper is, zal de resulterende schakeling daarom niet noodzakelijk goedkoper zijn. De extra kosten die met het invoeren of aanpassen van multiplexers gepaard gaan, moeten dus ook in rekening worden gebracht. \figref{minimal-functionals-multiplexer} illustreert dit concept. Aanvankelijk zijn er twee functionele eenheden $\mbox{FU1}$ en $\mbox{FU2}$. Elk met eigen in- en uitgangen. Wanneer we de functionele eenheden in een nieuwe functionele eenheid $\mbox{FU1\&2}$ onderbrengen, introduceren we twee multiplexers. Ook de controller zal extra stuursignalen moeten implementeren: het bedienen van de multiplexers vereist extra logica. Soms zal de samengestelde functionele eenheid ook extra stuursignalen vereisen. Dit is het geval wanneer de twee functionele eenheden niet dezelfde bewerking uitvoeren, en de controller dus moet bepalen welke opdracht zal worden uitgevoerd. Indien sommige registers dezelfde zijn (bijvoorbeeld $A$ en $C$ op de figuur), kunnen we multiplexers weglaten.
\importtikzfigure{minimal-functionals-multiplexer}{Minimaliseren van bewerkingen introduceert multiplexers.}
\paragraph{Minimum en Maximum bewerking}
In het leidend voorbeeld berekenen we het minimum en maximum. Deze schakelingen hebben we niet gedefinieerd in \chpref{combinatoric}. De schakeling is echter vrij eenvoudig: we berekenen eerst het verschil tussen de twee getallen. Indien het verschil positief is, zullen we met behulp van multiplexers het tweede getal teruggeven, in het andere geval geven we de waarde terug op de eerste invoer. In het geval van het maximum draaien we de multiplexer om. Een schematische voorstelling staat op \figref{minimal-functionals-minimum}
\begin{figure}[hbt]
\centering
\subfigure[Minimum]{\importtikz{minimal-functionals-minimum}}
\subfigure[Carry-logica]{\importtikz{minimal-functionals-carry}}
\subfigure[Minimax]{\importtikz{minimal-functionals-minimax}}
\caption{Implementatie van een minimum-component.}
\end{figure}
geeft dit proces schematisch weer. Vermits we bij het verschil enkel ge\"interesseerd zijn in het teken, hoeven we de overige bits niet uit te rekenen. Bijgevolg moeten we ook enkele de carry-logica implementeren. \figref{minimal-functionals-carry} toont een mogelijke implementatie van deze carry-logic. De schakeling is bijgevolg een gereduceerde versie van \figref{fulladder-implementation}. Wanneer we de kostprijs van de carry-logic uitrekenen komen we uit op 20 transistors per bit\footnote{Op een constante kostprijs na, die we hier niet beschouwen. Als het aantal bits vrij groot is, maakt de constante kostprijs meestal niet veel uit.}. Een multiplexer kost 12 transistors per bit. Bijgevolg kost een minimum- of maximum-component 32 transistors per bit.
\paragraph{}
De compatibiliteitsgrafe op \figref{compatibilitygraph-functionals-original} toont dat het samenvoegen van de $\mbox{min}$ en $\mbox{max1}$ bewerking niet mogelijk is. Beide bewerkingen worden immers in dezelfde toestand gebruikt. We kunnen echter wel $\mbox{max1}$ en $\mbox{max2}$ samenvoegen. Door het samenvoegen sparen we 32 transistors per bit uit. We dienen echter rekening te houden met de mogelijke introductie van multiplexers. In dit geval hebben we geluk: $\mbox{max1}$ neemt als invoer variabelen $t_3$ en $t_4$ voorgesteld door registers $t_1$ en $t_2$. $\mbox{max2}$ neemt als invoer variabelen $t_5$ en $t_{10}$ die ook worden opgeslagen in registers $t_1$ en $t_2$. In dit geval dienen we dus geen extra multiplexers te introduceren, de totale winst is dan ook 32 transistors per bit zoals aangeduid op \figref{compatibilitygraph-functionals-final}.
\paragraph{}
We kunnen er ook voor opteren om $\mbox{min}$ en $\mbox{max2}$ samen te voegen. \figref{minimal-functionals-minimax} toont een functionele eenheid die zowel het minimum als maximum kan uitrekenen. De schakeling verschilt slechts op \'e\'en punt van \figref{minimal-functionals-minimum}: er staat een XNOR\footnote{Een XOR-poort is ook correct indien het stuursignaal wordt ge\"inverteerd. Alleen kost een XOR-poort meer.}-poort tussen het tekenen van het verschil een de multiplexer. Met behulp van het stuursignaal kan men dus beslissen om het signaal uit het verschil om te draaien of te behouden. Door het signaal om te draaien, kan men dus de andere data-ingang bij de multiplexer selecteren. Omdat de XNOR-poort maar eenmalig in de schakeling moet worden verwerkt, stijgt de kostprijs per bit niet. Ook hier hebben we geluk met de registers: beide functionele eenheden halen hun invoer uit dezelfde registers. De totale winst is dus opnieuw 32 transistors per bit.
\paragraph{Abs\&max}
Een volgende combinatie die we uitproberen is het berekenen van een absolute waarde en het maximum. Omdat we in de vorige paragraaf besproken hebben dat het minimum en maximum zeer gelijkaardig zijn, gelden alle uitspraken ook voor een component $\mbox{abs\&min}$. Allereerst dienen we een component te introduceren die de absolute waarde kan uitrekenen. Een mogelijke implementatie staat op \figref{minimal-functionals-abs}.
\begin{figure}[hbt]
\centering
\subfigure[Abs]{\importtikz{minimal-functionals-abs}}
\subfigure[Abs per bit]{\importtikz{minimal-functionals-absbit}}
\subfigure[Abs\&max per bit]{\importtikz{minimal-functionals-absmax}}
\caption{Implementatie van abs en abs\&max.}
\end{figure}
De meest significante bit bepaald in complementvoorstelling het teken van het getal. Het teken stuurt vervolgens de selectie-ingang van een multiplexer aan. Indien het getal negatief is, wordt het tegengestelde van het oorspronkelijke getal doorgelaten. In het andere geval wordt het getal zelf doorgelaten. We moeten per bit het tegengestelde uitrekenen en voor elke bit dienen we ook een multiplexer te voorzien. Deze implementatie per bit wordt voorgesteld op \figref{minimal-functionals-absbit}. Op basis van deze schakeling zetten we de kostprijs van een $\mbox{abs}$ bewerking op 32 transistors per bit. We kunnen beide componenten samenvoegen, door per bit een schakeling te implementeren zoals op \figref{minimal-functionals-absmax}. De component berekent ofwel het maximum van de twee data-ingangen, of de absolute waarde van de tweede data-ingang. De schakeling werkt als volgt: de full adders berekenen samen met de NOT-poort het verschil tussen $D$ (voorgesteld door de $d_i$-bits) en $B$ (voorgesteld door de $b_i$-bits), het resultaat is een getal $S$ (voorgesteld door de $s_i$-bits). Indien we de absolute waarde berekenen (dit betekent dat het signaal $\mbox{Max/Abs*}$ laag staat), is $D$ gelijk aan $0$, bijgevolg is $S=-B$. In het geval we het maximum berekenen is $S=A-B$. Indien we de absolute waarde berekenen, zullen we op basis van het teken van $B$ de multiplexer aansturen, en beslissen of we ofwel $B$ doorlaten (in het geval $B$ positief is), ofwel $S=-B$. Dit hangt louter af van de hoogste bit van $B$ zoals voorgesteld in \tblref{minimal-functionals-absmaxmux}. In het geval we het maximum berekenen zal het tekenen van $S$ de multiplexer aansturen. Indien de teken-bit $s_{n-1}$ laag is ($S$ is positief), laten we $A$ door. $S$ is immers positief wanneer $A\geq B$. In het andere geval laten we $B$ door. Op basis van de tekens van $B$, $S$ en de operatie die we willen uitvoeren (absolute waarde of maximum), moeten we dus de stuursignalen $m_0$ en $m_1$ voor de multiplexer bepalen. Deze stuursignalen zijn voor de multiplexers van alle bits gelijk. Omdat we de logica dus maar eenmalig moeten implementeren zullen we de kostprijs hiervan verwaarlozen. De kost per bit omvat echter de multiplexer (18 transistors per bit), de full adder (36 transistors per bit), de AND-poort (6 transistors per bit) en de NOT-poort (2 transistors per bit). In totaal maakt dit dus 62 transistors per bit.
\importtabulartable{minimal-functionals-absmaxmux}{Multiplexer selectie-ingangen bij abs\&max.}
\paragraph{}
Nu we een samengesteld component hebben ge\"implementeerd, kunnen we de mogelijke winst berekenen. Er zijn verschillende manieren hoe we een dergelijk component kunnen gebruiken. Hieronder beschouwen we alle mogelijkheden:
\begin{equation}
\acclarray{
\accl{\mbox{abs1},\mbox{min}},\accl{\mbox{abs1},\mbox{max1}},\accl{\mbox{abs1},\mbox{max2}}\\
\accl{\mbox{abs2},\mbox{min}},\accl{\mbox{abs2},\mbox{max1}},\accl{\mbox{abs2},\mbox{max2}}\\
\accl{\mbox{abs1},\mbox{max1},\mbox{max2}},\accl{\mbox{abs2},\mbox{max1},\mbox{max2}}}
\eqnlab{minimal-functionals-combinations}
\end{equation}
We zullen niet alle combinaties bespreken. De mogelijke winsten staan op \figref{compatibilitygraph-functionals-final}. Enkel de scenario's die interessant zijn worden besproken. Als eerste voorbeeld nemen we $\mbox{abs1}$ en $\mbox{max1}$. De twee functionele eenheden apart kosten $64$ transistors per bit. Wanneer we de twee functionele eenheden samenvoegen, dan verwachten we slechts $62$ transistors per bit te betalen. Een probleem vormt echter de registers: $\mbox{abs1}$ haalt de gegevens uit register $t_1$, terwijl $\mbox{max1}$ deze uit de registers $t_1$ en $t_2$ haalt. Vermits het samengestelde component voor de absolute waarde de tweede ingang neemt, betekent dit dat we een multiplexer moeten plaatsen die kiest tussen register $t_1$ en $t_2$. Dit concept is ge\"illustreerd in \figref{minimal-functionals-absmax-mux}.
\importtikzfigure{minimal-functionals-absmax-mux}{Het combineren van $\mbox{abs1}$ en $\mbox{max1}$ introduceert een multiplexer.}
We kunnen echter een eigenschap van het maximum uitbuiten: deze operator is commutatief. Het maakt dus niet uit of deze component het maximum van $t_1$ en $t_2$ berekent, of van $t_2$ en $t_1$. Door de bedrading van het maximum om te draaien, vervalt de noodzaak om een multiplexer in te voeren. De totale winst bij het combineren is dus $2$ transistors per bit. Ditzelfde principe kan men toepassen bij het samenvoegen van alle combinaties in \eqnref{minimal-functionals-combinations} met twee bewerkingen.
\paragraph{}
We kunnen ook drie functionele eenheden in een nieuwe functionele eenheid combineren. Vermits we twee functionele eenheden hebben die de absolute waarde uitrekenen en twee die het maximum uitrekenen kunnen we verschillende combinaties uitproberen. Dit kan zolang niet beide absolute waardes door dezelfde component worden uitgerekend. De twee bewerkingen zijn immers incompatibel. Als voorbeeld nemen we $\mbox{abs1}$, $\mbox{max1}$ en $\mbox{max2}$. Vermits $\mbox{max1}$ en $\mbox{max2}$ commutatieve operaties zijn maakt de volgorde van de data-ingangen niet uit. Omdat de bewerkingen die we combineren allemaal op dezelfde registers werken, zullen we dus geen extra multiplexers moeten introduceren. De oorspronkelijke kostprijs per bit was 32 transistors voor $\mbox{abs1}$, 32 transistors voor $\mbox{max1}$ en 32 voor $\mbox{max2}$. Wat dus samen $96$ transistors per bit betekent. De totale winst is dus $34$ transistors per bit. Een probleem is hoe we de winsten met drie of meer functionele eenheden in een compatibiliteitsgrafe voorstellen. We kunnen immers geen boog tussen drie knopen trekken. Twee bogen die de drie knopen verbinden kan dan weer tot verwarring leiden: de twee bogen kunnen als twee afzonderlijke prioriteitsranden beschouwd worden. Daarom voert men meestal een gevulde knoop die de prioriteitsrand voorstelt. Vanuit de knoop vertrekken bogen naar de bewerkingen op dewelke de prioriteitsrand betrekking heeft. Op de knoop zelf noteren we de winst. Deze notatie staat ook op \figref{compatibilitygraph-functionals-final}.
\paragraph{Add\&Sub}In \figrefpag{addsub-twocomplement-implementation} bespraken we reeds hoe we een component kunnen realiseren die zowel een optelling als een aftrekking kan uitvoeren. Op basis van een stuur-signaal kan men beslissen welke operatie uiteindelijk zal worden uitgevoerd. In deze paragraaf bespreken we de mogelijke winst die dit oplevert. Allereerst berekenen we de kostprijs per bit van een aparte opteller en aftrekker. Een opteller kost per bit een full adder, dit komt dus neer op een kostprijs van $36$ transistors per bit. Een aftrekker voorziet per bit een full adder en een NOT-poort. De kostprijs is bijgevolg $38$ transistors per bit. De totale kostprijs van de twee afzonderlijke componenten is dus $74$ transistors per bit. Bij een component die beide bewerkingen uitvoert vervangen we de NOT-poort door een XOR-poort zodat we kunnen kiezen of we de bits inverteren. De kostprijs per bit wordt dus opgedreven naar $48$ transistors per bit. Dit betekent dus een winst van $26$ transistors per bit. Wanneer we echter naar de bedrading kijken, zien we dat de optelling data uitleest uit registers $t_1$ en $t_2$. De aftrekking werkt met de registers $t_6$ en $t_2$. Bijgevolg moeten we een multiplexer introduceren die $12$ transistors per bit kost. De totale winst wordt dus $14$ transistors per bit.
\begin{figure}[hbt]
\centering
\subfigure[Opteller per bit]{\importtikz{minimal-functionals-addbit}}
\subfigure[Aftrekker per bit]{\importtikz{minimal-functionals-subbit}}
\subfigure[Opteller/aftrekker per bit]{\importtikz{minimal-functionals-addsubbit}}
\subfigure[Opteller/aftrekker per bit met multiplexer]{\importtikz{minimal-functionals-addsubbitmux}}
\caption{Implementatie van een opteller, aftrekker en opteller/aftrekker.}
\end{figure}
\paragraph{Add\&Max}We kunnen ook proberen om een optelling een het maximum in \'e\'en bewerking samen te voegen. Rekenkundig zijn beide bewerkingen niet direct met elkaar verwant. We kunnen echter op \figref{minimal-functionals-absmax} vaststellen dat we om het teken van het verschil te berekenen, full adders kunnen gebruiken. Vermits deze componenten de basis vormen van de optelling, is er een vermoeden dat in termen van poorten beide operaties een deel van de logica delen. \figref{minimal-functionals-addmaxbit}
\importtikzfigure{minimal-functionals-addmaxbit}{De implementatie van $\mbox{add\&max}$ per bit.}
toont de implementatie per bit voor een zo'n component. Indien we een optelling willen uitrekenen, zal de XOR-poort voor elke bit eenvoudigweg de waarde van $b_i$ doorlaten. Bijgevolg zullen de full adders de optelling van $A$ (voorgesteld door de $a_i$'s) en $B$ (voorgesteld door de $b_i$'s) uitrekenen. Het komt er op neer in dat stadion de waarde $S$ (voorgesteld door de $s_i$'s) bij alle multiplexers door te laten. In het geval we het maximum berekenen is het signaal $\mbox{Max/Add*}$ hoog. Bijgevolg doet de XOR-poort dienst als een NOT-poort voor $b_i$. Het gevolg is dat $S=A-B$. Het teken van $S$ wordt mee in de logica van de selectie-ingangen voor de multiplexer genomen en bepaald dus of we ofwel $A$ zullen doorlaten (indien $S\geq 0$) ofwel $B$ (indien $S<0$). Het bepalen van de selectie-ingangen van de multiplexer vereist natuurlijk ook logica. \tblref{minimal-functionals-addmaxmux}
\importtabulartable{minimal-functionals-addmaxmux}{Multiplexer selectie-ingangen bij $\mbox{add\&max}$.}
beschrijft in welke omstandigheden welke selectie-ingangen vereist zijn. Een eenvoudige implementatie die geen extra kosten met zich meebrengt is $m_1=\mbox{Max/Add*}$ en $m_1=s_{n-1}$.
\paragraph{}
De kosten van een $\mbox{add\&max}$-component bestaan uit drie delen: de multiplexer kost $18$ transistoren per bit, de XOR-poort $12$ transistoren per bit en de full adder $36$ transistoren per bit. Samen maakt dit dus $66$ transistoren per bit. Nu we de kostprijs berekend hebben, kunnen we bepalen wat een dergelijk component kan opleveren. We kunnen de component op drie verschillende manieren gebruiken:
\begin{equation}
\accl{\accl{+,\mbox{max1}},\accl{+,\mbox{max2}},\accl{+,\mbox{max1},\mbox{max2}}}
\end{equation}
Bij alle combinaties komen we hetzelfde probleem tegen: de opteller haalt data uit registers $t_6$ en $t_2$ terwijl de beide componenten die het maximum berekenen dit halen uit de registers $t_1$ en $t_2$. Doordat beide bewerkingen commutatief zijn kunnen we de ingangen schikken zoals we wensen. Het resultaat blijft echter dat we minimum $1$ multiplexer moeten introduceren aan $12$ transistors per bit. De kostprijs van een opteller is $36$ transistors per bit en het maximum kost $32$ transistors per bit. Indien we dus een opteller samenvoegen met \'e\'en maximum-component bekomen we dus een verlies van $10$ transistors per bit. Dit duidt ook meteen aan dat het samenvoegen van bewerkingen niet altijd goedkoper is. Het rendeert echter wel wanneer we de opteller combineren met twee maximum componenten. In dat geval maken we $22$ transistoren per bit winst. Deze winst staat ook op \figref{compatibilitygraph-functionals-final}. Verliezen worden niet op een compatibiliteitsgrafe gezet. Men zal immers dergelijke prioriteitsranden altijd vermijden omdat het de kostprijs alleen maar verder verhoogt.
\paragraph{Abs\&Add\&Max}Tot dusver hebben we telkens twee types bewerken samengevoegd, eventueel is er wel sprake van meerdere functionele eenheden. Het is echter ook mogelijk meer dan twee types bewerkingen samen te voegen in een nieuwe functionele eenheid. Componenten die drie of meer bewerkingen werden al kort besproken onder de naam ``Arithmetic Logic Unit (ALU)'' in \sscrefpag{alu}. Het valt op dat combinaties met de absolute waarde, de opteller en het maximum op \figrefs{minimal-functionals-absmax,minimal-functionals-addmaxbit} een soortgelijk patroon vertonen: een full adder met daaronder een multiplexer die ofwel het resultaat van de full adder ofwel dit van de originele ingangen doorlaat. Meestal zijn de ingangen van de full adder ook voorzien van poorten. Het samenvoegen van bewerkingen kan dus mogelijks de transistoren uitsparen die betrokken zijn in de full adder en de multiplexer. \figref{minimal-functionals-absaddmaxbit} toont de implementatie per bit van een component die zowel een optelling, absolute waarde en maximum kan uitrekenen.
\importtikzfigure{minimal-functionals-absaddmaxbit}{De implementatie van $\mbox{abs\&add\&max}$ per bit.}
Hiervoor werkt de component met twee stuursignalen: $\mbox{Abs}$ en $\mbox{Add}$. Indien $\mbox{Abs}$ hoog staat en $\mbox{Add}$ laag, wordt de absolute waarde uitgerekend. Bij $\mbox{Abs}$ laag en $\mbox{Add}$ hoog berekent de component de optelling. Wanneer beide laag staan wordt het maximum uitgerekend. Dit systeem werkt als volgt: wanneer $\mbox{Abs}$ hoog staat, zal het resultaat van de AND-poort $0$ zijn\footnote{Bemerk dat we de negatie nemen van $\mbox{Abs}$ als tweede invoer op de AND-poort.}. Verder zal in alle gevallen waarbij $\mbox{Add}$ laag staat, wordt het signaal ge\"inverteerd (dit zijn dus alle toestanden behalve de optelling). Hieruit kunnen we drie situaties afleiden: bij een optelling is $S=A+B$, bij het berekenen van een absolute waarde is $S=-B$ en bij het berekenen van het maximum stellen we $S=A-B$. Het komt er dus op neer dat de selectie-ingangen van de multiplexer op basis van enerzijds het soort bewerking, en anderzijds het teken van $S$ en $B$, de juiste data moeten doorlaten. De relevante selectie-ingangen voor de verschillende situaties staan op \tblref{minimal-functionals-absaddmaxmux}.
\importtabulartable{minimal-functionals-absaddmaxmux}{Multiplexer selectie-ingangen bij $\mbox{abs\&add\&max}$.}
De kostprijs per bit voor deze component bedraagt $72$ transistors per bit: $18$ voor de multiplexer, $36$ voor de full adder, $12$ voor de XOR-poort en $6$ voor de AND-poort. We kunnen verder opnieuw heel wat mogelijkheden beschouwen bij het combineren van functionele eenheden:
\begin{equation}
\acclarray{\accl{\mbox{abs1},\mbox{max1},+},\accl{\mbox{abs1},\mbox{max2},+},\accl{\mbox{abs1},\mbox{max1},\mbox{max2},+}\\\accl{\mbox{abs2},\mbox{max1},+},\accl{\mbox{abs2},\mbox{max2},+},\accl{\mbox{abs2},\mbox{max1},\mbox{max2},+}}
\end{equation}
Een eigenschap die we zeker kunnen uitbuiten is dat alle beschouwde bewerkingen commutatief zijn. $\mbox{abs1}$ haalt bijvoorbeeld data uit register $t_1$. Dit betekent dus dat we dit register op de tweede ingang moeten aanleggen. Door de operanden van de maximum-bewerkingen om te draaien vermijden we echter dat we hiervoor een multiplexer moeten invoeren. De optelling leest de gegevens echter uit registers $t_6$ en $t_2$. We zullen dus sowieso een multiplexer moeten introduceren die kiest tussen het register $t_6$ en een ander register ($t_1$ of $t_2$ afhankelijk van de gekozen combinatie). Een dergelijke multiplexer komt met een kost van $12$ transistors per bit. Indien we bijvoorbeeld $\mbox{abs2}$ combineren met de twee maximum-bewerkingen en de opteller, kunnen we de schakeling implementeren zoals op \figref{minimal-functionals-absaddmaxbitmux}. In deze schakeling wordt de data van $B$ (voorgesteld door de $b_i$'s) opgehaald uit register $t_2$. Het getal $A$ (voorgesteld door de $a_i$'s) halen we uit register $t_1$ en $D$ (voorgesteld door de $d_i$'s) uit register $t_6$. De $2$-naar-$1$ multiplexer kiest bijgevolg uit een van de twee registers (register $t_6$ wordt enkel gekozen bij een optelling).
\begin{figure}[hbt]
\centering
\importtikzsubfigure{minimal-functionals-absaddmaxbitmux}{Generische multiplexer.}
\importtikzsubfigure{minimal-functionals-absaddmaxbitmuxand}{Samenstelling met poort.}
\caption{Methodes bij het introduceren van een multiplexer bij een samengestelde functionele eenheid.}
\end{figure}
Een aspect die we echter kunnen uitbuiten is de AND-poort die vlak onder deze multiplexer staat. Het resultaat van de AND-poort bestaat immers uit drie gevallen: $a_i$, $d_i$ of $0$. We kunnen dus de AND-poort en de $2$-naar-$1$ multiplexer samenvoegen in een $3$-naar-$1$ multiplexer. \figref{minimal-functionals-absaddmaxbitmuxand} toont de implementatie door de AND-poort en de multiplexer samen te drukken. De kostprijs blijft in dit geval constant (al is dat niet altijd zo). Al kan er nog een reden zijn om dit te doen: de tijdsduur van het kritische pad. Wanneer het signaal door een $3$-naar-$1$ multiplexer gaat, is dit sneller dan door een $2$-naar-$1$ multiplexer gevolgd door een AND-poort. Vermits het kritische pad de te realiseren kloksnelheid bepaald, is dit een niet onbelangrijke factor.
\paragraph{}
Concreet realiseren we de volgende winsten: wanneer we een combinatie maken met \'e\'en maximum-bewerking is de originele kostprijs $100$ transistors per bit. Het samengestelde component kost $72$ transistors per bit: $18$ transistors voor de multiplexer, $36$ voor de full adder, $12$ voor de XOR-poort en $6$ voor de AND-poort. Verder dienen we ook een multiplexer van $12$ transistors per bit te implementeren. De winst is dus $16$ transistors per bit. Wanneer we een samenstelling maken met de twee maximum-bewerkingen, besparen we alle transistoren betrokken in de tweede maximum bewerking: $32$ transistors per bit. De totale winst is dus $48$ transistors per bit.
\paragraph{Abs\&Add\&Max\&Sub}In de component die de absolute waarde, de optelling en het maximum samenvoegt in \'e\'en functionele eenheid valt op dat in sommige omstandigheden de full adder het verschil van de twee operanden uitrekent. Dit effect is afkomstig van het maximum: we berekenen immers het verschil om te bepalen welk register we zullen doorlaten. Door minimale veranderingen kunnen we een component implementeren die in het geval we een verschil willen uitrekenen, de onderste multiplexer het resultaat van het verschil doorlaat. Hiervoor zullen we teruggrijpen naar \figref{minimal-functionals-absaddmaxbit}. We beschouwen ook een extra stuursignaal: $\mbox{Sub}$. We dienen de schakeling per bit niet aan te passen, deze is dus volledig identiek aan \figref{minimal-functionals-absaddmaxbit}. We dienen alleen de logica die de selectie-ingangen van de multiplexer berekend aan te passen. De nieuwe logica staat in \tblref{minimal-functionals-absaddmaxsubmux}.
\importtabulartable{minimal-functionals-absaddmaxsubmux}{Multiplexer selectie-ingangen bij $\mbox{abs\&add\&max\&sub}$.}
\paragraph{}
De kostprijs verschilt niet van deze van \mbox{Abs\&Add\&Max}: $72$ transistors per bit.
\begin{equation}
\acclarray{\accl{\mbox{abs1},\mbox{max1},+,-},\accl{\mbox{abs1},\mbox{max2},+,-},\accl{\mbox{abs1},\mbox{max1},\mbox{max2},+,-}\\\accl{\mbox{abs2},\mbox{max1},+,-},\accl{\mbox{abs2},\mbox{max2},+,-},\accl{\mbox{abs2},\mbox{max1},\mbox{max2},+,-}}
\end{equation}
\paragraph{Min\&Sub} We onderzoeken ook het samenvoegen van minimum- en verschil-bewerkingen. Uit de implementatie van minumum en maximum weten we al dat we daarvoor het teken van het verschil moeten berekenen. Daarom verwachten we dat het samenvoegen van de componenten winst kan opleveren. \figref{minimal-functionals-minsubbit} toont de implementatie per bit van zo'n component.
\importtikzfigure{minimal-functionals-minsubbit}{De implementatie van $\mbox{min\&sub}$ per bit.}
De full adder rekent altijd het verschil uit: $S=A-B$. Ofwel wordt vervolgens op basis van het teken van $S$ een getal doorgelaten (in het geval van een minimum-bewerking), ofwel wordt $S$ zelf doorgelaten (wanneer we het verschil uitrekenen).
\importtabulartable{minimal-functionals-minsubbitmux}{Multiplexer selectie-ingangen bij $\mbox{min\&sub}$.}
\tblref{minimal-functionals-minsubbitmux} toont hoe we de stuur-signalen van de multiplexer moeten implementeren.
\paragraph{}
De kostprijs per bit bestaat uit de multiplexer ($18$ transistors), de full adder ($36$ transistors) en de NOT-poort ($2$ transistors). Dit maakt dus samen $56$ transistors per bit. In de schakeling komt maar \'e\'en minimum-bewerking en \'e\'en verschil bewerking voor. Dit is ook de enige combinatie die we kunnen uitproberen voor deze samengestelde functionele eenheid. Beide originele functionele eenheden halen bovendien gegevens uit dezelfde registers, bijgevolg worden geen extra multiplexers ge\"introduceert. De kostprijs van de twee afzonderlijke functionele eenheden is $70$ transistors per bit. We maken dus $14$ transistors winst per bit.
\paragraph{Abs\&Min\&Sub}
Bij het berekenen van de absolute waarde maken we een keuze tussen het ingevoerde getal en de negatie van dat getal. Vermits we in de schakeling van \mbox{Min\&Sub} de negatie van een ingevoerd getal berekenen, hopen we deze hardware te kunnen hergebruiken. We proberen dus een \mbox{Abs\&Min\&Sub} component te implementeren. We moeten echter wel extra logica voorzien: het is immers de bedoeling dat de full adders de negatie zullen leveren, maar dan moet de eerste operand wel op $0$ worden gezet. Dit kan men doen door over een AND-poort te voorzien die wanneer het stuur-signaal op de AND-poort $0$ is, zal de AND-poort ook $0$ op de linkse ingang van de full adder aanleggen. We bekomen dus de schakeling op \figref{minimal-functionals-absminsubbit}.
\begin{figure}[hbt]
\centering
\importtikzsubfigure{minimal-functionals-absminsubbit}{Algemene implementatie.}
\importtikzsubfigure{minimal-functionals-absminsubbitmux}{Met multiplexer.}
\caption{De implementatie van $\mbox{abs\&min\&sub}$ per bit.}
\end{figure}
\tblref{minimal-functionals-absminsubbitmux} ten slotte, toont de logica bij de selectie-ingangen van de multiplexer. We beschouwen een verschil-bewerking wanneer zowel \mbox{Abs} en \mbox{Min} allebei op $0$ staan.
\importtabulartable{minimal-functionals-absminsubbitmux}{Multiplexer selectie-ingangen bij $\mbox{abs\&min\&sub}$.}
\paragraph{}
De kostprijs per bit van dit samengesteld component is $62$ transistoren: $36$ van de full adder, $6$ van de AND-poort, $2$ voor de NOT-poort en $18$ voor de multiplexer. Er zijn twee verschillende combinaties van functionele eenheden die we kunnen samenstellen met deze samengestelde functionele eenheid:
\begin{equation}
\accl{\accl{\mbox{abs1},\mbox{min},-},\accl{\mbox{abs2},\mbox{min},-}}
\end{equation}
Indien we \mbox{abs1} betrekken in de combinatie, ontstaat er echter een probleem met de ingangen. De aftrekker berekent immers het verschil tussen de waarde in registers $t_1$ en $t_2$. Omdat een verschil operator niet commutatief is, mogen we dus de volgorde van de ingangen niet kiezen. \mbox{abs1} berekent echter de absolute waarde van $t_1$. Hierdoor moet $t_1$ op de tweede ingang worden aangelegd. Het gevolg is dus dat we een multiplexer moeten introduceren zoals op \figref{minimal-functionals-absminsubbitmux}. Dit drijft de kostprijs op naar $74$ transistors per bit. De originele kostprijs van de afzonderlijke functionele eenheden is $102$ transistoren: $32$ voor de absolute waarde, $32$ voor de minimum en $38$ voor het verschil. De winst is dus $40$ transistoren per bit voor een combinatie met \mbox{abs1} en $28$ transistoren per bit voor een combinatie met \mbox{abs2}.
\paragraph{Combinaties met schuif-operaties over een vast aantal bits}In de processor komen ook twee schuif-bewerkingen voor: $\mbox{shr 1}$ en $\mbox{shr 3}$. Kunnen we deze operaties niet combineren met andere functionele eenheden? Vermits een schuif-bewerking over een vast aantal bits\footnote{Merk op dat dit niet geldt voor een schuif-operatie over een variabel aantal bits, hiervoor wordt een schuifoperator gebruikt (zie \sscrefpag{shiftoperators}).} wordt ge\"implementeerd met behulp van bekabeling, worden er geen transistoren gebruikt. Het combineren met enig ander component zal bijgevolg nooit winst opleveren (het combineren zal immers altijd nieuwe logica introduceren, tenzij de originele bewerkingen niet optimaal werden ge\"implementeerd).
\paragraph{Uiteindelijk compatibiliteitsgrafe} Nu we een groot aantal combinaties hebben uitgeprobeerd bekomen we de compatibiliteitsgrafe of \figref{compatibilitygraph-functionals-final}. Op de prioriteitsranden staat de winst in transistors per bit.
\importtikzfigure{compatibilitygraph-functionals-final}{De uiteindelijke compatibiliteitsgrafe bij het samenvoegen van bewerkingen.}
\paragraph{Optimaliseren van een meervoudige compatibiliteitsgraaf}
Bij de compatibiliteitsgraaf hebben we de max-cut methode gebruikt om tot een goed resultaat te komen. Dit algoritme is echter minder geschikt voor compatibiliteitsgrafes met prioriteitsranden met drie of meer knopen. Het algoritme werkt minder goed vanaf het moment dat de winst op een prioriteitsrand voor drie of meer knopen verschilt van de som van de winsten van prioriteitsranden tussen twee van de knopen. Daarom zullen we gebruik maken van een gretig\footnote{Engels: greedy.} algoritme. We zoeken de prioriteitrand met de meeste winst. Alle betrokken knopen worden vervolgens in een aparte partitie ondergebracht. De knopen samen met prioriteitsranden die betrokken zijn in minstens \'e\'en van de knopen worden vervolgens uit de grafe verwijdert waarna we het proces herhalen.
\paragraph{}
\importalgorithmicalgorithm{compatibilitygraph-multiple}{Het bepalen van sterke samengestelde functionele eenheden.}
\algoref{alg:compatibilitygraph-multiple} formaliseert het principe. In het algoritme beschouwen we $V$ de verzameling van oorspronkelijke bewerkingen. In het geval van het leidend voorbeeld is $V$ dus:
\begin{equation}
V_0=\accl{\mbox{abs1},\mbox{abs2},\mbox{min},\mbox{max1},\mbox{max2},-,+,\mbox{shr1},\mbox{shr3}}
\end{equation}
$E$ is de verzameling van prioriteitsranden. We stellen een element $e\in E$ voor als een verzameling knopen $e\subseteq V$. Bijgevolg is $E\subseteq\powset{V}$. $l:E\rightarrow\NNN$ ten slotte is een functie die de prioriteitsranden afbeeldt op een geheel getal die de winst\footnote{De winst drukt men niet noodzakelijk uit in transistoren per bit: in het geval van een ontwerp op een FPGA zal men bijvoorbeeld het aantal logische blokken nemen.} voorstelt. In het leidend voorbeeld zijn $E$ en de functie $l$ dus:
\begin{equation}
E_0:l=\acclarray{\accl{\mbox{abs1},\mbox{min}}:2,\accl{\mbox{abs1},\mbox{min},-}:28,\accl{\mbox{abs2},\mbox{max1},\mbox{max2},-,+}:92
\\\accl{\mbox{abs2},\mbox{max1},\mbox{max2}}:34,\accl{\mbox{abs2},\mbox{max1},\mbox{max2},+}:54,\accl{\mbox{min},-}:14
\\\accl{\mbox{max1},\mbox{max2}}:32,\accl{\mbox{max1},\mbox{max2},+}:22,\accl{-,+}:14}
\end{equation}
Met $X:f$ de set $X'$ waarbij elk element $x\in X$ verreikt is met de waarde $\fun{f}{x}$. Het element met de hoogste waarde in $E_0$ is $\accl{\mbox{abs2},\mbox{max1},\mbox{max2},-,+}$ met $92$ transistoren winst per bit. We nemen bijgevolg deze prioriteitsrand en voegen deze toe aan de partitie-verzameling:
\begin{equation}
\calP_1=\accl{\accl{\mbox{abs2},\mbox{max1},\mbox{max2},-,+}}
\end{equation}
We verwijderen vervolgens de knopen uit $V$ en de interfererende prioriteitsranden uit $E$:
\begin{eqnarray}
V_1&=&\accl{\mbox{abs1},\mbox{min},\mbox{shr1},\mbox{shr3}}\\
E_1:l&=&\accl{\accl{\mbox{abs1},\mbox{min}}:2}
\end{eqnarray}
Er blijft nog \'e\'en prioriteitsrand over die we bijgevolg selecteren. We voegen deze toe aan de partitie en verwijderen de relevante knopen en bogen:
\begin{eqnarray}
\calP_2&=&\accl{\accl{\mbox{abs2},\mbox{max1},\mbox{max2},-,+},\accl{\mbox{abs1},\mbox{min}}}\\
V_2&=&\accl{\mbox{shr1},\mbox{shr3}}\\
E_2:l&=&\emptyset
\end{eqnarray}
We merken dat de verzameling met prioriteitsranden $E_2$ leeg is. We voegen de resterende knopen bijgevolg als enkelvoudige partities toe:
\begin{equation}
\calP_3=\accl{\accl{\mbox{abs2},\mbox{max1},\mbox{max2},-,+},\accl{\mbox{abs1},\mbox{min}},\accl{\mbox{shr1}},\accl{\mbox{shr3}}}
\end{equation}
Nu we deze partitionering hebben gerealiseerd, kunnen we een nieuwe processor implementeren. \figref{sprocessor-minfun} toont de realisatie na het minimaliseren van de functionele eenheden.
\importtikzfigure{sprocessor-minfun}{Implementatie van het datapad na optimalisatie van de bewerkingen.}
\subsubsection{Verbindingen samenvoegen (``bus sharing'')}
Niet alleen transistoren bepalen de kostprijs in een processor. Ook de bedrading vormt een niet te onderschatten kost. Doorgaans bevatten registers echter een groot aantal bits: $16$, $32$ of $64$ bits vormen geen uitzondering. Wanneer we dus een groot aantal verbindingen moeten leggen tussen de registers en de functionele eenheden loopt de kostprijs dus wel op. Daarom tracht men meestal dezelfde verbinding te gebruiken om data van de registers naar de functionele eenheden te brengen. Dit concept wordt dan ook \termen{bus sharing} genoemd: het delen van verbindingen. Om het aantal verbindingen te verminderen, moeten we echter multiplexers of tri-state buffers introduceren. Het verminderen van de verbindingen komt dus met een kostprijs. Anderzijds bevatten sommige functionele eenheden zelf al multiplexers. Wanneer we twee of meer verbindingen kunnen samenvoegen die op de data-ingangen staan van een multiplexer van een functionele eenheid, kan deze multiplexer worden gereduceerd of zelfs worden ge\"elimineerd.
\importtikzfigure{minimal-bus-merge}{Het samenvoegen van verbindingen kan multiplexers elimineren.}
\figref{minimal-bus-merge} illustreert het principe van bus sharing. In plaats van de originele twee verbindingen bekomen we na de transformatie \'e\'en verbinding die gedeeld wordt door de twee registers. Om te voorkomen dat beide registers tegelijk data op de bus plaatsen voorzien we multiplexers. Indien een functionele eenheid aanvankelijk met behulp van een multiplexer kiest vanuit welk register data kan worden uitgelezen, kan deze bij het samenvoegen van de verschillende registers worden ge\"elimineerd. In dit opzicht is het soms zelfs mogelijk om van deze operatie een neutrale operatie te maken. Merk echter op dat een 2-naar-1 multiplexer 12 transistoren per bit kost, terwijl een tri-state buffer 10 transistoren per bit kost. In het voorbeeld maken we dus 8 transistoren per bit verlies. Dit impliceert echter niet dat we altijd verlies maken: stel dat verschillende functionele eenheden met behulp van een multiplexer het uit te lezen register bepaalden, kan het samenvoegen van verbindingen zelfs winst opleveren.
\paragraph{Compatibiliteitsgrafe}Net als de twee vorige optimalisaties zullen we opnieuw gebruik maken van een compatibiliteitsgrafe. In deze compatibiliteitsgrafe zijn de knopen logischerwijs de verbindingen. Een eerste probleem is dat er twee soorten verbindingen zijn: \termen{operanden} (van registers naar functionele eenheden) en \termen{resultaten} (van functionele eenheden naar registers). Omdat de twee soorten verbindingen geen effect op elkaar hebben, worden ze apart geoptimaliseerd. We zullen dus twee compatibiliteitsgrafes aanmaken. Twee verbindingen zijn incompatibel wanneer ze in dezelfde toestand actief\footnote{Een verbinding wordt uiteraard altijd gebruikt en zal altijd een signaal overbrengen. Met actief bedoelen we dat het resultaat ook effectief zal gebruikt worden. Bij bijvoorbeeld de resultaten betekent dit dat de data in de registers zal worden geladen.} gebruikt worden en gegevens van een verschillende aansturing dragen. Met aansturing bedoelen we een register bij een operand en een functionele eenheid bij een resultaat. Tot slot dienen we ook de prioriteitsranden te introduceren. Er is een voorkeur om twee verbindingen te gebruiken wanneer ze de aansturing delen (en dus tri-state buffers besparen) of eenzelfde gebruiker\footnote{Een gebruiker is een functionele eenheid in het geval van een operand en een register in het geval van een resultaat.} hebben (en dus multiplexers besparen).
\paragraph{}
Om een compatibiliteitsgrafe op te stellen, zullen we twee nieuwe tabellen introduceren: een \termen{operand-tabel} en een \termen{resultaten-tabel}. Deze tabellen bevatten in de verticale dimensie de verbindingen van het specifieke type (operanden of resultaten) en in de horizontale dimensie de verschillende toestanden die het algoritme doorloopt. Een cel wordt gemarkeerd wanneer in de gegeven toestand de verbinding actief gebruikt wordt. We zullen dit doen op basis van het algoritme op \figrefpag{asmSqrt} en de verbindingen op \figref{sprocessor-minfun}.
\paragraph{}
\begin{table}[hbt]
\centering
\importtabularsubtable{minimal-bus-operand}{Operand-tabel.}
\importtabularsubtable{minimal-bus-result}{Resultaten-tabel.}
\caption{Operand- en resultaten-tabel van het leidend voorbeeld.}
\tbllab{minimal-bus-all}
\end{table}
In toestand $S_0$ wordt de data ingelezen in de registers $t_1$ en $t_2$. Bijgevolg gebruiken we $2$ verbindingen: $r_1:\mbox{In1}\rightarrow t_1$ en $r_4:\mbox{In2}\rightarrow t_2$. We markeren dan ook deze cellen in \tblref{minimal-bus-result}. In toestand $S_2$ wordt de absolute waarde van beide registers berekend en wordt het resultaat opnieuw in de registers opgeslagen. Hiervoor wordt de data van registers $t_1$ en $t_2$ respectievelijk naar \mbox{FU1} en \mbox{FU2} geleid. Bijgevolg zijn de verbindingen $o_2:t_1\rightarrow\mbox{FU1}.1$ en $o_6:t_2\rightarrow\mbox{FU2}.2$ actief. Omdat het resultaat ook terug in de registers moet worden geladen, zijn ook resultaten-verbindingen actief: $r_2:\mbox{FU1}\rightarrow t_1$ en $r_5:\mbox{FU2}\rightarrow t_2$. Analoog berekenen we de actieve verbindingen van de andere verbindingen en bekomen de waarden in \tblrefs{compatibilitygraph-connections-a,compatibilitygraph-connections-b}.
\paragraph{}
\begin{figure}[hbt]
\centering
\importtikzsubfigure{compatibilitygraph-connections-a}{Operanden (registers naar functionele eenheden).}
\importtikzsubfigure{compatibilitygraph-connections-b}{Resultaten (functionele eenheden naar registers).}
\caption{De compatibiliteitsgrafes bij het samenvoegen van verbindingen.}
\end{figure}
Op basis van de opgestelde tabellen kunnen we de compatibiliteitsgrafes opstellen. Twee verbindingen zijn immers incompatibel wanneer er een kolom bestaat waar beide verbindingen tegelijk actief zijn en de aansturing verschilt. Zo kunnen we uit \tblref{minimal-bus-operand} dat $o_1$ compatibel is met alle andere verbindingen: de verbinding is actief in toestand $S_7$, een toestand waarin geen enkele andere verbinding actief is. $o_2$ is echter wel incompatibel met andere verbindingen. $o_2$ is immers actief in toestanden $S_1$ en $S_2$. In die toestanden zijn volgende verbindingen actief: $o_3$, $o_4$, $o_6$. Van deze verbindingen wordt $o_4$ ook aangestuurd door register $t_1$. De verbindingen $o_3$ en $o_6$ worden echter aangestuurd door register $t_2$. Bijgevolg is $o_2$ incompatibel met $o_3$ en $o_6$.  Analoog berekenen we alle incompatibiliteitsranden van de operanden en de resultaten:
\begin{eqnarray}
I_o&=&\accl{\tupl{o_2,o_3},\tupl{o_2,o_6},\tupl{o_3,o_4},\tupl{o_4,o_6},\tupl{o_5,o_6},\tupl{o_7,o_8}}\\
I_r&=&\accl{\tupl{r_1,r_4},\tupl{r_2,r_5},\tupl{r_3,r_7},\tupl{r_6,r_8}}
\end{eqnarray}
Vervolgens berekenen we de prioriteitsranden. $o_1$ wordt aangestuurd door register $t_1$. Andere operanden die door dit register worden aangestuurd zijn $o_2$, $o_4$ en $o_7$. We hoeven bij eenzelfde aansturing niet te controleren op het bestaan van een incompatibiliteitsrand. Eenzelfde aansturing is immers een voldoende voorwaarde om te stellen dat deze niet zal bestaan. $o_1$ wordt verder gebruikt door \mbox{Out}. Vermits dit de enige verbinding is die hiervan gebruik maakt, levert het delen van de gebruiker geen extra prioriteitsranden op. We tekenen bijgevolg prioriteitsranden tussen $o_1$ en $o_2$, $o_4$ en $o_7$. Vervolgens beschouwen we verbinding $o_4$ op basis van de aansturing vinden we prioriteitsranden met $o_1$, $o_2$ en $o_7$. Daarnaast is er een andere verbinding die dezelfde gebruiker deelt: $o_5$. Omdat beide verbindingen niet tegelijk actief zijn, is er geen sprake van een incompatibiliteitsrand. Daarom tekenen we ook een prioriteitsrand tussen $o_4$ en $o_5$. Indien we dit proces verder zetten bekomen we de volgende prioriteitsranden:
\begin{eqnarray}
P_o&=&\accl{\tupl{o_1,o_2},\tupl{o_1,o_4},\tupl{o_1,o_7},\tupl{o_2,o_4},\tupl{o_2,o_7},\tupl{o_3,o_6},\tupl{o_4,o_5},\tupl{o_4,o_7},\tupl{o_5,o_8}}\\
P_r&=&\accl{\tupl{r_1,r_2},\tupl{r_1,r_3},\tupl{r_2,r_3},\tupl{r_2,r_7},\tupl{r_3,r_5},\tupl{r_4,r_5},\tupl{r_4,r_6},\tupl{r_5,r_6},\tupl{r_7,r_8}}
\end{eqnarray}
De compatibiliteitsgrafen van de twee types verbindingen worden voorgesteld in \figrefs{compatibilitygraph-connections-a,compatibilitygraph-connections-b}. Nadat we de max-cut methode op de grafen toepassen bekomen we de volgende partities:
\begin{eqnarray}
\calP_o&=&\accl{\accl{o_1,o_2,o_4,o_5,o_7},\accl{o_3,o_6,o_8}}\\
\calP_r&=&\accl{\accl{r_1,r_2,r_7,r_8},\accl{r_3,r_4,r_5,r_6}}
\end{eqnarray}
\paragraph{Uiteindelijke implementatie}
\importtikzfigure{sprocessor-mincon}{De implementatie van het datapad na het minimaliseren van de verbindingen.}
\subsubsection{Registers samenvoegen in registerbank (``register port sharing'')}
Registers voorzien niet enkel logica om gegevens op te slaan. Er is ook logica vereist om nieuwe data op te slaan of om de data uit te lezen. We kunnen echter opmerken dat niet steeds alle registers hun data in elke toestand beschikbaar moeten stellen of nieuwe gegevens moeten opslaan. Door registers te groeperen in een registerbank kunnen we besparen op de implementatie van deze functies. Verder kunnen we het aantal multiplexers en tri-state buffers mogelijk reduceren omdat het aantal ingangen en uitgangen van een registerbank beperkter zijn. Dit proces, ook wel \termen{register port sharing} genoemd, zullen we in deze subsectie bespreken.
\paragraph{}
Alvorens we het aantal registers kunnen reduceren dienen we eerst te onderzoeken hoeveel lees- en schrijfpoorten\footnote{Met poort wordt geen logische (AND, OR, NOT,...) poort bedoelt, maar een toegangspunt waar men data kan uitlezen of wegschrijven.} we dienen te introduceren voor de registerbank(en). Hiervoor stellen we eerst een \termen{registertoegangstabel} ofwel ``\termen{Register Access Table}'' op. Op basis van deze tabel kunnen we dan opnieuw minimaliseren met bijvoorbeeld de max-cut methode of in eenvoudige gevallen zelfs exhaustief zoeken.
\paragraph{Registertoegangstabel}Een registertoegangstabel is een tabel met in de verticale dimensie de verschillende registers en de horizontale dimensie de verschillende toestanden. Een cel wordt gemarkeerd als leeg, \termen{lees (R)}, \termen{schrijf (W)} of \termen{lees/schrijf (RW)}. De cel specificeert of we in de gegeven toestand uit het gegeven register gegevens uitlezen en/of wegschrijven. We kunnen een registertoegangstabel opstellen met behulp van het ASM-schema op \figref{asmSqrt} of bijvoorbeeld de operand- en resultaten-tabellen op \tblref{minimal-bus-all}. Wanneer een operand-verbinding aangestuurd door een register $t_i$ actief is in toestand $S_j$, dan zetten we een R in de overeenkomstige cel van de registertoegangstabel. Wanneer er een resultaten-verbinding actief is in deze toestand, plaatsen we een W in de cel.
\paragraph{}
Zo zien we in \tblref{minimal-bus-operand} dat in toestand $S_0$ dat geen enkele verbinding actief is. Bijgevolg plaatsen we in de kolom van toestand $S_0$ geen R. In toestand $S_1$ zijn de operand-verbindingen $o_2$ en $o_6$ actief. De overeenkomstige register die aansturen zijn $t_1$ en $t_2$. We plaatsen dus in de rijen $t_1$ en $t_2$ in kolom $S_1$ een R. Analoog berekenen we de leestoegang van de andere cellen:
\begin{equation}
R=\acclarray{
\tupl{t_1,S_1},\tupl{t_1,S_2},\tupl{t_1,S_3},\tupl{t_1,S_4}\\
\tupl{t_1,S_6},\tupl{t_1,S_7},\tupl{t_2,S_1},\tupl{t_2,S_2},\tupl{t_2,S_4}\\
\tupl{t_2,S_5},\tupl{t_2,S_6},\tupl{t_6,S_3},\tupl{t_6,S_5}
}
\end{equation}
We berekenen op een gelijkaardige manier de schrijftoegang van de verschillende registers. Met behulp van \tblref{minimal-bus-results} kunnen we vaststellen dat in toestand $S_0$ de resultaten-verbindingen $r_1$ en $r_4$ actief zijn. Vermits $r_1$ en $r_4$ data laten inlezen in register $t_1$ en $t_2$, schrijven we W in de cellen $\tupl{t_1,S_0}$ en $\tupl{t_2,S_0}$. In toestand $S_1$ zien we opnieuw activiteit op de  verbindingen naar $t_1$ en $t_2$ bijgevolg schrijven we opnieuw W in de volgende kolom. Wanneer we dit proces verder toepassen, bekomen we de volgende lijst van schrijftoegangen:
\begin{equation}
W=\acclarray{
\tupl{t_1,S_0},\tupl{t_1,S_1},\tupl{t_1,S_2},\tupl{t_1,S_6}\\
\tupl{t_2,S_0},\tupl{t_2,S_1},\tupl{t_2,S_3}\\
\tupl{t_2,S_4},\tupl{t_2,S_5},\tupl{t_6,S_2},\tupl{t_6,S_3}
}
\end{equation}
De uiteindelijke registertoegangstabel staat in \tblref{minreg-readwrite}.

\importtabulartable{minreg-readwrite}{Registertoegangstabel van het leidend voorbeeld.}
\paragraph{}
Een registerbank bevat een set van registers. Een belangrijke vraag is hoeveel lees en schrijfpoorten we moeten voorzien in een registerbank vermits dit in grote mate de kostprijs zal bepalen. De aantallen worden bepaald door de registers die de registerbank vertegenwoordigt. In elke toestand wordt er immers data ingelezen en weggeschreven in deze registers. Het aantal leespoorten is dan ook gelijk aan het maximale aantal leesoperaties van de betrokken registers in een bepaalde toestand. Analoog stelt men het aan schrijfpoorten gelijk aan het maximale aantal schrijfoperaties van deze registers in een toestand.
\paragraph{}
We zullen dit concept illustreren met twee voorbeelden. Stel een registerbank opstellen die de registers $t_1$ en $t_6$ groepeert. Vermits in toestand $S_2$ naar beide registers wordt geschreven, moeten we twee schrijfpoorten voorzien. Vermits de registerbank slechts twee registers vertegenwoordigt kan dit aantal onmogelijk groter worden.  Verder kunnen we ook vaststellen dat in toestand $S_3$ uit de twee registers data wordt uitgelezen. Bijgevolg is het aantal leespoorten ook gelijk aan twee.
\paragraph{}
Indien we een registerbank met de drie registers $t_1$, $t_2$ en $t_6$ beschouwen is de situatie echter anders. Er bestaat immers geen enkele toestand waar uit alle registers tegelijk data wordt uitgelezen. Evenmin bestaat er een toestand waarin data in alle betrokken registers wordt weggeschreven. In elke toestand zijn er maximum twee leesbewerkingen en twee schrijfbewerkingen. We dienen dus $2$ leespoorten en $2$ schrijfpoorten te voorzien, terwijl we drie registers beschouwen.
\paragraph{}
Alvorens de registers samen te voegen in een registerbank, zullen we eerst onderzoeken of het wel interessant is om alle registers in eenzelfde registerbank onder te brengen. Misschien is het interessanter om bijvoorbeeld twee of drie registerbanken te voorzien. Het feit dat we slechts drie register beschouwen, laat ons toe alle mogelijkheden te onderzoeken. Er zijn in totaal $5$ partities:
\begin{equation}
\calP_{\mbox{\small registerbank}}=\acclarray{
\accl{\accl{t_1},\accl{t_2},\accl{t_6}}\\
\accl{\accl{t_1},\accl{t_2,t_6}}\\
\accl{\accl{t_1,t_2},\accl{t_6}}\\
\accl{\accl{t_1,t_6},\accl{t_2}}\\
\accl{\accl{t_1,t_2,t_6}}
}
\end{equation}
Het aantal lees- en schrijfpoorten per partitionering in registerbanken staan in \tblref{minreg-config}. We kunnen vaststellen dat het groperen van alle registers in \'e\'en registerbank de meeste voordelige implementatie is.
\importtabulartable{minreg-config}{Registerbank-configuraties voor het leidend voorbeeld.}
\subsubsection{Vergelijking van de verschillende optimalisaties}
\importtabulartable{minimal-comparison}{Vergelijking van de kostprijs na de verschillende optimalisaties.}
Om de verschillende optimalisaties samen te vatten, presenteren we \tblref{minimal-comparison} die de implementaties na de verschillende optimalisaties samenvat. We drukken de kostprijs uit volgens twee systemen: het aantal transistoren bij een ASIC implementatie en het aantal logische blokken wanneer we de processor implementeren op een FPGA. Verder bepalen we ook het aantal verbindingen. We delen het aantal transistoren verder op in vier categorie\"en: registers, functionele eenheden, registers naar functionele eenheden en functionele eenheden naar register. De laatst twee bepalen het aantal transistoren die betrokken zijn in de tri-state buffers aan de uitgangen van de aansturing en de multiplexers aan de ingangen van de ontvangers. In het geval van een FPGA kunnen we enkel een onderscheid maken tussen registers en functionele eenheden. De reden is dat een FPGA altijd multiplexers voorziet aan de in en uitgangen en we bijgevolg er de logica tussen registers en functionele eenheden geen extra logische blokken vereist. Merk op dat de kostprijs telkens is uitgedrukt per bit. Indien de registers bijvoorbeeld 32 bit getallen voorstellen moet men de kostprijs ongeveer vermenigvuldigen met $32$ om een benaderende kostprijs uit te rekenen.
\subsection{Andere optimalisaties}
\label{ss:syntheseFSMDOptimal}
\subsubsection{Minimaal instructiewoord}
De verzameling van stuursignalen noemt men doorgaans het \termen{instructiewoord}. Het is immers een verzameling van bits die voor het datapad de operatie die moet worden uitgevoerd verder specificeert. Een typisch datapad bestaat uit \'e\'en of meer registerbanken, een paar tellers, een paar afzonderlijke registers, \'e\'en of meer aritmetische logische eenheden (ALU), een schuifoperator en een vergelijker. Al deze componenten introduceren stuursignalen. \figref{instructionword-example} toont een voorbeeld van hoe zo'n instructiewoord er kan uitzien. Per bit bevat de figuur een korte beschrijving van de betekenis van de bit.
\importtikzfigure{instructionword-example}{Een voorbeeld van een instructiewoord.}
\paragraph{}
In de meeste gevallen kan men echter het instructiewoord verkleinen met behulp van verschillende assumpties. Een eerste assumptie is bijvoorbeeld dat niet alle functionele eenheden (ALU, schuifregister,...) tegelijk actief zullen zijn. In plaats van voor elke functionele eenheid een reeks bits in het instructiewoord te voorzien, kunnen we enkele bits reserveren die bepalen of een sequentie aan stuursignalen bedoelt zijn voor de ene functionele eenheid of de andere.
\paragraph{}
Een andere reductie omvat het gebruik van tri-state buffers. Tri-state buffers verhinderen dat uitvoer van registers of functionele eenheden om tegelijk op dezelfde verbindingen worden gezet. Bijgevolg zijn de stuursignalen van de tri-state buffers exclusief: slechts \'e\'en van de stuursignalen van de tri-state buffers is tegelijk actief. In plaats van voor iedere tri-state buffer een bit in het instructiewoord te voorzien, kunnen we in bits een getal uitdrukken die bepaalt welke tri-state buffer nu de data op de verbinding moet zetten. Voor $n$ tri-state buffers wordt het instructiewoord dus gereduceerd van $n$ bits naar $\ceil{\log_2n}$ bits.
\paragraph{}
Bij wijze van voorbeeld zullen we nog enkele voorbeelden geven gebaseerd op het instructiewoord van \figref{instructionword-example}.
\begin{enumerate}
 \item De eerste operandbus bevat ofwel gegevens afkomstig van de de registerbank (leespoort 1) ofwel het register.We kunnen dus ofwel \mbox{RFOE1} ofwel \mbox{ROE} elimineren. We besparen $1$ bit.
 \item De tweede operandbus bevat ofwel gegevens afkomstig van de de registerbank (leespoort 2) ofwel de teller.We kunnen dus ofwel \mbox{RFOE2} ofwel \mbox{COE} elimineren. We besparen $1$ bit.
 \item In een registerbank is het signaal ``read-enabled (RE)'' altijd gelijk aan het stuursignaal op het tri-state buffer (RFOE). We kunnen dus voor beide leespoorten ofwel \mbox{RE$i$} ofwel \mbox{RFOE$i$}. We besparen $2$ bits (we beschouwen immers twee leespoorten).
 \item De resultaatbus bevat het resultaat van ofwel de ALU ofwel de schuifoperator. We kunnen dus opnieuw het stuursignaal van \'e\'en van de tri-state buffers wegwerken: ofwel \mbox{AOE} ofwel \mbox{SOE}. We besparen $1$ bit.
 \item De ALU en barrelshifter worden nooit tegelijk gebruik (de resultaten kunnen immers niet tegelijk op de bus worden geplaatst). Op basis van de tri-state buffer die de output op de bus plaatst, weten we welk component actief is. De overige bits die voor de instructie instaan (\mbox{F0}, \mbox{F1}, \mbox{F0}, \mbox{SH2}, \mbox{SH1}, \mbox{SH0} en \mbox{D}), kunnen worden gedeeld. We besparen bijgevolg $3$ bits.
 \item Bij de teller zijn de increment (\mbox{C}) en het inladen van de nieuwe waarde (\mbox{L}) exclusief. We besparen $1$ bit.
\end{enumerate}
Op basis van deze maatregelen hebben $9$ bits op het instructiewoord bespaard.
\subsubsection{``Chaining'': meerdere bewerkingen per klokcyclus}
Naast de eerder beschreven optimalisaties, kunnen we ook het ASM-schema zelf aanpassen. De argumentatie om dit te doen is een analyse van het tijdsgedrag. Elke klokcyclus wordt de data van de registers uitgelezen en voeren de functionele eenheden hierop bewerkingen uit. Niet alle functionele bewerkingen vereisten echter dezelfde hoeveelheid tijd om het resultaat te berekenen. We kunnen bijvoorbeeld denken aan de schuifoperaties met een vast aantal bits. Vermits hierbij geen logica moet worden ge\"implementeerd, wordt zo'n operatie in theorie onmiddellijk uitgevoerd. We kunnen echter analyseren wat er vervolgens met de gegevens van de functionele eenheid zal worden gedaan. Indien het mogelijk is om de volgende operatie in dezelfde klokcyclus uit te voeren kan dit tot besparingen leiden. Het groeperen van operaties die na elkaar moeten worden uitgevoerd noemen we \termen{chaining}. \figref{chaining} illustreert dit principe.
\paragraph{}
\begin{figure}[hbt]
\centering
\importtikzsubfigure{chaining-asm}{ASM-schema.}
\importtikzsubfigure{chaining}{Datapad.}
\caption{De chaining-transformatie.}
\figlab{chaining-general}
\end{figure}
Wanneer we het ASM-schema en het datapad aanpassen zoals op \figref{chaining-general}, voeren we een chaining transformatie uit. In het ASM-schema worden twee operaties $f_1$ en $f_2$ uitgevoerd. We kunnen de twee operaties samenvoegen in een operatie $f$ met $\fun{f}{x,y}=\fun{f_2}{\fun{f_1}{x},y}$. Alvorens we een chaining-transformatie kunnen uitvoeren, moeten we echter \'e\'en voorwaarde controleren: alle operanden moeten beschikbaar zijn alvorens we de gegroepeerde functie uitvoeren. Indien $z$ bijvoorbeeld nog niet beschikbaar is in de eerste toestand, kunnen we $f$ nog niet berekenen.
\subsubsection{``Multicycling'' en ``Pipelining'': meerdere klokcycli per bewerking}
\paragraph{Multicycling}
Ook in omgekeerde richting kunnen we mogelijk winst boeken. Door een instructie ``uit te smeren'' over verschillende toestanden kunnen we mogelijk de schakeling goedkoper maken, of de kloksnelheid verhogen. Dit concept noemen we \termen{multicycling}. In het leidend voorbeeld dienen we bijvoorbeeld het verschil te berekenen tussen twee getallen. Het verschil wordt doorgaans gerealiseerd met een functionele eenheid die dit in twee ``stappen'' zal uitrekenen: eerst nemen we de negatie van het tweede getal, waarna we vervolgens met behulp van een keten van full adders het eerste getal en de negatie van het tweede getal optellen. Omdat de twee stappen een sequentieel karakter vertonen\footnote{Men kan de twee stappen niet tegelijk uitvoeren of ze laten overlappen}, zouden we ervoor kunnen opteren om twee toestanden te voorzien. We voorzien dus geen aftrekker, maar voorzien twee functionele eenheden: een opteller en een componenten die de negatie van het getal kan berekenen.
\paragraph{}
\begin{figure}[hbt]
\centering
\importtikzsubfigure{multicycling-asm}{ASM-schema.}
\importtikzsubfigure{multicycling}{Datapad.}
\caption{De multicycling-transformatie.}
\figlab{multicycling-general}
\end{figure}
\figref{multicycling-general} beschrijft de transformatie die gepaard gaat met de introductie van multicycling. In het ASM-schema wordt een complexe functie $f$ uitgerekend die we kunnen opsplitsen in twee functies $f_1$ en $f_2$ zodat $\fun{f}{x,y}=\fun{f_1}{\fun{f_2}{x},y}$. In het ASM-schema wordt de functie uitgesmeerd over twee toestanden. Dit hoeft niet te betekenen dat we de tweede toestand moeten introduceren: we kunnen bijvoorbeeld ook de $f_2$-bewerking naar de volgende (al bestaande) toestand schuiven. In het datapad wordt in het algemeen een register ge\"introduceerd. Dit hoeft echter niet de betekenen dat we ook effectief een nieuw register zullen nodig hebben. Door de minimalisatie van de variabelen, kunnen we eventueel het ge\"introduceerde register elimineren. Als voorwaarde om deze transformatie uit te voeren stellen we \'e\'en voorwaarde: de bewerking moet zijn uitgevoerd alvorens andere bewerkingen het resultaat zullen gebruiken als operand.
\paragraph{}
De voordelen van multicycling zijn tweeledig. Allereerst kunnen we mogelijk hardware besparen in de functionele eenheden. Een opteller is immers goedkoper dan een aftrekker. Wanneer elders in het algoritme bijvoorbeeld nog een negatie moet worden berekend, kunnen we mogelijk de ge\"introduceerde functionele eenheid delen waardoor dit geen extra kosten met zich meebrengt. Verder valt op te merken dat wanneer we operaties kunnen herleiden naar kleinere basis-instructies, de kans groter is dat complexe operaties een deel van die basis-instructies zullen delen. Een eerste winst is dus mogelijk de kostprijs van de functionele eenheden.
\paragraph{}
Daarnaast kunnen we mogelijk de kloksnelheid opdrijven. Stel bijvoorbeeld dat de aftrekker de traagste operatie is in heel het proces, zal deze functionele eenheid de kloksnelheid bepalen. Het is echter mogelijk dat alle overige functionele eenheden significant sneller resultaten berekenen. Door een verschil-operatie op te splitsen in twee operaties die apart sneller presteren kunnen we mogelijk de kloksnelheid opdrijven. Indien de verschil-operatie vrij zeldzaam is, kan dit dus de doorvoer van het algoritme significant verhogen.
\paragraph{}
Multicycling komt in de meeste gevallen echter met een kostprijs. Allereerst valt in de praktijk de snelheidswinst meestal tegen. Stel bijvoorbeeld dat in het algoritme alleen verschil-bewerkingen worden uitgevoerd. De periode van het kloksignaal is in de realisatie van dit algoritme $1~\mbox{v}$\footnote{We introduceren hier een tijdseenheid \mbox{v}: de tijd die het proces nodig heeft om een verschil uit te rekenen en in te klokken.}. We kunnen deze periode vervolgens opdelen in $3$ delen\footnote{De verhoudingen van de delen is louter fictief en dient enkel ter illustratie.}: $0.35~\mbox{v}$ om de negatie te berekenen, $0.35~\mbox{v}$ voor de optelling en $0.3~\mbox{v}$ om de waarde in het register in te klokken. Merk op dat de tijd in dit fictieve voorbeeld mooi verdeeld is over de twee deeloperaties en dus ideaal is voor multicycling. Toch zullen we de kloksnelheid niet kunnen verdubbelen: de tussenresultaten (de negatie van het tweede getal) moeten immers ook in een register worden opgeslagen. De periode van een deelbewerking wordt bijgevolg $0.65~\mbox{v}$. We besparen bijgevolg $35\%$ op de klokfrequentie.
\paragraph{}
Een tweede kost heeft betrekking op de kostprijs. De kostprijs van de functionele eenheden kan in principe afnemen, maar introduceert extra logica ter hoogte van de controller. We introduceren immers een nieuwe ``deel-toestand'': een toestand waarin we de negatie van de tweede operand hebben berekend. Verder dienen we mogelijk nieuwe registers te introduceren om tussenresultaten in weg te schrijven\footnote{In het geval van een verschil-bewerking is dit niet het geval omdat we na de negatie van de tweede operand, de originele operand niet meer nodig hebben. In het algemeen is dit echter niet het geval.}. Bovendien introduceren we mogelijk nieuwe verbindingen, multiplexers en tri-state buffers om de resultaten van de extra functionele eenheden in de correcte registers weg te schrijven.
\paragraph{}
Multicycling is echter wel een populaire techniek voor operaties die buitenproportioneel veel tijd vragen. Dit is meestal het geval bij bijvoorbeeld RAM-geheugen. Men merkt op dat sinds de jaren~'80 de processorsnelheid significant werd opgedreven terwijl de RAM-geheugens deze evolutie niet hebben gevolgd. Indien we dus een toestand zouden voorzien om een cel uit het RAM-geheugen uit te lezen, zou de processorsnelheid hoogstens gelijk zijn aan deze van het RAM-geheugen. Een processor en geheugen werken dan ook eerder onafhankelijk van elkaar. Op geregelde tijdstippen geeft de processor een opdracht aan het RAM-geheugen om een cel uit te lezen en vervolgens enkele toestanden later te controleren of de gegevens reeds in de registers of cache zitten. De \verb+80x86+ instructieset kent zelfs ``prefetching''-instructies: instructies waarmee men de opdracht kan geven om data in de cache in te laden zodat op het moment dat de data effectief nodig is, deze meteen kan worden uitgelezen.
\paragraph{Pipelining}
Het uitsmeren van instructies over verschillende toestanden maakt ook een ander effect mogelijk: \termen{pipelining}. Pipelining is een uitvoeringsstrategie naast \termen{sequenti\"ele uitvoering} en \termen{parallelle uitvoering}.
\begin{figure}[hbt]
\centering
\importtikzsubfigure{execution-sequential}{Sequentieel.}
\importtikzsubfigure{execution-parallel}{Parallel.}
\importtikzsubfigure{execution-pipelining}{Pipelining.}
\caption{De verschillende uitvoeringsstrategie\"en: sequentieel, parallel en pipelining.}
\figlab{execution-models}.
\end{figure}
\paragraph{}
\figref{execution-models} toont de verschilende uitvoeringsstrategie\"en naast elkaar. Als voorbeeld gebruiken we het wassen van kledij. In het geval van sequenti\"ele uitvoer beschikken we over \'e\'en wasmachine, \'e\'en droogkast en \'e\'en strijkijzer. Kledij wordt eerst in de wasmachine gestopt, vervolgens gedroogd en daarna gestreken. Wanneer we een grote hoeveelheid kledij moeten wassen, herhalen we dit mechanisme. Indien elke opdracht $\Delta t$ in beslag neemt, betekent dit dus dat we $\bigoh{3\cdot k\cdot\Delta t}$ tijd nodig hebben om alles te verwerken met $k$ de hoeveelheid kledij. Dit proces staat beschreven in \figref{execution-sequential}.
\paragraph{}
Men kan de doorvoer echter opdrijven door meer infrastructuur te voorzien. Indien we elke machine in het $n$-voud aankopen en over $n$ arbeiders beschikken, kunnen we de doorvoer opdrijven met ongeveer $\bigoh{n}$. De gemiddelde tijd die het wassen van alle kledij in beslag neemt wordt dus gereduceerd tot $\bigoh{3\cdot k\cdot\Delta t/n}$. Het probleem is echter dat de kosten ook stijgen met een factor $\bigoh{n}$. Bovendien wordt de versnelling niet altijd gerealiseerd. Meestal is er een vorm van boekhouding nodig die bepaald wie welke taak precies zal uitvoeren\footnote{De maximaal te realiseren versnelling wordt geformaliseerd met de ``Wet van Amadahl''.}. Dit proces staat beschreven in \figref{execution-parallel}.
\paragraph{}
Tot slot is er pipelining. In het geval van pipelining beschikken we opnieuw over \'e\'en wasmachine, droogkast en strijkijzer. In plaats van echter alles na elkaar uit te voeren, werken de machines continu. Op het moment dat de wasmachine immers klaar is, verhuist de gewassen kledij naar de droogkast, maar stopt men nieuwe kledij in de wasmachine. Het gevolg is dat wanneer er $s$ stappen in het proces zijn, we de doorvoer kunnen opdrijven met $\bigoh{s}$ ten opzichte van het sequentieel proces, zonder de kosten voor de infrastructuur op te drijven. Dit proces staat beschreven in \figref{execution-pipelining}. Men kan pipelining ook vergelijken met het principe van de lopende band: we voorzien telkens \'e\'en machine voor elke bewerking en het resultaat schuift van machine naar machine.
\paragraph{}
\begin{figure}[hbt]
\centering
\importtikzsubfigure{pipelining-asm}{ASM-schema.}
\importtikzsubfigure{pipelining}{Datapad.}
\caption{De pipelining-transformatie.}
\figlab{pipelining-general}
\end{figure}
\figref{pipelining-general} beschrijft het transformatieproces bij pipelining. Wanneer twee identieke operaties dicht op elkaar volgen, kunnen we beide bewerkingen uitsmeren in de tijd en tegelijk parallelle uitvoer introduceren. De transformatie van het datapad is erg gelijkaardig aan dat van multicycling. Merk echter op dat we ook een nieuw register introduceren voor $z$. De reden is dat terwijl we de $f_2$ voor de eerste keer uitvoeren, we ook de operanden van de tweede instructie moeten uitvoeren. We kunnen een pipelining-transformatie uitvoeren op voorwaarde dat alle resultaten zijn uitgerekend alvorens ze door andere bewerkingen worden uitgevoerd en de instructies betrokken in het pipelining proces onafhankelijk van elkaar kunnen worden uitgevoerd.
\paragraph{}
We kunnen pipelining op vier niveaus toepassen:
\begin{enumerate}
 \item Functionele eenheden: in dit niveau splitsen we de bewerkingen op in deelbewerkingen en bewaren we de tussenresultaten in registers.
 \item Datapad: indien we werken met een trage registerbank (bijvoorbeeld RAM-geheugen) gebruiken we meestal ook pipelining.
 \item Controller.
 \item Het ASM-schema: we kunnen het ASM-schema herschrijven. Bijvoorbeeld door de verschil-instructie op te splitsen in een negatie-instructie en een optelling-instructie. Dit hoeft niet te betekenen dat we meer toestanden introduceren: sommige omstandigheden laten toe dat we een onderdeel van de instructie gewoon doorschuiven naar de volgende toestand. Op basis van het aangepaste schema zullen we mogelijk meer hardware introduceren, maar door het minimaliseren van het datapad kunnen we dit mogelijk terug ontdubbelen.
\end{enumerate}
In moderne processoren implementeert men pipelining op nagenoeg alle niveaus in de zogenaamde \termen{generische instructiecyclus}. De generische instructiecyclus is een beschrijving in welke onderdelen een gegeven instructie wordt uitgevoerd:
\begin{enumerate}
 \item Lees een instructie.
 \item Bereken de adressen (indien de instructie operanden uit bepaalde geheugenadressen haalt).
 \item Lees de operand(en) uit.
 \item Voer de bewerking uit.
 \item Schrijf het resultaat weg.
\end{enumerate}
\paragraph{}
Pipelining komt echter ook met enkele problemen. Om de snelheid effectief met \bigoh{s} op te drijven, dienen de deelstappen precies evenveel tijd te vragen. In de praktijk is het niet eenvoudig een proces zo op te delen. Wanneer dit niet te realiseren valt geldt de ``Wet van de zwakste schakel'': het traagste deelproces bepaald hoelang het duurt alvorens data zal worden uitgelezen. Verder is pipelining ook enkel interessant wanneer we een kort na elkaar een groot aantal maal de operatie moeten uitvoeren. Indien we slechts \'e\'enmaal de operatie uitvoeren blijft de rekentijd ongeveer dezelfde. We kunnen dus stellen dat de doorvoer (``\termen{throughput}'') wordt verhoogt, maar de vertraging (``\termen{latency time}'') ongeveer dezelfde blijft. Net als bij multicycling dienen we ook opnieuw extra registers te voorzien om de tussenresultaten in op te slaan. Deze register verhogen de kostprijs en zullen bovendien een deel van de tijdswinst tenietdoen omdat het wegschrijven van tussenresultaten in de registers ook tijd vergt.
\paragraph{}
\importtikzfigure{execution-pipelining-feedback}{Terugkoppeling verhindert pipelining.}
We kunnen ook niet zomaar elk proces uitvoeren volgens de pipelining uitvoeringsstrategie. Pipelining vereist dat de opeenvolgende stappen eerder onafhankelijk van elkaar kunnen werken. Stel bijvoorbeeld dat de tweede operatie echter een operand nodig heeft die in de vorige instructie wordt berekend, treed er een probleem op. Bij deze vorm van terugkoppeling verliezen we de performantie-winst. De pipeline wordt gestopt tot het resultaat van de eerste operatie beschikbaar is om vervolgens de tweede operatie uit te voeren. \figref{execution-pipelining-feedback} illustreert dit principe. Stel dat instructie $2$ bijvoorbeeld $t_4\leftarrow t_1-t_3$ berekent en instructie $3$ vervolgens $t_5\leftarrow t_2-t_4$. We kunnen opmerken dat instructie $3$ als operand een resultaat gebruikt dat wordt berekend door instructie $2$. Bijgevolg worden er geen nieuwe instructies in de pipeline ge\"injecteerd, tot instructie 2 volledig is uitgevoerd.
\subsection{Besluit}
Als algemene conclusie kunnen we stellen dat het zeer moeilijk is om de optimale implementatie te vinden. Dit aspect wordt bovendien bemoeilijkt omdat de optimalisaties op verschillende niveaus elkaar be\"invloeden: de minimalisatie van het aantal registers kan ertoe leiden dat we de functionele eenheden op een andere manier zullen optimaliseren. Daarom zullen we ons meestal tevreden stellen met heuristieken: benaderen methodes zoals bijvoorbeeld het max-cut algoritme. Sommige methodes zijn echter meer geschikt voor optimalisatie bij het ene niveau tegenover het andere. Bovendien is de beste keuze afhankelijk van de gekozen technologie: als we een schakeling implementeren op een FPGA zullen we logische blokken optimaliseren in plaats van transistoren.
\paragraph{}
Het algoritme herschrijven biedt ook veel mogelijkheden. Soms kunnen we hierdoor pipelining mogelijk maken of een functionele eenheid alsnog hergebruiken. Wanneer we echter het algoritme aanpassen, moeten we ook opnieuw het datapad minimaliseren.
\paragraph{}
Minimaliseren van het datapad is bijgevolg een complex proces waarbij het nagenoeg onmogelijk is om een globaal optimale implementatie te bekomen.
\section{Tijdsgedrag}
\seclab{timeFSMD}
\label{s:timeFSMD}
Hoewel een niet-programmeerbare processor in wezen een sequenti\"ele schakeling is, en we het tijdsgedrag omtrent sequenti\"ele schakelingen al hebben besproken, zullen we in deze sectie tijdsaspecten bespreken die enkel relevant zijn voor niet-programmeerbare processoren.
\subsection{Kritisch pad}
Een belangrijk aspect is deze van het kritisch pad. We hebben al besproken dat het kritisch pad bepaalt hoe snel een verandering aan de ingang effect heeft op de uitgang. In een processor is er niet altijd een in- of uitgang. Wel moet op het moment van een klokflank de correcte data op de ingang van de registers staan. Bovendien dienen we ook de controller in rekening te brengen die waar nodig de functionele eenheden configureert, en ook bepaald welke data er bijvoorbeeld uit de registers wordt uitgelezen.
\paragraph{}
Concreet komt het kritische pad nog altijd neer op het langste combinatorische pad. De verleiding is groot om hierbij enkel het datapad in rekening te brengen: namelijk vanuit een register door een functionele eenheid naar een register. Om te beantwoorden aan de formele definitie dient men dus de controller mee in rekening te brengen. Bovendien is het nagenoeg altijd zo dat het kritische pad zowel door de controller en het datapad loopt. Een typisch kritisch pad vertrekt dan ook vanuit bijvoorbeeld de ingang van een register die de toestand van de controller bijhoudt. Hierdoor wordt een nieuwe toestand in de registers van de controller geklokt. Vervolgens loopt het kritische pad doorheen de uitvoer-logica van de controller naar bijvoorbeeld een register van waaruit data dan door een functionele eenheid loopt om vervolgens ofwel terug te keren als status-signaal naar de controller of als resultaat-signaal naar de flipflop.
\paragraph{}
Het concreet bepalen van het kritisch pad is geen sinecure. Meestal stelt men tabellen op per functionele eenheid/register om te bepalen hoelang het signaal onderweg is, en wordt op basis van de tabellen dan het kritische pad bepaald. Meestal kan men het kritisch pad ook niet meteen vertalen naar de klokfrequentie: men dient meestal enige buffer te nemen die te wijten is aan meta-stabiliteit, lange lijnen, enzovoort.
\subsection{Verschoven kloksignalen (``clock skew'')}
\ssclab{clockSkew}
In het vorige hoofdstuk hebben we het reeds gehad over ``skew''. Een speciaal geval van skew die in de context van processoren belangrijk wordt is \termen{clock skew}, dit is de verschuiving van het kloksignaal. Processoren zijn immers grote en complexe schakelingen. Een constante doorheen de schakeling is echter dat de controller, de registers en eventuele RAM-geheugens allemaal afhankelijk zijn van \'e\'en en hetzelfde kloksignaal. Het gevolg is dat een verandering die aanvangt bij de klok zelf niet altijd op alle plaatsen in de schakeling op hetzelfde moment wordt waargenomen. Oorzaken van de vertraging zijn:
\begin{enumerate}
 \item Vertraging op de verbinding
 \begin{enumerate}
  \item Een verbinding die over de volledige schakeling loopt is relatief lang. Hoewel de snelheid van een elektrisch signaal twee derde van de lichtsnelheid bedraagt, is de frequentie zeer hoog en kan een relatief kleine afstand toch een significant tijdsverschil betekenen in verhouding tot de periode van de klok.
  \item Wanneer we werken met een FPGA doorloopt een kloksignaal meestal \'e\'en of meerdere schakelmatrices. Deze matrices zijn opgebouwd uit poorten die een significante vertraging teweeg brengen.
 \end{enumerate}
 \item Combinatorische logica op het klokpad
 \begin{enumerate}
  \item Sommige schakelingen werken met een clock-enabled. Een clock-enabled werkt door het kloksigaal door een AND-poort te sturen. Deze AND-poort levert een extra vertraging op.
  \item Een \termen{Klokbuffer}: een poort mag slechts een beperkte fan-out hebben. Het kloksignaal wordt echter gepropageerd naar een zeer groot aantal componenten. Men propageert het signaal dan ook door een hi\"erarchische structuur van klokbuffers. Wanneer componenten echter niet aangesloten zijn op dezelfde diepte van deze hi\"erarchie, ontstaat er een tijdsverschil.
 \end{enumerate}
 \item Verschillende daal- en stijgtijden: door verschillende capacitieve belasting (tussen bijvoorbeeld twee klokbuffers) zijn de vertragingen op lange lijnen -- zelfs al zijn ze identiek in lengte -- niet altijd gelijk. Wanneer een bepaalde lijn bijvoorbeeld een grote oppervlakte aan metaal vertegenwoordigt, stijgt de capaciteit van de virtuele condensator.
\end{enumerate}
\paragraph{Clock enabled bij laadbare registers}
In registers werken we vaak met een clock enabled. Dit is een belangrijke oorzaak van clock skew en kan zelfs leiden tot het inklokken van foute waarden: wanneer de data aan de ingang zelf afhankelijk is van de klok en sneller wordt aangepast dan de het signaal door de AND-poort propageert, wordt soms de nieuwe waarde opgeslagen.
\begin{figure}[hbt]
\centering
\importtikzsubfigure{loadregi-mux}{Met multiplexer.}
\importtikzsubfigure{loadregi-cle}{Met clock enabled.}
\caption{Laadbaar registers}
\figlab{loadregi}
\end{figure}
Er zijn dan ook grofweg twee technieken om een selectief op bepaalde klokflanken nieuwe waarden op te slaan:
\begin{enumerate}
 \item Met behulp van een multiplexer (zoals bij \figref{loadregi-mux});
 \item Met behulp van een AND-poort (zoals bij \figref{loadregi-cle}).
\end{enumerate}
Zoals zo vaak is geen van de twee technieken altijd te verkiezen: beide implementaties hebben hun voor- en nadelen. Zo kost register met multiplexer $6$ transistoren per bit meer dan een AND-poort. Ook het energie-verbruik ligt hoger: het kloksignaal komt immers in de flipflop terecht waar een groot aantal transistoren zullen schakelen. Dit verbruik is nutteloos wanneer we geen nieuwe gegevens willen inklokken en we dus opnieuw dezelfde waarde op de data-ingang aanbieden. Anderzijds is een multiplexer een vrij veilige optie. De nadelen van een AND-poort zijn dat dit typisch clock-skew introduceert omdat het kloksignaal duidelijk vertraging oploopt wanneer het door de AND-poort propageert. Wanneer het kloksignaal hoog is, mag de ``clock enabled'' bovendien niet zomaar worden verandert: de flipflop zou dit immers interpreteren als een veranderend kloksignaal. Het vereist dus wat ontwerpervaring om implementatie met een AND-poort veilig te stellen.
\subsection{Synchroniseren van asynchrone ingangen}
Ondanks voorzichtig ontwerp van een niet-programmeerbare processor, komt invoer doorgaans van buiten de schakeling. Deze data wordt misschien ook door een synchrone schakeling -- bijvoorbeeld een processor -- aangeboden, maar het synchroniseren van verschillende klokken is onmogelijk. Zelfs in eenzelfde synchrone schakeling kunnen bovendien synchronisatieproblemen ontstaan, bijvoorbeeld wanneer bepaalde delen van de schakeling sneller resultaten berekenen. Dergelijke effecten maken het echter niet evident om logica te ontwikkelen: meestal gaan we ervan uit dat tijdens berekeningen van de processor de ingang niet zomaar een andere waarde aanneemt. Stel dat de processor afwisselend de twee getallen aan de ingang optelt en aftrekt maken we meestal de assumptie dat halverwege de optelling de operanden niet veranderen. Om asynchrone veranderingen op te vangen gebruikt men doorgaans flipflops die de waarden aan de ingangen op vaste momenten inlezen. De data-uitgangen van deze flipflops zijn gedurende een klokflank stabiel.
\paragraph{}
Dit is echter de theorie en ook hier kunnen dingen foutlopen. Een signaal moet immers een tijdje op de ingang van een flipflop worden aangelegd voor de klokflank om het correct in te klokken. Indien de gegevens op tijd worden aangelegd is er dus geen probleem. Wanneer de gegevens echter kort voor de klokflank worden aangeleverd kunnen er verschillende problemen ontstaan:
\begin{itemize}
 \item De gegevens komen te laat waardoor de oude gegevens na de klokflank op de uitgang van de flipflops worden aangelegd. De gegevens komen dan meestal met vertraging van \'e\'en klokflank wel beschikbaar. Dit wordt ge\"illustreerd in periode $a$ op \figref{synchronization-problems}.
 \item Een deel van de gegevens wordt correct ingeklokt, een ander deel wordt foutief ingelokt, dit kan ertoe leiden dat de flipflops invoer aanleggen die zelfs niet mogelijk werd geacht. Dit wordt ge\"illustreerd in periode $b$ op \figref{synchronization-problems}.
 \item \'E\'en of meer flipflops komen in een metastabiele toestand terecht: er wordt noch $0$ noch $1$ op de ingang aangelegd. Het duurt een zekere tijd voor een flipflop uit dergelijke toestand gaat. Dit effect kan ook optreden wanneer de nieuwe data al lang op de flipflop wordt aangelegd. Dit wordt ge\"illustreerd in periode $c$ op \figref{synchronization-problems}.
\end{itemize}
Aan de eerste twee problemen is weinig te verhelpen. In de praktijk zal men dit meestal oplossen met een synchronisatiesignaal: men stuurt bijvoorbeeld het kloksignaal mee op \'e\'en van de verbindingen tussen de twee circuits, of in het geval er maar af en toe data op de ingang komt te staan kondigt men dit aan door de ingangen met de regelmaat van de klok te laten veranderen van signaal\footnote{Dit is het geval bij Ethernet (IEEE 802.3).}.
\begin{figure}[hbt]
\centering
\importtikzsubfigure{synchronization-circuit}{Schakeling.}
\importtikzsubfigure{synchronization-problems}{Problemen met synchronisatie.}
\figlab{synchronization}
\caption{Synchronisatie.}
\end{figure}
\paragraph{}
Metastabiliteit is echter een probleem waar men wel iets aan kan doen. Uit \sscref{flipflop} weten we immers dat de kans dat een schakeling zich na een zekere tijd nog steeds in een metastabiele toestand bevindt, exponentieel daalt met de tijd. Als we dus een redelijk termijn wachten, kunnen we vrij zeker stellen dat de de binnenkomende signalen niet metastabiel zijn. Deze tijd noemt men de \termen{metastability resolution time $t_r$}. Logischerwijs wensen we dat $t_r\geq t_{\mbox{\small meta}}$ met $t_{\mbox{\small meta}}$ de tijd om uit een metastabiele toestand te geraken. $t_r$ hangt echter van drie andere parameters af:
\begin{equation}
t_r=t_{\mbox{\small clock}}-t_{\mbox{\small comb}}-t_{\mbox{\small set-up}}
\end{equation}
Met $t_{\mbox{\small clock}}$ de periode van de klok, $t_{\mbox{\small clock}}$ de tijd die het signaal nodig heeft om door de combinatorische schakeling te propageren van het circuit en $t_{\mbox{\small set-up}}$ de set-up tijd om de resultaten op te slaan in de flipflops van het circuit\footnote{Bijvoorbeeld om op basis van de invoer een nieuwe toestand aan te nemen.}.
\importtikzfigure{synchronization-single}{De verschillende componenten van de ``metastability resolution time''.}
\paragraph{}
De klokperiode $t_{\mbox{\small clock}}$ staat in principe niet vast: we kunnen zelf de kloksnelheid bepalen, maar wensen meestal een hoge kloksnelheid. Bovendien wordt de kloksnelheid ook beperkt in de mate dat invoer beschikbaar komt: wanneer we bijvoorbeeld geluid verwerken moet we $40~\mbox{kHz}$ aan metingen kunnen aanbieden. In het geval we echter een niet-programmeerbare processor beschouwen, staat de toepassing meestal vast en dus ook de invoersnelheid van die toepassing. De kloksnelheid verder opdrijven heeft dan geen zin. We kunnen dus de klokperiode altijd proberen te maximaliseren.
\paragraph{}
Men kan ook inzetten op technologie. Snelle flipflops hebben een kortere set-up tijd en bovendien verkleint ook de halfwaardetijd: hierdoor zullen we dus ook sneller de metastabiele toestand verlaten. Bijgevolg wordt $t_{\mbox{\small meta}}$ kleiner. Een nadeel van deze flipflops is dat ze meer vermogen verbruiken.
\paragraph{}
Extra hardware kan ook een bijdrage leveren. Wanneer de processor de data moet verwerken met een bepaalde doorvoersnelheid, maar de vertraging niet zo belangrijk is\footnote{De vertraging is bijvoorbeeld wel van belang bij real-time applicaties zoals live televisie-uitzendingen en \emph{Skype}.} kan men in plaats van \'e\'en flipflop, een opeenvolging aan flipflops plaatsen. Men geeft hierdoor meer tijd aan het signaal om uit de metastabiele toestand te komen maar behoudt tegelijk een sterke (interne) klokfrequentie.
\begin{figure}[hbt]
\centering
\importtikzsubfigure{synchronization-multi}{Zelfde kloksignaal.}
\importtikzsubfigure{synchronization-freq}{Trager kloksignaal.}
\caption{Metastabiliteit oplossen door een opeenvolging van flipflops.}
\end{figure}
\paragraph{}
Indien de ingangssignalen zelf met hoge frequentie worden aangeleverd kunnen we $n$ flipflops plaatsen die de synchronisatie uitvoeren. Een signaal propageert dus eerst door $n$ flipflops en heeft dus meer tijd om uit de metastabiele toestand te geraken. In dat geval is de resolutie-tijd:
\begin{equation}
\begin{array}{cr}
t_r=n\cdot\brak{t_{\mbox{\small clock}}-t_{\mbox{\small set-up}}}-t_{\mbox{\small comb}}&\mbox{(\figref{synchronization-multi})}
\end{array}
\end{equation}
Meestal volstaan $2$ of $3$ flipflops. Een abstracte implementatie staat beschreven in \figref{synchronization-multi}.
\paragraph{}
Soms verandert het signaal niet zo snel. Denk bijvoorbeeld aan een processor die een deling moet uitrekenen. Meestal zal de interne klok significant sneller klokken omdat er veel operaties moeten worden uitgevoerd om het quoti\"ent uit te rekenen. Dit terwijl de invoer trager bijvoorbeeld maar om de 10 klokflanken wordt aangepast. In dat geval kunnen we werken met twee flipflops die aan een lagere (in dit geval $\dfrac{1}{10}$ van de originele klokfrequentie nieuwe waardes opslaan. De resolutie-tijd wordt dan:
\begin{equation}
\begin{array}{cr}
t_r=n\cdot t_{\mbox{\small clock}}+t_{\mbox{\small clock}}-t_{\mbox{\small comb}}-2\cdot t_{\mbox{\small set-up}}&\mbox{(\figref{synchronization-freq})}
\end{array}
\end{equation}
Een abstracte implementatie staat beschreven in \figref{synchronization-freq}.
\chapter{Programmeerbare Processoren}
\chplab{programmableprocessors}
\chapterquote{Mensen hebben met computers gemeen dat ze ook niets bereiken zonder goede programmeurs.}{Toon Verhoeven, Nederlands aforist (??-??)}
\begin{chapterintro}
In dit hoofdstuk bespreken we programmeerbare processoren. Deze processoren verschillen van de niet-programmeren processoren omdat een gebruiker een programma -- een reeks van instructies -- kan uitvoeren op de processor en bijgevolg in grote mate het algoritme die de processor uitvoert zelf kan bepalen. Centrale processoreenheden (CPU's) zijn hiervan slechts een subset van de programmeerbare processoren. We bespreken eerst de hoe een instructie eruitziet en hoe we een schakeling kunnen ontwerpen om deze instructies uit toe voeren. Cruciaal hierbij zijn de verschillende adresseermodi. Vervolgens beschouwen we de twee grote instructie-families: RISC en CISC. We zullen voor beide een instructieset ontwerpen en de verschillende aspecten die hierbij komen kijken bespreken.
\end{chapterintro}
\minitoc[n]
\section{De Programmeerbare Processor}
We zullen in deze sectie het verschil bespreken tussen een niet-programmeerbare processor (zie \chpref{nonprogramming}) en een programmeerbare processor. Algemeen is deze grens eerder vaag. Ook bij een niet programmeerbare processor zullen de controller en het datapad meestal be\"invloed worden door signalen van buiten de schakeling. Sommige technici kunnen door deze invoer te manipuleren de processor een algoritme laten uitvoeren waarvoor de schakeling niet ontworpen was.
\paragraph{}
Een duidelijke grens kunnen we trekken bij de controller. Bij conventie stellen we dat een processor niet-programmeerbaar is wanneer de schakeling wordt aangestuurd met een vaste controller. In het geval we dus een ander algoritme willen uitvoeren zullen we een ander controller moeten implementeren om het uit te voeren. Bij een programmeerbare processor dient men de controller niet aan te passen. De controller beschikt immers over een geheugen waarin het programma geladen kan worden. Door het geheugen aan te passen zal de controller het datapad anders aansturen waardoor een ander algoritme kan worden uitgevoerd. We kunnen dus stellen dat de eindige toestandsautomaat die in de controller werd ge\"implementeerd vast staat, onafhankelijk van het ingeladen programma. Het geheugen waaruit zo'n controller leest noemen we het \termen{programmageheugen}. De gegevens die van de controller naar dit geheugen stuurt worden het \termen{adres}, de \termen{programmateller} of de \termen{program counter (PC)} genoemd. De data van het programmageheugen naar de controller noemen we de \termen{instructie}.
\paragraph{}
Merk op dat de definitie hierbij geen concrete uitspraak doet over hoe een programma of instructie er precies dient uit te zien. De definitie impliceert bijvoorbeeld niet dat de processor elk te beschrijven algoritme moet kunnen uitvoeren. Hierbij kunnen we bijvoorbeeld denken aan een ``Graphical Processing Unit (GPU)''. Een GPU is een programmeerbare processor die gespecialiseerd is in grafische taken. De instructieset is dan ook eerder beperkt tot grafische operaties. Hoewel men dergelijk processoren meestal niet kan programmeren om bijvoorbeeld Dijkstra's algoritme uit te voeren, is de processor wel programmeerbaar.
\paragraph{}
\importtikzfigure{processor-programming}{De structuur van een programmeerbare processor.}
\figref{processor-programming} toont hoe een programmeerbare processor er in grote lijnen uitziet. Merk op dat het enige verschil tussen deze figuur en \figrefpag{processorInformationStreams} de introductie van een programmageheugen is.
\section{Instructies en Velden}
Nu we de structuur van een programmeerbare processor hebben voorgesteld, zullen we in deze sectie de verschillende aspecten en terminologie van een programma bespreken.
\subsection{Programma}
Een algemeen aanvaarde definitie voor een \termen{programma} is een sequentie van instructies. De instructies zijn op zo'n manier bepaald en geordend dat ze samen een complexe (en nuttige) taak uitvoeren. In dit opzicht bevat een sequentie van instructies dus dezelfde informatie als de eindige toestandsmachine van de controller bij een niet-programmeerbare processor.
\subsection{Instructie}
Een instructie is een reeks van bits die de informatie van \'e\'en toestand in het ASM-schema of \'e\'en toestand in de eindige toestandsautomaat van de controller voorstellen. Door pipelining kan een instructie of multicycling kan een instructie echter meerdere klokcycli duren. Dit kan men implementeren door bijvoorbeeld een andere volgende instructie te kiezen. Om een toestand voor te stellen zijn drie types informatie vereist:
\begin{itemize}
 \item De aansturing van het datapad: het bepalen van de stuursignalen naar de verschillende functionele eenheden, registers, tri-state buffers en multiplexers betrokken in het datapad.
 \item \termen{Data-uitwisseling}: sommige instructies bepalen welke informatie er ingeladen of weggeschreven wordt naar (externe) geheugens.
 \item De volgende instructie: de meeste algoritmes bevatten lussen en voorwaardelijke gedeeltes. Deze controle wordt ge\"implementeerd doordat de instructies (impliciet) bepalen wat de volgende instructie zal zijn.
\end{itemize}
Een instructie moet niet elk type informatie expliciet specificeren. We kunnen bijvoorbeeld denken aan het bepalen van de volgende toestand: meestal wordt een programma zo gestructureerd dat de volgende instructie bijna altijd op het volgende adres staat. In dat geval zullen enkel instructies die afwijken van deze regel dit moeten specificeren.
\paragraph{Notatie van een instructie}
Om instructies uit te drukken bestaan er twee typische notaties: de \termen{mnemonische notatie} en de \termen{actie-notatie}. In de mnemonische notatie specificeert men eerst de operatie gevolgd door het doel\footnote{De plaats waar het resultaat zal worden opgeslagen.} en de operanden. Verder worden zowel het doel en de operanden gespecificeerd aan de hand van adressen. Een typische instructie is bijgevolg \verb+add A B C+. Dit voorbeeld is een instructie uit de \verb+80x86+ instructieset. Mnemonische notatie wordt dan ook vaak gebruikt in assembleertalen (\verb+80x86+, ). Actie-notatie specificeert daarentegen eerst het doel, meestal gevolgd door bijvoorbeeld een pijl met daarna de functie en de operand. Een concreet voorbeeld van een instructie volgens deze notatie is \verb/Mem[A] <- Mem[B]+Mem[C]/. Actie-notatie is populair bij hardwarespecificatietalen.
\subsection{Instructieformaat}
De mnemonische notatie en de actie-notatie zijn manieren om instructies voor te stellen zodat ze leesbaar zijn voor mensen. In digitale logica wordt een instructie enkel voorgesteld door een sequentie aan bits. De ontwerper van een processor dient dan ook een \termen{instructieformaat} te specificeren: een beschrijving hoe een sequentie bits een instructie bepaalt. Men kan dit formaat natuurlijk vrij bepalen, maar meestal beoogt men een structurele opbouw: de instructie wordt onderverdeeld in \termen{velden}: groepen van bits waar een betekenis of een subtaak aan wordt toegekend. In de meeste instructieformaten komen volgende velden voor:
\begin{itemize}
 \item \termen{Instructietype}: een groep bits die de klasse van de instructie aangeeft (bijvoorbeeld een sprongbevel, een bewerking, een geprivilegieerde instructie,...)
 \item \termen{Opcode} ofwel \termen{operation code}: een groep bits die de bewerking voorstellen (bijvoorbeeld een optelling, vermenigvuldiging,...)
 \item Adres: een groep bits die de locatie van een operand of een resultaat specificeert. Een adres hoeft echter niet beperkt te zijn tot de locatie in een (extern) geheugen: ook registers en registerbanken kunnen soms worden geadresseerd.
 \item \termen{Adresseermode}: een adresseermode bepaalt hoe het adres gespecificeerd in bits wordt omgezet in een fysisch adres. Zo kan de adresseermode bijvoorbeeld bepalen dat het adres een register uit de registerbank specificeert of dat de adressen indirect\footnote{Bij indirecte adressering leest men de waarde van het geheugen uit op de gegeven locatie. Die waarde bepaalt dan de locatie van de effectieve waarde.} moeten worden berekend.
 \item Constante: bewerkingen zoals een optelling tellen soms een constante op bij een register. In dat geval moet de constante in de instructie worden ingebed.
\end{itemize}
Men dient op te merken dat een veld niet noodzakelijk een vaste lengte heeft of slechts \'e\'enmaal voorkomt. Stel bijvoorbeeld dat de adresseermode bepaalt dat de gegevens uit een registerbank moeten uitgelezen worden, verwachten we dat het adres korter zal zijn dan wanneer we het adres uit een RAM-geheugen halen. Verder zal men bij sommige operaties twee of meer operanden moeten selecteren. In dat geval is het dus mogelijk dat het adresveld meerdere keren voorkomt. Sommige processoren voorzien ook een instructieformaat waarbij men twee of meer bewerkingen kan specificeren die dan parallel worden uitgevoerd. In dat geval komt de opcode dus twee of meer keer voor.
\subsection{Generische Instructiecyclus}
Een processor voert een instructie doorgaans uit in vijf stappen. Deze stappen noemt men de generische instructiecyclus. De stappen zijn:
\begin{enumerate}
 \item Lees de instructie in: het programmageheugen wordt uitgelezen en de instructie wordt in het \termen{instructieregister (IR)} opgeslagen. De programmateller wordt verhoogd.
 \item Bereken de adressen: Op basis van de adresseermodi en de adressen in de instructie worden de echte adressen bepaald. Een adres uit een registerbank zal er dus anders uitzien dan een adres uit het RAM-geheugen.
 \item Lees de operanden: de adressen van de operanden worden uitgelezen en weggeschreven in tijdelijke registers.
 \item Voer de bewerking uit: op basis van de data in de operand-registers en de opcode kan men de relevante bewerking op de relevante data uitvoeren. Het resultaat wordt in een tijdelijk register geplaatst.
 \item Schrijf het resultaat weg: het resultaat wordt weggeschreven in een registerbank of RAM-geheugen (indien dit relevant is voor de bewerking).
\end{enumerate}
Deze cyclus wordt eindeloos herhaald en leent zich meestal erg goed tot pipelining: enkel wanneer de volgende instructie een operand moet inlezen die bepaald wordt door een instructie die kort ervoor is uitgevoerd moet de pipeline worden onderbroken. Men kan een sprongbevel uitvoeren door in de instructie de programmateller aan te passen.
\subsection{Uitvoeringssnelheid}
De uitvoeringssnelheid van een instructie hangt in grote mate af van twee factoren:
\begin{itemize}
 \item De snelheid van het datapad
 \item Het aantal toegangen tot extern geheugen.
\end{itemize}
We kunnen de snelheid van het datapad doorgaans opdrijven door meer hardware te voorzien die bijvoorbeeld bewerkingen in parallel uitvoeren. Het aantal toegangen tot extern geheugen kunnen we dan weer verlagen door kleine en simpele instructies te voorzien waardoor elke instructie maar een beperkt aantal operaties op het geheugen uitvoert.
\paragraph{}
De grootte van een instructie is echter een trade-off. Wanneer we grote instructies voorzien met een groot aantal bits laten we de programmeur toe om een complexe taak in zo'n instructie te specificeren. Bijgevolg verwachten we dat een programma uit een klein aantal van dergelijke instructies zal bestaan. Omdat de instructies echter complex zijn, verwachten we een traag datapad en veel geheugentoegangen. Wanneer de processor enkel simpele instructies aanbiedt kan het datapad deze snel uitvoeren, maar een programma zal een groot aantal instructies bevatten. De complexiteit van een \termen{instructieset} wordt dan ook soms uitgedrukt in het aantal adresvelden.
\subsection{Adresvelden}
In deze subsectie zullen we enkele instructiesets bespreken volgens het aantal adresvelden. Bij de verschillende instructiesets zullen we aantal geheugentoegangen berekenen die nodig zijn om de functie $\brak{a+b}\times\brak{a-b}$ uit te rekenen. Deze operatie wordt doorgaans niet als \'e\'en instructie aangeboden (tenzij bij processoren die taken uitvoeren waarbij deze bewerking zeer regelmatig zou voorkomen). We zullen dan ook uitgaan van een algemene instructieset die de optelling (\termen{Add-instructie}), aftrekking (\termen{Sub-instructie}) en vermenigvuldiging (\termen{Mul-instructie}) voorziet.
\paragraph{}
Alvorens we de instructiesets met elkaar kunnen vergelijken, zullen we eerst enkele aannames moeten maken over hoe gegevens en instructies kunnen worden ingelezen. We zullen uitgaan van een woordlengte\footnote{De woordlengte is het aantal bits die in een geheugen onder \'e\'en adres worden opgeslagen.} $w$. We maken de assumptie dat een instructie zonder geheugenadressen in \'e\'en woord\footnote{Een woord is een sequentie van $w$ bits met $w$ de woordlengte.} kan worden opgeslagen. Het geheugen omvat $2^w$ adressen, bijgevolg telt het geheugen $w\cdot 2^w$ bits en is elk adres voor te stellen met een woord. Een instructie met $k$ geheugenadressen kan dus worden opgeslagen in $k+1$ woorden. We vergelijken de instructiesets op basis van geheugentoegangen. Dit zijn dus het aantal toegangen om de instructie uit te lezen samen met het uitlezen en wegschrijven van gegevens die in de instructie worden gespecificeerd.
\subsubsection{Instructies met 3 adresvelden}
Een instructieset met drie adresvelden bepaalt meestal \'e\'en adres voor het resultaat en twee adressen voor de operanden. Bijvoorbeeld de \verb+Add a b c+ instructie berekent de optelling van de gegevens die op de geheugenplaatsen $b$ en $c$ staan en plaatst het resultaat dus in adres $a$. Om $\brak{a+b}\times\brak{a-b}$ dus uit te rekenen zullen we volgend programma uitvoeren:
\begin{verbatim}
Add c a b
Sub x a b
Mul c c x
\end{verbatim}
Per instructie voorzien we dus $4+3$ geheugentoegangen. $4$ instructietoegangen per instructie, $2$ leesoperaties voor de operanden en $1$ schrijfoperatie. Omdat we $3$ instructies uitvoeren vereist het programma dus $21$ geheugentoegangen. We kunnen echter opmerken dat in de laatste instructies we tweemaal hetzelfde adres vermelden. Dit soort instructies vormen dan ook de argumentatie om soms instructies met twee adresvelden te gebruiken.
\subsubsection{Instructies met 2 adresvelden}
Instructiesets met twee adresvelden zijn vrij populair. \verb+80x86+ is een voorbeeld van zo'n instructieset. In zo'n instructieset vertegenwoordigen het eerste en tweede adresveld de adressen van de operanden en het eerste veld ook het adres van het resultaat. Het eerste adres wordt bijgevolg overschreven.
\paragraph{}
Een probleem bij dit mechanisme is dat we soms na de bewerking de operanden willen kunnen hergebruiken om andere operaties uit te voeren. Zo willen we na het berekenen van $a+b$ nog over $a$ en $b$ kunnen beschikken om $a-b$ uit te rekenen. We kunnen de waarde kopi\"eren om dit probleem te vermijden. We kunnen de waarde van $x$ kopi\"eren naar adres $y$ met behulp van twee instructies: \verb+Sub y y+ en \verb+Add y x+. De meeste processoren voorzien echter een kopieer instructie: de \termen{Mov-instructie}. Het voordeel van deze instructie is dat naast het ophalen van de instructie slechts twee geheugentoegangen vereist zijn.
\paragraph{}
We kunnen het algoritme dan ook als volgt implementeren:
\begin{verbatim}
Mov c a
Add c b
Mov x a
Sub x b
Mul c x
\end{verbatim}
De Mov-instructie vereist in totaal $5$ geheugentoegangen, de overige instructies vereisen $6$ geheugentoegangen. In totaal vereist het uitvoeren van het algoritme dus $28$ geheugentoegangen. We dienen echter wel op te merken dat de instructies sneller zullen worden uitgevoerd.
\subsubsection{Instructies met 1 adresvelden}
We kunnen de instructieset verder reduceren tot \'e\'en adresveld per instructie. Dit doen we met behulp van een \termen{accumulator (ACC)}. Een accumulator is een speciaal register die dienst doet als zowel de eerste operand en het register waar het resultaat in wordt geplaatst. Men kan deze instructieset dus vergelijken met de eerste instructieset, maar waarbij de eerste operator altijd een vast adres voorstelt. Het voordeel van een accumulator is de implementatie door middel van een register: de accumulator inlezen of resultaten wegschrijven vereist bijgevolg geen geheugentoegang.
\paragraph{}
Ook wanneer we een accumulator gebruiken zullen we soms tussenresultaten tijdelijk in een andere variabele willen opslaan om de waarde later in te lezen. We kunnen hier geen gebruik maken van de Mov-instructie omdat het resultaat altijd vast staat: de accumulator. Daarom worden twee nieuwe instructies ge\"introduceerd: de \termen{Load-instructie} leest het adres uit en plaatst de waarde in de accumulator, de \termen{Store-instructie} schrijft de waarde van de accumulator weg in het opgegeven adres. De Store-instructie is bijgevolg een instructie waar de accumulator geen dienst doet als de ontvanger van het resultaat.
\paragraph{}
We kunnen het algoritme realiseren met volgende code:
\begin{verbatim}
Load a
Add b
Store x
Load a
Sub b
Mul x
Store c
\end{verbatim}
Elke instructie vereist telkens $3$ geheugentoegangen: $2$ om de instructie uit te lezen en $1$ geheugentoegang om het adres uit te lezen of de resultaten weg te schrijven. Omdat het algoritme in $7$ instructies kan worden ge\"implementeerd, zijn er in totaal $21$ geheugentoegangen vereist.
\subsubsection{Instructies zonder adresvelden}
Sommige instructiesets bevatten geen adressen als operanden. In een dergelijk systeem moeten we echter wel een systeem implementeren die zelf de adressen voorstelt.
\paragraph{}
Een populaire methode werkt met een stapelgeheugen. In zo'n systeem beschouwen we een stapel die groeit bij een een Load-instructie. Wanneer een instructie een berekening uitvoert worden de operanden uit de bovenste elementen van de stapel gehaald. Deze elementen worden van de stapel gehaald en het resultaat wordt vervolgens op de stapel gezet. Ook in een dergelijk systeem is er soms nood aan het opslaan van tussenresultaten om deze later te hergebruiken. Ook hiervoor gebruiken we de store operatie. De operatie neemt als argument een adres van de stapel: het aantal elementen onder de top. Ook dit kan men als een adres zien. Men maakt echter meestal de assumptie dat deze waarde niet buitengewoon groot is en de waarde dus in de instructie kan worden meegenomen. Bijgevolg bevat de instructie dus geen adres. Omdat het bovenste gedeelte van de stapel meestal met een registerbank wordt ge\"implementeerd, vereisen de bewerkingen bijgevolg geen geheugentoegangen.
\paragraph{}
We kunnen het algoritme implementeren met volgende instructies:
\begin{verbatim}
Load a
Load b
Add
Load a
Load b
Sub
Mul
Store c
\end{verbatim}
Om de impact te berekenen moeten we opnieuw een onderscheid maken tussen twee soorten instructies: Load- en Store instructies vereisen twee geheugentoegangen ($1$ geheugentoegang om de instructie op te halen en $1$ instructie om het resultaat op te halen of weg te schrijven) de overige instructies vereisen slechts \'e\'en geheugentoegang. In totaal vereist dit programma dus $13$ toegangen tot het geheugen.
\subsubsection{Instructies met registerbank-adressen: dubbele adressering}
Tot slot dienen we nog te vermelden dat een adresveld niet noodzakelijk altijd een geheugenadres moet omvatten. Men kan bijvoorbeeld een bit in dit veld voorzien die bepaalt of het adres een geheugenadres specificeert of het adres van een registerbank. Daarnaast kan de interpretatie van een dergelijk adres ook afhangen van de instructie. Moderne processoren interpreteren bijvoorbeeld vaak de adressen bij bewerkingen als registerbank-adressen en de tweede operand bij een Load- of Store-instructie als een geheugenadres. Vermits een algoritme vooral met tussenresultaten zal rekenen verwachten we een tijdswinst omdat we niet telkens de operanden uit het geheugen moeten uitlezen.
\paragraph{}
Een dergelijke instructieset kan ook instructies met variabele lengte beschouwen. Sommige instructies omvatten immers geen geheugenadressen, andere wel. In dat geval kan de eerste instructie bijvoorbeeld aanleiding geven om het volgende woord in een instructieregister in te lezen, of de instructie meteen uit te voeren.
\paragraph{}
Registeradressen worden meestal voorgesteld met een prefix \verb+R+. In het geval van een dergelijke instructieset ziet het programma er als volgt uit:
\begin{verbatim}
Load R1 a
Load R2 b
Add R3 R1 R2
Sub R4 R1 R2
Mul R5 R3 R4
Store c R5
\end{verbatim}
Bewerking instructies worden voorgesteld in \'e\'en woord en vereisten geen extra geheugentoegang. Bijgevolg vereisen ze \'e\'en geheugentoegang. Load- en Store-bewerkingen vereisen $3$ geheugentoegangen: $2$ bij het inlezen van de instructie en $1$ bij het inlezen of wegschrijven van de data. In totaal vereist het programma dan ook $12$ keer toegang tot het geheugen.
\subsubsection{Besluit}
Er zijn verschillende instructiesets mogelijk met een verschillend aantal adresvelden. Instructies met een groot aantal adresvelden laten compacte programma's toe maar vereisen soms onnodig toegang tot het geheugen. Deze extra belasting komt in twee vormen: het inlezen van de adresvelden van de instructie en het uitlezen of wegschrijven van data in de vermelde geheugenadressen.
\paragraph{}
Een instructieset met minder adresvelden zal minder vaak onnodig toegang tot het geheugen aanvragen. Anderzijds wordt het programma langer waardoor de processor meer instructies uit het geheugen met uitlezen.
\subsection{Adresseermodi}
In de loop der jaren zijn er verschillende manieren ontwikkeld om een geheugenadres te bepalen op basis van een adresveld. Deze methodes noemen we adresseermodi. Adresseermodi hebben als primair doel het aantal bits te reduceren die het adresveld in beslag neemt. Anderzijds laten verschillende adresseermodi soms toe om meer te realiseren per instructie en dus tot compactere programma's te schrijven.
\paragraph{}
Ook in hogere programmeertalen maakt men impliciet gebruik van adresseermodi. We kunnen bijvoorbeeld denken aan arrays. Wanneer men in \verb+Java+ het tiende element van een array \verb+a+ wil uitlezen schrijft men \verb+a[10]+. Impliciet stelt men echter het adres dat tien plaatsen verder dan het begin van het record voor \verb+a+ ligt. Adresseermodi kunnen programma's dus ook leesbaarder maken.
\subsubsection{Impliciete adressering}
\termen{Impliciete adressering} is een adresseermode waarbij een adresveld niet vermeld wordt in de instructie maar door de processor zelf kan worden berekend.
\paragraph{}
Dit is bijvoorbeeld het geval bij processoren met een stapelgeheugen. In een dergelijke instructieset worden de adressen afgeleid uit de toestand van de stapel: de bovenste elementen bevatten de data van de operanden. De processor rekent deze aspecten dus zelf uit.
\paragraph{}
Ook bij processoren die werken met een accumulator is dit het geval: men vermeldt immers de bestemming van de data niet. Dergelijke processoren bieden meestal een instructie aan om de gegevens uit de accumulator te verwijderen: de \termen{CLRA-instructie}.
\subsubsection{Onmiddellijke adressering}
Behalve in het adresveld de plaats in het geheugen te specificeren waar de data moet worden opgehaald, kunnen we ook de data zelf opslaan in een adresveld. Deze vorm van adressering noemen we \termen{onmiddellijke adressering}. Dit is bijvoorbeeld het geval wanneer we werken met constanten.
\subsubsection{Directe adressering}
Meestal lezen we operanden uit met behulp van een adres. Dit adres verwijst ofwel naar een geheugenadres ofwel naar een index van een registerbank. In beide gevallen spreken we over \termen{directe adressering}. Het is meestal voordelig om een registerbank te gebruiken. Dit omwille van twee redenen: een registerbank bevat minder adressen en de toegang is sneller. Omdat een registerbank typisch $8$ tot $256$ registers bevat, is bijgevolg de lengte van een registeradres tussen de $3$ en $8$ bits, terwijl een geheugenadres typisch tussen de $32$ en $64$ bits lang is. We besparen dus ook tijd bij het inladen van de instructie.
\subsubsection{Indirecte adressering}
Soms kennen we de plaats waar de data in het geheugen staat niet expliciet. Stel bijvoorbeeld dat we werken met een heap\footnote{Een heap is een geheugenruimte waar men dynamisch records kan toevoegen en terug verwijderen.}, hangt de plaats waar de data staat meestal af van de manier hoe het programma eerder werd doorlopen. Het adres wordt dan ook ergens anders bewaard in het geheugen of een registerbank.
\paragraph{}
Indirecte adressering is een belangrijke vorm in programma's. Naast werken met een onbekend adres laat het ook toe het adres mee te geven wanneer we bijvoorbeeld een subroutine uitvoeren die een ``pointer'' vereist.
\paragraph{}
\termen{Indirecte adressering} maakt het mogelijk om in \'e\'en instructie de operand in te lezen waarvan het adres ergens in het geheugen staat. In dat geval dienen we dus het adres te specificeren waar het adres staat.
\begin{figure}[hbt]
\centering
\importtikzsubfigure{address-indirect}{Indirecte adressering}
\importtikzsubfigure{address-indirect-register}{Register-indirecte adressering}
\caption{Indirecte adressering}
\end{figure}
\paragraph{}
Tweemaal het geheugen uitlezen in \'e\'en instructie is echter een dure operatie. Daarom zal men de adressen doorgaans in een registerbank opslaan. In dat geval spreken we dan ook van \termen{register-indirecte adressering}.
\subsubsection{Relatieve adressering}
Hedendaagse besturingssystemen ondersteunen doorgaans multiprogramming: het tegelijk uitvoeren van verschillende programma's. Bijgevolg worden verschillende programma's tegelijk in het register geladen. De compiler van een programma weet doorgaans op voorhand niet waar het programma in het geheugen zal worden ingeladen. Wanneer een programma echter een sprongbevel uitvoert, moet het adres van het sprongbevel wijzen naar het correcte adres.
\paragraph{}
Het aangehaalde voorbeeld is \'e\'en van de redenen om \termen{relatieve adressering} toe te passen. Bij relatieve adressering houdt de processor een \termen{basisadres} bij in een impliciet register. Dit basisadres wordt bij het gegeven adres opgeteld om het werkelijke adres te bepalen. Het adresveld wordt in dat geval de \termen{offset} genoemd.
\begin{figure}[hbt]
\centering
\importtikzsubfigure{address-relative}{Relatieve adressering}
\importtikzsubfigure{address-relative-register}{Register-relatieve adressering}
\caption{Relatieve adressering}
\figlab{address-relative-general}
\end{figure}
\paragraph{}
Naast het verzekeren van correcte spronginstructies kan relatieve adressering ook tot kortere instructiewoorden leiden. We kunnen bijvoorbeeld een registerbank voorzien die verschillende basisadressen voorziet. De basisadressen kunnen bijvoorbeeld wijzen naar datastructuren die vaak gebruikt worden. Naast een veld die het register in de registerbank specificeert, dient men een veld te voorzien die de offset vanaf het basisadres voorstelt. Een dergelijke vorm van adressering noemen we \termen{register-relatieve adressering}.
\subsubsection{Ge\"indexeerde adressering}
We kunnen ook de twee velden omdraaien door het basisadres in de instructie onder te brengen en de offset in een impliciet register of een registerbank op te slaan. Deze vormen van adressering noemen we \termen{ge\"indexeerde adressering}. Ge\"indexeerde adressering is vooral interessant voor gegevensstructuren met eigen indexsystemen. We denken hierbij bijvoorbeeld aan arrays, matrices, stapels, wachtrijen,... In het register staat immers de index van de datastructuur zelf. Bij een array is dit bijvoorbeeld $i$ indien we het $i$-de element willen uitlezen.
\paragraph{}
\begin{figure}[hbt]
\centering
\importtikzsubfigure{address-indexed}{Ge\"indexeerde adressering}
\importtikzsubfigure{address-indexed-register}{Register-ge\"indexeerde adressering}
\caption{Ge\"indexeerde adressering}
\figlab{address-indexed-general}
\end{figure}
De structuur van ge\"indexeerde adressen op \figref{address-indexed-general} lijkt sterk op de structuur van relatieve adressen op \figref{address-relative-general}. Het verschil is echter dat een basisadres een volwaardig adres is en dus makkelijk $32$ bits telt. De offset daarentegen telt doorgaans slechts $3$ tot $8$ bits. Bij ge\"indexeerde adressering zijn de instructies dus groter.
\paragraph{}
Ook bij deze vorm van adressering kunnen we gebruik maken van een register bank. In dat geval wordt het adres van de registerbank bepaald door een veld in de instructie. Uit dit adres wordt dan de offset opgeslagen in registerbank berekend. Deze vorm van adressering noemen we naar analogie \termen{register-ge\"indexeerde adressering}.
\subsubsection{Ge\"indexeerde adressering met autoincrement/autodecrement}
Meestal leest men een bepaald element uit een array, matrix,... uit binnen een lus in het programma. De meeste programma's zullen dan ook de volledige array of een significant deel overlopen. Dit gebeurt doorgaans in een logische volgorde van links naar rechts of omgekeerd. Men kan het mechanisme achter ge\"indexeerde adressering uitbreiden met \termen{autoincrement} en/of \termen{autodecrement}. Wanneer we deze adresseringmode toepassen wordt het impliciete register die de offset bijhoudt na het uitvoeren van de opdracht automatisch opgehoogd of verlaagd. Hierdoor kan men bij een volgende instructie meteen het volgende of vorige element in de array uitlezen zonder eerst zelf manueel het register op te hogen. Dit kan voordelig zijn omdat het uitrekenen van de nieuwe waarde voor het offset-register parallel kan gebeuren met het uitvoeren van de instructie zelf.
\paragraph{}
\importtikzfigure{address-index-increment}{Ge\"indexeerde adressering met autoincrement/autodecrement.}
\figref{address-index-increment} illustreert het principe. Op basis van de een de instructie kiest men welke waarde door de multiplexer stroomt, deze waarde wordt bij de originele offset opgeteld en vervolgens in het register ingeladen. Door dezelfde instructie vervolgens daarna uit te voeren, zullen we een ander geheugenadres uitlezen bij het bepalen van de operand.
\paragraph{}
Soms bestaat een array niet uit bytes maar uit bijvoorbeeld long integers (8 bytes). Bij sommige processoren kan men dan ook de staplengte bepalen door deze in een veld in de instructie te specificeren, of weg te schrijven in een speciaal hiervoor bestemd register.
\section{Processorontwerp}
Nu we de belangrijkste aspecten van een instructieset hebben besproken, zullen we het ontwerp van de processor zelf in detail bespreken. We zullen eerst de algemene ontwerpcyclus waarna we het ontwerp bespreken voor de twee grote families van instructiesets: CISC en RISC.
\subsection{RISC en CISC}
In de vorige sectie werden instructiesets vergeleken volgens het aantal adresvelden per instructie. Het aantal adresvelden vormt dan ook de belangrijkste basis om instructiesets in te delen in \'e\'en van de twee families: de \termen{Complex Instruction Set Computer (CISC)} of \termen{Reduced Instruction Set Computer (RISC)}.
\subsubsection{CISC}
In het geval van CISC beschouwt men een grote instructieset met complexe en trage instructies. De instructieset omvat meestal een groot aantal verschillende operaties waarbij men ook een groot aantal adresseermodi voorziet. Dit leidt meestal tot een complex datapad met veel functionele eenheden registers en complexe verbindingen die de datastroom controleren. Het gevolg is dat dergelijke processoren aan een lage klokfrequentie werken. De programma's zijn echter vrij kort en men hoopt meestal snelheidswinst te boeken door de operaties die worden uitgevoerd bij een operatie in parallel op het datapad uit te voeren. Typische CISC instructiesets zijn \verb+System/360+, \verb+PDP-11+, \verb+VAX+, \verb+Motorola 68k+ en \verb+80x86+.
\subsubsection{RISC}
Een RISC processor omvat een kleine instructieset die uit een klein aantal eenvoudige instructies bestaat. Een instructie omvat hoogstens \'e\'en adresveld en meestal zijn slechts enkele adresmodi beschikbaar. Een eenvoudige instructieset leidt echter tot een eenvoudig datapad waardoor men een hoge kloksnelheid kan aanbieden. De programma's zijn vrij lang maar men hoopt meestal om hierdoor neven-operaties die soms ongewild worden uitgevoerd op een CISC processor te vermijden. Typische RISC instructiesets zijn \verb+DEC Alpha+, \verb+ARM+, \verb+SPARC+, en \verb+MIPS+.
\subsubsection{Evolutie}
Men merkt dat er een evolutie is van CISC processoren naar RISC processoren. Deze evolutie werd vooral gepromoot door onderzoek bij IBM dat aantoonde dat meestal een beperkte subset van de aangeboden instructies effectief door programma's werd gebruikt. Dit effect werd ook versterkt door de komst van compilers die er meestal niet in slagen alle aspecten van een CISC instructieset automatisch uit te buiten. Andere aspecten die voor het gebruik van RISC processoren pleiten zijn de grootte van de chip, de energieconsumptie en de kostprijs.
\subsection{Ontwerpcyclus}
\importtikzfigure{processor-design-cycle}{De processorontwerp-cyclus.}
\figref{processor-design-cycle} beschrijft de vijf fases bij het ontwerpen van een processor:
\begin{enumerate}
 \item Ontwerp van de instructiecyclus: in deze fase stelt men een set van mogelijke instructies op, meestal gaan de instructies gepaard met een informele beschrijving van het effect van de instructie.
 \item Instructieset-stroomschema: in deze fases worden per instructie alle betrokken operaties beschreven. De meeste instructies zullen immers verschillende effecten teweegbrengen. Bij de \verb+80x86+-instructieset zal bijvoorbeeld naast de bewerking ook de programmateller worden opgehoogd en zullen bepaalde registers op basis van het resultaat worden aangepast.
 \item Allocatie van het datapad: op basis van de vereiste bewerkingen kunnen we bepalen welke componenten we in het datapad zullen moeten voorzien. Dit omvat een beschrijving van de vereiste registers en functionele eenheden. Indien dit tot een duur datapad leidt kan men beslissen om de instructieset opnieuw te herbekijken en begint men dus terug met fase 1.
 \item Op basis van de beschikbare datapad componenten kan men een ASM-schema opstellen. Doorgaans wordt een instructie niet in \'e\'en klokcyclus uitgevoerd. De verschillenden vereiste operaties worden dus uitgevoerd in verschillende toestanden in het ASM-schema. Dit schema beschrijft dan ook de verschillende registertransfers per klokcyclus.
 \item Ontwerp van de controller en het datapad: op basis van het ASM-schema kunnen we vervolgens een datapad en controller synthetiseren. De methodologie van deze synthese staat beschreven in \chpref{nonprogramming}.
\end{enumerate}
\subsection{Complex Instruction Set Computer (CISC)}
In deze subsectie zullen we het proces doorlopen bij het ontwerpen van een CISC processor.
\subsubsection{Ontwerp Instructieset}
In de eerste stap zullen we een CISC instructieset bepalen. We bepalen eerst de specificaties van het geheugen waarmee de processor werkt. Bij wijze van voorbeeld zullen we een $2^{16}\times 16$-bit geheugen beschouwen. De woordlengte is dus $16$ bit en elk adres kan voorgesteld worden met behulp van $16$ bits. We beschouwen ook een registerbank met $8$ registers. Elk registeradres kan dus voorgesteld worden met $3$ bits.
\paragraph{}
Uit de specificaties van het geheugen kunnen we afleiden dat een adresveld bij voorkeur ook $16$ bit groot is. De rest van de instructie zullen we ook voorstellen met behulp van $16$ bit. We beschouwen vier verschillende types van instructies:
\begin{enumerate}
 \item \termen{Registerinstructies}: instructies die bewerkingen uitvoeren met $1$ of $2$ operanden en $1$ doelregister.
 \item \termen{Verplaatsinstructies}: instructies die de gegevens van een register of geheugenadres kopi\"eren naar een register of geheugenadres.
 \item \termen{Spronginstructies}: instructies die de programmateller aanpassen. Deze instructies bepalen bijgevolg het verloop van het programma.
 \item Overige instructies: dit zijn instructies zoals \termen{No Operation (NOP)}\footnote{De ``No Operation'' is een instructie die geen effect heeft. Deze instructie wordt gebruikt bij pipelining om ruimte op te vullen in de pipeline zodat instructies kunnen wachten tot andere instructies zijn uitgevoerd.}.
\end{enumerate}
We voorzien in een instructie dan ook twee bits om het type voor te stellen. De instructies vereisen hoogstens drie registers. In elke registers zullen we daarom drie velden van telkens $3$ bits voorzien om het doelregister -- \mbox{Bestemming} -- en de twee operandregisters \mbox{Operand 1} en \mbox{Operand 2} voor te stellen. Vermits een instructie $16$ bits telt en we in totaal reeds $11$ bits hebben toegewezen, blijven er $5$ bits over voor de opcode. We hebben bijgevolg een instructieformaat gedefinieerd zoals weergegeven op \figref{cisc-bitstructure-general}.
\importtikzfigure{cisc-bitstructure-general}{De bitstructuur van de CISC instructieset.}
\paragraph{}
Het type en de opcode bepalen samen welke instructie zal worden uitgevoerd. We dienen echter nog bewerkingen toe te kennen aan de verschillende waardes. Hiervoor voorzien we \tblrefs{cisc-registerinstructions,cisc-moveinstructions,cisc-jumpinstructions,cisc-otherinstructions} die de instructies per type beschrijven.
\paragraph{Registerinstructies}
\importtabulartable{cisc-registerinstructions}{De registerinstructies van de CISC-processor (type $00$).}
De registerinstructies worden allemaal uitgevoerd op registeradressen en worden opgedeeld in drie categorie\"en: aritmetische, logische en schuifoperaties. De categorie\"en worden bepaald door de eerste twee bits: wanneer $o_4$ laag is beschouwen we een schuifoperatie, in het andere geval beschouwen we ofwel een aritmetische of logische operatie. Bij een schuifoperatie bepaalt $o_3$ of we naar links of naar rechts schuiven. De overige drie bits bepalen vervolgens het aantal bits waarover we schuiven. In het andere geval bepaalt $o_3$ of we aritmetische (laag) of logische operatie (hoog beschouwen). We delen de aritmetische operaties verder op in optelling/aftrekking-bewerkingen en vermenigvuldiging/deling operaties. $o_2$ deelt de operaties verder in zodat bij bewerkingen met twee operanden $o_1$ laag is en we in het andere geval bewerkingen met \'e\'en operand beschouwen. We beschouwen volgende operaties: optelling, aftrekking, increment, decrement, vermenigvuldiging, deling, \termen{vierkantswortel} en negatie. In het geval van logische bewerkingen bepalen de overige drie bits welke operatie we uitvoeren: AND, NAND, OR, NOR, XOR, XNOR, \termen{mask} en invert. De verschillende instructies worden voorgesteld in \tblref{cisc-registerinstructions} samen met een formele beschrijving van de bewerking.
\paragraph{Verplaatsinstructies}
\importtabulartable{cisc-moveinstructions}{De verplaatsinstructies van de CISC-processor (type $01$).}
Verplaatsinstructies zijn instructies die gegevens tussen het geheugen en de registerbank kopi\"eren. Er zijn in dit geval slechts twee basisinstructies mogelijk: kopi\"eren naar het geheugen (\termen{store}) of kopi\"eren naar de registerbank (\termen{load}). We bepalen dit met behulp van de eerste bit $o_0$. Met de overige vier bits kunnen we vervolgens de adresseermode bepalen. De eerste bit van de adresmodus bepaalt of de instructie een adresveld voorziet. Dit is nuttig omdat de hardware snel moet kunnen beslissen of dit adresveld ook moet worden uitgelezen. Indien $m_3$ dus laag is maken we gebruik van een adresveld, in het andere geval kan men het adres afleiden uit de gegevens in de registervelden. De overige bits bepalen vervolgens de soort adressering: onmiddellijk ($000$), direct ($001$), indirect ($010$), relatief ($011$), ge\"indexeerd ($100$) en registerkopie ($101$). Afhankelijk van het feit of we een adresveld beschouwen is de adressering dan bijvoorbeeld relatief of register-relatief. Verder zijn sommige combinaties niet mogelijk: we kunnen bijvoorbeeld onmogelijk de waarde van een register naar een constante kopi\"eren wanneer we onmiddellijke adressering zouden gebruiken. Wanneer we de verschillende mogelijkheden uitproberen bekomen we de instructieset in \tblref{cisc-moveinstructions}.
\paragraph{Spronginstructies}
\importtabulartable{cisc-jumpinstructions}{De spronginstructies van de CISC-processor (type $10$).}
Doorgaans beschouwt men vier verschillende soorten sprongbevelen:
\begin{itemize}
 \item \termen{Ongeconditioneerde sprong (JMP)}: in dit geval wordt de programmateller gezet op het opgegeven adres.
 \item \termen{Geconditioneerde sprong (CJMP)}: dit is een spronginstructie die enkel wordt uitgevoerd als aan een bepaalde voorwaarde wordt voldaan. Hiervoor houdt de processor een \termen{statusregister} bij. Een bit in het statusregister noemt men meestal een \termen{statusvlag}\footnote{Met het concept vlag introduceert men meestal een terminologie: de vlag is gehesen wanneer de bit op $1$ staat.}. Sommige instructies die we verder zullen beschouwen kunnen op basis van een bepaalde voorwaarde een $0$ of een $1$ naar dat register schrijven. Enkel wanneer het statusregister op $1$ staat wordt de programmateller op het opgegeven adres gezet. In het andere geval wordt de programmateller opgehoogd.
 \item \termen{Subroutine sprong (JSR)}: Wanneer het programma een subroutine wil uitvoeren, wordt het terugkeeradres eerst op een stapel gezet. Dit terugkeeradres is de opgehoogde programmateller. Nadat de stapel is verhoogd wordt de programmateller vervolgens aangepast zodat de processor aan het begin van de subroutine staat.
 \item \termen{Subroutine terugkeer (RTS)}: Eenmaal we de subroutine hebben uitgevoerd is het de bedoeling dat het processor terugkeert naar het stuk code net achter de plaats waar de subroutine werd opgeroepen. Omdat verschillende stukken code een subroutine kunnen oproepen werd daarom dit adres op de stapel geplaatst. Door het adres uit te lezen van de stapel en de stapel met \'e\'en element te reduceren bereikt de processor de correcte toestand.
\end{itemize}
Vermits we slechts vier instructies beschouwen kunnen we de instructies voorstellen met behulp van twee bits. De overige bits zijn dan ook niet relevant. \tblref{cisc-jumpinstructions} beschrijft het formaat van verschillende spronginstructies samen met een formele beschrijving.
\paragraph{Overige instructies}
\importtabulartable{cisc-otherinstructions}{De overige instructies van de CISC-processor (type $11$).}
De voorwaardelijke sprong leest het statusregister uit. Tot nu toe hebben we echter nog geen instructies ge\"introduceerd die de waarde van het statusregister aanpast. Hiervoor voorzien we zes vergelijking-operaties: \termen{groter dan (GT)}, \termen{groter dan of gelijk aan (GE)}, \termen{kleiner dan (LT)}, \termen{kleiner dan of gelijk aan (LE)}, \termen{gelijk aan (EQ)} en \termen{niet gelijk aan (NE)}. Indien aan de voorwaarde wordt voldaan wordt het statusregister op $1$ gezet, in het andere geval op $0$. Soms wenst men ook complexere condities te berekenen. Daarom voorzien we ook twee overige instructies die het statusregister aanpassen: \termen{zet status (SetS)} en \termen{reset status (ClrS)}. Tot slot introduceren we ook twee andere operaties: de no operation (NOP) en \termen{reset register (ClrR)}. No-operation kan worden gebruikt als we werken met pipelining om de pipeline op te vullen met instructies waardoor de instructies na de no operation-instructie wachten tot de instructie ervoor is afgelopen. Het op $0$ zetten van een register is soms nuttig alvorens we bijvoorbeeld de som van een array willen berekenen en vormt een alternatief voor de het inladen van een constante vermits de laatste instructie twee woorden vereist. Het instructieformaat van de overige instructies staat in \tblref{cisc-otherinstructions}.
\paragraph{Machinetaal}
Het binaire patroon van de instructies wordt ook wel de \termen{machinetaal} genoemd. Het is immers een formele beschrijving hoe communicatie met de processor verloopt. Een programmeur kan een programma schrijven in machinetaal door elementen van deze taal in het geheugen in te laden. Zo zal bijvoorbeeld de de instructie $0010000111110000_2$ een optelling realiseren die de som van registers $R6$ en $R0$ berekend en opslaat in register $R7$. We kunnen dit signaal als volgt decoderen:
\begin{equation}
\underbrace{00}_{\mbox{registerbewerking}}\overbrace{10}^{\mbox{aritmetisch}}\underbrace{000}_{\mbox{optelling}}\overbrace{111}^{\mbox{D}=R7}\underbrace{110}_{\mbox{s1}=R6}\overbrace{000}^{\mbox{s2}=R0}
\end{equation}
Deze \termen{interpretatie} van de machinecode wordt op hardware niveau uitgevoerd met behulp van logische poorten in de controller. Om schrijffouten te vermijden zullen programmeurs doorgaans de broncode in hexadecimale cijfers schrijven. Dit maakt de instructies compacter en vermijdt schrijffouten door bijvoorbeeld een $1$ of $0$ te vergeten. De instructie kunnen we dus ook voorstellen als $21\mbox{F}0_{16}$.
\paragraph{Assembleertaal}
Hoewel de hexadecimale voorstelling de meeste schrijffouten voorkomt, vormt dit geen gemakkelijke basis om programma's te schrijven. De hexadecimale code is immers niet makkelijk leesbaar omdat de bits niet altijd in groepen van vier zijn gestructureerd en we bijgevolg niet snel kunnen bepalen of een bepaald binair getal overeenkomt met een optelling. Daarom maakt men meestal gebruik van een \termen{assembleertaal}. Een assembleertaal is een mnemonische voorstelling van de instructie en vormt een \'e\'en-op-\'e\'en relatie met de binaire instructies. \tblrefs{cisc-registerinstructions,cisc-moveinstructions,cisc-jumpinstructions,cisc-otherinstructions} bevatten daarom ook een kolom die het equivalent in een zelf ontwikkelde assembleertaal voorstelt.
\paragraph{Assembleerprogramma}
Een programma geschreven in de assembleertaal naar een equivalent programma in machinetaal gebeurt door een \termen{assembleerprogramma}. Dergelijke programma's werken met een eenvoudige opzoektabel die per mnemonische instructie het binaire equivalent bevat. De meeste assembleerprogramma's bieden bovendien extra gebruiksgemak aan: symbolische adressen en labels.
\paragraph{}
Wanneer we gegevens uit het geheugen lezen of wegschrijven dienen we een adres op te geven. De programmeur moet daarom bijhouden op welke plaats welke data precies staat. De meeste assembleerprogramma's nemen deze taak echter gedeeltelijk over met \termen{symbolische adressen}. Een programmeur gebruikt hierbij een symbolische naam en het programma kent zelf een adres toe. Dit leidt niet alleen tot gebruiksgemak maar voorkomt ook dat programmeurs per ongeluk een fout adres opgeven.
\paragraph{}
Ook wanneer we sprongbevelen uitvoeren dienen we een adres op te geven. Het probleem met expliciete adressen is dat wanneer we een instructie invoegen alle adressen erna opschuiven. Om dit probleem te verhelpen werken assembleerprogramma's met een \termen{label}. Een label is een symbolische markering bij een bepaalde instructie. Wanneer men vervolgens een sprong-instructie invoegt zet de programmeur de symbolische markering op de plaats waar het adres moet staan. Het assembleerprogramma houdt vervolgens zelf bij met welk adres dit label overeenkomt en vervangt de symbolische markering bij sprongbevel door een expliciet adres.
\paragraph{Voorbeeld}
Bij wijze van voorbeeld zullen we een programma beschrijven in zowel een hoog-niveau programmeertaal, de assembleertaal en de machinetaal. Het programma berekent tegelijk het minimum, maximum en de som van 1024 getallen in een array \verb+A+. Wanneer we dit in bijvoorbeeld \verb+Pascal+ implementeren schrijven we volgende code:
\importlisting[Pascal]{pascal-minmaxsum}{Hoog niveau beschrijving van het min-max-sum algoritme.}
In het codefragment in \lstref{pascal-minmaxsum} maken we de assumptie dat een ander deel van het programma waardes in \verb+A+ zal laden. We houden verder drie integers bij: \verb+min+, \verb+max+ en \verb+sum+. Vervolgens overlopen we in een for-lus alle items. Ieder item wordt bij \verb+sum+ opgeteld en ondergaat vervolgens twee testen: indien het getal groter is dan \verb+max+ passen we \verb+max+ aan. Analoog berekenen we \verb+min+. Bijgevolg bevatten op het einde van de lus \verb+min+ en \verb+max+ respectievelijk het minimum en maximum.
\paragraph{}
Het omzetten van dit codefragment naar assembleertaal valt buiten het bereik van deze cursus maar wordt besproken in cursussen over compilerconstructie. We presenteren echter een equivalent programma in assembleertaal in \lstref{asm-minmaxsum}.
\importlisting[ciscasm]{asm-minmaxsum}{Het min-max-sum algoritme in assembleertaal.}
Het codefragment bestaat grofweg uit twee delen: data en programma. \verb+ORG DATA+ en \verb+ORG PROGRAM+ zijn dan ook directieven voor het assembleerprogramma. Het zijn geen instructies die omgezet worden in machinecode maar aanwijzingen om de assembleercode correct om te zetten in machinecode. Onder \verb+ORG DATA+ plaatst men een lijst met variabelen samen met hun datatype. Het is de bedoeling dat het assembleerprogramma ruimte voorziet voor de variabelen en de symbolische adressen in de code vervangt door hun fysieke waarde. Het type van de variabele bepaalt hoeveel adressen er gereserveerd moeten worden in het geheugen. In de micro-assembleertaal beschouwen we twee types: \verb+Word+ en \verb+Array+. Een \verb+Word+ omvat \'e\'en woord en is dus een hoeveelheid data die kan worden binnengehaald met \'e\'en adres. Het assembleerprogramma zal bijgevolg adres $0_{16}$ toekennen aan \verb+min+, $1_{16}$ aan \verb+max+ en $2_{16}$ aan \verb+sum+. In het geval van de processor die we hier ontwikkelen is dat $16$ bit. Een \verb+Array+ duidt op een aantal woorden met het aantal gespecificeerd tussen vierkante haken. Dit betekent dus dat de variabele \verb+A+ zal worden opgeslagen in adressen $3_{16}$ tot $402_{16}$.
\paragraph{}
Na \verb+ORG PROGRAM+ begint het eigenlijke program. De instructies staan dus vanaf adres $403_{16}$ vermits de vorige adressen worden ingenomen door de variabelen. Indien we een nieuwe variabele toevoegen, zal het assembleerprogramma alle adressen automatisch aanpassen. Het programma begint met het zetten van de drie registers waarin we \verb+min+ ($R0$), \verb+max+ ($R1$) en \verb+sum+ ($R2$). Zo laden we de \verb+maxint+ constante in in $R0$ en $0$ in de overige registers. Daarna begint de initialisatie van de for-lus. We gebruiken hiervoor register $R3$ als lus-variabele (\verb+i+). $R4$ gebruiken we als register die de constante bijhoudt om de for-lus te be\"eindigen. Tot slot gebruiken we register $R5$ die het adres bijhoudt van het element die we op dat moment moeten uitlezen. Vanaf het \verb+body+-label beginnen de instructies in de for-lus. Zo laden we eerst de data op de specifieke plaats in de array in register $R6$. De data wordt bij het som-register opgeteld waarna we een eerste test uitvoeren. Indien de waarde kleiner of gelijk is dan het op dat moment geldende maximum, slaan we het aanpassen van de \verb+max+ variabele over. In het andere geval kopi\"eren we het getal naar het \verb+max+-register. In de tweede test vergelijken we de waarde met het minimum. Indien de waarde groter is dan het minimum slaan we het aanpassen van het \verb+min+-register over. Na de twee testen bereiken we het \verb+last+-label: de laatste instructies binnen de for-lus. In dit gedeelte hogen we de lus-teller op samen met adres binnen de array. Vervolgens vergelijken we de for-teller en de eindconditie. Wanneer de teller kleiner is of gelijk aan het einde van de array voeren we de lus opnieuw uit: we springen dus terug naar het \verb+body+-label. In het andere geval is de for-lus afgelopen. Tot slot dienen we nog de waardes binnen het register te kopi\"eren naar de variabelen. Dit doen we met behulp van drie store-instructies. Merk op dat het voorgestelde algoritme niet volledig equivalent is: we slaan immers eerst de waarde van \verb+A[i]+ in een register op in plaats van deze telkens uit te lezen. Het voordeel van deze werkwijze is echter dat we load-instructies en toegang tot het geheugen uitsparen en bijgevolg de snelheid van het programma opdrijven.
\paragraph{}
\importlisting[bin]{bin-minmaxsum}{Het min-max-sum algoritme in machinetaal.}
We kunnen de code in assembleertaal handmatig omzetten in binaire machinetaal. Het resultaat na deze omzetting staat in \lstref{bin-minmaxsum}. In de listing beschouwen we twee formaten: \verb+-- ----- --- --- ---+ beschrijft instructies en \verb+---- ---- ---- ----+ beschrijft adresvelden. In werkelijkheid is er natuurlijk geen verschil bij de instructies. Verder beschouwen we in het bestand don't care bits. Men kan deze bits uiteraard niet voorstellen in een geheugen. In dat geval dienen we gewoon een waarde -- $0$ of $1$ -- in te vullen. Doorgaans kiest een assembleerprogramma $0$. We zien dat een assembleerprogramma zowel de labels als de variabelen vervangt door hun fysische equivalent. Verder worden de verschillende instructies omgezet en de opgegeven registers ingevuld. Samen met deze cursus komt het programma \verb+assembler-cisc.hs+ die in staat is om programma's geschreven in de gegeven assembleertaal om te zetten in de gegeven machinetaal. Men schrijft het programma in de assembleertaal en slaat het op in een bestand. Vervolgens roept men het programma aan met \verb+assembleer-cisc bestand program+. De uitvoer is een binair bestand genaamd \verb+program+ die de equivalente machinecode voorstelt. In \chpref{experiments} ten slotte vindt men terug hoe men de CISC-processor zelf kan realiseren. Op die manier kan men dus zelfgeschreven programma's uitvoeren.
\paragraph{}
\begin{table}[hbt]
\centering
\importtabularsubtable{minmaxsum-program}{Programma.}
\importtabularsubtable{minmaxsum-execution}{Uitvoer.}
\caption{Frequentietabel bij het uitvoeren van het voorbeeldprogramma.}
\tbllab{freq-minmaxsum}
\end{table}
We zullen tot slot op basis van het programma cijfermateriaal berekenen over het aantal uitgevoerde instructies. Dit is nuttig wanneer we bijvoorbeeld een sneller datapad willen ontwikkelen en statistische gegevens nodig hebben in verband met het voorkomen van instructies. Bij het verzamelen van statistische gegevens kunnen we een onderscheid maken tussen enerzijds het programma en de uitvoer. Bij de uitvoer van een programma kunnen sommige instructies immers meermaals worden uitgevoerd. \tblref{minmaxsum-program} toont dan ook het gebruik van instructies in een programma. Wat opvalt is het hoge aantal verplaatsinstructies. Wanneer we de frequentie van de instructies willen berekenen zullen de meeste instructies $1024$ keer in de lus worden uitgevoerd: we denken bijvoorbeeld aan de increment instructies. Een probleem vormen echter de register-kopieer instructies (COPY). De uitvoer van deze instructies hangt immers af van de gegevens in de array. We kunnen echter stellen dat wanneer we reeds $k$ elementen hebben onderzocht, de kans om nog een minimum of maximum te bekomen gelijk is aan:
\begin{equation}
\fun{P_{\mbox{nieuw min/max}}}{k}=\displaystyle\frac{1}{1+k}
\end{equation}
In dat geval verwachten we dat we $15.018$ keer deze instructie zullen moeten uitvoeren. \tblref{minmaxsum-execution} beschrijft de frequentietabel bij de uitvoer. Wat opvalt is dat het aandeel aan verplaatsinstructies significant lager is. Dit is typisch: buiten de lussen voert men vooral instructies uit die gegevens inladen en wegschrijven. Binnen de lussen voert men meestal rekenwerk uit.
\subsubsection{Instructieset-stroomschema}
Na het opstellen van een instructieset volgt de volgende stap: het opstellen van een \termen{instructieset-stroomschema} ofwel \termen{Instruction Set Flowchart}. Een dergelijk stroomschema is een visuele voorstelling van alle instructies samen met de bijbehorende registertransfers. Meestal wordt dit schema dan ook gebruikt om later de \termen{instructiedecoder} te ontwikkelen: een onderdeel van de controller die de instructie omzet in de juiste stuursignalen voor het datapad.
\paragraph{}
Bij het defini\"eren van de instructieset hebben we verondersteld dat we beschikken over:
\begin{itemize}
 \item een geheugen (Mem)
 \item een registerbank (RF)
 \item een programmateller (PC)
 \item een instructieregister (IR)
 \item een statusvlag (Status)
\end{itemize}
\paragraph{}
Het stroomschema kent twee concepten: instructieblokken en beslissingsblokken. Een instructieblok wordt voorgesteld door een rechthoek terwijl een beslissingsblok wordt voorgesteld door een driehoek. Op het bovenste hoekpunt wordt de variabele gezet die betrokken is in de beslissing of een booleaanse expressie. Op de basis van de driehoek worden vervolgens de verschillende resultaten van de variabele of booleaanse expressie geplaatst en vertrekt vanuit elk resultaat een pijl. Men leest een stroomschema dan ook door de relevante pijl te volgen.
\paragraph{}
De beschrijving van een stroomschema lijkt op dat van een ASM-schema maar verschilt op enkele punten. Zo omvat een stroomschema geen concrete toestanden: we doen geen uitspraak over wat precies in welke klokcyclus moet worden afgehandeld, soms zullen bewerkingen in het stroomschema dus verder onderverdeeld worden. Verder kent een stroomschema wel een sequenti\"ele volgorde toe aan instructies. Instructies die vermeld staan in eenzelfde blok worden conceptueel na elkaar uitgevoerd\footnote{Dit hoeft echter niet te betekenen dat de opdrachten ook effectief na elkaar moeten worden uitgevoerd. Indien beide instructies niet interfereren bijvoorbeeld kan men de instructies omdraaien of tegelijk uitvoeren.}. In een ASM-schema worden alle instructies in een blok tegelijk uitgevoerd. Stel bijvoorbeeld dat men in een blok volgende instructies tegenkomt:
\begin{equation}
\begin{array}{l}
a\leftarrow a+1\\
b\leftarrow\mbox{Mem}[a]
\end{array}
\end{equation}
In het geval deze instructies in een ASM-schema staat zal $b$ de waarde van het geheugen inlezen dat op het adres van de oude waarde van $a$ staat. In het geval van een stroomschema wordt de waarde op het adres ernaast uitgelezen.
\paragraph{}
Een stroomschema voor een instructieset omvat alle mogelijke instructies die kunnen worden uitgevoerd. Centraal staat doorgaans een instructieblok die beschrijft hoe de instructie wordt opgehaald en de programmateller wordt opgehoogd. Vanuit dit blok worden de stroom vervolgens opgedeeld op basis van mogelijke instructies. Elke vertakking voert vervolgens de instructie uit en keert terug naar het centrale blok zodat de volgende instructie kan worden opgehaald.
\paragraph{}
\importtikzfigure{flowchart-cisc}{Het instructieset-stroomschema van de CISC processor.}
\figref{flowchart-cisc} toont het stroomschema van alle 44 instructies in \'e\'en afbeelding. De verschillende types instructies zijn in verschillende richtingen opgebouwd. Centraal zien we een instructieblok met twee instructies: $\texttt{IR}\leftarrow\texttt{Mem[PC]}$ laadt de instructie op het adres van de programmateller in het instructieregister en $\texttt{PC}\leftarrow\texttt{PC}$ hoogt de instructieteller op. Na het inladen van de instructie vertakt de stroom zich aan de hand van de data in het instructieregister. Allereerst maken we op basis van $\mbox{IR}_{15}$\footnote{Met deze notatie bedoelen we de $15$-de bit van het instructieregister.} en $\mbox{IR}_{14}$ een onderscheid tussen de verschillende types. Eenmaal voldoende vertakt beschrijft een instructieblok de concrete uitvoer van de bewerking. Het stroomdiagram toont ook details die niet in de tabellen met de instructies vermeld staan. We denken hierbij bijvoorbeeld aan het inlezen van het adresveld. In het geval de instructie gepaard gaat met een adresveld wordt wordt dit ook uitgelezen met $\texttt{Mem[PC]}$\footnote{Merk op dat alvorens we deze instructie bereiken, de programmateller al is opgehoogd. Bijgevolg staat de programmateller op dat moment op het adresveld.} waarna de programmateller nogmaals wordt opgehoogd. Na het uitvoeren van de instructies komen de verschillende stromen terug samen bij het instructieblok die de volgende instructie inleest.
\subsubsection{Allocatie datapadcomponenten}
Nadat per instructie een formele beschrijving werd opgesteld met betrekking tot de verschillende operaties, bestudeert men in een volgende fases het datapad. Op basis van de uitgevoerde bewerkingen, kunnen we immers de vereiste hardware vastleggen. Zo vereist het stroomdiagram vijf soorten geheugen:
\begin{itemize}
 \item 	Een extern geheugen \texttt{Mem}: $2^{16}\times 16$ ofwel $128\mbox{ kiB}$ met $1$ lees/schrijfpoort.
 \item Een registerbank \texttt{RF}: $2^3\times 16$ met $2$ leespoorten en $1$ schrijfpoort.
 \item Een statusvlag \texttt{Status}
 \item Een instructieregister \texttt{IR}
 \item Een programmateller \texttt{PC}
\end{itemize}
De laatste twee registers behoren echter tot de controller en zullen we in deze stap niet beschouwen. We kunnen verder voor elke operatie een functionele eenheid voorzien. Anderzijds zal elke klokcyclus er hoogstens \'e\'en instructie actief zijn. Alle bewerkingen die moeten worden uitgevoerd kunnen we uitvoeren met drie algemene functionele eenheden:
\begin{itemize}
 \item Aritmetische Logische Eenheid (ALU)
 \item Vergelijker
 \item $16$-bit bidirectionele schuifoperator.
\end{itemize}
We maken de assumptie dat de ALU alle basisbewerkingen (ADD, SUB, MUL, DIV, ROOT, NEG, AND, NAND, OR, NOR, XOR, XNOR, MASK en INV) kan uitvoeren, dat de vergelijker iedere conditie (GT, GE, LT, LE, EQ en NE) met betrekking tot twee registers kan uitrekenen en dat het schuifregister de twee schuifinstructies (ASR, ASL) kan uitvoeren. Instructies die niet onder het type registerinstructies vallen voeren doorgaans geen bewerkingen uit behalve de optelling. In dat geval is de ALU echter vrij waardoor dit geen probleem vormt.
\paragraph{}
In deze fase is het echter de bedoeling te onderzoeken of we de snelheid niet kunnen opdrijven door extra hardware te voorzien. Wanneer men echter aanpassingen aanbrengen aan het datapad zal men ook de instructieset moeten herzien.
\paragraph{}
Wanneer we de snelheid willen opdrijven zullen we dit altijd doen ter hoogte van het kritisch pad. Het heeft immers weinig zin andere instructies sneller uit te voeren wanneer de kloksnelheid onder druk staat van een andere instructie. We hebben reeds vermeld dat geheugen doorgaans een bottleneck vormt dus is het meestal interessant vooral naar instructies te kijken met een groot aantal geheugentoegangen.
\paragraph{}
Op basis van het stroomdiagram kunnen we opmerken dat impliciete adressering het meeste geheugenoproepen kent. De instructie vereist immers volgende bewerking:
\begin{equation}
\texttt{RF[d]}\leftarrow\texttt{Mem[Mem[Mem[PC]]]}
\end{equation}
Dit komt bijgevolg neer op $4$ geheugentoegangen (inclusief het inlezen van de instructie). Stel dat het RAM-geheugen een toegangstijd heeft van $50~\mbox{ns}$, dan vereist de instructie dus minstens $200~\mbox{ns}$. Indien we dus de volledige instructie in \'e\'en klokcyclus uitvoeren, dan is de maximale klokfrequentie bijgevolg:
\begin{equation}
f_{\mbox{max}}=\displaystyle\frac{1}{4\cdot 50~\mbox{ns}}=5~\mbox{MHz}
\end{equation}
We kunnen de impliciete adressering onmogelijk versnellen met de gegeven toegangstijd. Anderzijds impliceert het uitvoeren van de instructie in \'e\'en klokcyclus dat alle instructies in $200~\mbox{ns}$ worden uitgevoerd, ook registeroperaties die geen geheugenoperaties vereisen. Dit betekent dat het uitvoeren van het voorbeeldprogramma $2\ 052\ 800~\mbox{ns}$ vereist.
\paragraph{}
We kunnen een multicycling transformatie uitvoeren waarin instructies worden uitgesmeerd zodat per klokcyclus er \'e\'en geheugentoegang wordt uitgevoerd. In dat geval is een klokcyclus ongeveer $50~\mbox{ns}$ lang. Wanneer we dus een LOAD instructie uitvoeren met behulp van impliciete adressering zal dit nog steeds $200~\mbox{ns}$ vereisen. Anderzijds zullen bijvoorbeeld registeroperaties in $1$ klokcyclus worden uitgevoerd. Uit \tblref{minmaxsum-execution} kunnen we echter afleiden dan het grootste deel van de instructies enkel toegang tot het geheugen vereisen om de instructie in te laden. De uitvoertijd van het programma zal dus ongeveer verviervoudigen.
\paragraph{}
Met de multicycling transformatie dienen we echter een nieuw register te introduceren: het \termen{adresregister (AR)}. Dit register houdt het adres bij tijdens de uitvoeren van verplaatsinstructies. We hebben slechts nood aan \'e\'en register omdat bij bijvoorbeeld impliciete adressering we dit register enkele malen kunnen overschrijven tot de instructie is uitgevoerd.
\subsubsection{ASM-schema}
Na het alloceren van componenten bij het datapad zullen we een ASM-schema opstellen. We stellen dit proces op op basis van enerzijds het stroomdiagram op \figref{flowchart-cisc} en anderzijds de allocatie van de componenten op het datapad. Met deze stap dienen we echter te beslissen welke bewerkingen we in welke toestand uitvoeren. Concreet betekent dit dat we de instructieblokken van het stroomdiagram opsplitsen in zo weinig mogelijk ASM-blokken. In principe kan men alle instructies altijd tegelijk uitvoeren. Anderzijds hebben we de componenten op het datapad reeds vastgelegd. Concreet betekent dit dat we in dit geval over slechts \'e\'en ALU beschikken, dat we slechts \'e\'en adres van het geheugen tegelijk kunnen uitlezen, we \'e\'en woord per toestand naar de registerbank kunnen wegschrijven, enzovoort. Bij wijze van voorbeeld zullen we enkele gevallen bespreken.
\paragraph{Aritmetisch schuiven naar rechts (ASR)}
\begin{figure}[hbt]
\centering
\importtikzsubfigure{flowchart-asr}{Stroomschema.}
\importtikzsubfigure{asm-asr}{Tussentoestand.}
\caption{Stroomschema naar ASM-schema voor de ASR-instructie.}
\label{flowasm-asr}
\end{figure}
Om de voorbeelden te bespreken zullen we telkens eerst de relevante delen van het stroomschema voorstellen. In het geval van aritmetisch schuiven naar rechts (ASR) is dit \figref{flowchart-asr}. De cyclus omvat slechts twee instructieblokken: het ophalen van een instructie en het uitvoeren. Dit is het geval bij alle registerinstructies. Hardwarematig is er geen probleem om de twee bewerkingen in \'e\'en klokcyclus uit te lezen: we kunnen de instructie uit het geheugen inlezen en deze tegelijk uitvoeren en in het instructieregister plaatsen. Er zijn echter drie redenen om dit niet te doen: in dat geval heeft het plaatsen van de instructie in het register geen nut: de volgende instructie zal de waarde immers overschrijven zonder dat deze ooit werd uitgelezen. Verder beschouwen we ook een Moore-machine als controller. Dat betekent dat de stuursignalen afhankelijk zijn van de data die in een vorige klokcyclus werd uitgevoerd. Tot slot is het ophalen van de instructie een bewerking die met alle overige instructies wordt gedeeld. De logica wordt meestal eenvoudiger als men bij het ophalen van de instructie niet voor sommige instructies meteen de relevante bewerkingen uitvoert.
\paragraph{}
We stellen dus twee verschillende toestanden: een toestand waarin we de instructie ophalen en een toestand waarin we de schuifoperatie naar rechts uitvoeren. In het ASM-schema moeten we dus een leeg toestandskader introduceren na het toestandskader die de instructie ophaalt. Onder dit toestandskader plaatsen we vervolgens beslissingskaders die naar een conditioneel kader leiden waar de instructie wordt uitgevoerd. Een ASM-schema dat enkel relevant is voor de ASR-instructie is getekend in \figref{asm-asr}. Merk op dat in een ASM-schema zaken die in hetzelfde toestandskader gebeuren tegelijk worden uitgevoerd, dit in tegenstelling tot een stroomschema.
\paragraph{Laden van een constante (LOAD)}
\begin{figure}[hbt]
\centering
\importtikzsubfigure{flowchart-load1}{Stroomschema.}
\importtikzsubfigure{asm-load1}{ASM-schema.}
\caption{Stroomschema naar ASM-schema voor de LOAD constante-instructie.}
\label{flowasm-load1}
\end{figure}
We beschouwen vervolgens de LOAD-instructie met onmiddellijke adressering die een constante uit het programma in een register inladen. Een compact deel van het stroomschema staat op \figref{flowchart-load1}. In het geval van deze instructie zijn we verplicht om twee toestanden te voorzien. Dit komt omdat het geheugen slechts \'e\'en leespoort heeft en we bijgevolg geen twee woorden tegelijk uit het geheugen kunnen laden. Verder wordt ook de programmateller tweemaal opgehoogd terwijl de controller slechts \'e\'en increment-operator omvat. We introduceren dus opnieuw twee toestanden: \'e\'en toestand waarbij we de instructie inlezen en \'e\'en toestand waarin we het adresveld inlezen en wegschrijven naar de registerbank. Dit leidt tot het ASM-schema in \figref{asm-load1}.
\paragraph{Laad direct uit het geheugen (LOAD)}
\begin{figure}[hbt]
\centering
\importtikzsubfigure{flowchart-load2}{Stroomschema.}
\importtikzsubfigure{asm-load2}{Tussentoestand.}
\caption{Stroomschema naar ASM-schema voor de LOAD direct-instructie.}
\label{flowasm-load2}
\end{figure}
Naast het inladen van een constante bevat de instructieset ook een instructie om gegevens in te laden waarbij het adres gespecificeerd is. Opnieuw kunnen we omwille van het feit dat het geheugen slechts \'e\'en leespoort telt deze operatie enkel met minstens drie toestanden uitvoeren. Bij een multicycling-transformatie dienen we echter de tussenresultaten op te slaan. Hiervoor werd echter het adresregister ge\"introduceerd. Dit adresregister zal na toestand twee het adres bevatten dat moet worden uitgelezen. In de derde toestand wordt de waarde meteen terug uitgelezen om het correcte adres uit het geheugen uit te lezen.
\paragraph{Laad indirect uit het geheugen (LOAD)}
\begin{figure}[hbt]
\centering
\importtikzsubfigure{flowchart-load3}{Stroomschema.}
\importtikzsubfigure{asm-load3}{Tussentoestand.}
\caption{Stroomschema naar ASM-schema voor de LOAD indirect-instructie.}
\label{flowasm-load3}
\end{figure}
De laatste LOAD-instructie die we beschouwen is de impliciete adressering. Deze instructie vereist in totaal viermaal toegang tot het geheugen. Bijgevolg vereist deze operatie minimum vier toestanden. Ook hier beschouwen we dus een multicycling-transformatie waarbij we \'e\'en instructieblok van het stroomschema op in drie toestandsblokken in het ASM-schema. We zullen bijgevolg de tussenresultaten opnieuw moeten wegschrijven. In een algemeen geval betekent dit dat we twee registers moeten voorzien\footnote{In een algemeen geval kan het immers voorkomen dat het finale resultaat afhangt van alle tussenresultaten berekend in de vorige toestanden.}. In dit geval hebben we slechts \'e\'en extra register nodig. We dienen immers enkel de resultaten van de vorige toestand te onthouden. Dit betekent dat we in de vierde toestand de resultaten van de tweede toestand niet meer nodig hebben.
\paragraph{Sprong naar een subroutine (JSR)}
\begin{figure}[hbt]
\centering
\importtikzsubfigure{flowchart-jsr}{Stroomschema.}
\importtikzsubfigure{asm-jsr}{Tussentoestand.}
\caption{Stroomschema naar ASM-schema voor de JSR-instructie.}
\label{flowasm-jsr}
\end{figure}
Ook in het geval van een sprong naar een subroutine moeten we het uitvoeren van de eigenlijke instructie opsplitsen in twee delen. De volledige set instructies staat op \figref{flowchart-jsr}. Dit komt opnieuw door de toegang tot het geheugen: we moeten zowel het adresveld uitlezen en de programmateller na uitlezen in het geheugen plaatsen. Het geheugen omvat echter \'e\'en lees/schrijfpoort. We kunnen dus niet tegelijk lezen en schrijven. Om de toestand te overbruggen zullen we bovendien opnieuw gebruik moeten maken van de adresregister. In de eerste toestand lezen we dus eenvoudigweg het adresveld uit en hogen we de programmateller op. In de tweede toestand schrijven we de programmateller weg naar het geheugen, hogen we de stapel op en stellen we tot slot de programmateller gelijk aan het adresregister. Men kan opmerken dat er nog een tweede reden is om de instructies in twee blokken op te splitsen: we dienen tweemaal een increment uit te voeren. In het geval van de programmateller valt deze verantwoordelijkheid echter onder de controller. De controller beschikt over een eigen increment-operator waardoor men in principe tegelijk een optelling met de ALU en een increment van de programmateller kan realiseren. Het ASM-schema die de twee blokken voorstelt staat in \figref{asm-jsr}.
\paragraph{Terugkeer uit een subroutine (RTS)}
\begin{figure}[hbt]
\centering
\importtikzsubfigure{flowchart-rts}{Stroomschema.}
\importtikzsubfigure{asm-rts}{Tussentoestand.}
\caption{Stroomschema naar ASM-schema voor de RTS-instructie.}
\label{flowasm-rts}
\end{figure}
Tot slot beschouwen we ook nog de terugkeer instructie waarvan het stroomschema is afgebeeld op \figref{flowchart-rts}. Deze instructie voeren we uit in \'e\'en toestand. Men kan opmerken dat we eerst de waarde van een register in de registerbank moeten aftellen om vervolgens de relevante geheugenplaats in te lezen. We kunnen echter zonder gebruik te maken van extra hardware een chaining-transformatie uitvoeren en dus niet alleen het resultaat na de decrement-operatie opslaan in de registerbank maar deze ook gebruiken om het adres te bepalen in het geheugen. We kunnen bijgevolg alle instructies in \'e\'en klokcyclus uitvoeren. Dit leidt dan ook tot het ASM-schema op \figref{asm-rts}.
\subsubsection{Ontwerp Controller}
Na het ontwerpen van een (gedeeltelijk) ASM-schema, kunnen we een specifieke controller ontwikkelen voor onze processor. In de eerste plaats beschouwen we altijd de controller uit \figrefpag{subroutineStructureController}. In deze controller beschouwen we echter enkele algemene componenten zoals een toestandsregister, next-state logica en output logica. In deze subsubsectie ondernemen we een poging deze componenten in te vullen.
\paragraph{}
De eerste component die we zullen bespreken is het toestandsregister. Dit register omvat normaal de toestand waarin de controller zich bevindt. In het geval van een processor is de toestand echter de plaats in het programma waarin we ons bevinden. Deze plaats wordt gemarkeerd door de programmateller: deze teller bevat immers het geheugenadres van waaruit we de instructie zullen uitlezen. Het stroomschema en het ASM-schem specificeren verder ook dat we op basis van de waarde van de programmateller de relevante instructie uit het geheugen inlezen en deze vervolgens in het instructieregister plaatsen. We voegen bijgevolg het geheugen en register tussen de programmateller enerzijds en de next-state- en output-logica anderzijds. De implementatie van de controller na deze stap is ge\"illustreerd in \figref{cisc-controller-pcmemir}.
\importtikzfigure{cisc-controller-pcmemir}{Het ontwerp van een CISC-controller met programmateller, geheugen en instructieregister.}
\paragraph{}
Door het toestandsregister te vervangen met de programmateller introduceren we echter een belangrijk probleem: de waarde van het toestandsregister in een controller beschrijft immers volledig de toestand. Dit zou dus betekenen dat we op basis van de programmateller of de overeenkomstige instructie alle bewerkingen kunnen afleiden. Dit is echter niet het geval: bij sommige bewerkingen werd een multicycling transformatie uitgevoerd. Bij bijvoorbeeld een indirecte laad-instructie doorlopen we vier verschillende toestanden. Alleen in de eerste twee toestanden wordt de programmateller opgehoogd. Dit betekent dus dat we geen onderscheid kunnen maken tussen bijvoorbeeld toestand $S_3$ en $S_4$ in \figrefpag{asm-load3}. We kunnen dit probleem echter oplossen met behulp van de componenten die de volgende toestand en de uitvoer-logica berekenen. Klassiek zijn dit twee combinatorische schakelingen. We kunnen echter ook een eindige toestandsautomaat implementeren om deze componenten voor te stellen. Deze automaat omvat dan een intern toestandsregister die de verschillende deeltoestanden van een instructie bijhoudt. In sommige processoren is er zelfs sprake van een volledige eindige toestandsautomaat-controller met een micro-programmateller en micro-programmageheugen.
\importtikzfigure{cisc-controller-nsolfsm}{Het ontwerp van een CISC-controller met interne eindige toestandsautomaat.}
\paragraph{}
De controller laat toe om naast de programmateller op te hogen ook op twee andere manieren een nieuwe waarde in te laden: de logica die de volgende toestand uitrekent kan zelf een waarde voorstellen of de waarde kan worden afgeleid uit een LIFO-stack (ook wel de \termen{execution stack} of \termen{uitvoerstapel} genoemd) in het geval van subroutines. Op basis van het stroomschema kunnen we echter opmerken dat in het geval van een spronginstructie het adres expliciet in het programma wordt vermeld. We hebben bijgevolg geen extra logica nodig om het adres verder te decoderen. Daarom vervangen we de verbinding vanuit de next-state logica naar de multiplexer door een verbinding vanuit het geheugen naar de multiplexer.
\paragraph{}
Wanneer we een subroutine uitvoeren doet een deel van het geheugen zelf dienst als stapel. Dit zien we bijvoorbeeld in het stroomschema waar de instructies $\texttt{Mem[RF[s1]]}\leftarrow\texttt{PC}$ en $\texttt{RF[s1]}\leftarrow\texttt{RF[s1]}+1$ respectievelijk eerst de waarde van de programmateller op de juiste plaats in de stapel plaatst en vervolgens een register in de registerbank die dienst doet als \termen{stapelteller} ophoogt. In het geval we terugkeren uit een subroutine geldt een omgekeerde procedure. De controller die we op dit moment beschouwen voorziet hiervoor op dit moment zelf een LIFO-stapel. We kunnen vervolgens deze stapel elimineren en de verbindingen afleiden naar het geheugen. Dit betekent dat de uitgang van de increment die oorspronkelijk het adres van de LIFO bepaalde nu in bepaalde omstandigheden het geheugenadres bepaalt. Verder werd al een verbinding vanuit het geheugen naar de multiplexer voorzien. We elimineren dus een verbinding en reduceren ook de multiplexer tot een 2-naar-1 MUX. De controller na het doorvoeren van deze twee wijzigingen staat op \figref{cisc-controller-lifo}.
\importtikzfigure{cisc-controller-lifo}{Controller met samengevoegd geheugen en stapelgeheugen.}
\paragraph{}
Tot slot dienen we op te merken dat het geheugen van de ontwikkelde processor geen onderdeel is van de controller zelf: ook het datapad heeft toegang tot hetzelfde geheugen om bijvoorbeeld programmaconstanten uit te lezen. Een architectuur waar men een centraal geheugen beschouwt die zowel gegevens als programma's beschouwt noemen we Von Neumann-architectuur, in het geval beide geheugens gescheiden zijn spreken we over Harvard-architectuur. We verplaatsen dus het geheugen uit de controller. Dit doen we door twee bussen te beschouwen: een \termen{adresbus} is een reeks verbindingen waarmee men de adresingangen van het geheugen kan aansturen en met behulp van de \termen{databus} leest men gegevens in of schrijft men deze weg in het geheugen. De databus is bijgevolg een bus die in twee richtingen werkt. We dienen bijgevolg tri-state buffers te voorzien voor zowel lees- als schrijfoperaties. Ook op de adresbus brengen we tri-state buffers aan: het datapad moet immers ook adressen kunnen aanleggen. De uiteindelijke controller beschrijven we in \figref{cisc-controller-memex}.
\importtikzfigure{cisc-controller-memex}{Controller met extern geheugen (Von Neumann-architectuur).}
\subsubsection{Ontwerp Datapad}
De laatste stap is het ontwerpen van een datapad. We hebben reeds enkele stappen geleden de vereiste componenten voor het datapad vastgelegd. We dienen echter nog verbindingen te introduceren om de verschillende instructies te kunnen uitvoeren. Deze verbindingen omvatten ook tri-state buffers aan de ingangen van een bus en multiplexers aan de uitgangen. De vereiste verbindingen kunnen we uiteraard afleiden uit het ASM-schema.
\paragraph{}
Het is mogelijk dat er conflicten ontstaan tussen enerzijds het ASM-schema en anderzijds de componenten en verbindingen op het datapad. Dit kunnen we oplossen door terug te keren op \'e\'en van de volgende stappen:
\begin{itemize}
 \item De allocatie van de componenten: in dat geval voorzien we extra componenten zoals een extra ALU of registerbank. We dienen vervolgens wel opnieuw een ASM-schema op te stellen en mogelijk ook een nieuwe controller ontwerpen.
 \item Het ASM-schema: we kunnen extra toestanden introduceren zodat we de componenten op het datapad meermaals kunnen gebruiken. Hiervoor dienen we het ASM-schema en mogelijkerwijs de controller aan te passen. Dit leidt uiteraard tot meer klokcycli voor (sommige) instructies.
\end{itemize}
\paragraph{}
Bij het omzetten van het ASM-schema naar de verbindingen van het datapad is het belangrijk een onderscheid te maken tussen bewerkingen die onder de verantwoordelijkheid van de controller vallen en deze van het datapad. Zo beschrijft toestand $S_0$ het ophalen van een instructie en deze vervolgens in het instructieregister plaatsen. Zowel de programmateller en het instructieregister vallen echter onder de verantwoordelijkheid van de controller. Bijgevolg dienen we voor deze bewerkingen geen verbindingen te introduceren. Ook het interpreteren van de instructie is de verantwoordelijkheid van het datapad.
\paragraph{}
\importtikzfigure{cisc-datapad-basic}{De basis datapad-ontwerp van de CISC-processor.}
Bij het ontwerp beginnen we met een datapad waar we enkel de componenten tekenen. Verder worden ook externe componenten (behalve de controller) die interageren met het datapad getekend. In dit concrete geval is dit het centrale geheugen samen met de adres- en databus. We beschrijven ook de tri-state buffers van de databus van en naar het geheugen. Deze aspecten werden ook al beschreven bij het ontwerp van de controller. Dit resulteert in de afbeelding op \figref{cisc-datapad-basic}.
\paragraph{}
We realiseren een controller door de verschillende instructies toe te voegen en waar nodig extra bussen in het datapad aan te brengen. Het is daarom nuttig om onmiddellijk bij het introduceren van een bus met tri-state buffers bij de aansturing en eventueel ontvangers aan de kant van de gebruikers te plaatsen. Indien later blijkt dat de bus enkel door \'e\'en register wordt aangestuurd of de multiplexer slechts data ontvangt van \'e\'en bus, kan men deze componenten gewoon verwijderen. We zullen in de rest van deze subsubsectie enkele instructies vertalen naar het datapad. Natuurlijk dienen we in realiteit alle instructies te beschouwen. Het resulterende datapad staat in \figrefpag{cisc-datapad-connections} en de uiteindelijke controller in \figrefpag{cisc-controller-final}. We zullen vanaf hier de synthese van deze figuren beschouwen.
\paragraph{Schuifoperatie (SHR)}
We beschouwen opnieuw de schuifoperatie naar rechts. Zoals reeds aangehaald is het inlezen en interpreteren van de instructie de verantwoordelijkheid van de controller. Bijgevolg rest alleen nog verbindingen te voorzien voor de laatste bewerking in de instructie: $\texttt{RF[d]}\leftarrow\texttt{RF[s1]>{}>N}$. Het bepalen van de juiste adressen in de registerbank (\texttt{d} en \texttt{s1}) valt hierbij ook onder de verantwoordelijkheid van de controller. Dit doen we door verbindingen te plaatsen tussen de output logica eindige toestandsautomaat en de adres-ingangen van de registerbank en de schuiflengte-ingangen van het schuifregister. We dienen dus enkel bussen te realiseren die gegevens vanuit de registerbank naar het schuifregister transporteert en terug. De eerste bus wordt via een tri-state buffer aangestuurd vanuit de eerste lees-poort van de registerbank\footnote{We kunnen ook opteren voor de tweede leespoort. Vermits momenteel nog geen enkele leespoort in gebruik is, is de keuze arbitrair.}. Verder loopt vanuit het schuifregister een bus terug naar de registerbank.
\paragraph{Optelling (ADD)}
In het geval van een optelling dienen we opnieuw slechts de verbindingen verantwoordelijk voor de optelling zelf te realiseren. Dit betekent dat we de twee leespoorten moeten aanleggen op de twee ingangen van de ALU. We hebben echter al \'e\'en operandbus voorzien: deze bus wordt aangestuurd door het register bepaald door het \verb+s1+ veld van de instructie. We verwachten dat de implementatie van de controller vereenvoudigt wanneer we dit principe consistent toepassen. Bijgevolg zullen we ook bij de optelling deze bus aansturen met de inhoud van \verb+RF[s1]+. We dienen echter nog een andere operand-bus te voorzien: een bus om \verb+RF[s2]+ te verplaatsen. Deze bus wordt dus aangestuurd door de tweede leespoort van de registerbank. We dienen tot slot ook de twee ingangen van de ALU aan de operandbussen te koppelen. Het datapad voorziet ook reeds een resultaatbus. Deze resultaatbus wordt aangestuurd door de schuifoperator en gebruikt door de registerbank. Omdat we het schuifregister niet tegelijk met de ALU gebruiken, kunnen we bijgevolg de resultaatbus hergebruiken. We voorzien dus een tri-state buffer tussen de uitgang van de ALU en de resultaatbus. De overige registerinstructies worden allemaal ofwel door het schuifregister ofwel door de ALU uitgevoerd. Zonder verder in detail te gaan kunnen we dus stellen dat de twee operandbussen en de resultaatbussen dus volstaan om al deze instructies te ondersteunen.
\paragraph{Onmiddellijk laden (LOAD)}
Naast register-instructies beschouwen we ook instructies die data in en uit het geheugen halen. Dit geheugen bevindt zich buiten het datapad, maar het datapad kan wel communiceren met dit geheugen via de adres- en databus. Wanneer we een constante inladen betekent dit dat we volgende bewerking uitvoeren: $\texttt{RF[d]}\leftarrow\texttt{Mem[PC]}$. We beschikken evenmin over de inhoud van de programmateller, maar de controller kan wel de inhoud op de adres-bus plaatsen. In dat geval staat de inhoud die moet worden ingeladen dus op de data-bus. Wanneer we dus een tri-state buffer plaatsen tussen de data-bus en de resultaat-bus, kunnen we de data in de registerbank inladen.
\paragraph{Direct laden (LOAD)}
Het ASM-schema op \figrefpag{asm-load2} beschrijft hoe gegevens direct uit het geheugen worden ingeladen. Het proces komt neer op twee bewerkingen waar het datapad een rol in speelt: $\texttt{AR}\leftarrow\texttt{Mem[PC]}$ en $\texttt{RF[d]}\leftarrow\texttt{Mem[AR]}$. De eerste bewerking heeft veel gemeen met het onmiddellijk inladen van een constante: we laden immers opnieuw $\texttt{Mem[PC]}$ in, alleen is de gebruiker nu het adresregister \verb+AR+. We kunnen opnieuw gebruik maken van de resultaatbus vermits deze bus opnieuw door geen enkel ander component gebruikt wordt in dezelfde toestand. We dienen hierbij wel het adresregister aan te sluiten op de resultaatbus. Hiervoor hoeven we op dit moment uitsluitend een eenvoudige verbinding te gebruiken: het adresregister dient immers niet te kiezen tussen data van verschillende bussen waardoor een multiplexer overbodig is. Vervolgens beschouwen we de tweede bewerking. In deze bewerking dient de inhoud van het adres-register op de adres-bus te verschijnen. Hiervoor voorzien we dus een verbinding tussen de lees-poort van het adresregister enerzijds en de adresbus anderzijds. In dit geval dienen we wel gebruik te maken van een tri-state buffer: er zijn immers nog registers en componenten die hun inhoud op de adres-bus kunnen plaatsen.
\paragraph{Ge\"indexeerd laden (LOAD)}
Als laatste laad commando beschouwen we ge\"indexeerd laden. Bij ge\"indexeerd laden voorzien we opnieuw twee relevante bewerkingen: de eerste bewerking stelt: $\texttt{AR}\leftarrow\texttt{Mem[PC]+RF[s1]}$, en in een volgende toestand $\texttt{RF[d]}\leftarrow\texttt{Mem[AR]}$. De tweede bewerking is identiek aan een bewerking bij het direct laden van een waarde. Bijgevolg dienen we geen extra verbindingen te voorzien. De eerste operatie omvat echter een nieuw aspect: we laden niet enkel de waarde van de programmateller in, maar tellen er ook nog de registerwaarde \texttt{RF[s1]} bij. Bijgevolg dienen we de waarde die op de databus verschijnt eerst aan te leggen op een ALU die vervolgens de som berekent. Het resultaat van de ALU wordt dan aangelegd op de resultaat-bus en wordt op die manier in de registerbank ingeklokt. Om de data aan te leggen kunnen we gebruik maken van een operand-bus: slechts \'e\'en van de twee wordt immers gebruikt om de data van het register \texttt{RF[s1]} te verplaatsen. Omdat in alle voorgaande bewerkingen de eerste operandbus deze waarde transporteerde, zullen we de waarde van de data-bus dan ook aanleggen op de tweede operand-bus. De optelling is immers commutatief dus maakt de concrete toewijzing niks uit. Concreet introduceren we dus \'e\'en nieuw tri-statebuffer tussen de databus en de tweede operand-bus. Vermits alle overige laad-instructies enkel gebruik maken van bewerkingen die reeds werden beschouwd, zullen deze geen nieuwe verbindingen introduceren.
\paragraph{Kopi\"eren van een register (COPY)}
Verplaatsinstructies verplaatsen niet enkel gegevens tussen registers en het geheugen, maar ook tussen de registers onderling. Dit is de zogenaamde COPY-instruction. We kunnen in de eerste plaats denken om in het geval van een copy-instruction een operandbus te verbinden met de resultaat-bus met behulp van een tri-state buffer. Anderzijds kan deze instructie worden ondersteund zonder extra verbindingen te voorzien. Wanneer we immers schuiven over $0$ bits naar links of rechts, zal dezelfde waarde op de resultaat-bus worden aangelegd die op de eerste operand bus stond. De meeste ALU's laten bovendien toe dat met behulp van een bepaalde functiecode de waarde aan \'e\'en van de ingangen eenvoudigweg wordt aangelegd op de resultatenbus. We kunnen kiezen welke methode moet worden ge\"implementeerd in de logica van de controller, maar de operatie introduceert geen nieuwe verbindingen op het datapad.
\paragraph{Direct stockeren (STOR)}
De verplaats-instructies die data uit het geheugen inladen hebben op het inladen van een constante na een tegenhanger die data wegschrijft naar het geheugen. Vermits het berekenen van het adres gelijkaardig is, zullen we hiervoor geen nieuwe instructies moeten voorzien. De bewerking die het uiteindelijke resultaat wegschrijft -- $\texttt{Mem[AR]}\leftarrow\texttt{RF[s1]}$ dienen we echter wel te beschouwen. In deze bewerking leggen we de inhoud van het adresregister aan op de adresbus. De verbindingen hiervoor zijn reeds voorzien omdat we dit ook moeten doen om bijvoorbeeld gegevens direct in te laden. Anderzijds dienen we de inhoud van het gespecificeerde register aan te leggen op de registerbank. Tot nu werd altijd de eerste leespoort gebruikt om \verb+RF[s1]+ uit te lezen. Om de implementatie van de controller eenvoudig te houden zullen we deze conventie verder toepassen. We dienen echter de leespoort te verbinden met de data-bus. We voorzien dus een tri-state buffer tussen de eerste leespoort en de databus. Met deze STOR-instructie beschouwen we alle overige verplaats-instructies ook als ge\"implementeerd.
\paragraph{Sprong naar subroutine (JSR)}
We zullen ook enkele sprong-instructies beschouwen. De eerste instructie is de sprong naar een subroutine. We beschouwen deze instructie omdat we hierdoor ook tegelijk de bewerkingen van de onvoorwaardelijke (JMP) en voorwaardelijke (CJMP) sprong beschouwen. Het ASM-schema van deze instructie staat op \figrefpag{asm-jsr}. We zien op het ASM-schema twee toestanden die van belang zijn. In de eerste toestand wordt het geheugen op het adres van de programmateller ingelezen in het adresregister. Deze bewerking hebben we echter ook beschouwd bij het inladen van geheugendata met directe adressering. In de volgende toestand beschouwen we drie bewerkingen: $\texttt{Mem[RF[s1]]}\leftarrow\texttt{PC}$, $\texttt{RF[s1]}\leftarrow\texttt{RF[s1]+1}$ en $\texttt{PC}\leftarrow\texttt{AR}$. De increment van het register $\texttt{RF[s1]}$ is eenvoudig te implementeren. Dit is immers een register-instructie en bijgevolg dienen we geen extra verbindingen te voorzien. We dienen verder ook de waarde van de programmateller weg te schrijven in het geheugen op het adres aangegeven door een register. Hiervoor dienen we verbindingen te voorzien die de programmateller verbindt met de databus en de eerste leespoort met de adresbus. Merk op dat we bijgevolg zowel de controller als het datapad moeten aanpassen. Tot slot dienen we de waarde van het adresregister naar de programmateller kopi\"eren. Een probleem is echter dat alle beschikbare bussen reeds bezet zijn. We kunnen weliswaar de tweede operandbus gebruiken, maar deze heeft op dit moment geen verbinding met de programmateller. We kunnen dit oplossen door een verbinding te voorzien vanuit het adresregister naar de programmateller, al da niet via de tweede operandbus, ofwel door het ASM-schema aan te passen. We kunnen bijvoorbeeld een nieuwe toestand voorzien waarin -- bijvoorbeeld via de vrijgekomen databus -- de waarde van het adresregister aanlegt op de multiplexer van de programmateller. In dit geval kiezen we voor het tweede. Dit leidt tot de finale controller in \figref{cisc-controller-final}.
\importtikzfigure{cisc-controller-final}{De uiteindelijke controller van de CISC-processor.}
\paragraph{Terugkeer uit subroutine (RTS)}
De laatste spronginstructie die we beschouwen is de terugkeer uit een subroutine ofwel de RTS-instructie. Het relevante ASM-schema bij deze instructie staat in \figref{asm-rts}. Hiervoor dienen twee bewerkingen tegelijk worden uitgevoerd: $\texttt{RF[s1]}\leftarrow\texttt{RF[s1]-1}$ en $\texttt{PC}\leftarrow\texttt{Mem[RF[s1]-1]}$. Een terugkeer-instructie heeft veel gemeen met een sprong naar een subroutine. Toch is er een fundamenteel verschil: het geheugenadres wordt bepaald door een verschil berekend in dezelfde klokcyclus. Dit verschil wordt berekend door de ALU. Het resultaat van de ALU dient niet enkel teruggeschreven worden naar de registerbank, maar moet ook op de adresbus worden aangelegd (om de waarde op te vragen die in de programmateller moet worden weggeschreven). Daarom voorzien we een tri-state buffer tussen de resultaat-bus en de adresbus. Het wegschrijven van de waarde van de programmateller is vervolgens de verantwoordelijkheid van de controller. De vereiste verbindingen in de controller werden reeds gerealiseerd.
\paragraph{Groter dan (GT) en andere vergelijkingen}
Tot slot beschouwen we \'e\'en van de overige instructies: de groter dan operatie (hoewel iedere operatie kan worden beschouwd). De vergelijker in het datapad is een algemene vergelijker die op basis van stuursignalen \'e\'en van de voorwaarden bepaalt. De vergelijker vergelijkt de waarde in de registers \texttt{RF[s1]} en \texttt{RF[s2]}. Vermits hiervoor reeds operandbussen werden voorzien dienen we enkel verbindingen tussen de bussen en de ingangen van de vergelijker te voorzien. Verder dienen we vervolgens het resultaat in het statusregister weg te schrijven. We kunnen hiervoor een bit van de resultaatbus gebruiken. In dat geval dienen we echter een tri-state buffer te voorzien. Vermits het statusregister slechts \'e\'en bit bijhoudt, is de vereiste verbinding slechts \'e\'en bit breed. Daarom voorzien we een aparte verbinding vanuit de vergelijker naar het statusregister. De toestand van het statusregister vormt ook een signaal voor de controller (bijvoorbeeld bij een voorwaardelijk sprong). We kunnen andere condities testen door de stuursignalen van de vergelijker aan te passen. We beschouwen geen verdere overige instructies omdat deze geen nieuwe verbindingen introduceren. Deze instructies zetten of resetten immers registers, iets wat we kunnen realiseren met behulp van een multiplexer aan de ingang van het statusregister en met behulp van de reset-ingang bij verschillende registers. We bekomen dan ook het uiteindelijke datapad op \figref{cisc-datapad-connections}.
\importtikzfigure{cisc-datapad-connections}{Het datapad-ontwerp na het aanbrengen van verbindingen van de CISC-processor.}
\subsubsection{De 8086 Microprocessor}
Tot slot kijken we naar enkele populaire CISC-processoren: de 8086 microporocessor en de 8051 microcontroller. De structuur van deze processoren staat respectievelijk in \figref{processor-8086} en \figref{processor-8051}.
\paragraph{}
Vooral de 8086 is een populaire processor in computers en vormt de basis voor alle Intel CPU's\footnote{Hoewel de verdere processoren bijvoorbeeld de bitlengte van de registers verder hebben opgedreven is de algemene structuur nog steeds dezelfde.}. Op de figuur staat een component waarvan de vorm nog niet werd ge\"introduceerd: een trapezium met in het midden een inkeping wordt meestal gebruikt om een ALU aan te duiden.
\paragraph{}
In het blokdiagram van Intel 8086 zien we twee ALU's staan. De bovenste ALU rekent enkel het adres uit bij impliciete adressering en staat dan ook bekend als de \emph{adress adder}. De onderste ALU voert verschillende bewerkingen uit zoals optellen, aftrekken, vermenigvuldigen,... Verer zien we onderaan links ook een registerbank. Deze registerbank houdt de acht registers van elk 16 bit. Vier van de registers: AX, BX, CX en DX kunnen verder worden opgedeeld als registers van 8-bit. De overige vier registers zijn registers met een speciaal doel: bijvoorbeeld SP houdt de \emph{stack pointer} bij. Deze registers kan men dan ook niet aanwenden om eender welke instructie op toe te passen en de waarde wordt meestal impliciet aangepast.
\paragraph{}
Onder de ALU zien we nog een register die de \emph{Operand Flags} bijhoudt. Het is een 16 bit registers waarvan 9 bits dienstdoen om een voorwaarde met betrekking tot het resultaat berekend in de ALU bij te houden. Bijvoorbeeld als het resultaat van de ALU gelijk is aan $0$ wordt de \emph{Zero Flag} op $1$ gezet. Het voordeel van dit register is dat men niet altijd eerst een expliciete vergelijk-instructie moet uitvoeren.
\paragraph{}
Boven de central bus zijn we een andere registerbank. Deze registerbank houdt registers bij met betrekking tot de adressering zoals \emph{Instruction Pointer (IP)} \emph{Code Segment (CS)}, \emph{Data Segment (DS)}, \emph{Extra Segment (ES)} en \emph{Stack Segment (ES)}. Behalve de instructionpointer (het equivalent van de programmateller) valt de betekenis van de overige registers buiten het bereik van deze cursus. Men kan echter stellen dat ze nuttig zijn om het adres in een instructie uit te rekenen.
\paragraph{}
Tot slot vinden we bovenaan centraal ook nog een registerbank. Deze registerbank noemen we de \emph{Instruction Queue}. Het is een geheugen dat enkele instructies bevat die waarschijnlijk binnenkort zullen worden uitgevoerd. Door deze in een register te bewaren hoopt men geen wachttijden te introduceren om instructies uit het geheugen op te halen. Onder deze wachtrij bevindt zich ten slotte de controle-eenheid ofwel de controller die op basis van de instructie de gegevens op de juiste manier door de bussen laat stromen.
\importtikzfigure{processor-8086}{De structuur van de 8086 microprocessor.}
\subsubsection{De 8051 Microcontroller}
De 8051-microcontroller is in tegenstelling tot de Intel 8086 een Harvard-machine met een onderscheid tussen datageheugen en instructiegeheugen. In tegenstelling tot de 8086 waar enkel een rekeneenheid met registers wordt voorzien, voorziet de 8051 een rekeneenheid, geheugen, ROM-geheugen, Input-Output (I/O), interrupt logica en een timer.
\paragraph{}
De processor werkt hoofdzakelijk met een woordlengte van $8$-bit. Zo voorziet men een $8$-bit ALU en $8$-bit registers. De adresbus is echter $16$-bit waardoor een groter geheugen kan worden uitgelezen. Wat opvalt is dat de processor zowel een klein RAM-geheugen voorziet van 128 bytes waar de programmagegevens in worden ondergebracht terwijl de chip ook over een grote hoeveelheid EPROM of ROM geheugen beschikt waar het programma in wordt ingelezen. EPROM kan slechts eenmalig worden beschreven. Het is dan ook de bedoeling om deze chip in een specifieke applicatie te gebruiken door er een programma in onder te brengen zonder hier later wijzigingen in aan te brengen. Verder kan men ook opmerken dat de chip voor verschillende speciale registers extra functionaliteit voorziet. Zo zien we een increment-operator die enkel dienst doet om de programmateller op te hogen.
\importtikzfigure{processor-8051}{De structuur van de 8051 microcontroller.}
\subsection{Reduced Instruction Set Computer (RISC)}
\subsubsection{Ontwerp}
Het ontwerp van een RISC processor verloopt ongeveer gelijkaardig aan dat van een CISC processor. Een programma zal echter uit meer instructies bestaan dan bij een CISC processor, daarom is de klokfrequentie een zeer belangrijke factor. Pipelining vormt dan ook een zeer belangrijk aspect in de implementatie van een RISC processor. Zo worden de instructies zelf in nagenoeg alle processoren uitgevoerd volgens het pipelining principe\footnote{Zie voorbeelden van de ARM7 en ARM11 microprocessor verder in deze subsectie.}.
\paragraph{}
Pipelining zal er meestal toe leiden dat men extra beperkingen op de instructieset plaatst. We denken bijvoorbeeld aan het feite dat de meeste instructies even lang zijn. Op de ARM7 processor worden bijvoorbeeld alle instructies voorgesteld met 32 bit (inclusief verplaats instructies). Door geen optioneel adresvelden te voorzien kan een instructie in \'e\'en toestand worden opgehaald. Dit is belangrijk wanneer we bijvoorbeeld een instructie willen inlezen in \'e\'en toestand. Met optionele velden toe te laten moeten we op basis van de instructie beslissen of we de programmateller verder ophogen. Dit verhindert echter dat we in de volgende toestand dus een nieuwe instructie kunnen ophalen. Ook de controller en het datapad zelf werken volgens het pipelining-principe. Dit is belangrijk omdat de pipeline in het geval van sprongbevelen bijvoorbeeld moet worden onderbroken. Andere instructiesets voorzien geen hardware om de pipeline te onderbreken maar een No-Operation instruction die tussen twee instructies kunnen worden geplaatst waardoor de tweede instructie wacht tot de eerste instructie volledig is uitgevoerd.
\subsubsection{De ARM7 Microprocessor}
Bij wijze van voorbeeld zullen we twee microprocessoren bespreken: de ARM7 en ARM11 microprocessor. Voor beide processoren zullen we de structuur bespreken samen met de instructieset en de vormen van pipelining die werden ge\"implementeerd.
\paragraph{}
\importtikzfigure{processor-arm7}{De structuur van de ARM7 microprocessor.}
\figref{processor-arm7} beschrijft de structuur van een ARM7 processor. Op de figuur staan opnieuw enkele vormen die nog niet werden ge\"introduceerd. Zo wordt een barrel shifter typisch voorgesteld aan de hand van een parallellogram en een vermenigvuldiger aan de hand van een zeshoek waarbij de linkse en rechtse knoop inwendige hoeken zijn.
\paragraph{}
Net als in het geval van de Intel 8086 omvat de processor zelf geen RAM geheugen behalve enkele registers. Instructies en data worden dan ook uit een geheugen gelezen naast de ARM processor die verbonden is via een adresbus \texttt{A[31:0]} en een databus \texttt{D[31:0]}. De ARM7 processor omvat grofweg drie delen: een geheugeninterface: dit omvat het \emph{adresregister}, de \emph{adres incrementer}, en het \emph{write data register}; een rekenkundige eenheid met een registerbank, vermenigvuldiger, ALU en barrel shifter en de controle eenheid die de \emph{instruction pipeline} en de \emph{instruction decoder} omvat.
\subsubsection{De ARM11 Microprocessor}
\importtikzfigure{processor-pipeline-arm11}{De pipelining-structuur van de ARM11 microprocessor.}
\part{Very High Speed Integrated Circuit Hardware Description Language}
\chapter{VHDL}
\chapterquote{In redelijke taal weerklinkt wat in werkelijkheid gebeurt.}{Gerard Bolland, Nederlands taalkundige en filosoof (1854-1922)}
\section{Elementen van de VHDL-taal}
In deze sectie zullen we eerst de woordenschat in de VHDL taal bespreken, waarna we de belangrijkste types zullen bespreken samen met het 
\subsection{Lexicale elementen (woordenschat)}
Elk code-bestand\footnote{Onafhankelijk van de taal.} bestaat uit een sequentie van lexicale elementen die men \termen{tokens} oemt. We zullen eerst de verschillende lexicale elementen die in VHDL-code kunnen voorkomen beschrijven. Deze kunnen we dan als terminologie gebruiken. De tokens kunnen verder worden onderverdeeld in: \termen{literals} (deze vertegenwoordigen bepaalde waardes zoals getallen, tekst,...), \termen{identifiers} (dit zijn namen van variabelen, functies, componenten,... gebruikt om naar te refereren en defini\"eren), \termen{scheidingstekens}\footnote{Engels: \emph{delimiters}.} (haakjes en andere symbolen die waardes en identifiers combineren) en \termen{commentaar}. We beschrijven de verschillende soorten tokens verder in de volgende subsecties.

\subsection{Literals}
We onderscheiden vier soorten literals: getallen, karakters, karakterreeksen en bitreeksen. We overlopen de verschillende types. In VHDL is er ook nog sprake van \vhdltermen{enumeration literals} en van de \vhdltermen{NULL literal}, deze literals worden niet beschouwd.

\paragraph{Getal}Een getal (ook wel ``\vhdltermen{abstract literal}'' genoemd) is een numerieke waarde. Net als bij programmeertalen onderscheid men verschillende soorten numerieke waarden:
\begin{itemize}
  \item \vhdltermen{universal\_integer}: dit zijn numerieke literals die geen decimale komma (\vhdltermen{.}) bevatten, bijvoorbeeld \verb+1425+. Deze getallen zijn vergelijkbaar met het \texttt{int}-type in \texttt{Java}\footnote{De vergelijking gaat echter niet geheel op: een \texttt{int} is immers beperkt tot $32$ bit en heeft een specifieke representatie.}.
  \item \vhdltermen{universal\_real}: dit zijn numerieke getallen die wel een decimale komma bevatten. Bijvoorbeeld \verb+1425.1917+. Merk op dat indien er enkel nullen achter de komma staan dat het getal dezelfde waarde heeft als een \vhdltermen{universal\_integer}: \verb+1425.0+, maar daarom niet dezelfde representatie. In \texttt{Java} zouden we dit type voorstellen met bijvoorbeeld een \texttt{float} of \texttt{double}.
  \item \vhdltermen{physical types}: dit zijn fysische waarden. Fysische waarden bestaan uit een getal gevolgd door een fysische eenheid. Er wordt een spatie tussen het getal en de eenheid gezet. Bijvoorbeeld: \verb+1818 ns+. Dit type heeft geen echt equivalent in de meeste programmeertalen. Immers redeneren programmeertalen meestal niet over grootheden.
\end{itemize}
Getallen worden standaard in decimale notatie uitgedrukt. Men kan ze ook in exponenti\"ele vorm schrijven door het toevoegen van een '\vhdltermen{E}' of `\vhdltermen{e}' op het einde van het getal gevolgd door de exponent, bijvoorbeeld: \verb+14e2+. Om een andere basis te gebruiken, wordt volgend formaat gebruikt: \vhdltermen{base\#literal\#exp}, met de basis tussen 2 en 16. Zo kunnen we $1425$ schrijven als \verb+16#591#+, of $4864$ als \verb+16#13#E2+. Tot slot mag men ook vrij \emph{underscores} (\vhdltermen{\_}) toevoegen in een getal om de code leesbaar te houden. Bijvoorbeeld: \verb+19_171_425+.


\paragraph{Karakter} Een karakter (ofwel ``\vhdltermen{character literal}'') slaat een \termen{karakter} (ofwel \termen{character}) op: de eenheid van tekst. Deze literals worden tussen enkele aanhalingstekens (\vhdltermen{'}) geplaatst, bijvoorbeeld \verb+'K'+. In \texttt{Java} zou het equivalent van dit type een \texttt{char} zijn.

\paragraph{Karakterreeks}
Een \termen{karakterreeks} (ofwel ``\vhdltermen{string literal}'') slaat een sequentie karakters op, dus een tekstfragment. In \texttt{Java} noemt men een dergelijk type een \texttt{String}. Deze reeks wordt tussen dubbele aanhalingstekens (\vhdltermen{"}) geplaatst, bijvoorbeeld \texttt{"Hello World!"}. Een probleem doet zich voor wanneer we de dubbele aanhalingstekens (\texttt{"}) zelf in het tekstfragment willen gebruiken. In dat geval dienen we de aanhalingstekens dubbel te plaatsen, bijvoorbeeld:
\begin{quote}\verb+"""The answer?"" said Deep Thought. ""The answer to what?"""+\cite[\S25]{Adams81BOOK54}\end{quote}
\paragraph{Bitreeksen}
Een \vhdltermen{bitreeks} (ofwel \vhdltermen{``bit string literal}'') is een sequentie aan bits (dit is een eenheid van informatie die ofwel \texttt{0} ofwel \texttt{1} is). Een bitreeks wordt geschreven als een sequentie van enen en nullen tussen dubbele aanhalingstekens (\vhdltermen{"}). Zoals we al vroeger in deze cursus hebben opgemerkt is zo'n sequentie niet effici\"ent, leesbaar en praktisch. Daarom laat men ook octale  en hexadecimale notatie toe. Hiertoe wordt er een \vhdltermen{B} (binair), \vhdltermen{O} (octaal) of \vhdltermen{X} (hexadecimaal) voor de dubbele aanhalingstekens geplaatst om de bitreeks in een bepaald formaat te lezen. Bij hexadecimale getallen gebruikt men \vhdltermen{a}, \vhdltermen{b}, \vhdltermen{c}, \vhdltermen{d}, \vhdltermen{e} en \vhdltermen{f} om respectievelijk $10$ tot $15$ voor te stellen. Net als bij getallen kan de gebruiker \termen{underscores} (\vhdltermen{\_}) toevoegen om tot meer leesbare code te komen.
\subsection{Identifiers}
Een \vhdltermen{identifier} is een referentie naar een variabele, functie, component,... Dit is vergelijkbaar met de naam van een variabele, methode, klasse,... in \texttt{Java}. Identifiers beginnen met een letter en mogen letters en cijfers bevatten. Ook de underscore (\vhdltermen{\_}) teken is toegelaten, indien dit niet het eerste of laatste karakter van de identifier vormt. Identifiers zijn niet hoofdletter gevoelig. In VHDL'93 wordt het begrip van een identifier verder uitgebreid. Deze uitbreiding wordt hier niet beschouwd. Voor meer details \cite[p. 4]{hardi00}.
\paragraph{Gereserveerde woorden}Niet alle namen die aan de hierboven beschreven regels mag men zelf gebruiken. Sommige identifiers zijn immers \termen{sleutelwoorden} die een reeds ingebouwde functie vevullen in VHDL: ze verwijzen naar functies die in de VHDL compiler zijn ingebakken. De woorden in \tblref{vhdl-reserved-words} mogen niet gebruikt worden als identifiers.

\importtabulartable{vhdl-reserved-words}{Gereserveerde woorden in VHDL.}

We onderscheiden verschillende soorten sleutelwoorden: enerzijds zijn er compiler-directieven zoals bijvoorbeeld \vhdltermen{case} en \vhdltermen{downto}, daarnaast zijn er basis-operaties zoals \vhdltermen{nand} en \vhdltermen{sla}, tot slot zijn er ook type-directieven zoals \vhdltermen{array} en \vhdltermen{signal}.

\subsection{scheidingstekens}
Een \vhdltermen{scheidingsteken} (ofwel ``\vhdltermen{delimiter}'') tenslotte wordt gebruikt om operaties op gegevens uit te voeren, bijvoorbeeld een optelling maar ook een index. De
scheidingstekens van VHDL worden weergegeven in \tblref{vhdl-delim}.

\importtabulartable{vhdl-delim}{Scheidingstekens in VHDL.}

We onderscheiden opnieuw verschillende soorten scheidingstekens: 

\subsection{Commentaar}
Commentaar zijn delen van de code die genegeerd worden door de VHDL-compiler, maar die nuttig zijn voor programmeurs: ze bevatten gegevens over het project en richtlijnen die een belangrijke rol kunnen hebben tijdens het project. Commentaar plaatst men na twee horizontale strepen \verb+--+ en reikt tot het einde van de lijn. Dit is vergelijkbaar met de dubbele slash (\verb+//+) in \texttt{Java}.

\section{Typesysteem}
VHDL is getypeerd: een variabele slaat data op volgens een bepaald type, dit type geeft een interpretatie aan zowel de data en de operatoren die erop gedefinieerd zijn. De types kunnen bovendien door de programmeur uitgebreid worden. Men vertrekt echter steeds van basistypes.

\paragraph{}
Ook in de VHDL-secties werden sommige van deze types reeds gebruikt. \tblref{vhdl-type} geeft een overzicht van de meest populaire hoofdtypes.
\begin{table}[hbt]
\centering
\importtabularsubtable{vhdl-type}{Overzicht van belangrijke types in VHDL.}
\importtabularsubtable{vhdl-type-deriv}{Overzicht van belangrijke afgeleide types in VHDL.}
\caption{Overzicht van belangrijke types en afgeleide types in VHDL.}
\end{table}
Naast deze reeks van basistypes bevat VHDL ook standaard enkele afgeleide types. Het zijn types die een gereduceerd bereik uit de basistypes vertegenwoordigen. Deze subtypes staan in \tblref{vhdl-type-deriv} samen met hun gereduceerd bereik.
\paragraph{Zelf types defini\"eren} Hoe defini\"eren we nu zelf types? Er zijn enkele manieren om zelf types te defni\"eren:
\begin{itemize}
 \item Defini\"eren door opsomming
 \item Defini\"eren door subtypering (beperk het bereik)
 \item Defini\"eren van fysische types
 \item Afgeleide types met matrices en vectoren
\end{itemize}
\paragraph{Defini\"eren met opsomming} We kunnen een type specificeren door alle mogelijke toestanden op te sommen. Deze methode wordt toegepast voor bijvoorbeeld de types \vhdltermen{bit} en \vhdltermen{byte}. De namen of karakters die de toestand bepalen worden tussen haakjes opgesomd na het sleutelwoord \vhdltermen{is}. \vhdlref{lst:typeEnum} toont mogelijke definities van \vhdltermen{bit} en \vhdltermen{byte}.
\begin{vhdlcode}[hbt]
\begin{lstlisting}
type bit is ('0','1');
type boolean is (false,true);
\end{lstlisting}
\caption{Defini\"eren van types door opsomming.}
\label{lst:constante}
\end{vhdlcode}

\begin{vhdlcode}[hbt]
\paragraph{Defini\"eren met subtypering}
\begin{lstlisting}
subtype byte is integer range 0 to 255;
subtype lowercase is character range 'a' to 'z';
\end{lstlisting}
\caption{Defini\"eren van types door subtypering.}
\label{lst:constante}
\end{vhdlcode}

\begin{vhdlcode}[hbt]
\begin{lstlisting}
type bit is ('0','1');
type boolean is (false,true);
\end{lstlisting}
\caption{Defini\"eren van fysische types.}
\label{lst:constante}
\end{vhdlcode}
\subsubsection{Data-objecten}
Een object in VHDL is een benoemd item met een specifieke waarde. We onderscheiden 3 soorten objecten: \vhdltermen{constante}, \vhdltermen{variabele} en \vhdltermen{signaal}.
\paragraph{Constante}Een constante is een object die slechts \'e\'enmaal een waarde kan toegekend worden. Ze geven een interpretatie aan de waarde, en maken de VHDL-code daarom beter verstaanbaar en aanpasbaar. We gebruiken hiervoor het sleutelwoord \vhdltermen{constant}. \vhdlref{lst:constante} toont de declaratie van verschillende constanten. Merk op data VHDL niet hoofdlettergevoelig is in programma-syntax. Alleen de waarde die in karakters en karakterreeksen wordt opgeslagen zijn hoofdlettergevoelig.
\begin{vhdlcode}[hbt]
\begin{lstlisting}
constant pi : real := 3.14159265;
constant byte_length : natural := 8;
constant word_length : natural := 4*byte_length;
\end{lstlisting}
\caption{Werken met constanten.}
\label{lst:constante}
\end{vhdlcode}
Het algemene formaat is dus \verb+constant <identifier> : <type> := <waarde>;+.
\paragraph{Variabele}De syntax van een variabele is ongeveer dezelfde. Alleen wordt het sleutelwoord \vhdltermen{variable}. Verder is het onmiddellijk toekennen van een waarde optioneel. \vhdlref{lst:variabele} toont het gebruik van variabelen in VHDL.
\begin{vhdlcode}[hbt]
\begin{lstlisting}
variable index : integer;
index := 0;
index := index + 1;
variable andere_index : natural := 12;
variable antwoord : natural := 4*andere_index-6*index;
\end{lstlisting}
\caption{Werken met variabelen.}
\label{lst:variabele}
\end{vhdlcode}
\paragraph{Signaal}Een signaal ten slotte is het VHDL-equivalent van een fysische verbinding of een groep van verbindingen in de hardware. Een signaal wordt geconstrueerd met behulp van het sleutelwoord \vhdltermen{signal}. Toewijzingen aan een signaal gebeuren met de \vhdltermen{<=} operator. De toewijzing bij een signaal werkt anders dan bij een variabele en constante. Variabelen en constanten worden onmiddellijk toegewezen. Dit betekent dat expressies als \verb/index := index + 1;/ mogelijk zijn. Signalen veranderen echter wanneer elementen uit hun toewijzing veranderen. Beschouw het voorbeeld uit \vhdlref{lst:signal}.
\begin{vhdlcode}[hbt]
\begin{lstlisting}
signal a : bit;
signal b : bit;
signal y : bit;
a <= '1';
b <= '0', '1' after 100 ns;
y <= a and b;
\end{lstlisting}
\caption{Werken met signalen.}
\label{lst:variabele}
\end{vhdlcode}
We zien drie signalen \verb+a+, \verb+b+ en \verb+y+. \verb+b+ zal na $100\mbox{ ns}$ een 1 aanleggen. Dit heeft als implicatie dat ook \verb+y+ van waarde zal veranderen. Indien we dit met variabelen en constanten zouden hebben berekend zou variabele \verb+y+ na de toewijzing een waarde toegewezen krijgen, en deze tot een volgende toewijzing behouden. Merk verder op dat we hier ook de betekenis van het codewoord \vhdltermen{after} tonen. Een signaal kan ook een initiele waarde krijgen, hiervoor gebruiken we de \vhdltermen{:=} operator.
\subsubsection{Bibliotheken}
\subsubsection{Bewerkingen}
\section{Combinatorische schakelingen in VHDL}
\label{s:combinatorischVHDL}
Nu we verschillende schakelingen hebben gebouwd zullen we deze proberen te beschrijven met VHDL. In deze sectie zullen we eerst een formeel overzicht geven van de VHDL-syntax. Vervolgens zullen we in subsectie \ref{ss:combinatorischVHDLHardware} een methode ontwikkelen om combinatorische elementen te beschrijven met deze syntax. In elk hoofdstuk zullen we de methodologie uitbreiden zodat we de nieuwe componenten ook kunnen beschrijven.
\subsection{Hardwarebeschrijving met VHDL}
\subsubsection{Structurele beschrijving}
\begin{vhdlcode}[hbt]
\centering
\begin{lstlisting}
-- 2-naar-1 Multiplexer
--
architecture Struct of MUX2 is
  signal U,V,W : bit;
  component AND2 is
    port (X,Y: in bit;
          Z: out bit);
  end component AND2;
  component OR2 is
    port (X,Y: in bit;
          Z: out bit);
  end component OR2;
  component INV is
    port (X: in bit;
          Z: out bit);
    end component INV;
begin
  Gate1: component INV  port map (X=>S,Z=>U);
  Gate2: component AND2 port map (X=>A,Y=>S,Z=>W);
  Gate3: component AND2 port map (U,B,V);
  Gate4: component OR2  port map (X=>W,Y=>V,Z=>Y);
end Struct;
\end{lstlisting}
\caption{2-naar-1-multiplexer.}
\label{vhdl:bToAMultiplexer}
\end{vhdlcode}
\subsubsection{Gedragsbeschrijving}
\begin{vhdlcode}[hbt]
\centering
\begin{lstlisting}
-- Opteller
--
library ieee;
use ieee.std_logic_signed.all;

entity adder is
  generic(n : positive := 4);
  port(Cin : in std_logic;
       X,Y : in std_logic_vector(n-1 downto 0);
       S   : in std_logic_vector(n-1 downto 0);
       Cout, Overflow : out std_logic);
end adder;

architecture behav of adder is
  signal Sum : std_logic_vector(n downto 0);
begin
  Sum <= (X(n-1) & X) + Y + Cin;
  S <= Sum(n-1 downto 0);
  Cout <= Sum(n);
  Overflow <= Sum(n) xor X(n-1) xor Y(n-1) xor Sum(n-1);
end behav;
\end{lstlisting}
\caption{$n$-bit Opteller.}
\label{vhdl:adder}
\end{vhdlcode}
\subsubsection{Repetitieve structuren}
\label{ss:combinatorischVHDLHardware}
\part{Digitale Elektronica in de Praktijk}
\chapter{Circuits Schakelen}
\chplab{circuitsschakelen}
\begin{chapterintro}
Tot slot bieden we een laatste gedeelte over het bouwen van digitale schakelingen in de praktijk. Dit gedeelte is optioneel. Het is geen leerstof, staat niet in de presentaties en vorm zeker geen onderdeel tijdens het examen. Het is dan ook eerder tot stand gekomen voor ``enthousiastelingen'' die de opgedane kennis in de praktijk willen omzetten. In dit hoofdstuk beschrijven we welke ge\"integreerde circuits men zich dient aan te schaffen om de schakelingen te implementeren. In de rest van het deel worden enkele projecten besproken die men kan realiseren.
\end{chapterintro}
\section{Anatomie van een Ge\"integreerd circuit}
We beginnen het hoofdstuk met de terminologie die men hanteert bij ge\"integreerde circuits.
\subsection{DIP packing}
Vermits het de bedoeling is dat men de circuits zelf implementeert, zullen we ook enkel werken met ge\"integreerde circuits die men zelf kan ineenzetten. Meestal kiest men daarvoor voor ``DIP packing''. DIP packing beschouwt rechthoekige chips met verbindingen in de lengte. \figref{dip8-packing} geeft dit concept grafisch weer.
\begin{figure}[hbt]
\centering
\importtikzsubfigure{dip8-packing}{Structuur}
\importtikzsubfigure{dip8-front}{Vooraanzicht}
\importtikzsubfigure{dip8-side}{Zijaanzicht}
\caption{DIP packing}
\end{figure}
Een verbinding noemt men een ``pin'' of ``pootje''. Omdat chips correct geori\"enteerd moeten worden maakt men gebruik van twee systemen. De ``notch'' is een cirkelvormige inkeping aan \'e\'en kant van de chip. Daarnaast zet men op de meeste chips ook een ``index marker''. Een index marker is een cirkel in de buurt van pin 1. Wanneer we de notch aan de linkerkant plaatsen (zoals op de figuur), worden de pootjes onderaan van links naar rechts genummerd. De nummering van de bovenste pootjes verloopt van rechts naar links. Elke digitale chip beschikt telkens over minstens twee pootjes: de power pin (ofwel \mbox{Vcc}) en de ground pin (ofwel \mbox{GND}). Bij de meeste chips stelt pin 1 de ground pin door. Meestal stelt de laatste pin (pin 8 op \figref{dip8-packing}) de power pin voor. Soms gebruikt men ook de eerste pin van de bovenste rij (pin 5 op \figref{dip8-packing}). De powerpin moet worden verbonden met de positieve pool van de voeding, de ground pin met de negatieve pool. In een schematische voorstelling van een ge\"integreerd circuit geeft men vaak de nummers van de verschillende pinnen weer. Een diagram die de functie van de verschillende pinnen beschrijft noemt men een ``pinout''. \figref{dip8-front} en \figref{dip8-side} tonen het voor- en zijaanzicht van een DIP ge\"integreerd circuit. Doorgaans zijn de pootjes dikker rond het midden van de chip dan het einde. De structuur van DIP vereist bijna dat het aantal pootjes altijd even is. Indien dit niet het geval is zal men meestal een veelvoud van de componenten in een chip implementeren. Zo bestaan er bijvoorbeeld chips die $4$ NAND-poorten aanbieden.
\subsection{Populaire ge\"integreerde circuits}
Nu we de structuur van een ge\"integreerd circuit hebben voorgesteld, zullen we een bondige samenvatting geven over de belangrijkste ge\"integreerde circuits. Belangrijke ge\"integreerde circuits werden gestandaardiseerd met een nummer. Op gebied van digitale elektronica spreekt men over twee ``ge\"integreerde circuit-families'': de 74000 serie en de 4000 serie. De 74000 serie omvat digitale componenten ge\"implementeerd met transistor-transistor logic. De 4000 serie implementeert dezelfde schakelingen in CMOS-logica. Indien we dus Zo stelt ge\"integreerd circuit 4068 een 8-NAND poort voor in CMOS logica. 47068 biedt dezelfde poort aan, maar volgens een andere technologie. Voor een volledige lijst van de ge\"integreerde circuits in de 74000 en 4000 families: zie \cite{74000icfamily,4000icfamily}.
\paragraph{}
\importtabulartable{popular-ic}{Lijst met populaire ge\"integreerde circuits.}
In \tblref{popular-ic} staat een lijst met populaire ge\"integreerde circuits. Naast de digitale circuits omvat de tabel ook enkele analoge schakelingen. De werking van deze circuits wordt verder in dit hoofdstuk verduidelijkt.
\section{Analoge componenten}
Naast ge\"integreerde circuits zal men in de praktijk in een digitale schakeling ook analoge componenten aantreffen: weerstanden, condensatoren, transistoren en in mindere mate spoelen. Deze componenten zijn meestal nodig om effectief iets aan te sturen (bijvoorbeeld een led) of bij de captatie van invoer (bijvoorbeeld radiogolven). In deze sectie geven we een kort overzicht.
\subsection{Weerstanden}
Een weerstand is een component die de stroomsterkte in een circuit verlagen en bovendien lokaal de spanningsniveaus verminderen. Een circuit gedraagt zich dan ook volgens de ``\emph{Wet van Ohm}''. Deze wet stelt dat tussen elke twee punten in een elektronische schakeling, het verschil in potentiaal $\Delta V$ lineair schaalt met zowel de stroomsterkte $I$ als de \emph{weerstand} $R$:
\begin{equation}
\Delta V=I\cdot R
\end{equation}
De meeste weerstanden in een schakeling hebben een vaste weerstand $R$, er bestaan echter ook weerstanden zoals \emph{lichtgevoelige weerstanden}, waar de weerstand afhangt van de lichtintensiteit, temperatuur, ... In deze cursus zullen we enkel weerstanden met een vaste weerstand beschouwen.
\subsubsection{Kleurcode}
De weerstand wordt traditioneel niet numeriek op het component genoteerd, maar aan de hand van vier tot zes banden die elk een specifieke kleur hebben. Deze sequentie beschrijft de weerstand, de \emph{toleratie}\footnote{Dit is een betrouwbaarheidsinterval waarin de werkelijke weerstandswaarde zit.} en optioneel de \emph{temperatuursco\"effici\"ent}\footnote{De werkelijke waarde van de weerstand is altijd enigszins afhankelijk van de temperatuur van de weerstand op dat moment.}. Afhankelijk van het aantal banden, heeft elke band ook een verschillende betekenis: getal, multiplicatie, toleratie en temperatuursco\"effici\"ent. \tblref{resistor-colorcode} toont hoe de verschillende kleuren overeenstemmen met hun betekenis.

\importtabulartable{resistor-colorcode}{De verschillende kleurcodes bij een weerstand.}
\paragraph{}
De eerste band is altijd aangebracht op een plaats waar de straal van de weerstand groter is. In het geval de weerstand uit zes banden bestaat, is de zesde band aangebracht aan de andere kant van de weerstand waar de straal ook groter is. Om toch het onderscheid te kunnen maken tussen de eerste en de laatste band zijn de tweede en de derde band altijd aan dezelfde kant van de weerstand. Vermits er hoogstens zes banden aangebracht zijn op een weerstand kan men dus door te rekenen vanuit het midden van de weerstand, altijd de eerste band lokaliseren door de kant te nemen die de meeste banden bevat. De posities samen met de betekenis van de banden staat schematisch voorgesteld op \figref{resistor-colorband}.
\importtikzfigure{resistor-colorband}{De uitlijning en betekenis van de verschillende kleurbanden.}%TODO finish
\paragraph{}
In het algemeen beschrijven de eerste twee (in het geval van vier banden) of drie banden de cijfers die samen een getal vormen. Dit getal moet dan vermenigvuldigt worden met de multiplicator. De toleratie drukt aan de hand van een percentage uit wat het betrouwbaarheidsinterval is. Indien de weerstand bijvoorbeeld $10~\mbox{k}\Omega$ is met een toleratie van $5\%$, dan is het betrouwbaarheidsinterval: $\fbrk{9.5~\mbox{k}\Omega ; 10.5~\mbox{k}\Omega}$.
\paragraph{}
In het geval de weerstand vier banden bevat zijn de eerste twee banden cijfers $c_1$ en $c_2$, de derde de multiplicatie $m$ en de laatste de toleratie $t$, in dat geval is de weerstandswaarde dus:
\begin{equation}
\begin{array}{lr}
R=\brak{10\cdot c_1+c_2}\cdot m~\Omega&\mbox{($4$ banden)}
\end{array}
\end{equation}
In het geval van vijf en zes banden zijn de eerste drie banden cijfers $c_1$, $c_2$ en $c_3$, de vierde de multiplicatie $m$, de vijfde de tolerantie $t$ en optioneel de zesde de temperatuurco\"effici\"ent. In dat geval is de weerstandswaarde dus:
\begin{equation}
\begin{array}{lr}
R=\brak{100\cdot c_1+10\cdot c_2+c_3}\cdot m~\Omega&\mbox{($5$ en $6$ banden)}
\end{array}
\end{equation}
%TODO: finish
\subsection{Condensatoren}
\emph{Condensatoren} en \emph{spoelen} zijn componenten die streven naar een behoudt van respectievelijk spanning en stroomsterkte. In het geval van wisselstroom of een periodiek veranderende spanning, kan men ze bovendien zien als weerstanden waarbij de weerstand afhangt van de fase op dat moment.
\paragraph{}
Een condensator heeft tot doel spanning te stabiliseren, dit doet een condensator door elektronische spanning om te zetten in een elektrisch veld, wanneer de spanning vervolgens oploopt of afneemt, zal het elektrische veld een tegenbeweging uitvoeren. Het garanderen van de spanning is van cruciaal belang in een groot digitaal circuit. Zo hebben computers meestal enkele condensatoren die dicht bij de voeding het spanningsverschil moeten waarborgen. Immers is het mogelijk dat door het aan- en uitschakelen van andere apparaten, de spanning licht fluctueert. In analoge circuits is dit meestal niet problematisch. In een digitaal circuit kan dit echter het verschil maken tussen een 0 of een 1. Omdat in sequenti\"ele circuits een fout ge\"interpreteerd signaal tot een opeenvolging van foute invoer en foute toestanden kan leiden is een foute interpretatie dus zeer gevaarlijk.
\paragraph{}
Een condensator wordt ook vaak gebruikt in een RC-keten: een keten waarbij een weerstand een condensator in serie worden geplaatst. Wanneer er dan spanning wordt aangelegd op het circuit, zal de spanning op de condensator traag evolueren. Dit kan men dus gebruiken om signalen met enige vertraging te laten propageren. Dit is belangrijk in analoge circuits alsook in digitale, bijvoorbeeld om een kloksignaal te generen (zie \secref{clocksignal}).
\paragraph{}
Een condensator wordt als volgt gerealiseerd. Men voorziet doorgaans twee of meer platen die dicht bij elkaar liggen, maar niet verbonden zijn zoals weergegeven op \figref{capacitor}.

\importtikzfigure{capacitor}{De realisatie van een condensator.}%TODO

Wanneer er spanning wordt aangelegd op de condensator, worden de platen elektrisch geladen. Als gevolg ontstaat er een elektrisch veld dat gelijk is aan:
\begin{equation}
E=\dfrac{V}{d}
\end{equation}
De meeste condensatoren worden niet gerealiseerd aan de hand van twee parallelle platen, maar aan de hand van bijvoorbeeld een opgerold structuur waarbij de twee platen gescheiden worden door een di\"elektrische stof.
\subsection{Spoelen}
Een spoel tracht de stroomsterkte te behouden, dit doet men door stroomsterkte om te zetten in magnetisch veld.

\paragraph{}
Een spoel wordt doorgaans weinig gebruikt bij een digitaal circuit. Men kan een spoel in een RL-keten gebruiken om een vertraging te realiseren op de stroomsterkte, bovendien kan men een LC-keten realiseren om een oscillerende keten te bouwen.

\paragraph{}
In de analoge elektronica wordt een spoel typische gebruikt bij het filteren van bepaalde frequenties uit een signaal, bijvoorbeeld uit een gecapteerd radiosignaal zodat enkel de signalen van een bepaald radiostation de luidspreker aanleggen.

\paragraph{}
Men realiseert een spoel door door de stroomsterkte die doorheen een geleider vloeit om te zetten in een magnetisch veld. Men bereikt een maximaal effect wanneer men deze geleider op een spiraal-vormige manier buigt zoals weergegeven op \figref{inductor}.

\importtikzfigure{inductor}{De realisatie van een spoel.}%TODO

\subsection{Transistoren}
We hebben transistoren al in \chpref{} besproken, maar dan vooral in een digitale context. Transistoren worden echter ook in een analoge context gebruikt: bijvoorbeeld als versterker waarbij de weerstand tussen de collector en de emittor wordt bepaald door de spanning die op de basis wordt aangelegd. Soms ook als basis om een feedback-circuit te bouwen: als men bijvoorbeeld een RC-keten laat op- en ontladen op basis van een transistor, en deze transistor vervolgens ook aangestuurd wordt door deze RC-keten, kan men wisselstroom realiseren of bijvoorbeeld een kloksignaal.
\subsection{Diodes en Operationele versterkers}
\emph{Diodes} en \emph{operationele versterkers} zijn componenten die toelaten om bij een bepaalde drempelwaarden, stroom door te laten. In het ideale geval betekent dit dat een diode wanneer de spanning correct gepolariseerd is, een weerstand van $0~\Omega$ voorstelt. In het andere geval is de weerstand $\infty~\Omega$.
\paragraph{}
\section{Bouw van een elektronische schakeling}
Om de verschillende componenten die we hier besproken hebben samen te schakelen, dienen we geleiders te voorzien. We kunnen dit natuurlijk realiseren door geleiders zoals koperdraad te voorzien en dit te solderen aan de connectoren van de verschillende componenten. Dit zou echter tot een chaotische realisatie leiden die weinig compact is. Men maakt dan ook meestal gebruik van bijvoorbeeld een \emph{matrixbord}, \emph{europrintplaat} of \emph{printplaat} die de geleiders structureren: meestal wordt \'e\'en zijde van een plaat voorzien voor de componenten en de andere zijde voor de geleiders. In het geval van een printplaat wordt soms de zijde van de componenten ook gebruikt om geleiders op te zetten. Indien geleiders enkel aan \'e\'en kan van de plaat automatisch worden aangebracht, zal men soms ook manueel nog extra geleiders moeten aanbrengen, meestal met behulp van koperdraad. Wanneer men aan de twee kanten van de plaat geleiders toelaat, kan men in feite elke schakeling realiseren zonder extra geleiders te moeten aanbrengen.
\subsection{Matrixbord}
Een \emph{matrixbord} ook wel \emph{breadboard} genoemd is een paneel waar de geleiders reeds op voorhand op een gestructureerde wijze zijn aangebracht. De componentenkant van een matrixbord bestaat uit vooraf gefabriceerde gaten, waar componenten eenvoudigweg ingeplugd kunnen worden. \figref{breadboard} beschrijft een typisch matrixbord.
\importtikzfigure{breadboard}{Een voorbeeld van een matrixbord.}
\paragraph{}
Op de figuur zien we centraal twee matrices van $57\times 5$ pinnen. Deze pinnen zijn horizontaal verbonden. Aan beide kanten zijn ook lange lijnen. Dit zijn $25\times 2$ matrices die verticaal verbonden zijn. Bij grote breadboards zijn er (zoals ook de figuur) meerdere onderbroken lange lijnen. Ook bestaan er breadboards met meer centrale matrices. Bovenaan ten slotte voorziet men verbindingen voor bijvoorbeeld stroombronnen. De meeste lange lijnen worden verbonden met de polen van de spanningsbron. Tussen twee centrale matrices loopt een opening van $0.2~\mbox{inch}$. Dit laat toe om typische chips in DIP verpakking op het breadboard te plaatsen. Immers kan men dit niet op een centrale matrix plaatsen: anders zouden de pootjes met elkaar verbonden worden.
\paragraph{}
Een matrixbord is erg geschikt om snel een prototype mee uit te testen. Na gebruik kan men immers de componenten terug uitpluggen. Een matrixbord is echter niet geschikt om compacte en blijvende realisaties mee te bouwen. Immers men de componenten zo ori\"enteren dat de juiste connectoren met elkaar verbonden worden. Dit leidt er toe dat bijvoorbeeld DIP-componenten altijd onder elkaar moeten worden geplaatst en bovendien in het midden van een kolom. Doorgaans kan men door eerst de componenten te plaatsen en dan de geleiders te minimaliseren tot een compacter ontwerp komen dan omgekeerd. Een matrixbord biedt echter comfort aan bij het schakelen van elementen en is daarom nuttig voor een eerste realisatie.
\subsection{Europrintplaat}
Een europrintplaat is een stap tussen een matrixbord en speciaal ontwikkelde printplaat. In een europrintplaat beschouwen we een rechthoekige bord van kunststof. Op deze plaat brengt men in een rasterstructuur rechthoekige plaatjes van koper aan die men \emph{eilanden} noemt. Zo'n plaatje bevat drie gaten waarop men de verbindingen van de componenten kan solderen. Hierdoor zijn de connectoren die worden gesoldeerd aan eenzelfde eiland automatisch verbonden. Deze structuur is ge\"illustreerd op \figref{europrint}.
\importtikzfigure{europrint}{Structuur van een europrintplaat.}
\paragraph{}
Doordat de eilanden relatief klein zijn, levert dit vrij compacte schakelingen op. Immers verliest men bij een matrixbord meteen een volledige rij wanneer men op een bepaalde plaats een component inplugt. Bij een europrintplaat zijn dit slechts drie verbindingen. De componenten dienen op het bord gesoldeerd te worden, wat dus tot een permanente realisatie leidt. Het realiseren van een schakeling is echter arbeidsintensief en bovendien is het moeilijk om een component terug te verwijderen. In \sscref{solderinghints} bespreken we hoe men componenten op een (euro)printplaat solderen.
\subsection{Printed Circuit Board (PCB)}
Wanneer men een schakeling heeft ontwikkeld die men wil realiseren in een klein formaat gebruikt men meestal een ``Printed Circuit Board (PCB)'' of ``printplaat''. Een printplaat is een op maat gemaakte plaat waarbij de verbindingen gerealiseerd worden door een koperen laag aan de ene kant van deze plaat. De componenten worden op deze plaat bevestigt door openingen in de plaat. De componenten worden aan de andere kant bevestigd waarbij de verbindingen door middel van kleine openingen en de connectoren worden op de koperen kant gesoldeerd. Moderne printplaten laten ook toe om verbindingen aan de twee kanten van de plaat te plaatsen.
\importtikzfigure{pcbexample}{Een voorbeeld van een printed circuit board (pcb).}
\paragraph{}
Men kan printplaten op twee manieren bekomen: ze bestellen bij een bedrijf die hierin gespecialiseerd is of ze zelf maken. Bij beide dient men eerst de printplaat te ontwerpen. Op maat gemaakte printplaten zijn relatief goedkoop: ongeveer $\mbox{\euro}\ 10.00$ voor een prototype en $\mbox{\euro}\ 00.25$ per stuk bij massaproductie.
\subsubsection{Ontwerp}
Het ontwerp van een printplaat is een proces waarin men bepaald hoe de koper-laag er zal uitzien samen met de locaties van de openingen.
\subsubsection{Productieproces}
In deze subsubsectie zullen we het productieproces om zelf printplaten te maken bondig beschrijven. Meer informatie over dit proces is te vinden in \cite{fabricatingprintedcircuitboards}.
\paragraph{}
Het proces bestaat grofweg uit vier fases: ``blootstelling'', ``etsen'', ``tin betegeling'' en ``boren''\cite[p. 69]{fabricatingprintedcircuitboards}. In de volgende paragrafen zullen we de fases bespreken.
\paragraph{Blootstelling}
\paragraph{Etsen}
\paragraph{Tin betegeling}
\paragraph{Boren}
\subsection{Plaatsen van componenten}
\subsection{Tips bij het solderen}
\ssclab{soleringhints}
\section{Implementatie van een kloksignaal}
\seclab{clocksignal}
In de cursus hebben we telkens abstractie gemaakt van de implementatie van een kloksignaal: een signaal die met een bepaalde frequentie afwisselend $0$ en $1$ aanlegt. Hoe kunnen we een dergelijk signaal implementeren? Een eerste antwoord zou kunnen zijn dat men een sequenti\"ele schakeling bouwt die dit signaal zal aanleggen. Het probleem is dat een dergelijke schakeling zelf een kloksignaal nodig heeft. We zullen dus een ander component nodig hebben. Twee populaire keuzes zijn het gebruik van een $555$-timer en een kristal-oscillator. We zullen beide technieken kort bespreken.
\subsection{555-timer}
De $555$-timer\footnote{Soms uitgesproken als ``triple five'' of ``triple five timer''.} is een ge\"integreerd circuit die gebruikt wordt om allerhande periodieke functies te implementeren. Het component kost los rond de $\$ 2.50$. Een 555-timer kan kloksignalen produceren tot ongeveer $300\mbox{ kHz}$.
\subsubsection{Pinout en interface}
Een $555$-timer is een component met $8$ verbindingen. \figref{pinout-555} geeft de pinout van de chip met DIP packing weer.
\begin{figure}[hbt]
\centering
\importtikzsubfigure{pinout-555}{555-timer pinout}
\importtikzsubfigure{interface-555}{555-interface}
\importtikzsubfigure{pinout-556}{556-timer pinout}
\caption{De pinout van de 555-timer en 556-timer.}
\end{figure}
In een blokdiagram tekent men een $555$-timer meestal met vaste posities voor de verschillende in- en uitgangen. \figref{interface-555} toont deze posities samen met de nummers van de pinout.
\subsubsection{Werking}
De werking van de $555$-timer is niet eenvoudig te verklaren. Dit komt omdat men een dergelijk timer implementeert met complexe analoge elektronica zoals operationele versterkers. Zonder in detail te treden zullen we verklaren hoe we een $555$-timer als kloksignaal kunnen gebruiken. De ground pin en de power pin leggen een spanning aan op het component zodat het kan functioneren. Daarnaast bevat de $555$-timer een interne flipflop. De toestand van deze flipflop ($0$ of $1$) wordt aangelegd op de output pin. Twee ingangen dienen vervolgens om het signaal om te wisselen: de \mbox{trigger} en de \mbox{threshold}. De \mbox{trigger} is een ingang die de flipflop op $1$ zet op het moment dat de spanning hoger wordt dan $1/3$  van de spanning die op de \mbox{Vcc} staat. De \mbox{threshold} werkt omgekeerd: de ingang is gevoelig voor spanning boven $2/3$ van de spanning op de \mbox{Vcc}. Indien op dat moment de trigger een spanning heeft groter dan $1/3$, wordt de flipflop terug op $0$ gezet. De overige pinnen zijn van minder belang. De controle pin kan de waarde van de threshold aanpassen. Wanneer we dus een vaste kloksignaal willen genereren is het belangrijk dat de control pin ten alle tijde op een vaste spanning staat. Daarom wordt de controle pin meestal via een condensator verbonden met de negatieve pool van de stroombron. Vermits een condensator behoudt van spanning nastreeft, is dit dus de beste garantie op een constante spanning op de controle-ingang. De \mbox{discharge} ingang is een verbinding die stroom naar de ground laat vloeien op het moment dat de \mbox{output} hoog is. Wanneer de \mbox{output} laag is, is de \mbox{discharge} hoog impedant. De \mbox{reset} pin ten slotte zal de output terug op $0$ zetten wanneer men een lage spanning\footnote{Beneden een spanning van $0.8\mbox{ V}$ tegenover de ground.} aanlegt. Meestal is de \mbox{reset}-ingang aangesloten op de positieve pool van de stroombron om dit te vermijden.
\paragraph{}
\importtikzfigure{clock-555}{Schakeling voor de implementatie van een kloksignaal met een $555$-timer}
Op basis van de beschrijving kunnen we een eenvoudige schakeling met een RC-keten ontwerpen die een kloksignaal zal genereren. Deze schakeling staat beschreven in \figref{clock-555}. Op het moment dat men een spanning op de schakeling aanlegt, is de condensator $C_1$ niet opgeladen en het \mbox{output}-signaal van de $555$-timer is hoog. Bijgevolg wordt de condensator opgeladen door de RC-keten. Op het moment dat de spanning van de condensator (de spanning tussen pin 1 en pin 6) $2/3$ van de totale spanning bereikt, wordt de \mbox{output} op laag gezet. Vanaf dat moment begint de condensator te ontladen. De spanning over de condensator neemt bijgevolg af en op het moment dat spanning onder $1/3$ van de totale spanning gaat, detecteert de \mbox{trigger} dit. Op dat moment wordt de \mbox{output} terug hoog en begint de cyclus opnieuw. Omdat het opladen en ontladen van tussen $1/3$ en $2/3$ van de spanning in de RC-keten even lang duurt, is de duty cycle bijgevolg 50\%. \figref{clock-555} toont ook de spanning van de \mbox{output} en de condensator op verschillende momenten in de tijd. De stippellijnen op de grafiek van de condensator tonen het verdere verloop van de functie indien de \mbox{output}-niet zou worden omgedraaid.
\paragraph{}
De frequentie van het kloksignaal hangt duidelijk af van de parameters van de RC-keten. Een condensator wordt op- en ontladen volgens volgende functies:
\begin{eqnarray}
\fun{U_{\mbox{op}}}{t}&=&U_f-\brak{U_f-U_0}\cdot e^{-t/R_1\cdot C_1}\\
\fun{U_{\mbox{ont}}}{t}&=&U_0\cdot e^{-t/R_1\cdot C_1}
\end{eqnarray}
Met $U_0$ de beginspanning en $U_f$ de spanning die op de keten wordt aangelegd. In een halve klokcyclus wordt de spanning dus opgeladen van $1/3\ V$ naar $2/3\ V$ of ontladen van $2/3\ V$ naar $1/3\ V$. De tijd die hierbij verstrijkt is dus gelijk aan:
\begin{equation}
\Delta t=-R_1\cdot C_1\cdot\fun{\ln}{1/2}=\fun{\ln}{2}\cdot R_1\cdot C_1\approx 0.693147181 R_1\cdot C_1
\end{equation}
De frequentie van het kloksignaal is bijgevolg gelijk aan:
\begin{equation}
f=\displaystyle\frac{1}{2\Delta t}=\displaystyle\frac{1}{2\cdot\fun{\ln}{2}\cdot R_1\cdot C_1}=\displaystyle\frac{1}{\fun{\ln}{4}\cdot R_1\cdot C_1}\approx\displaystyle\frac{0.72134752}{R_1\cdot C_1}
\end{equation}
Meer informatie over de implementatie, de werking en concrete oscillator-schakelingen is te vinden in \cite{ne555}.
\subsubsection{556-timer}
Tot slot introduceren we ook nog een andere populaire ge\"integreerde schakeling: de $556$-timer. Deze ge\"integreerde schakeling is eigenlijk niets anders dan twee $555$-timers in \'e\'en chip. Het voordeel van dergelijke chips is dat bepaalde in- en uitgangen gedeeld kunnen worden. In het geval van de $556$-timer is dat het geval voor de power pin en ground pin.
\subsection{Astabiele multivibrator}
Een andere manier om een kloksignaal te implementeren is door een oscillator te implementeren. De meeste oscillatoren werken op basis van het volgende principe: een we beschouwen een transistor die een zekere weerstand tussen de collector en emitor aanbrengt. De weerstand wordt bepaald door de spanning tussen de basis en de emitor. Door een schakeling te implementeren die de weerstand negatief terugkoppelt naar de basis en hierover een zekere tijd laat verstrijken kunnen we een oscillator bouwen. Een dergelijke feedback schakeling moet dus in het geval van een PNP-transistor bij een lage weerstand de spanning op de basis verlagen en bij een hoge weerstand de spanning aan de basis opdrijven.
\paragraph{}
\importtikzfigure{twopnp}{Astabiele multivibrator.}
Een typische manier om dit te realiseren is een symmetrische schakeling met twee PNP-transistoren en twee RC-ketens zoals ge\"illustreerd op \figref{twopnp}. Deze schakeling werkt als volgt. We beschouwen een toestand waarbij de eerste transistor gesloten is\footnote{Een transistor is gesloten wanneer er stroom vloeit van de collector naar de emitor. In het ander geval is de transistor open.} en de andere transistor gesloten.
\subsection{Kristal-oscillator}
Een nadeel van de $555$-timer is dat de periodieke functie meestal niet nauwkeurig wordt aangelegd. Componenten op basis van transistoren en weerstanden zijn bijvoorbeeld onderhevig aan de temperatuur. Indien men dus een kloksignaal met een zekere frequentie aanlegt kan men fluctuaties op die frequentie verwachten. Zolang de frequentie laag is, levert dit weinig problemen op: de componenten rekenen immers zo snel dat de data lang op de ingangen van de registers staat alvorens ze worden ingeladen. Wanneer men echter een processor implementeert, wil men een hoge kloksnelheid die zo weinig mogelijk tijd de data onbenut laat. Ook bij hoge frequenties blijft er echter sprake van ruis. In dat geval kan de ruis het verschil maken tussen de correcte data die aan de ingang van een register staat, of oude of tijdelijke data.
\paragraph{}
In dergelijke gevallen zal men opteren voor een kristal-oscillator. Een kristal-oscillator werk op basis van pi\"ezo-elektromagnetisme. De fysica achter dit proces valt buiten het bereik van deze cursus. Men kan echter stellen dat het een component is die op basis van een kwartskristal met een zeer vaste frequentie van weerstand varieert. De afwijkingen worden dan ook uitgedrukt in ``parts per billion (ppb)''.
\section{Printed Circuit Board (PCB) Layout}
In vakbladen zal men meestal naast de schakeling ook een ``printed circuit board (pcb) layout'' weergeven. Dit is een schematische weergave hoe men de schakeling compact kan realiseren op een printplaat. Een dergelijke afbeelding dient niet om de schakeling te analyseren, maar enkel om de schakeling zelf op een effici\"ente manier te realiseren. Een probleem met dergelijke plannen is dat men andere symbolen gebruikt om de componenten voor te stellen: meestal wordt de basisvorm van het relevante component weergegeven, bovendien dient men ook twee lagen weer te geven: de boven- en onderkant van de printplaat.
\paragraph{}
De meeste afbeeldingen lossen het probleem van de twee lagen op met behulp van kleur: in deze cursus zullen we de voorzijde afbeelden in het zwart en de achterzijde in het grijs. Het probleem met het toewijzen van componenten wordt meestal aangepakt door de componenten te labelen. Meestal wordt hierbij het type component (weerstand, condensator, ...) weergegeven, maar de meest relevante eenheid van het component. Zo zal bij een condensator de capaciteit in micro-Farad worden weergegeven.
\paragraph{}
Bij wijze van introductie toont \figref{pcb-components} een overzicht van de symbolische weergaven van enkele populaire componenten.
\begin{figure}[hbt]
\centering
\importtikzsubfigure{pcb-rcy}{Condensator.}
\importtikzsubfigure{pcb-alf}{Diode.}
\importtikzsubfigure{pcb-dip}{DIP-chip.}
\importtikzsubfigure{pcb-wire}{Draad.}
\importtikzsubfigure{pcb-hc49}{Kristal-oscillator.}
\importtikzsubfigure{pcb-led}{LED.}
\importtikzsubfigure{pcb-tact}{Schakelaar.}
\importtikzsubfigure{pcb-to92}{Transistor.}
\importtikzsubfigure{pcb-usb}{USB-verbinding.}
\importtikzsubfigure{pcb-acy}{Weerstand.}
\caption{PCB-weergave van populaire componenten.}
\figlab{pcb-components}
\end{figure}
\paragraph{}
Een PCB-layout kan ook digitaal worden ingevoerd. Bijvoorbeeld met \emph{gEDA} of \emph{Gerber}. Bovendien kan deze software op basis van de gegevens een lijst met specificaties maken waar gaten in de printplaat moeten worden geboord, waar de geleiders moeten worden aangebracht en welke componenten worden gebruikt en waar deze moeten worden aangebracht. Dit proces kan bovendien worden geautomatiseerd door relevante apparatuur aan te sturen. Dit ligt buiten het bereik van deze cursus.
\section{Oproep aan de lezers}
De auteur roept enthousiaste lezers op om projecten te delen zodat deze in deze cursus kunnen worden gepubliceerd als een hoofdstuk in dit deel. Men kan echter niet elk project als nuttig beschouwen. Ingediende projecten moeten aan enkele voorwaarden voldoen:
\begin{enumerate}
 \item De componenten in het project dienen in de cursus vermeld te worden. Het is niet de bedoeling om ``exotische componenten'' te introduceren, in het bijzonder denken we dan aan componenten uit de analoge elektronica (operationele versterker, spoel, ...). Sommige projecten kunnen een klein aantal van dit soort componenten bevatten. In dat geval dient men een korte beschrijving van de werking bij te voegen.
 \item Het project moet realiseerbaar zijn. Zowel op een op maat gemaakte printplaat als bijvoorbeeld een europrintplaat. Verder is het evenmin de bedoeling dat het project veel werk vereist en het resultaat weinig inzichten zal verwerken (hierbij denken we bijvoorbeeld aan een 1024-bit opteller).
 \item De effecten die in het project beschreven worden moeten te verklaren zijn, en dit op basis van de cursus.
\end{enumerate}
Indien het project aan deze voorwaarden voldoet maakt het kans om opgenomen te worden. Een lezer kan een project indienen op volgend adres: \texttt{http://goo.gl/rzIlr3}. Een ``aanvraag'' bestaat uit \'e\'en of meerdere schema's samen met een verslag. Dit verslag bevat een lijst van benodigde componenten, aanwijzingen bij de bouw van de schakeling en een tekst die de werking verklaart. Het verslag mag figuren bevatten die de werking verder uitleggen. Omdat een dergelijke aanvraag veel werk vraagt, kan men ook een ``voor-aanvraag'' indienen (op hetzelfde webadres). In een voor-aanvraag specificeert men kort het project in een tekst van maximaal \'e\'en pagina. Op basis van de reactie van de auteur kan men dan beslissen om al dan niet een aanvraag in te dienen.
\chapter{Experimenten}
\chplab{experiments}
\chapterquote{Het leven is niets dan een experiment. Hoe meer je experimenteert, hoe beter.}{Ralph Waldo Emerson, Amerikaans dichter en filosoof (1803-1882)}
\begin{chapterintro}
In dit hoofdstuk beschrijven we enkele experimenten die we kunnen uitvoeren om praktische en nuttige digitale schakelingen te realiseren voor dagelijks gebruik.
\end{chapterintro}
\minitoc[n]
\section{Oneindig vermenigvuldigen}
\'E\'en van de voordeling van delen en vermenigvuldigen met constanten, is dat de overdracht of het lenen ook beperkt is. Wanneer we een binair getal bijvoorbeeld vermenigvuldigen met $3$, kan de carry nooit groter worden dan $4$. Wanneer we dit veralgemenen naar een vermenigvuldiging in het binair stelsel met $n$, kan de overdracht nooit groter worden dan $2\cdot n-1$. We kunnen dit uitbuiten om een sequenti\"ele vermenigvuldiger te maken die over bit-stromen met oneindige lengte werkt. We realiseren dus een schakeling waar als invoer een stroom aan bits en het bijbehorende kloksignaal binnenkomt, en als uitvoer een andere stroom bits geklokt volgens hetzelfde kloksignaal maar waarbij de bits het $k$-voud voorstellen van de inkomende bitstroom. We werken volgen big-endian encodering: de eerste bit die in de schakeling binnenkomt is altijd de minst significante.
\subsection{Benodigdheden}
Dit hangt grotendeels af van de constante $k$ waarmee we willen vermenigvuldigen. Voor een constante $k=5$ zijn de benodigdheden de volgende:
\begin{enumerate}
 \item $2$ leds;
 \item $3$ npn-transistoren;
 \item $1$ pnp-transistoren;
 \item $2$ drukschakelaars.
\end{enumerate}
Samen kosten deze componenten (zonder gereedschap, printplaten en soldeersel) ongeveer $\mbox{\euro}\ ??.??$\footnote{Gebaseerd op prijzen bij Conrad.}.
\subsection{Model}
In de inleiding hebben we de werking van de component niet gespecificeerd. Dit zullen we in deze sectie doen. Wanneer een getal bit-na-bit wordt ingelezen in het circuit, dient het circuit dit getal de vermenigvuldigen. We kunnen echter de eigenschap uitbuiten dat de overdracht die bij de vermenigvuldiging optreedt, altijd eindig zal zijn. Bij voorbeeld zullen we een vermenigvuldiging met $k=5$ modelleren.
\paragraph{}
\importtikzfigure{infinitemul-carry5}{Een toestandsdiagram voor $k=5$.}
De machine begint in een toestand waar er geen overdracht is, formeler $c=0$. Dit is de toestand $q_0$ op \figref{infinitemul-carry5}. Wanneer er nu een $0$ binnenkomt, is het resultaat logischerwijs $0$ en blijft de overdracht ook op $0$ staan. We trekken dus een lus vanuit $q_0$ met boog $0/0$. Wanneer er echter een $1$ op de invoer wordt aangelegd, dan weten we dat de uitvoer-bit ook $1$ moet worden, en dat we vanaf dat moment met overdracht $c=4$ moeten rekening houden. Nu hebben we alle invoer voor $q_0$ gemodelleerd, maar $q_4$ dienen we verder in te vullen. In het algemeen geldt de regel voor een vermenigvuldiging van $k$ dat we een boog van $q_i$ naar $q_j$ trekken met label $b_i/b_o$ zodat:
\begin{eqnarray}
b_i\in\accl{0,1}\\
r=\displaystyle\frac{i}{2}+k\cdot b_i\\
b_o=r\mod 2\\
j=r-b_o
\end{eqnarray}
Voor $q_4$ betekent dit dus dat we een boog $0/0$ naar $q_2$ trekken en een boog $1/1$ naar $q_6$. We dienen vervolgens ook bogen uit deze toestanden te trekken en dit tot er geen nieuwe toestanden meer opduiken. We bekomen vervolgens een diagram zoals op \figref{infinitemul-carry5}. We kunnen dit voor iedere $k$ doen.
\subsubsection{Implementatie-specificaties}
\paragraph{Reset}
Getallen hebben altijd een eindig aantal cijfers en bijgevolg ook een eindig aantal bits. We willen de schakeling dan ook niet noodzakelijk gebruiken om een product uit te rekenen over een oneindige sequentie bits. Maar wel op een snelle manier getallen van willekeurige lengte vermenigvuldigen. We dienen echter aan het circuit duidelijk te maken dat we een nieuw getal willen vermenigvuldigen. We kunnen dit op twee manieren doen:
\begin{enumerate}
 \item We lezen de resterende bits uit door telkens $0$ op de ingang aan te leggen tot het circuit zich terug in toestand $q_0$ bevindt, dit duurt hoogstens $\fun{\log_2}{n}+1$ stappen. Dit is ook in principe de correcte methode omdat het resultaat van een vermenigvuldiging nu eenmaal een getal met meer bits kan opleveren. Een nadeel is echter wanneer we dit circuit in een processor zouden gebruiken, waarbij de registers een beperkt aantal bits hebben, we meestal ge\"interesseerd zijn in de $n$ minst significante bits\footnote{Analoog met de overloop bij bijvoorbeeld een optelling.}, en we dus klokflanken ``verspillen''.
 \item We voorzien een reset-ingang die de schakeling terug in de grondtoestand brengt. De meest significante bits gaan verloren.
\end{enumerate}
In dit voorbeeld kiezen we voor de eerste variant. Dit houdt immers ook het circuit eenvoudig en goedkoop.
\paragraph{In- en Uitvoer}
Normaal kunnen we een dergelijk circuit gebruiken in bijvoorbeeld een processor, of kunnen we het in een ethernet-kaart implementeren wanneer data moet worden aangepast. Wanneer we echter de schakeling op zichzelf willen implementeren, dienen we deze te kunnen testen. Bijgevolg zullen we ook invoer-uitvoer voorzien. Communicatie tussen mens en modules wordt meestal gerealiseerd met behulp van leds en schakelaars. Om duidelijk te maken welk signaal we zelf zullen aanleggen zullen we ook leds aan de ingang gebruiken. Een probleem die zich soms stelt is dat er een groot aantal bits aan de invoer kunnen worden aangelegd, en het dus niet eenvoudig is om met drukschakelaars een bepaalde configuratie aan te leggen. In dat geval kunnen we gebruik maken van een tuimelschakelaar. Een tuimelschakelaar kan worden aangekocht, maar ook ge\"implementeerd met behulp van een drukschakelaar, twee weerstanden, een pnp- en npn-transistor. Een dergelijke implementatie staat beschreven in \figref{toggleswitch-transistor}.
\importtikzfigure{toggleswitch-transistor}{Implementatie van een tuimelschakelaar.}
\subsection{Realisatie}
\subsubsection{Toestanden en soorten schakeling}
In de vorige subsectie hebben we besproken hoe we een diagram kunnen opstellen voor de overdracht. Dit diagram is precies het diagram voor de Mealy-machine om onze schakeling te realiseren. We kunnen het diagram dus rechtstreeks gebruiken om een toestandstabel op te stellen zoals in \tblref{infinitemul-state5}.
\importtabulartable{infinitemul-state5}{Toestandstabel van de oneindige vermenigvuldiger.}
\subsubsection{Toestandscodering}
We dienen nu enkel nog een codering te bedenken om de verschillende toestanden op te slaan. Een logische manier kan zijn om $q_i$ op te slaan als het binaire equivalent van $i/2$. Dit biedt bovendien in dit geval vrij mooie overgangen: bij twee overgangen verandert er geen bit, bij vijf overgangen slechts \'e\'en bit, \'e\'en overgang brengt twee veranderingen teweeg en \'e\'en overgang drie. We kunnen echter $q_8$ ook encoderen als $111$. Dit zorgt ervoor dat er geen overgangen met drie bits meer zijn. Dit is natuurlijk heuristisch: er is geen echte garantie dat dit effectief tot een goedkopere implementatie zal leiden, al wordt het wel verondersteld. De coderingstabel wordt dan zoals voorgesteld als in \tblref{infinitemul-code5}
\importtabulartable{infinitemul-code5}{Coderingstabel van de oneindige vermenigvuldiger.}
\subsubsection{Type flipflop}
Het is niet echt duidelijk welk type flipflop hier nodig is. We zullen de schakeling met een D-flipflop implementeren.
\importtabulartable{infinitemul-combine5}{De oneindige vermenigvuldiger.}
Wanneer we D-flipflops gebruiken bekomen dienen we combinatorische schakelingen te implementeren zoals op \tblref{ininitemul-combine5} een grafische voorstelling met behulp van Karnaugh-kaarten staat op \figref{infinitemul-karnaugh5}.
\importtikzfigure{infinitemul-karnaugh5}{Karnaugh-kaarten voor de oneindige vermenigvuldiger.}
\subsubsection{Implementatie}
Tot slot dienen we enkel nog de schakeling te implementeren. Het logische circuit staat om dit te realiseren staat op \figref{infinitemul-impl5}.
\importtikzfigure{infinitemul-impl5}{Mogelijke implementatie voor de oneindige vermenigvuldiger.}
In de praktijk worden poorten meestal niet per eenheid verkocht: het is goedkoper om vier poorten op \'e\'en chip te zetten dan vier afzonderlijke chips te produceren. Bijgevolg dienen we nog een keuze te maken hoe we deze schakeling realiseren.
\paragraph{}
Wanneer we naar de schakeling op \figref{infinitemul-impl5} kijken, kunnen we de verschillende types poorten tellen:
\begin{enumerate}
 \item $3\times$ NOT poort;%
 \item $2\times$ 2-NOR poorten;
 \item $1\times$ 3-NOR poort;
 \item $4\times$ 2-NAND poorten;%
 \item $2\times$ 3-NAND poorten; en%
 \item $3\times$ D-flipflops.%
\end{enumerate}
Op chipniveau zullen we daarom werken met volgende poorten:
\begin{enumerate}
 \item $1\times$ 7400 ($4\times$ 2-NAND poorten), DIP14, ongeveer $\mbox{\euro}\ 00.14$ per stuk;
 \item $1\times$ 7404 ($6\times$ NOT-poorten), DIP14, ongeveer $\mbox{\euro}\ 00.17$ per stuk;
 \item $1\times$ 7410 ($3\times$ 3-NAND poorten), DIP14, ongeveer $\mbox{\euro}\ 00.32$ per stuk;
 \item $1\times$ 7427 ($3\times$ 3-NOR poorten), DIP14, ongeveer $\mbox{\euro}\ 00.32$ per stuk; en
 \item $2\times$ 7474 ($2\times$ D-flipflops), DIP14, ongeveer $\mbox{\euro}\ 00.14$ per stuk;
\end{enumerate}
Hieruit kunnen we dan een printplaat samenstellen zoals op \figref{infinitemul-pcb5}.
\importtikzfigure{infinitemul-pcb5}{Printplaat-ontwerp voor de oneindige vermenigvuldiger.}
\section{Fietslicht}
Men kan vandaag fietslichten kopen met verschillende knippermodes. Bij wijze van experiment zullen we een knipperlicht implementeren gebaseerd op een ledlamp van \emph{Hema}.
\paragraph{}
Het fietslicht bestaat uit drie leds en kent 4 standen: uit, links-rechts, knipperen en aan. \figref{bicycle} beschrijft de verschillende modi. Men verandert de modus door een drukschakelaar in te duwen. In het geval dat men in knipper-modus staat, is de frequentie waarmee men van toestand van led verandert, slechts de helft.
\importtikzfigure{bicycle}{De verschillende standen van een \emph{Hema}-fietslicht}
\subsection{Benodigdheden}
Volgende componenten heeft men nodig om deze schakeling te implementeren:
\begin{itemize}
 \item $3$ leds
 \item $3$ NPN transistoren. Bij voorkeur \verb+BC547C+
 \item $3$ weerstanden van $1~\mbox{k}\Omega$.
 \item $1$ drukschakelaar.
 \item $1$ europrintplaat.
\end{itemize}
Alle componenten zijn verkrijgbaar bij Conrad. De richtprijs is ongeveer $\mbox{\euro}\ ??.??$.
\subsection{Implementatie-specificaties}
In deze subsectie bespreken we hoe we de schakeling zullen implementeren. Dit betekent dat we onder meer de in- en uitvoer vastleggen.
\paragraph{In- en Uitvoer}
Naast de drie leds van het fietslicht is er geen uitvoer: we gaan er immers vanuit dat een gebruiker door op de drukschakelaar te duwen de werking van het licht zelf kan leren. Als invoer voorzien we \'e\'en enkele drukschakelaar. We realiseren de in- en uitvoer aan de hand van de schakeling op \figref{bicycle-io}.
\importtikzfigure{bicycle-io}{De IO-module van het fietslicht}
\paragraph{Toestanden en soort schakeling}
We dienen ook te beslissen wat soort schakeling we zullen implementeren. Vermits het aantal toestanden en de mogelijkheden nogal beperkt is, is het implementeren van een processor niet nodig: we kunnen deze schakeling met een eenvoudige sequenti\"ele schakeling realiseren.
\paragraph{}
Verder dienen we te beslissen of we een synchrone of asynchrone schakeling zullen bouwen. Er speelt natuurlijk een tijdsaspect: we willen niet dat we meteen in de volgende toestand geraken wanneer een led aan of uitgaat. Anders is het verschil tussen knipperen en aan moeilijk te zien. We kunnen echter ook het kloksignaal als een ingangsignaal beschouwen. Dit laatste pleit voor een asynchrone schakeling: er is sprake van twee soorten ingangen: de schakelaar van het fietslicht die aangeeft dat we een volgende toestand willen kiezen, en het kloksignaal die wanneer het opkomt betekent dat we de bij links-rechts en knipperen leds aan of uit moeten zetten. Een asynchrone schakeling is echter moeilijker te implementeren. Wanneer we voor een synchrone schakeling opteren zal de gebruiker wanneer hij de schakelaar induwt moeten wachten tot het volgende kloksignaal voor de volgende stand van het fietslicht wordt geactiveerd. Het tijdverlies valt echter goed mee en is meestal niet levensbedreigend. In dit boek kiezen we daarom voor een synchrone schakeling.
\paragraph{}
Verder rest ons nog het aangeven van de verschillende toestand en wat we doen wanneer een gebruiker de schakelaar indrukt. In principe zijn hier drie mogelijkheden. Iedere keer wanneer de gebruiker de schakelaar indrukt, dan komen we in de begintoestand van de stand. We kunnen ervoor opteren terecht te komen in een toestand die bijvoorbeeld globaal bepaald is: wanneer er $k$ ticks geweest zijn komen we in de $k\mod n$-de toestand terecht, met $n$ het aantal toestanden van die stand. Een laatste optie is dat het niet uitmaakt, zolang we maar in een toestand terechtkomen die geldig is voor die stand. We opteren in deze cursus voor de eerste optie, dit is ook zo bij het echte fietslicht van \emph{Hema}.
\subsection{Realisatie}
\subsubsection{Toestandstabel}
Op basis van de specificaties zullen we een toestandstabel opstellen. De uitgangssignalen van de sequenti\"ele schakeling zijn drie bits: voor elke led betekent $1$ dat de overeenkomstige led zal branden, in het geval van $0$ brandt de led natuurlijk niet. Verder voorzien we \'e\'en ingangssignaal: dit signaal is $1$ wanneer de gebruiker de schakelaar indrukt. In het geval de schakelaar niet ingedrukt is, is het signaal $0$. Dit levert ons de toestandstabel op zoals \tblref{bicycle-state0}.
\importtabulartable{bicycle-state0}{Toestandstabel van het fietslicht.}
In de tabel hebben we de toestanden geannoteerd met de stand in subscript ($0$ uit, $1$ links-rechts, $2$ knipperen, $3$ aan). In stand $2$ moest de frequentie gehalveerd worden. Dit hebben we opgelost door vier toestanden te voorzien: twee wanneer alle leds uit zijn, en twee wanneer alle leds branden. Op die manier kost het twee klokflanken vooraleer de leds veranderen.
\subsubsection{Toestandscodering}
In totaal hebben we $18$ toestanden nodig om de schakeling te realiseren. Een toestand zal dus worden ge\"encodeerd op minstens $5$ bits. We kunnen er ook voor opteren om de toestand van de leds rechtstreeks in het geheugen te encoderen. In dat geval hebben we drie bits nodig om de toestand van de leds weer te geven en vier overige bits\footnote{Het komt immers $9$ keer voor dat alle leds uit zijn. Om een onderscheid te maken tussen de verschillende toestanden hebben we dus minstens $4$ extra bits nodig}. Dit zou neerkomen op zeven bits wat vrij duur is. We zullen de toestand voorstellen aan de hand van $5$ bits.
\paragraph{}
Nadat we het aantal bits bepaald heb dienen we nog voor elke toestand een codering te voorzien. Hier proberen we de transitie- en uitvoerlogica mee te vereenvoudigen. We zien dat de links-rechts staat de meeste toestanden gebruikt. We zullen dit dan ook encoderen op zo'n manier dat de eerste bit $1$ aanwijst en de overige bits werken volgens het principe van een $4$-bit gray-teller. De overige standen zullen allemaal met een $0$ beginnen. Ook hier proberen we enige logica in de codering te stoppen. Zo spreekt het voor zich dat we de uit-toestand code $00000_a$ toewijzen: we kunnen logica voorzien die wanneer bepaalde bits passief zijn, alle leds meteen niet branden. De eerste en tweede toestand van de knipper-stand geven we dan weer de coderingen $01000_n$ en $01001_n$. Alle leds moeten branden in de laatste twee toestanden van de knipper-stand en de enige toestand van de aan-stand. We coderen deze toestanden respectievelijk met $01101_p$, $01100_q$ en $01110_r$. De volledige coderingstabel staat beschreven in \tblref{bicycle-state1}.
\importtabulartable{bicycle-state1}{Coderingstabel van het fietslicht.}
\subsubsection{Type flipflop}
Na het opstellen van de coderingstabel dienen we het type flipflop uit te kiezen.
\subsubsection{Implementatie in poortlogica}

\section{Tic tac toe-machine}
\emph{Tic tac toe} is een spel waarbij twee spelers beurtelings een zet spelen. Tijdens een zet schrijft een speler een cirkel of een kruis op het scorebord, een $3\times 3$ raster. De eerste speler zet altijd een cirkel, de tweede een kruis. Elke speler probeert een situatie te bekomen waarbij drie van de eigen symbolen op een rij staan. Het spel is simpel en kan daarom goedkoop op zelf in electronica ge\"implementeerd worden. We zullen bij dit experiment twee schakelingen ontwerpen en realiseren: een schakeling zodat twee personen tegen elkaar spelen, en een schakeling waarbij het ook mogelijk is om te spelen tegen een artifici\"ele intelligentie\footnote{Deze AI-bot zal perfect spelen: als eerste speler zal hij altijd winnen}.
\subsection{Benodigdheden}
Volgende componenten heeft men nodig om deze schakeling te implementeren.
\begin{enumerate}
 \item $22$ leds waaronder $11$ rode en $11$ groene.
 \item $11$ NPN transistoren. Bij voorkeur \verb+BC547C+.
 \item $11$ weerstanden van $1~\mbox{k}\Omega$.
 \item $6$ drukschakelaars.
 \item $1$ normale schakelaar.
 \item $1$ europrintplaat.
\end{enumerate}
Alle componenten zijn verkrijgbaar bij Conrad. De richtprijs is ongeveer $\mbox{\euro}\ ??.??$
\subsection{Implementatie-specificaties}
Alvorens we dit spel kunnen implementeren, zullen we eerst moeten specificeren hoe we bijvoorbeeld met de gebruiker(s) gaan communiceren, hoe we de toestand van het bord gaan voorstellen, hoe het spel verloopt, enzovoort.
\paragraph{Uitvoer}
De uitvoer is een $3\times 3$ bord waar op elke tegel in principe een cirkel of kruis kan worden geplaatst. Dit geeft dus ongeveer\footnote{Niet alle toestanden zijn mogelijk: wanneer bijvoorbeeld de tweede speler wint, is er altijd \'e\'en vakje niet toegekend. Een situatie waarbij alle vakjes zijn opgevuld en er drie kruiss op \'e\'en rij staan is bijgevolg niet mogelijk.} $3^9$ toestanden. De uitvoeren realiseren we aan de hand van leds. Op elk vakje brengen we twee leds aan, bijvoorbeeld een rode en groene. We maken de afspraak dat indien de rode led brandt, er een cirkel op het respectievelijke vak staat. Wanneer de groene led brandt stelt dit een kruis voor. Wanneer geen van de twee leds brandt is het vakje leeg. In het vorige hoofdstuk hebben we reeds beargumenteerd dat men niet zomaar leds aan de uitgang van een poort kan schakelen. We voorzien dus een relay-mechanimsme met behulp van transistoren\footnote{We kunnen in principe ook echte relays gebruiken. Leds hebben echter een laag verbruik waardoor de taak ook door transistoren kan worden uitgevoerd.}. We dienen dus een schakeling te realiseren zoals op \figref{ledtttmatrix}.
\paragraph{}
Om te communiceren met de spelers is meer hardware nodig. Zo is het nuttig dat de schakeling aangeeft wie aan zet is, of de gespeelde zet legaal is en wie er gewonnen heeft. Daarom voorzien we nog twee leds: een rode en een groene. Wanneer \'e\'en rode let brandt, is de eerste speler aan zet, in het geval van de groene led speelt de tweede speler. Wanneer een speler gewonnen is knippert de overeenkomstige led. Wanneer een speler een onmogelijke zet speelt, branden beide leds voor een korte periode allebei.
\importtikzfigure{ledtttmatrix}{Een led-matrix en invoer-component voor het tic tac toe spel.}
\paragraph{Invoer}
We kunnen ervoor opteren om per vak een schakelaar te voorzien. Wanneer een speler dan op de schakelaar drukt, zal de schakeling de overeenkomstige led laten branden. Dit betekent echter dat we $9$ schakelaars moeten voorzien. We hebben daarom besloten om aan de rand van de twee dimensies elk $3$ schakelaars te voorzien. $3$ schakelaars laten dus toe om de rij te specificeren, met de overige $3$ kan een gebruiker de kolom aangeven. Ook vanuit educatief standpunt is deze beslissing positief: men zal immers meer logica moeten voorzien om de invoer te interpreteren.
\paragraph{Spelverloop}

\appendix
\part{Appendices}
\chapter{Conventies en Schemas}
%\section{Circuit conventies}
%\begin{figure}[H]
%\centering
%\begin{tikzpicture}
%\node (H1) at (0,0) {\textbf{\large{Component}}};
%\node (H2) at (8,0) {\textbf{\large{Circuit Symbol}}};
%\end{tikzpicture}
%\caption{Circuit conventies.}
%\end{figure}
\section{Lijst van component interfaces}
\begin{figure}[H]
\centering
\begin{tikzpicture}[yscale=-1,circuit logic US]
\filldraw[fill=black!3,draw=black,thick] (0,0) rectangle ++(15,14);
\foreach \x in {1,2} {
  \draw[thick] (5*\x,0) -- ++(0,14);
}
\foreach \x/\y/\t in {0/0/Poorten,0/9.25/Rekenkundig,1/0/Rekenkundig,1/4.75/Andere basisschakelingen,2/0/Andere basisschakelingen} {
  \filldraw[thick,draw=black,fill=black!50] (5*\x,\y) rectangle ++(5,0.5);
  \draw[thick] (5*\x+2.5,\y+0.25) node {\textbf{\t}};
}
\foreach \x/\y/\t in {0/0.5/AND-poort,0/1.75/OR-poort,0/3/NOT-poort,0/4.25/NAND-poort,0/5.5/NOR-poort,0/6.75/XOR-poort,0/8/XNOR-poort,0/9.75/Halve Opteller,0/11.875/Volledige Opteller,1/0.5/Carry-Lookahead Opteller (CLA),1/2.625/Carry-Lookahead Opteller-Generator,1/5.25/Multiplexer,1/7.4375/Decoder,1/9.625/Demultiplexer,1/11.8125/Encoder,2/0.5/Vergelijker} {
  \draw (5*\x,\y) -- ++(5,0);
  \node[scale=0.75,anchor=north] (T) at (2.5+5*\x,\y) {\t};
}
\node[and gate] (A) at (2.5,1.25) {};
\draw (A.input 1) -- ++(-0.25,0) node[anchor=east,scale=0.75]{$x$};
\draw (A.input 2) -- ++(-0.25,0) node[anchor=east,scale=0.75]{$y$};
\draw (A.output) -- ++(0.25,0) node[anchor=west,scale=0.75]{$f$};

\node[or gate] (O) at (2.5,2.5) {};
\draw (O.input 1) -- ++(-0.25,0) node[anchor=east,scale=0.75]{$x$};
\draw (O.input 2) -- ++(-0.25,0) node[anchor=east,scale=0.75]{$y$};
\draw (O.output) -- ++(0.25,0) node[anchor=west,scale=0.75]{$f$};

\node[not gate] (N) at (2.5,3.75) {};
\draw (N.input) -- ++(-0.25,0) node[anchor=east,scale=0.75]{$x$};
\draw (N.output) -- ++(0.25,0) node[anchor=west,scale=0.75]{$f$};

\node[nand gate] (NA) at (2.5,5) {};
\draw (NA.input 1) -- ++(-0.25,0) node[anchor=east,scale=0.75]{$x$};
\draw (NA.input 2) -- ++(-0.25,0) node[anchor=east,scale=0.75]{$y$};
\draw (NA.output) -- ++(0.25,0) node[anchor=west,scale=0.75]{$f$};

\node[nor gate] (NO) at (2.5,6.25) {};
\draw (NO.input 1) -- ++(-0.25,0) node[anchor=east,scale=0.75]{$x$};
\draw (NO.input 2) -- ++(-0.25,0) node[anchor=east,scale=0.75]{$y$};
\draw (NO.output) -- ++(0.25,0) node[anchor=west,scale=0.75]{$f$};

\node[xor gate] (XO) at (2.5,7.5) {};
\draw (XO.input 1) -- ++(-0.25,0) node[anchor=east,scale=0.75]{$x$};
\draw (XO.input 2) -- ++(-0.25,0) node[anchor=east,scale=0.75]{$y$};
\draw (XO.output) -- ++(0.25,0) node[anchor=west,scale=0.75]{$f$};

\node[xnor gate] (XNO) at (2.5,8.75) {};
\draw (XNO.input 1) -- ++(-0.25,0) node[anchor=east,scale=0.75]{$x$};
\draw (XNO.input 2) -- ++(-0.25,0) node[anchor=east,scale=0.75]{$y$};
\draw (XNO.output) -- ++(0.25,0) node[anchor=west,scale=0.75]{$f$};

\node[halfadder,scale=0.75] (HA) at (2.5,11) {HA};
\draw (HA.y) -- ++(0,-0.2) node[anchor=south,scale=0.75]{$x$};
\draw (HA.x) -- ++(0,-0.2) node[anchor=south,scale=0.75]{$y$};
\draw (HA.s) -- ++(0,0.2) node[anchor=north,scale=0.75]{$s$};
\draw (HA.co) -- ++(-0.2,0) node[anchor=east,scale=0.75]{$c_o$};

\node[fulladder,scale=0.75] (FA) at (2.5,13.125) {FA};
\draw (FA.y) -- ++(0,-0.2) node[anchor=south,scale=0.75]{$x$};
\draw (FA.x) -- ++(0,-0.2) node[anchor=south,scale=0.75]{$y$};
\draw (FA.s) -- ++(0,0.2) node[anchor=north,scale=0.75]{$s$};
\draw (FA.co) -- ++(-0.2,0) node[anchor=east,scale=0.75]{$c_o$};
\draw (FA.ci) -- ++(0.2,0) node[anchor=west,scale=0.75]{$c_i$};

\node[cla,scale=0.75] (CLA) at (7.5,1.75) {};
\draw (CLA.x) -- ++(0,-0.2) node[anchor=south,scale=0.75]{$x$};
\draw (CLA.y) -- ++(0,-0.2) node[anchor=south,scale=0.75]{$y$};
\draw (CLA.g) -- ++(0,0.2) node[anchor=north,scale=0.75]{$g$};
\draw (CLA.p) -- ++(0,0.2) node[anchor=north,scale=0.75]{$p$};
\draw (CLA.c) -- ++(0.2,0) node[anchor=west,scale=0.75]{$c$};
\draw (CLA.s) -- ++(-0.2,0) node[anchor=east,scale=0.75]{$s$};

\node[clag3,scale=0.65] (CLAG) at (7.75,3.875) {};
\draw (CLAG.pa) -- ++(0,-0.2) node[anchor=south,scale=0.75]{$p_0$};
\draw (CLAG.ga) -- ++(0,-0.2) node[anchor=south,scale=0.75]{$g_0$};
\draw (CLAG.pb) -- ++(0,-0.2) node[anchor=south,scale=0.75]{$p_1$};
\draw (CLAG.gb) -- ++(0,-0.2) node[anchor=south,scale=0.75]{$g_1$};
\draw (CLAG.pc) -- ++(0,-0.2) node[anchor=south,scale=0.75]{$p_2$};
\draw (CLAG.gc) -- ++(0,-0.2) node[anchor=south,scale=0.75]{$g_2$};
\draw (CLAG.ca) -- ++(0,0.2) node[anchor=north,scale=0.75]{$c_1$};
\draw (CLAG.cb) -- ++(0,0.2) node[anchor=north,scale=0.75]{$c_2$};
\draw (CLAG.cc) -- ++(0,0.2) node[anchor=north,scale=0.75]{$c_3$};
\draw (CLAG.pab) -- ++(-0.2,0) node[anchor=east,scale=0.75]{$g_{0,2}$};
\draw (CLAG.gab) -- ++(-0.2,0) node[anchor=east,scale=0.75]{$p_{0,2}$};
\draw (CLAG.c) -- ++(0.2,0) node[anchor=west,scale=0.75]{$c$};
\node[mux4to1,scale=0.85] (MUX) at (7.5,6.59375) {};%+1.34375
\draw (MUX.data0) -- ++(0,-0.2) node[anchor=south,scale=0.75]{$d_0$};
\draw (MUX.data1) -- ++(0,-0.2) node[anchor=south,scale=0.75]{$d_1$};
\draw (MUX.data2) -- ++(0,-0.2) node[anchor=south,scale=0.75]{$d_2$};
\draw (MUX.data3) -- ++(0,-0.2) node[anchor=south,scale=0.75]{$d_3$};
\draw (MUX.selin0) -- (MUX.selin0 -| 6.65,0) node[anchor=east,scale=0.75]{$s_0$};
\draw (MUX.selin1) -- (MUX.selin1 -| 6.65,0) node[anchor=east,scale=0.75]{$s_1$};
\draw (MUX.selout0) -- (MUX.selout0 -| 8.35,0) node[anchor=west,scale=0.75]{$s_0$};
\draw (MUX.selout1) -- (MUX.selout1 -| 8.35,0) node[anchor=west,scale=0.75]{$s_1$};
\draw (MUX.output) -- ++(0,0.2) node[anchor=north,scale=0.75]{$f$};
\node[decoder2to4,scale=0.85] (DEC) at (7.5,8.25) {Decoder};
\node[demux1to4,scale=0.85] (DEM) at (7.5,10.25) {Demux};
\node[encoder4to2,scale=0.85] (ENC) at (7.5,12.25) {Encoder};
%\node[comp,scale=0.85] (COM) at (7.5,14.25) {Comp};
\end{tikzpicture}
\caption{Lijst van component interfaces.}
\end{figure}
\begin{figure}[H]
\centering
\begin{tikzpicture}[circuit logic US,yscale=-1]
\draw[thick] (0,0) rectangle (15,12);
\draw (0,0) node[anchor=north west] {\Large Basispoorten};
\begin{scope}[xshift=2.5 cm,yshift=1.5 cm]
\draw (0,-0.75) node {\underline{NOT}};
\node[not gate] (N) at (-1.25,0) {};
\draw (N.output) -- ++(0.25,0) node[anchor=west]{$z$};
\draw (N.input) -- ++(-0.25,0) node[anchor=east]{$x$};
\end{scope}
\begin{scope}[xshift=7.5 cm,yshift=1.5 cm]
\draw (0,-0.75) node {\underline{AND}};
\node[and gate] (A) at (-1.25,0) {};
\draw (A.output) -- ++(0.25,0) node[anchor=west]{$z$};
\draw (A.input 1) -- ++(-0.25,0) node[anchor=east]{$x$};
\draw (A.input 2) -- ++(-0.25,0) node[anchor=east]{$y$};
\end{scope}
\begin{scope}[xshift=12.5 cm,yshift=1.5 cm]
\draw (0,-0.75) node {\underline{OR}};
\node[or gate] (O) at (-1.25,0) {};
\draw (O.output) -- ++(0.25,0) node[anchor=west]{$z$};
\draw (O.input 1) -- ++(-0.25,0) node[anchor=east]{$x$};
\draw (O.input 2) -- ++(-0.25,0) node[anchor=east]{$y$};
\end{scope}
\draw (5,0.25) -- ++(0,2.5);
\draw (10,0.25) -- ++(0,2.5);
\draw[thick] (0,3) node[anchor=north west] {\Large{Complexe Poorten}} -- (15,3);
\begin{scope}[xshift=2.5 cm,yshift=4.5 cm]
\draw (0,-0.75) node {\underline{NAND}};
\node[nand gate] (N) at (-1.25,0) {};
\draw (N.output) -- ++(0.25,0) node[anchor=west]{};
\draw (N.input 1) -- ++(-0.25,0) node[anchor=east]{$x$};
\draw (N.input 2) -- ++(-0.25,0) node[anchor=east]{$y$};
\end{scope}
\begin{scope}[xshift=7.5 cm,yshift=4.5 cm]
\draw (0,-0.75) node {\underline{NOR}};
\node[nor gate] (A) at (-1.25,0) {};
\draw (A.output) -- ++(0.25,0) node[anchor=west]{};
\draw (A.input 1) -- ++(-0.25,0) node[anchor=east]{$x$};
\draw (A.input 2) -- ++(-0.25,0) node[anchor=east]{$y$};
\end{scope}
\begin{scope}[xshift=12.5 cm,yshift=4.5 cm]
\draw (0,-0.75) node {\underline{XOR}};
\node[xor gate] (O) at (-1.25,0) {};
\draw (O.output) -- ++(0.25,0) node[anchor=west]{};
\draw (O.input 1) -- ++(-0.25,0) node[anchor=east]{$x$};
\draw (O.input 2) -- ++(-0.25,0) node[anchor=east]{$y$};
\end{scope}
\draw (5,3.25) -- ++(0,2.5);
\draw (10,3.25) -- ++(0,2.5);
\end{tikzpicture}
\caption{Samenvattend schema: poorten en componenten (deel 1)}
\end{figure}
\section{Conventies}
\section{Poorten}
\subsection{Basispoorten}
\subsection{Complexe Poorten}
\label{ssc:appendixComplexePoorten}
\section{Componenten}
\subsection{Rekenkundige schakelingen}
\subsection{Geheugen schakelingen}
\subsection{Andere schakelingen}
\subsubsection{Multiplexer}
\begin{figure}
\centering
\subfigure[Multiplexer]{
\begin{tikzpicture}[circuit logic US,rotate=-90]
\node[or gate,inputs={normal,normal,normal,normal}] (O) at (0,0) {};
\draw (O.output) -- ++(0.5,0) node[anchor=north]{$f$};
\draw[dashed] (0.75,-1.75) -- (-2.75,-2.75) -- (-2.75,2.75)  to node[gray,rotate=-90,below,sloped]{MULTIPLEXER} (0.75,1.75) -- cycle;
\draw (-2.25,-3) node[anchor=east]{$s_0$} -- (-2.25,2);
\draw (-2.5,-3) node[anchor=east]{$s_1$} -- (-2.5,2);
\node[and gate,inputs={normal,inverted,inverted}] (A0) at (-1.5,1.5) {};
\draw (A0.input 1 -| -3.25,0) node[anchor=south]{$d_0$} -- (A0.input 1);
\draw (A0.input 2 -| -2.5,0) -- (A0.input 2);
\draw (A0.input 3 -| -2.25,0) -- (A0.input 3);
\node[and gate,inputs={normal,inverted,normal}] (A1) at (-1.5,0.5) {};
\draw (A1.input 1 -| -3.25,0) node[anchor=south]{$d_1$} -- (A1.input 1);
\draw (A1.input 2 -| -2.5,0) -- (A1.input 2);
\draw (A1.input 3 -| -2.25,0) -- (A1.input 3);
\node[and gate,inputs={normal,normal,invertedl}] (A2) at (-1.5,-0.5) {};
\draw (A2.input 1 -| -3.25,0) node[anchor=south]{$d_2$} -- (A2.input 1);
\draw (A2.input 2 -| -2.5,0) -- (A2.input 2);
\draw (A2.input 3 -| -2.25,0) -- (A2.input 3);
\node[and gate,inputs={normal,normal,normal}] (A3) at (-1.5,-1.5) {};
\draw (A3.input 1 -| -3.25,0) node[anchor=south]{$d_3$} -- (A3.input 1);
\draw (A3.input 2 -| -2.5,0) -- (A3.input 2);
\draw (A3.input 3 -| -2.25,0) -- (A3.input 3);
\draw (A0.output) -- ++(0.3,0) |- (O.input 1);
\draw (A1.output) -- ++(0.2,0) |- (O.input 2);
\draw (A2.output) -- ++(0.2,0) |- (O.input 3);
\draw (A3.output) -- ++(0.3,0) |- (O.input 4);
\end{tikzpicture}
}
\subfigure[Decoder]{
\begin{tikzpicture}[circuit logic US,rotate=-90]
\draw[dashed] (-0.75,-2.5) -- (-3,-2.5) -- (-3,2.5) to node[gray,rotate=-90,below,sloped]{DECODER} (-0.75,2.5) -- cycle;
\draw (-2.25,-2) -- (-2.25,2);
\draw (-2.5,-2) -- (-2.5,2);
\draw (-2.75,-3) node[anchor=east]{enable} -- (-2.75,2);
\draw (-3.25,1) node[anchor=south]{$a_0$} -- (-2.25,1);
\draw (-3.25,-1) node[anchor=south]{$a_1$} -- (-2.5,-1);
\node[and gate,inputs={normal,inverted,inverted}] (A0) at (-1.5,1.5) {};
\draw (A0.input 1 -| -2.75,0) -- (A0.input 1);
\draw (A0.input 2 -| -2.5,0) -- (A0.input 2);
\draw (A0.input 3 -| -2.25,0) -- (A0.input 3);
\draw (A0.output) -- ++(0.75,0) node[anchor=north]{$s_0$};
\node[and gate,inputs={normal,inverted,normal}] (A1) at (-1.5,0.5) {};
\draw (A1.input 1 -| -2.75,0) -- (A1.input 1);
\draw (A1.input 2 -| -2.5,0) -- (A1.input 2);
\draw (A1.input 3 -| -2.25,0) -- (A1.input 3);
\draw (A1.output) -- ++(0.75,0) node[anchor=north]{$s_1$};
\node[and gate,inputs={normal,normal,invertedl}] (A2) at (-1.5,-0.5) {};
\draw (A2.input 1 -| -2.75,0) -- (A2.input 1);
\draw (A2.input 2 -| -2.5,0) -- (A2.input 2);
\draw (A2.input 3 -| -2.25,0) -- (A2.input 3);
\draw (A2.output)-- ++(0.75,0) node[anchor=north]{$s_2$};
\node[and gate,inputs={normal,normal,normal}] (A3) at (-1.5,-1.5) {};
\draw (A3.input 1 -| -2.75,0) -- (A3.input 1);
\draw (A3.input 2 -| -2.5,0) -- (A3.input 2);
\draw (A3.input 3 -| -2.25,0) -- (A3.input 3);
\draw (A3.output) -- ++(0.75,0) node[anchor=north]{$s_3$};
\end{tikzpicture}
}
\subfigure[Demultiplexer]{
\begin{tikzpicture}[circuit logic US,rotate=-90]
\draw[dashed] (-0.75,-2.5) -- (-3,-2.5) -- (-3,2.5) to node[gray,rotate=-90,below,sloped]{DEMUX} (-0.75,2.5) -- cycle;
\draw (-2.75,-2) -- (-2.75,2);
\draw (-2.5,-3) node[anchor=east]{$s_1$} -- (-2.5,2);
\draw (-2.25,-3) node[anchor=east]{$s_0$} -- (-2.25,2);
\draw (-3.25,0) node[anchor=south]{$f$} -- (-2.75,0);
\node[and gate,inputs={normal,inverted,inverted}] (A0) at (-1.5,1.5) {};
\draw (A0.input 1 -| -2.75,0) -- (A0.input 1);
\draw (A0.input 2 -| -2.5,0) -- (A0.input 2);
\draw (A0.input 3 -| -2.25,0) -- (A0.input 3);
\draw (A0.output) -- ++(0.75,0) node[anchor=north]{$d_0$};
\node[and gate,inputs={normal,inverted,normal}] (A1) at (-1.5,0.5) {};
\draw (A1.input 1 -| -2.75,0) -- (A1.input 1);
\draw (A1.input 2 -| -2.5,0) -- (A1.input 2);
\draw (A1.input 3 -| -2.25,0) -- (A1.input 3);
\draw (A1.output) -- ++(0.75,0) node[anchor=north]{$d_1$};
\node[and gate,inputs={normal,normal,invertedl}] (A2) at (-1.5,-0.5) {};
\draw (A2.input 1 -| -2.75,0) -- (A2.input 1);
\draw (A2.input 2 -| -2.5,0) -- (A2.input 2);
\draw (A2.input 3 -| -2.25,0) -- (A2.input 3);
\draw (A2.output)-- ++(0.75,0) node[anchor=north]{$d_2$};
\node[and gate,inputs={normal,normal,normal}] (A3) at (-1.5,-1.5) {};
\draw (A3.input 1 -| -2.75,0) -- (A3.input 1);
\draw (A3.input 2 -| -2.5,0) -- (A3.input 2);
\draw (A3.input 3 -| -2.25,0) -- (A3.input 3);
\draw (A3.output) -- ++(0.75,0) node[anchor=north]{$d_3$};
\end{tikzpicture}}
\subfigure[Encoder]{
\begin{tikzpicture}[circuit logic US,rotate=-90]
\draw[dashed] (0,-3) -- (-3,-3) -- (-3,3) to node[gray,rotate=-90,below,sloped]{ENCODER} (0,3) -- cycle;
\coordinate (S0) at (-3,1.5);
\coordinate (S1) at (-3,0.5);
\coordinate (S2) at (-3,-0.5);
\coordinate (S3) at (-3,-1.5);
\coordinate (F0) at (0,1);
\coordinate (F1) at (0,-1);
\draw (-3.25,1.5) node[anchor=south]{$s_0$} -- (S0);
\draw (-3.25,0.5) node[anchor=south]{$s_1$} -- (S1);
\draw (-3.25,-0.5) node[anchor=south]{$s_2$} -- (S2);
\draw (-3.25,-1.5) node[anchor=south]{$s_3$} -- (S3);
\draw (F0) -- ++(0.5,0) node[anchor=north]{$f_0$};
\draw (F1) -- ++(0.5,0) node[anchor=north]{$f_1$};
\node[or gate,inputs={normal,normal,normal,normal},rotate=-90] (O1) at (-2.5,-2.15) {};
\draw (O1.output) -- ++(0,-0.5) node[anchor=east]{any};
\draw (S0) -| (O1.input 1);
\draw (S1) -| (O1.input 2);
\draw (S2) -| (O1.input 3);
\draw (S3) -| (O1.input 4);
\node[or gate] (O2) at (-1.25,1) {};
\draw (S2 -| O1.input 3) |- (O2.input 2);
\draw (S3 -| O1.input 4) |- (O2.input 1);
\draw (O2.output) -- (F0);
\node[and gate, inputs={normal,inverted}] (A1) at (-1.5,0) {};
\draw (A1.input 1 -| O1.input 2) -- (A1.input 1);
\draw (A1.input 2 -| O1.input 3) -- (A1.input 2);
\node[or gate] (O3) at (-0.5,-1) {};
\draw (O1.input 4 |- O3.input 2) -- (O3.input 2);
\draw (A1.output) -- ++(0.1,0) |- (O3.input 1);
\draw (O3.output) -- (F1);
\end{tikzpicture}}
\caption{Andere combinatorische schakelingen}
\end{figure}
\section{Kostprijs van de Componenten}
\importtabulartable{cost-components}{De kostprijs van de verschillende componenten.}
\input{appendix_software}
\input{appendix_oplossingen}
\backmatter
%\printbibliography
\listoftables
\listoffigures
\listof{vhdlcode}{Lijst van VHDL-Codes}
\begin{twocolumn}
\nocite{*}
\bibliographystyle{alpha}
\bibliography{bibliography}
\label{reference}
\end{twocolumn}
\label{glos}
\printglossaries
\label{idx}
\printindex
% \chapter*{``And Now For Something Completely Different''}
% \begin{figure}[H]
% \centering
% \begin{tikzpicture}[scale=15]
% \def\h{1};
% \def\t{0.02};
% \def\ts{10};
% \fill (0,0) rectangle ++(-\t,\h);
% \fill (-0.5*\t,0.5*\h-0.5*\t) rectangle ++(0.5*\h+1.5*\t,\t);
% \fill (\h,0) arc (270:90:0.5*\h) -- (\h,\h-\t) arc (90:270:0.5*\h-\t) -- (\h,0);
% \draw (0.75*\h,0.5*\h) node[scale=\ts]{$\mathfrak{K}$};
% \foreach\s/\n/\tx in {0.5/1/S,0.25/3/O,0.125/7/O,0.0625/15/M,0.03125/31/F,0.015625/63/M,0.0078125/127/T,0.00390625/255/U} {
%   \foreach \y in {0,...,\n} {
%     \begin{scope}[yshift=\y*\s*\h cm,scale=\s]
%     \fill (-0.5*\t,0.5*\h-0.5*\t) rectangle ++(0.5*\h+1.5*\t,\t);
%     \fill (\h,0) arc (270:90:0.5*\h) -- (\h,\h-\t) arc (90:270:0.5*\h-\t) -- (\h,0);
%     \draw (0.75*\h,0.5*\h) node[scale=\ts*\s]{$\mathfrak{\tx}$};
%     \end{scope}
%   }
% }
% \end{tikzpicture}
% \end{figure}
\end{document}
%Start 19/03/2011
%Chapter 1: 28/03/2011
%Chapter 2: 15/12/2013
%Chapter 3: 
%Chapter 4: 
%Chapter 5: 11/01/2014
%Chapter 6:
%Chapter 7:
%Chapter 8:
%Appendix A:
%Appendix B:
%Appendix C: